\section{Methodology}\label{sec-methodology}

To assess the performance of each set of rules, we required a collection of
correctly hyphenated Portuguese words. We employed word frequency data to
ensure we selected representative, commonly used words, avoiding excessively
rare ones. Our comprehensive word list was curated from online dictionaries,
and Portuguese corpora, and verified against Portuguese grammars to validate
hyphenation rules. We also performed manual corrections, where necessary.
We should not expect an entire corpus to follow a set of conventions for a written 
language nor uniformity among a variety of corpora, especially those harvested from the Internet.
``It is notoriously difficult to prescribe rules governing the use of a written language; it is even more difficult
to get people to `follow the rules.' This is in large part due to the nature of written language, in which
the conventions are not always in line with actual usage and are subject to frequent change'' \parencite[p.~14]{palmer2010}.



\subsection{Data collection}\label{sec-data-coll}

Initially, we selected
\href{https://www.linguateca.pt/cetenfolha/index_info.html}{CETENFolha}
as the source corpus, but this approach presented two issues: many
words were still missing, and the corpus was based on text from 1994, prior to
the implementation of the Orthographic Agreement. The Orthographic Agreement
of the Portuguese Language was conducted in 1990, and its transition period
started in 2009, becoming mandatory in 2016. However, even after the agreement,
some words continue to have different spellings in the participating countries.
To address these idiosyncrasies, we aim to develop hyphenation rules that
accommodate variations in spelling. For instance, the word \emph{reception}
might be written as \emph{receção} (Portugal) or \emph{recepção} (Brazil); the
word \emph{action} might be written as \emph{acção} (Portugal) or \emph{ação}
(Brazil); and the word \emph{project} might be written as \emph{projecto}
(Portugal) or \emph{projeto} (Brazil).

% We have used the Portuguese Wikipedia dump as our Portuguese corpus. Texts in Wikipedia might be written by Brazilian or Portuguese contributors, the both spellings may appear.

% cat /ms/downloads/samples/wikipedia/ptwiki-latest-pages-articles-multistream_wordlist.txt | tr -dc '0-9\n' | paste -sd+ | bc
% 302226482
% cat /ms/downloads/samples/wikipedia/ptwiki-latest-pages-articles-multistream_wordlist.txt | awk '{CUMCOUNT+=$1; if(CUMCOUNT < 287115158) print;}' >/tmp/ptwikitop95.txt
% cat /ms/downloads/samples/wikipedia/ptwiki-latest-pages-articles-multistream_wordlist.txt | awk '{CUMCOUNT+=$1; if(CUMCOUNT < 299204217) print;}' >/tmp/ptwikitop99.txt
% cat /tmp/ptwikitop99.txt | tr -d "0-9 " |  while read -r word; do if ! grep -q "^${word}," ../data/hyphenations.csv; then echo $word; fi; done > /tmp/ptwikitop99_notinlist.txt
% cat /home/leoca/ee/research/hyphenation/listapalavrasnet.txt | awk '{print tolower($0)}' | while read -r word; do if ! grep -qP "^\s*[0-9]+\s*${word}$" /ms/downloads/samples/wikipedia/ptwiki-latest-pages-articles-multistream_wordlist.txt; then echo $word; fi; done > /tmp/list_of_palavrasnet_not_in_wikipedia.txt

% wc -l /ms/downloads/samples/wikipedia/ptwiki-latest-pages-articles-multistream_wordlist.txt
% 1746947
% wc -l /tmp/ptwikitop99_notinlist.txt
% 42017
% from 1746947 of wikipedia, only 42017 are not in list

Due to the peculiarities mentioned, we made the decision to integrate the word
list from \href{https://www.palavras.net/}{Palavras NET}, and subsequently, we
augmented it with words sourced from the
\href{https://pt.wikipedia.org}{Portuguese Wikipedia} dump. From the Wikipedia
data, we narrowed down the selection to a subset of \num{50721} words,
accounting for 95\% of occurrences in the corpus. This threshold was set to
filter out typos and infrequent words. The initial word list became quite
extensive, comprising \num{\CorpusSize} words. However, we refined the list by
retaining only those words for which we could find hyphenation data in at least
one of the following online dictionaries:
\href{https://michaelis.uol.com.br/}{Michaelis},
\href{https://dicionario.priberam.org/}{Priberam},
\href{https://pt.wiktionary.org}{Wikcionário},
\href{https://aulete.com.br/}{Aulete},
\href{http://www.portaldalinguaportuguesa.org/}{Portal da Língua Portuguesa}, and
\href{https://www.dicio.com.br/}{Dicio}.
Consequently, this curation process resulted in a final dictionary containing
\num{\DictionarySize} words.  

Table \ref{tblstats} presents the number of distinct words with a given number
of hyphenations found in the dictionaries.  Additionally, it displays the
number of words that received the highest agreed-upon hyphenation.  For
instance, the word \emph{como} was consistently hyphenated as \emph{co-mo} in
all six dictionaries, therefore, it contributes to both the counts in the
column representing six hyphenations (first column).  On the other hand, the
word \emph{sua} received different hyphenations: \emph{su-a} in four
dictionaries, \emph{sua} in one dictionary, and one dictionary did not provide
any hyphenation result.  This leads to an increment in the first line, second
column (since hyphenations were found in five dictionaries) and the second
line, third column (since the most frequent hyphenation was found in four
dictionaries). It is evident now that the number of identical hyphenations can
exceed the total number of hyphenations found, especially for smaller values.

\begin{table}
\centering
\caption{Considering the six dictionaries used, the first line of this table
presents the number of words that a given number of hyphenations were found in the
dictionaries. The second line presents the number of words that have a given
number of hyphenation as its most frequent form found in the
dictionaries.}\label{tblstats}
\small
\begin{tabular}{@{ }l@{ } *{7}{@{ }S[table-format=6]@{ }}}
    hyphenations & 6 & 5 & 4 & 3 & 2 & 1 & 0 \\
    \hline
    hyphenations found & \NumberOfSixHyphens{} & \NumberOfFiveHyphens{} &
    \NumberOfFourHyphen{} & \NumberOfThreeHyphens{} & \NumberOfTwoHyphens{} &
    \NumberOfOneHyphens{} & \NumberOfNoHyphens{} \\

    same hyphenations & \NumberOfSixAgrees{} & \NumberOfFiveAgrees{} &
    \NumberOfFourAgrees{} & \NumberOfThreeAgrees{} & \NumberOfTwoAgrees{} &
    \NumberOfOneAgrees{} & n/a
\end{tabular}
\end{table}

In summary, our data compilation involved \emph{CETENFolha}, \emph{Palavras
NET}, and \emph{Wikipedia}, and hyphenated the words using the original \TeX{}
hyphenation rules.  Subsequently, we compared the hyphenated results with
those obtained from the dictionary. By conducting a thorough performance
appraisal, we systematized the errors and delved into the relevant literature
to identify potential rules that could complement and enhance the existing
hyphenation system.


