\section{Concluding Remarks and Future Directions}\label{sec-conclusion}

In this study, we explored the default TeX hyphenation rules and proposed
additional rules to improve hyphenation performance in Portuguese. By
addressing the limitations of the old rules and existing approaches within the
TeX typesetting system, and by incorporating morphological and phonological
considerations, we significantly reduced the number of hyphenation errors.

However, the patched set of rules is still not enough to achieve perfect
accuracy, and it is not even desired since our dataset is noisy and there are
many dubious hyphenation cases. We have opted to create a concise set of rules
that could better generalize and, therefore, align more closely with the
underlying hyphenation rules in the language.

We also tested rules created by patgen, which generated an extensive set of
rules that were unable to generalize effectively. In contrast, our handcrafted
rules performed superiorly in a validation set.

In conclusion, while our enhanced rule set demonstrates significant
improvements in hyphenation accuracy, there could remains room for further
refinement.

Wide character support and universal syllabic segmentation, as considered by
\cite{sojka2023roadmap}, are additional points for consideration. Incorporating
regular expressions, creating classes of characters, and supporting word stress
positions might further enhance the efficiency and generality of the
hyphenation system.


