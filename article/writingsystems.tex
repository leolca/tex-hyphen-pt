\section{Writing systems}\label{sec-writing-systems}

There is a diverse collection of writing systems, categorized into logographic,
syllabic, and alphabetic systems. Although distinct, these systems can be built
on an interplay of these categories \parencite{coulmas2003,palmer2010}. 
The principles guiding an alphabetic or phonemic writing system vary significantly 
based on the language and its history. Here are some key points worth highlighting:
\begin{enumerate}[label=\arabic*)]
    \item The most common principle involves representing the sounds of a
	language with written symbols, using one or a combination of symbols to
	represent each sound (e.g., the orthographic system of Portuguese uses
	an alphabet of 27 letters to represent different sounds of the
	language).
    \item Etymology also plays a crucial role, where the spelling of a word
	reflects its origins and historical development (e.g., the French
	language often reflects the Latin roots of words in its spelling).
    \item Morphology serves as a guide in structuring many languages, using
	specific letters or symbols to indicate word endings, prefixes, or
	suffixes (e.g., Russian uses different forms of the Cyrillic alphabet
	to indicate gender and case in its nouns).
    \item The evolution of a language over time also contributes to its written
	form (e.g., the spelling of many English words has changed over time to
	reflect changes in pronunciation).
\end{enumerate}
Orthographic systems for languages with logographic writing systems, such as
Chinese ideograms, cuneiform writing, and Egyptian hieroglyphs, differ
significantly from alphabetic writing systems. In logographic systems,
individual symbols or characters represent entire words or ideas, rather than
phonemes or sounds. As a result, the principles guiding their orthographic
systems are based more on semantic and visual principles than on phonemic
principles.
Hyphenation is a feature of alphabetic scripts where words are composed of letters, 
and spaces or hyphens are used to separate words, syllables, or parts of words. 
Logographic writing systems, such as the Chinese, 
function differently and do not employ hyphenation \parencite{honorof2006}.



