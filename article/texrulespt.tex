\section{\TeX{} hyphenation rules for portuguese}\label{sec-tex-hyphen-pt}
\textcite{rezende1987} was the first to create patterns for Portuguese hyphenation
in \TeX{}. The pattern set was last updated in 2015, incorporating
contributions from %José João Dias Almeida 
\textcite{hyphpt}, resulting in a total
of 307 rules\footnote{The set of rules for Portuguese hyphenation is short when
compared to rules in other languages. For example, English currently has 4938
rules, Russian has 7023 rules and German has 34011 rules.}. 
These rules effectively hyphenate the majority of Portuguese words.

However, in light of certain cases, we propose an analysis of the default rules
to identify areas for improvement. By addressing specific issues and
considering non-typical patterns, we aim to enhance hyphenation accuracy. The
methodology used for this analysis is described in Section
\ref{sec-methodology}.

%
% FILE="../data/default.TeX.pt-br.patterns"; grep "^1..$" "$FILE" | awk '{print substr($0, 0, 2)}' | sort | uniq -c | while read line; do p=$(echo $line | cut -d' ' -f2); c=$(grep "^$p[^0-9]$" "$FILE" | awk '{print substr($0, 3)}' | tr '\n' ' '); printf "$line $c \n"; done | cut -d' ' -f1 | paste -sd+ | bc
%
Out of the default \TeX{} rules, 252 (82\%) follow the pattern \verb|1CV|, which
represents the recurring CV syllables in Portuguese.  As for the consonants,
there are typically 18 considered (\verb|b|, \verb|c|, \verb|ç|, \verb|d|,
\verb|f|, \verb|g|, \verb|j|, \verb|k|, \verb|l|, \verb|m|, \verb|n|, \verb|p|,
\verb|r|, \verb|s|, \verb|t|, \verb|v|, \verb|x|, and \verb|z|) that can
combine with 14 vowels (\verb|a|, \verb|e|, \verb|i|, \verb|o|, \verb|u|,
\verb|á|, \verb|â|, \verb|ã|, \verb|é|, \verb|í|, \verb|ó|, \verb|ú|, \verb|ê|,
and \verb|õ|), resulting in these CV patterns that indicate favorable
hyphenation points before the consonants\footnote{The complete list of these
252 patterns is given here: \ExceptionRulesCV{}.}. 
It is worth noting that there might
be other rules or exceptions in the hyphenation patterns not covered by these
default rules.  To accommodate exceptions, \textcite{rezende1987} proposes the 
following rules: 
\begin{itemize}
    \item 20 rules created for cases involving consonants \verb|b|, \verb|c|, \verb|d|, \verb|f|, \verb|g|, \verb|k|, \verb|p|, \verb|t|, \verb|v|, or \verb|w| followed by \verb|l| or \verb|r|\footnote{The 20 additional rules are: \ExceptionRulesLR{}.};
    \item 3 rules for \verb|c|, \verb|l| or \verb|n| followed by \verb|h|\footnote{Those 3 rules are: \ExceptionRulesCLNH{}.};
    % FILE="../data/default.TeX.pt-br.patterns"; grep -v "^1..$" "$FILE" | grep "2" | awk '{print substr($0, 2, 1)}' | sort | uniq | while read line; doc=$(grep "^1${line}2.$" "$FILE" | tr -d '12' | awk '{print substr($0, 2, 1)}' | tr '\n' ' '); printf "$line $c\n"; done
    \item 23 distinct patterns were introduced to indicate hyphenation points between vowels or between \verb|c|'s, \verb|r|'s, and \verb|s|'s\footnote{Those patterns are: \ExceptionRulesThree{}.}; 
    % FILE="../data/default.TeX.pt-br.patterns"; grep -v "^1..$" "$FILE" | grep -v "2" | grep "3" | wc -l
    \item 8 patterns adhere to the \verb|1[gq]u4V| pattern, signaling beneficial hyphenation points before \verb|g| or \verb|q|, followed by a sequence of \verb|u| and a vowel, with an inhibiting point between the \verb|u| and the subsequent vowel\footnote{They are the following: \ExceptionRulesFour{}.};
    \item 1 pattern represented as \verb|1-|, denoting that a hyphen indeed serves as a beneficial hyphenation point.
\end{itemize}

% - contabilizar a quantidade de palavras que não são contempladas pela proposta do autor
% - analisar a necessidade de excluir alguma regra (ao invés de apenas acrescentar novas)


