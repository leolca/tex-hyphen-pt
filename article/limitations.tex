\section{Limitations}

In general, although the erroneous words share some common characteristics that
might allow for a reduction in hyphenation errors to some extent, there is a
limitation inherent in the way \TeX{}'s rules are conceived. Below, we present
the systematics found in these data that could be encompassed by a different
rule structure:

\begin{description} 
    \item [Morphological determination\label{morpho-det}] Prefixes such as \emph{re-},
	\emph{sub-}, \emph{ciber-}, \emph{hiper-}, and \emph{auto-}, among others, require separating the
	prefix from its stem, which can lead to phonological issues. For example, words like
	\emph{reiniciar}, \emph{sublinhar}, \emph{ciberespaço}, \emph{hiperalgesia}, and \emph{autoimagem} contain prefixes and could be hyphenated as 
	\emph{re-i-ni-ci-ar}, \emph{sub-li-nhar}, \emph{ci-ber-es-pa-ço}, \emph{hi-per-al-ge-si-a}, and \emph{au-to-i-ma-gem}, 
	respectively, to respect their morphological formation. However,
        considering that Portuguese hyphenates its words based on syllabic phonological correlates, 
	words that require morphological information, such as these, might not have their hyphenation performed correctly. 
	%In our corpus, there are twelve such words (eight with \emph{re-}, and one each with \emph{auto-}, \emph{meta-}, \emph{sub-}, \emph{cyber-}, \emph{geo-}, or \emph{neuro-}). 
	These examples are presented in \Cref{tab-mor-syl}.
	% $ { head -n 1 ../data/hyphenations.csv; grep "^sublinhar," ../data/hyphenations.csv; } | column -t -s,
	% su-bli-nhar (priberam, wiktionary, aulete)
	% sub-li-nhar (michaelis, portal, dicio)

	% $ { head -n 1 ../data/hyphenations.csv; grep "^auto[a-z]\+," ../data/hyphenations.csv | grep -v ",,,";} | column -t -s,

	\begin{table}
	\centering
	\scriptsize
	\caption{Hyphenation in the dictionaries of words having conflict between morphological and syllabic information.}
	\label{tab-mor-syl}
        \begin{tabular}{ll*{6}{c}}
	\toprule
	    word                      & hyphenation       & \multicolumn{6}{c}{dictionary} \\
             &                                            & Michaelis & Priberam & Wikitionary & Aulete & Portal & Dicio \\
	\midrule
        \multirow{2}{*}{sublinhar}    & su-bli-nhar       &           & x        & x           & x      &        &       \\
				      & sub-li-nhar       & x         &          &             &        & x      &       \\
	\midrule
	\multirow{2}{*}{reiniciar}    & rei-ni-ci-ar      &           &          & x           &        & x      &       \\
				      & re-i-ni-ci-ar     & x         & x        &             & x      &        & x     \\
	\midrule
	\multirow{2}{*}{ciberespaço}  & ci-be-res-pa-ço   &           & x        &             &        & x      &       \\
				      & ci-ber-es-pa-ço   & x         &          & x           & x      &        & x     \\
	\midrule
	\multirow{2}{*}{hiperalgesia} & hi-pe-ral-ge-si-a & x         & x        &             & x      & x      & x     \\
				      & hi-per-al-ge-si-a &           &          & x           &        &        &       \\
	\midrule
	\multirow{2}{*}{autoimagem}   & au-toi-ma-gem     &           &          &             &        & x      &       \\
				      & au-to-i-ma-gem    & x         &          &             & x      &        & x     \\
	\bottomrule
	\end{tabular}
	\end{table}

    \item [Foreignness\label{foreignness}] There is a group of words %20 words 
	in the corpus that are terminologies or words incorporated into the 
	Portuguese language without full phonological adaptation, such as 
	\emph{darwinismo}, \emph{quilowatt}, and \emph{esfiha}. The lack of
	adaptation makes the phonological pattern very specific to the word,
	making it impossible to incorporate their cases \TeX{}'s rules alongside the
	other rules. The solution is to add them to the exception word list.

    \item [Word-initial consonant clusters\label{word-init-cc}] Portuguese has few cases of
	consonant clusters at the beginning of a word. They are, in general, 
	etymological remnants and are currently unproductive in the language,
	since there are no neologisms with this pattern. Encounters like
	\emph{ps-} and \emph{pn-} are more frequent, as they are present in
	words like \emph{psicologia} and \emph{pneu}, which have moderate 
	frequency in the language. These can be predicted by specific rules that would cover  
	49 and 13 words in the corpus, respectively.
	However, there are consonant clusters that are
	found in very specific and low-frequency words. Although possible, it
	is not worth adding very specific rules for clusters found in words like
	\emph{dzeta}, \emph{gnu}, \emph{cnidário},
	\emph{ftálico}, and \emph{gnaisse} -- which amount to only five words.

    \item [Abbreviations, Acronyms, or Initialisms\label{abbrev}] 
       Whether for efficiency, convenience, clarity, or specialized jargon, 
       it is common to use shortened versions of words or phrases. \emph{Abbreviation} is 
       a method employed to achieve this shortening. In our Portuguese corpus, we find examples such as 
       \emph{etc.}\footnote{Latin expression \emph{et cetera}, meaning `and other similar things'.}, 
       \emph{Dr.}\footnote{\emph{doutor} (doctor, person with PhD title, but popularly used to designate an erudite individual)}, 
       \emph{Exmo.}\footnote{\emph{Excelentíssimo} (honourable)},
       \emph{cap.}\footnote{\emph{capítulo} (chapter)}, \emph{Univ.}\footnote{Universidade (university)}, 
       \emph{ed.}\footnote{\emph{edição} (edition)}, 
       \emph{s.n.}\footnote{sine nomine, Latin expression meaning `without a name', mostly used in the context of publishing.}.
       Another shortened form is an \emph{initialism}, which consists of using
       the initial letters of words to create a shortened version. However,
       initialisms may not always conform to the hyphenation rules described in
       this work, as they do not necessarily follow the orthographic or
       phonotactic standards of the language. Some abbreviations found in the
       corpus include 
       \emph{SESC}\footnote{Serviço Social do Comércio}, 
       \emph{INSS}\footnote{\emph{Instituto Nacional do Seguro Social} (National Institute of Social Security)}, 
       \emph{PCdoB}\footnote{\emph{Partido Comunista do Brasil} (Communist Party of Brazil)},
       \emph{PM}\footnote{\emph{Polícia Militar} (military police)}, and 
       \emph{UFRJ}\footnote{Universidade Federal do Rio de Janeiro}.
       \emph{Acronyms} are a specific type of shortening where the first letters (or
       groups of letters) of each word are combined to form a new pronounceable
       word. In the corpus, we encounter examples like 
       \emph{Anatel}\footnote{\emph{Agência Nacional de Telecomunicações} (National Telecommunications Agency)},
       \emph{Ovni}\footnote{\emph{Objeto voador não identificado} (unidentified flying object - UFO)},
       \emph{Sida}\footnote{\emph{Síndrome da Imunodeficiência Adquirida} (acquired immunodeficiency syndrome - AIDS)} (in Portugal) and
       \emph{Mercosul}\footnote{\emph{Mercado Comum do Sul} (Southern Common Market).}.
       These various shortened forms play an important role in written
       language, providing concise ways to represent longer words or phrases.
       It is important to note that abbreviations, acronyms, and initialisms
       are generally treated as single units and are not hyphenated.

\end{description}



