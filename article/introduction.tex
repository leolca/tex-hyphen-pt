\section{Introduction}\label{sec-intro}

Hyphenation in text wrapping was not used for a long time. Words should fit
entirely in a line, or they would be broken in arbitrary places.  Initially, no
markers were used to indicate word wrapping, leading to potential confusion and
unintended interpretations. As a result, orthographers advocated for the
introduction of a sign to indicate such breaks. Portuguese faced the same
gradual introduction of a hyphenation sign to mark words wrapping across lines.
Even though the usage of a hyphenation sign ($=$) was advocated by
orthographers \parencite{gandavo1574}, few documents used such sign until the end of
the 18th century \parencite{araujo2015}.

In some cases, hyphenation hinders smooth reading and should be avoided in
child literature. In opposition, both excessive spacing and insufficient
spacing between words also impose difficulty in the reading process, making
hyphenation fundamental when texts use short line lengths. As a line gets
shorter, the number of breaking candidates between words decreases, leading to
awkward spaces between words and among letters. For that matter, automatic
hyphenation plays an important role in good typesetting.

%Automatic hyphenation plays an important role in good typesetting and
%it is fundamental when a text uses short line lengths. As a line gets
%shorter, the number of breaking candidates between words decreases,
%creating awkward spaces between words and among letters.

\TeX{} is a typesetting system which carefully deals with these issues,
automatically arranging text on a page to create a good reading experience.
Automatic hyphenation is an important part on this process,
promoting an even-tempered distribution of elements on the page.
%Hyphenation is therefore an important part on this process.
%A good readability depends on an even-tempered distribution of elements
%on the page.
Line height and line length, paragraph length, font size and typeface,
letter-spacing, and word-spacing are some factors which influence the text legibility
and readability. Space between words should not be too long creating lakes and
rivers in the text, nor too tight, impairing the legibility and readability.

Another important matter to consider is ambiguities that might be created when
a word is partitioned during the hyphenation process. In English we should
avoid hyphenations such as \emph{re-cover}, \emph{re-form}, \emph{re-sign}, \emph{the-rapist},
\emph{depart-mental}, and \emph{mans-laughter} (in Portuguese, some examples are:
\emph{de-putada}, \emph{fede-ração}, \emph{acu-mula}, \emph{após-tolo},
\emph{cú-bico}). Hyphenations that might lead the reader to pronounce a word
incorrectly should also be prevented. That is the case of
\emph{considera-tion}, in Enslish (and \emph{pe-rigo} in Portuguese).

In some situations, hyphenation is also a matter of style. Some partitioning
choices sound better than others. These conflicting alternatives typically
arises when a words has many possible hyphenation points. Consider
\emph{ar-chae-ol-o-gist} (\emph{or ar-che-ol-o-gist}), which is preferably partitioned as
\emph{archae-ologist} (or \emph{arche-ologist}) in opposition to \emph{archaeol-ogist} (or
\emph{archeol-ogist}) or \emph{archaeolo-gist} (or \emph{archeolo-gist}). It is preferable to keep
whole morphemes together. In the previous example: \emph{archae} (or \emph{arche}, meaning
``ancient'', ``primitive'')
%\footnote{archae- comes from the Ancient Greek ἀρχαῖος (arkhaîos, `ancient',
%`primitive'), from ἀρχή (arkhḗ, `beginning').}
%https://en.wiktionary.org/wiki/archae-
and \emph{-ologist} (``one who studies the topic''). In Portuguese, it is also more
elegant to avoid splits between double consonants or vowels, even if an
hyphenation point do exists between those letters. For example,
\emph{pressu-rizar} is preferable over \emph{pres-surizar} and
\emph{empreen-dedor} is preferable over \emph{empre-endedor}. Even so,
exceptions exists, it is preferable to partition \emph{micro-organismo}
rather than \emph{microor-ganismo} (keeping morphemes together is favored
over splitting a double vowel). Numerous factors come into play when choosing
the optimal hyphenation point for a word.

The general rule for Portuguese hyphenation is to split a word into its
syllables. A syllable is made of a mandatory nucleus, filled by a vowel, and
optional peripheral consonants (before or after the nucleus). In some
situations, the syllabic division does not respect the ethnologic constituents.
The usage of the prefixes \emph{bis-} and \emph{in-} are examples of this
circumstance. The correct syllabifications are \emph{bi-sa-vô},
\emph{i-nobs-tan-te}\footnote{The rule of syllable
division could lead to two possible partitions: \emph{i-nobs-tan-te} and
\emph{in-obs-tan-te}, but the first is preferable.}, and \emph{i-na-ti-vo}, 
where the prefixes are split into two syllables. 
But, as pointed out previously, it is preferable to keep
morphemes together rather than splitting them apart. Therefore we should
favor the hyphenation \emph{bisa-vô} over \emph{bi-savô}, \emph{inobs-tante} 
over \emph{i-nobstante}, and \emph{ina-tivo} over \emph{i-nativo}. 
This last example could also lead to a misunderstanding
since the word \emph{nativo} (\emph{native} in English) emerges from this word
break. 

Each language has its hyphenation rules, which can be categorized into two
groups: those driven by morphology (etymology) and those driven by
pronunciation.  An algorithmic approach employs a logical system to analyze
words and apply the hyphenation rules of a specific language. Since hyphenation
rules vary significantly between languages, an algorithm must be
developed for each one. While a logic-based system can be efficient and
compact, it still needs to address exceptions through hard-coded rules.

%The automation of this process might use a dictionary based approach, which
%will restrict the hyphenation possibilities to those entries in the dictionary,
%an algorithmic base approach, which might be applied to whatever sequence found
%in a text, or a mix of both approaches. 
%%An algorithmic approach might use a logic system to analyse words and apply
%%the hyphenation rules of a given language. 
%A rule base model might include the recognition of prefixes,
%suffixes, morphemes, or some sequences suitable for the inclusion of a break.
%Another approach is the usage of pattern matching. Given a corpus of
%hyphenation examples in a language, it is necessary to identify those sequence
%of letters that define a point where hyphenation is suitable or not. Patterns
%might include prefixes, suffixes, exceptions and special hyphenation rules of a
%given language. Such approach might be used for different languages, by
%providing hyphenation examples in a target language to extract patterns and
%rules.

The automation of this process can employ various approaches, including:
\begin{description}
\item[Dictionary-Based Approach:] This method restricts hyphenation
    possibilities to entries found in the dictionary.
\item[Algorithmic-Based Approach:] This approach can be applied to any sequence
    encountered in a text.
\item[Rule-Based Model:] This model recognizes prefixes, suffixes, morphemes,
    or specific sequences suitable for hyphenation.
\item[Pattern Matching:] By using a corpus of hyphenation examples in a
    language, this approach identifies letter sequences that determine suitable
    hyphenation points. Patterns encompass prefixes, suffixes, exceptions, and
    special hyphenation rules of the language.
\item[Mixed Approach:] This approach combines two or more of the previously
    described methods to enhance hyphenation accuracy and flexibility.
\end{description}

Analyzing only the immediate surroundings may not always suffice to determine a
potential hyphenation point. For instance, consider \emph{de-moc-ra-cy} and \emph{dem-o-crat},
where the immediate surrounding of the letter \emph{e} does not provide a clear
indication of the hyphenation point. 
Even when facing chosen patterns that typically make hyphenation
straightforward, exceptions can arise. Take the sequence \emph{tion}, for example.
It may seem logical to place a break before this pattern, but the word \emph{cation}
is hyphenated as \emph{cat-ion}, highlighting the influence of etymology on the way
a word is split.

A word with multiple meanings may have distinct hyphenations based on its intended meaning.
`For instance, the Swedish word form \emph{glassko} has three different
meanings, and can be hyphenated as \emph{glas-sko} (glass shoe),
\emph{glass-ko} (ice cream cow) and in the non-standard way,
\emph{glass-sko} (ice cream shoe)' \parencite{nemeth2006}. In Portuguese, the
word \emph{sublinha} might be hiphenated in two different ways:
\emph{su-bli-nha} when representing the inflected form of the verb
\emph{sublinhar} (to underline) or \emph{sub-li-nha} when refering
to the line under (underline as a noun).

% hyphenation might change with syntactic function
% part of speech (PoS)

Machine learning techniques were employed to hyphenate Norwegian text
\parencite{kristensen2001}. The study revealed that, overall, the \TeX{} approach
outperformed a neural network.  Both methods demonstrated similar performance
in identifying correct hyphenation points and minimizing incorrect hyphenations
when tested on a small word list (with the neural network slightly
outperforming in correctly recognizing hyphenation points).  However, when
tested on a larger word list, the \TeX{} approach proved superior in avoiding
incorrect hyphenations compared to the neural network.


The original \TeX{} hyphenation algorithm, introduced by %Knuth in 1977
\textcite{knuth1977}, primarily focused on the English language and employed three
main rules:
\begin{enumerate*}[label=(\arabic*)]
 \item suffix removal; 
 \item prefix removal; and 
 \item the Vowel-Consonant-Consonant-Vowel (VCCV) breaking rule\footnote{The 
     VCCV rule in hyphenation patterns, places the syllable boundary between 
     two consecutive consonants when they appear between two vowels. For example, 
     in the word \emph{sudden}, the syllable break occurs between the \emph{d} and 
     the second \emph{d}, making it \emph{sud-den}. 
     The VCCV pattern is a relatively common syllable division pattern in English.
     This rule ensures accurate 
     hyphenation, maintaining word readability and pronunciation when words are 
     split at the end of a line in printed text.}. 
\end{enumerate*}
Tests demonstrated that it could identify 40\% of allowable hyphen locations
\parencite{liang1983}.  The \TeX{} hyphenation algorithm later adopted the approach
proposed by Frank M. Liang, which involves competing patterns.  The \TeX{}82
algorithm employs five alternating levels of hyphenating and inhibiting
patterns. The program for pattern generation, known as PATGEN, was created by
%Liang 
\parencite{liangbreitenlohner1999} and has been utilized to generate
hyphenating patterns for numerous languages
\parencite{sojka1995,sojka1995a,sojka2005thesis,sojka2003,scannell2003}. It involves
sweeping a database of hyphenated words in a language to identify both
hyphenating and inhibiting patterns, ultimately creating a list of competing
patterns for that specific language.

The effective hyphenation of words by \TeX{} will actually depends on the following factors:
\begin{enumerate*}[label=(\arabic*)]
    \item document language, which will determine which set of patterns to apply;
    \item characters used, since some might block hyphenation at their edges;
    \item the value of the internal variables \verb|lefthyphenmin| and \verb|righthyphenmin|\footnote{
	The variables \verb|lefthyphenmin| and \verb|righthyphenmin| are language dependant and
	are defined in \emph{tlpobj} files (\verb|/usr/local/texlive/20XX/tlpkg/tlpobj/hyphen-xxxxxx.tlpobj|). 
	Default values varies in the range from 1 to 3. 
    	English and Portuguese, for example, use \verb|lefthyphenmin=2| and \verb|righthyphenmin=3|.},
	% ls /usr/local/texlive/2023/tlpkg/tlpobj/*.tlpobj | while read -r filename; do grep -Po "lefthyphenmin=[0-9]\s+righthyphenmin=[0-9]" $filename; done | tr '\t' ' ' | tr -s ' ' | sort | uniq -c
	%      17 lefthyphenmin=1 righthyphenmin=1
        %       6 lefthyphenmin=1 righthyphenmin=2
        %       1 lefthyphenmin=1 righthyphenmin=3
        %      49 lefthyphenmin=2 righthyphenmin=2
        %       9 lefthyphenmin=2 righthyphenmin=3
        which defines the minimum sequence length of characters at the left and right borders
        before any hyphenation is allowed.
\end{enumerate*}

Despite the fact that \TeX{} hyphenation algorithm and rules are old, they are,
to these days, the most frequently used approach, even outside the \TeX{}'s
world. The grounds for this is Hunspell, a spell checker and morphological
analyzer that is adopted in many softwares (e.g. LibreOffice, OpenOffice.org,
Mozilla Firefox, Mozilla Thunderbird, Google Chrome, macOS, InDesign, memoQ,
Opera, Affinity Publisher, among others \parencite{hunspell}). Hunspell uses \TeX{}
hyphenation rules \parencite{hunspellhyphen,levien1998}, making \TeX{} hyphenation
widespread in the computer world. That is a result of \TeX{} approach
simplicity and versatility.  The algorithm works effectively, as it already
supports rules for 66 languages \parencite{texhyphenrules}, and offers the
flexibility to create rules for any currently unsupported languages.

Unfortunately, certain hyphenation rules cannot be implemented using the \TeX{}
hyphenation algorithm. For example, in German, hyphenation can lead to letter
change or insertion.  Additionally, compound words lack hyphens, resulting in
extended letter sequences without visible separation and even repetitions of
the same letter, as seen in examples like \emph{Wasserrinne} and
\emph{Schifffahrt}.  Furthermore, the German spelling reform made some changes,
making it necessary to create a different set of rules for German hyphenation.
For example, the word \emph{Schiffahrt} should be hyphenated as
\emph{Schiff-fahrt}, preserving the \emph{f}s from each word that makes this
compound. The hyphenation should insert an \emph{f} that is not part of the
written form. That was not a problem for the old written form of the word:
\emph{Schifffahrt}. Also, the old hyphenating rules of German grammar stated
the hyphenation \emph{Bäk-ker} for the word \emph{Bäcker}, \emph{Zuk-ker} for
the word \emph{Zucker} and \emph{pak-ken} for the word \emph{packen}. Nowadays, those
words are hyphenated as \emph{Bä-cker}, \emph{Zu-cker} and \emph{pa-cken},
respectively. Some words have also changed their hyphenation point after the
spelling reform. For example, \emph{Fen-ster} became \emph{Fens-ter} and
\emph{mei-stens} became \emph{meis-tens}. Some problems in compound word
hyphenation in \TeX{} are discussed in \textcite{sojka1995a}.


% Some hyphenation rules are not possible with TeX algorithm.
% For example, the german word Schiffahrt should be hyphenated as Schiff-fahrt.
% Although it is no longer the recommend written form, it was the most usual
% until 2003 (see graph in schifffart.png from google ngrams).
% Also the old hyphenating rules of German grammatic stated the following
% hyphenations Bäk-ker (for the word Bäcker, what is now hyphenated as Bä-cker),
% Zuk-ker (for the word Zucker, what is now hyphenated as Zu-cker),
% pak-ken (for packen, and now pa-cken) and trok-ken (for trocken, now tro-cken).


%which contain archeo can be alternatively spelt with archaeo or archæo.


%GÂNDAVO, Pero de Magalhães de. Regras que ensinam a maneira de escrever e
%orthographia da lingoa portuguesa, etc. Lisboa: Antonio Gonsalvez, 1574

%  HIFENIZAÇÃO EM PORTUGUÊS
% Antonio Martins de ARAÚJO and Toru MARUYAMA



Moreover, as mentioned earlier, certain hyphenations may be preferred for
stylistic reasons, or to avoid ambiguity, or for better reading experience.
Some words might have multiple hyphenations, depending on the intended meaning.
In such cases, \TeX{} may not efficiently address the hyphenation challenge, as
it would necessitate case-by-case handling.


