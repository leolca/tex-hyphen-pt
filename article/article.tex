% xelatex --shell-escape article.tex 
% pandoc -s article.tex --pdf-engine=xelatex --pdf-engine-opt=-shell-escape -o article.html
\documentclass{article}
\usepackage[english]{babel}
%\usepackage{testhyphens}
\usepackage{multicol}
\usepackage{verbatim}
\usepackage{hyperref}
\usepackage{amssymb}% http://ctan.org/pkg/amssymb
\usepackage{pifont}% http://ctan.org/pkg/pifont
\newcommand{\cmark}{\ding{51}}%
\newcommand{\xmark}{\ding{55}}%
\usepackage{longtable}
\usepackage[style=alphabetic, backend=biber]{biblatex}
\addbibresource{references.bib}
\usepackage[inline]{enumitem}
\usepackage{listings}
\usepackage{hyperref}
\usepackage{cleveref}
\usepackage{fancyvrb}
\usepackage{siunitx}

%%%%%%% https://tex.stackexchange.com/questions/16790/write18-capturing-shell-script-output-as-command-variable
\usepackage{xparse}
\ExplSyntaxOn
\NewDocumentCommand{\captureshell}{som}
 {
  \sdaau_captureshell:Ne \l__sdaau_captureshell_out_tl { #3 }
  \IfBooleanT { #1 }
   {% we may need to stringify the result
    \tl_set:Nx \l__sdaau_captureshell_out_tl
     { \tl_to_str:N \l__sdaau_captureshell_out_tl }
   }
  \IfNoValueTF { #2 }
   {
    \tl_use:N \l__sdaau_captureshell_out_tl
   }
   {
    \tl_set_eq:NN #2 \l__sdaau_captureshell_out_tl
   }
 }

\tl_new:N \l__sdaau_captureshell_out_tl

\cs_new_protected:Nn \sdaau_captureshell:Nn
 {
  \sys_get_shell:nnN { #2 } { } #1
  \tl_trim_spaces:N #1 % remove leading and trailing spaces
 }
\cs_generate_variant:Nn \sdaau_captureshell:Nn { Ne }
\ExplSyntaxOff
%%%%%%%



\title{Improving the portuguese hyphenation rules}
\author{Leonardo Araújo \and Aline de Lima Benevides}
\begin{document}
\VerbatimFootnotes
\maketitle

\captureshell*[\NumberOfDefaultRules]{wc -l ../data/default.TeX.pt-br.patterns | cut -d' ' -f1}

\captureshell*[\NumberOfNewRules]{wc -l ../data/patch.TeX.pt-br.patterns | cut -d' ' -f1}

% 6 dictionaries --- 5 commas
\captureshell*[\DictionarySize]{../scripts/dicstats.sh}

% https://www.texdev.net/2009/10/06/what-does-write18-mean/
\captureshell*[\NumberOfSixHyphens]{../scripts/dicstats.sh 6}
\captureshell*[\NumberOfFiveHyphens]{../scripts/dicstats.sh 5}
\captureshell*[\NumberOfFourHyphen]{../scripts/dicstats.sh 4}
\captureshell*[\NumberOfThreeHyphens]{../scripts/dicstats.sh 3}
\captureshell*[\NumberOfTwoHyphens]{../scripts/dicstats.sh 2}
\captureshell*[\NumberOfOneHyphens]{../scripts/dicstats.sh 1}
\captureshell*[\NumberOfNoHyphens]{../scripts/dicstats.sh 0}
% run to update values
%\immediate\write18{../scripts/hyphenationagreements.sh > ../data/hyphagreements}
\captureshell*[\NumberOfSixAgrees]{../scripts/gethyphenationagreements.sh 6}
\captureshell*[\NumberOfFiveAgrees]{../scripts/gethyphenationagreements.sh 5}
\captureshell*[\NumberOfFourAgrees]{../scripts/gethyphenationagreements.sh 4}
\captureshell*[\NumberOfThreeAgrees]{../scripts/gethyphenationagreements.sh 3}
\captureshell*[\NumberOfTwoAgrees]{../scripts/gethyphenationagreements.sh 2}
\captureshell*[\NumberOfOneAgrees]{../scripts/gethyphenationagreements.sh 1}

\newcounter{numberRulesGroups}


\begin{abstract}
    Portuguese hyphenation rules are available for more than 35
    years and have done a good job. Nonetheless they still make mistakes and
    leave some hyphenation points unmarked. Although most undetected
    hyphenation points are located near word boundaries, what will be
    irrelevant for \TeX{} typographic purposes, they are still useful to
    hyphenate proper nouns, new words or pseudoword, and for usage in other
    applications, such as text-to-speech conversion. A list of
    \DictionarySize{} hyphenated words acquired from online dictionaries was
    used along \emph{patgen} to create improved rules, leading to better hyphenation
    of Portuguese words.
\end{abstract}

\section{Introduction}\label{sec-intro}
Hyphenation in text wrapping was not used for a long time. Words should fit
entirely in a line, or they would be broken in arbitrary places. First no
marker at all was used to sign the wrapping of a word, what could create
confusion and undesirable meaning. Therefore, ortographers advocated for the
use of a sign to indicate this break. Portuguese faced the same gradual
introduction of a hyphenation sign to mark words wrapping across lines. Even
though the usage of a hyphenation sign ($=$) was advocated by orthographers
\cite{gandavo1574}, few documents used such sign until the end of the 18th
century \cite{araujo2015}.

In some cases, hyphenation hinders smooth reading and should be avoided in
child literature, for example. In opposition, large or small spaces between
words also impose difficulty in the reading process, making hyphenation
fundamental when texts use short line lengths. As a line gets shorter, the
number of breaking candidates between words decreases, leading to awkward
spaces between words and among letters. For that matter, automatic hyphenation
plays an important role in good typesetting.

%Automatic hyphenation plays an important role in good typesetting and
%it is fundamental when a text uses short line lengths. As a line gets
%shorter, the number of breaking candidates between words decreases,
%creating awkward spaces between words and among letters.

\TeX{} is a typesetting system which carefully deals with these issues,
automatically arranging text on a page to create a good reading experience.
Automatic hyphenation is an important part on this process,
promoting an even-tempered distribution of elements on the page.
%Hyphenation is therefore an important part on this process.
%A good readability depends on an even-tempered distribution of elements
%on the page.
Line height and length, paragraph length, font size and typeface,
letter and word-spacing, are some factors which influence the text legibility
and readability. Space between words should not be too long creating lakes and
rivers in the text, nor too tight, impairing the legibility and readability.

Another important matter to consider is ambiguities that might be created when
a word is partitioned during the hyphenation process. In English we should
avoid hyphenations such as \emph{re-cover}, \emph{re-form}, \emph{re-sign}, \emph{the-rapist},
\emph{depart-mental}, and \emph{mans-laughter} (in Portuguese, some examples are:
\emph{de-putada}, \emph{fede-ração}, \emph{acu-mula}, \emph{após-tolo},
\emph{cú-bico}). Hyphenations that might lead the reader to pronounce a word
incorrectly should also be prevented. That is the case of
\emph{considera-tion}, in Enslish (and \emph{pe-rigo} in Portuguese).

In some situations, hyphenation is also a matter of style. Some partitioning
choices sound better than others. These conflicting alternatives typically
arises when a words has many possible hyphenation points. Consider
\emph{ar-chae-ol-o-gist} (\emph{or ar-che-ol-o-gist}), which is preferably partitioned as
\emph{archae-ologist} (or \emph{arche-ologist}) in opposition to \emph{archaeol-ogist} (or
\emph{archeol-ogist}) or \emph{archaeolo-gist} (or \emph{archeolo-gist}). It is preferable to keep
whole morphemes together. In the previous example: \emph{archae} (or \emph{arche}, meaning
``ancient'', ``primitive'')
%\footnote{archae- comes from the Ancient Greek ἀρχαῖος (arkhaîos, `ancient',
%`primitive'), from ἀρχή (arkhḗ, `beginning').}
%https://en.wiktionary.org/wiki/archae-
and \emph{-ologist} (``one who studies the topic''). In Portuguese, it is also more
elegant to avoid splits between double consonants or vowels, even if an
hyphenation point do exists between those letters. For example,
\emph{pressu-rizar} is preferable over \emph{pres-surizar} and
\emph{empreen-dedor} is preferable over \emph{empre-endedor}. Even so,
exceptions exists, it is preferable to partition \emph{micro-organismo}
rather than \emph{microor-ganismo} (keeping morphemes together is favored
over splitting a double vowel).

The general rule for Portuguese hyphenation is to split a word into its
syllables. A syllable is made of a mandatory nucleus, filled by a vowel, and
optional peripheral consonants (before or after the nucleus). In some
situations, the syllabic division does not respect the ethnologic constituents.
The usage of the prefixes \emph{bis-} and \emph{in-} are examples of this
circumstance. The correct syllabifications are \emph{bi-sa-vô},
\emph{i-na-ti-vo}, and \emph{i-nobs-tan-te}\footnote{The rule of syllable
division could lead to two possible partitions: \emph{i-nobs-tan-te} and
\emph{in-obs-tan-te}, but the first is preferable.}, where the prefixes are
split into two syllables. But, as pointed previously, it is preferable to keep
morphemes together rather than splitting them apart, therefore the we should
favor the hyphenation \emph{bisa-vô} over \emph{bi-savô} and \emph{ina-tivo}
over \emph{i-nativo}. This last example could also lead to a misunderstanding
since the word \emph{nativo} (\emph{native} in English) emerges from this word
break. 

Each language has its own hyphenation rules. They might be grouped into two
groups: those which hyphenation is driven by morphology (etymology) and those
driven by pronunciation. An algorithmic approach might use a logic system to
analyse words and apply the hyphenation rules of a given language. An algorithm
must be written for each language, as hyphenation rules vary extensively.
Although a logic system might be fast and compact, inevitably it will have to
deal with exceptions through hard-coded rules.

The automation of this process might use a dictionary based approach, which
will restrict the hyphenation possibilities to those entries in the dictionary,
an algorithmic base approach, which might be applied to whatever sequence found
in a text, or a mix of both approaches. 
%An algorithmic approach might use a logic system to analyse words and apply
%the hyphenation rules of a given language. 
A rule base model might include the recognition of prefixes,
suffixes, morphemes, or some sequences suitable for the inclusion of a break.
Another approach is the usage of pattern matching. Given a corpus of
hyphenation examples in a language, it is necessary to identify those sequence
of letters that define a point where hyphenation is suitable or not. Patterns
might include prefixes, suffixes, exceptions and special hyphenation rules of a
given language. Such approach might be used for different languages, by
providing hyphenation examples in a target language to extract patterns and
rules.

Only the analysis of immediate surrounding might not be enough to determine a
potential break point. For example, consider \emph{dem-o-crat} and \emph{de-moc-ra-cy},
where immediate surrounding of \emph{e} does not suffice to determine the break
location. Even after chosen patterns that makes hyphenation pretty
straightforward, there might be exceptions. Consider the sequence \emph{tion}. It
seems safe to always place a break before this pattern. However, the word
\emph{cation} is hyphenated as \emph{cat-ion}, making is etymology determinant to the way
this word will be split. 

A word with different meanings may be hyphenated differently according to its
meaning. `For instance, the Swedish word form \emph{glassko} has three different
meanings, and can be hyphenated as \emph{glas-sko} (glass shoe),
\emph{glass-ko} (ice cream cow) and in the non-standard way,
\emph{glass-sko} (ice cream shoe).' \cite{nemeth2006}. In Portuguese, the
word \emph{sublinha} might be hiphenated in two different forms:
\emph{su-bli-nha} when representing the inflected form of verb
\emph{sublinhar} (underline as a verb) or \emph{sub-li-nha} when refering
to the line under (underline as a noun).

% hyphenation might change with syntactic function
% part of speech (PoS)

Machine learning approach was used by to hyphenate Norwegian text
\cite{kristensen2001}. It was found that the \TeX{} approach, in general,
performed better then a neural network. Both approaches had similar
performance on finding the correct hyphenation points and avoiding wrong
hyphenation points on a small word list (the neural network performed slightly
better on finding the correct hyphenation points). Nonetheless, \TeX{} approach overcame
the neural network on a large word list, regarding the avoidance of wrong
hyphenations.

The original \TeX{} hyphenation algorithm was introduced by Knuth 1977
\cite{knuth1977}. It was focused on the English language and used three main
rules:(1) suffix removal; (2) prefix removal; and (3)
vowel-consonant-consonant-vowel (VCCV) breaking. Tests showed it could find
40\% of the allowed hyphen locations \cite{liang1983}. The hyphenation
algorithm proposed by Frank M. Liang and adopted in \TeX{} uses the notion of
competing patterns. A database of hyphenated words is swept, looking for
hyphenating and inhibiting patterns. The algorithm introduced in \TeX{}82 uses
five alternating levels of hyphenating and inhibiting patterns. The program
for patten generation, based on a corpus, PATGEN was created Liang
\cite{liangbreitenlohner1999} and was used to create hyphenating patterns for
many languages
\cite{sojka1995,sojka1995a,sojka2005thesis,sojka2003,scannell2003}.

The effective hyphenation of words by \TeX{} will actually depends on the following factors:
\begin{enumerate*}[label=\arabic*)]
    \item document language, which will determine which set of patterns to apply;
    \item characters used, since some might block hyphenation at their edges;
    \item the value of the internal variables \verb|lefthyphenmin| and \verb|righthyphenmin|\footnote{
	The variables \verb|lefthyphenmin| and \verb|righthyphenmin| are language dependant and
	are defined in \emph{tlpobj} files (\verb|/usr/local/texlive/20XX/tlpkg/tlpobj/hyphen-xxxxxx.tlpobj|). 
	Default values varies in the range from 1 to 3. 
    	English and Portuguese, for example, use \verb|lefthyphenmin=2| and \verb|righthyphenmin=3|.},
	% ls /usr/local/texlive/2023/tlpkg/tlpobj/*.tlpobj | while read -r filename; do grep -Po "lefthyphenmin=[0-9]\s+righthyphenmin=[0-9]" $filename; done | tr '\t' ' ' | tr -s ' ' | sort | uniq -c
	%      17 lefthyphenmin=1 righthyphenmin=1
        %       6 lefthyphenmin=1 righthyphenmin=2
        %       1 lefthyphenmin=1 righthyphenmin=3
        %      49 lefthyphenmin=2 righthyphenmin=2
        %       9 lefthyphenmin=2 righthyphenmin=3
        which defines the minimum sequence length of characters at the left and right borders
        before any hyphenation is allowed.
\end{enumerate*}

Despite the fact that \TeX{} hyphenation algorithm and rules are old, they are,
to these days, the most frequently used even outside the \TeX{}'s world. The
grounds for this is Hunspell, a spell checker and morphological analyzer that
is adopted in may software (e.g. LibreOffice, OpenOffice.org, Mozilla Firefox,
Mozilla Thunderbird, Google Chrome, macOS, InDesign, memoQ, Opera, Affinity
Publisher, among others \cite{hunspell}). Hunspell uses \TeX{} hyphenation
rules \cite{hunspellhyphen,levien1998}, making \TeX{} hyphenation widespread in
the computer world. That is a result of \TeX{} approach simplicity and
versatility.  The algorithm works well, it currently has rules for 66 languages
\cite{texhyphenrules} and it is possible to design new rules for those still
missing.

Unfortunately, some specific hyphenation rules are not possible to be
implemented using the \TeX{} hyphenation algorithm. In German, for example,
when words are hyphenated, some letters might change or a letter might be
inserted.  Compound words have no hyphens, leading to long sequences of letters
with no visible separation and even repetitions of a same letter, such as in
\emph{Wasserrinne} and \emph{Schifffahrt}. Furthermore, the German spelling
reform made some changes, making it necessary to create a different set of
rules for German hyphenation.  For example, the word \emph{Schiffahrt} should
be hyphenated as \emph{Schiff-fahrt}, preserving the \emph{f}s from each word
that makes this compound. The hyphenation should insert a \emph{f} that is not
part of the written form. That was not a problem for the old written form of
the word: \emph{Schifffahrt}. Also, the old hyphenating rules of German grammar
stated the hyphenation \emph{Bäk-ker} for the word \emph{Bäcker},
\emph{Zuk-ker} for the word \emph{Zucker} and \emph{pak-ken} for the word
\emph{packen}. Now those words are hyphenated as \emph{Bä-cker}, \emph{Zu-cker}
and \emph{pa-cken}, respectively. Some words have also changed their
hyphenation point after the spelling reform. For example, \emph{Fen-ster}
became \emph{Fens-ter} and \emph{mei-stens} became \emph{meis-tens}. Some
problems in compound word hyphenation in \TeX{} are discussed in
\cite{sojka1995a}.


% Some hyphenation rules are not possible with TeX algorithm.
% For example, the german word Schiffahrt should be hyphenated as Schiff-fahrt.
% Although it is no longer the recommend written form, it was the most usual
% until 2003 (see graph in schifffart.png from google ngrams).
% Also the old hyphenating rules of German grammatic stated the following
% hyphenations Bäk-ker (for the word Bäcker, what is now hyphenated as Bä-cker),
% Zuk-ker (for the word Zucker, what is now hyphenated as Zu-cker),
% pak-ken (for packen, and now pa-cken) and trok-ken (for trocken, now tro-cken).


%which contain archeo can be alternatively spelt with archaeo or archæo.


%GÂNDAVO, Pero de Magalhães de. Regras que ensinam a maneira de escrever e
%orthographia da lingoa portuguesa, etc. Lisboa: Antonio Gonsalvez, 1574

%  HIFENIZAÇÃO EM PORTUGUÊS
% Antonio Martins de ARAÚJO and Toru MARUYAMA


\section{Patterns for \TeX{} hyphenation}\label{sec-patt4TeX} 
The patterns used in \TeX{} hyphenation are of the form: \texttt{4z1z2}. Each
pattern is made of a sequence of letters and numbers. Odd numbers indicate a
good hyphenation point, whereas even numbers indicate a bad place to break.
The given example states that the sequence has a good breaking point between
the first and the second \emph{z} and an hyphenation should be inhibited before
the first \emph{z} and after the second \emph{z}. For example, the hyphenation
of the word \emph{piz-za}, \emph{fiz-zle} and \emph{mez-zanine} use this rule,
where we see the hyphen placed between the two \emph{z}'s and no hyphen before,
nor after the \emph{z}'s. The pattern may also use period symbol (\emph{.}) to
indicate word boundaries. The pattern \texttt{.s2h2} applies to beginning of
words, implying that the \emph{s} and the \emph{h} should stick together in
beginning of a word and an hyphenation should also be inhibited after the
\emph{h}. For example, this pattern is used in \emph{Sher-lock}.

Hyphenation rules are organized in levels, from 1 to 9, where odd numbers
represent hyphenating levels and even numbers represent inhibiting levels. Each
level works as an exception level of it predecessor. For example, the rule
\texttt{sh1er} indicate a good hyphenation point between the \emph{h} and the
\emph{e} in the sequence \emph{sher}. A rule at a higher level, as
\texttt{.s2h2}, implies an exception to the lower level rule. When we see
\emph{sher}, in the beginning of a word, the rule \texttt{.s2h2} applies and
hyphenation proposed by the lower level rule \texttt{sh1er} should be hindered.
That is the case in the hyphenation of the word \emph{Sher-lock}. The full
example is provided in \Cref{sherlockhyphenation}, where we might see all
English rules that takes place in the hyphenation of \emph{Sher-lock}.

\begin{lstlisting}[language={}, caption={Example of rules applyied in the
hyphenation of the word \emph{Sherlock}. Example done using a port of \TeX{}'s
hyphenation algorithm to Go provided at
\url{https://github.com/speedata/hyphenation}.}, label=sherlockhyphenation]
   .   s   h   e   r   l   o   c   k   .
     0   0   2   |   |   |   |   |   |    .sh2
     0   2   0   |   |   |   |   |   |    s2h
     0   0   1   0   0   |   |   |   |    sh1er
     |   |   |   0   1   0   |   |   |    r1l
     |   |   |   0   3   0   4   |   |    r3lo4
     |   |   |   |   |   |   0   0   1    ck1
max: 0   2   2   0   3   0   4   0   1
final: s   h   e   r - l   o   c   k -
\end{lstlisting}


A pattern will consist of a string made of characters (from the language
alphabet) possibly with a number in between, expressing the
hyphenation/inhibition level and possibly word boundaries marker (the period)
at the pattern edges. When there is no number between characters in a pattern,
a zero is assumed, which means \emph{undefined} and no hyphenation point will
be suggest at that location.



\section{PATGEN}\label{sec-patgen}
PATGEN uses a list of hyphenated words to define rules at different
levels and lengths, based on the patterns found in the data. It start with
short patterns and increases until a maximum pattern length, allowed by the
user, is reached. As it advances, creating longer patterns, it also establishes
exceptions. In some cases, the analysis of long patterns might be necessary,
since some hyphenation points might depend on characters far away from the
breaking point\footnote{Some examples of hyphenation dependency on characters
far from the break point: \emph{dem-o-crat} and \emph{de-moc-ra-cy};
\emph{as-pi-rin} and \emph{aspir-ing}; \emph{de-mon-stra-tive} and
\emph{dem-on-stra-tion}.}.


PATEGEN works on glyph indices rather than character codes. Each glyph is
represented by a single byte. That amount to 256 indices, where 13 of them are
reserved for the digits 0-9 and the characters `.', `-' and `*'. The remaining 243
are used to represent symbols of the given language. To run PATGEN, it is required a
translation file, which defines the values of some language specific parameters
(in the first line) and defines the many forms in which language symbols might
appear (all following lines). In the first line, positions 1 and 2 are used to
set the value of \verb|lefthyphenmin|, positions 3 and 4 are for
\verb|righthyphenmin|. These values defines the shortest length of a string
that might be generated by an hyphenation procedure. In order to set a single
digit value, leave the first position blank, that is, put a blank space in
position 1 and 3 for \verb|lefthyphenmin| and \verb|righthyphenmin|
respectively. Positions 5, 6 and 7 are used to define alternative values for
the special characters `.', `-' and `*'. The following lines uses a delimiter
to surround each `letter' of the desired language alphabet and also its
alternative representations. The first position of the line defines the
delimiter and each symbol of the language may use as many positions as
necessary, since not using the value reserved for the delimiter. Consider the
example for defining the letter `e' in Portuguese: 
\begin{verbatim}
XeXEXéXêXÉXÊX\'{e}X\^{e}X\'{E}X\^{E}XX 
\end{verbatim}
It represents the many ways in which it might be found: lowercase, uppercase,
with or without acute or circumflex accents. In this definition, we have
assumed that the many forms in which we may find the character \emph{e} and
\emph{e} will be equivalent for hyphenation (pattern matching) purpose. As
another example, see the line defining the character $\pi$ (taken from
\cite{haralambous2021}): \begin{verbatim} #p#P#\varpi ## \end{verbatim}

The translate file and the dictionary file should use the same encoding. Even
if it is a multi byte encoding, the translation file express how to deal with a
sequence of bytes representing a glyph and PATGEN will work fine as long as
there are at most 243 symbols in the given language.
The dictionary file is a list of already hyphenated words from which PATGEN
will look for patterns and create hyphenation rules. 

Those rules are organized
in different levels, from 1 to 9, where odd numbers represent hyphenating
levels and even numbers represent inhibiting levels. Each level works as an
inhibiting level of it predecessor. 
%A pattern will consist of a string made of
%characters (from the language alphabet) possibly with a number in between,
%expressing the hyphenation/inhibition level and possibly word boundaries marker
%(the period) at the pattern edges. When there is no number between characters
%in a pattern, a zero is assumed, which means \emph{undefined} and no
%hyphenation point will be suggest at that location.




\section{\TeX{} hyphenation rules for portuguese}\label{sec-tex-hyphen-pt}

\cite{rezende1987} created the first patters for Portuguese hyphenation in
\TeX{}. It was lastly updated in 2015 with the contributions from José João
Dias Almeida \cite{hyphpt}. The set has \NumberOfDefaultRules{}
rules\footnote{The set of rules for Portuguese hyphenation is short when
compared to rules in other languages. For example, English currently has 4938
rules, Russian has 7023 rules and German has 34011 rules.} and are able to
correctly hyphenate the majority of Portuguese words. 
% - detalhar as regras de Rezende e Dias 
% - contabilizar a quantidade de palavras que não são contempladas pela proposta do autor
% - analisar a necessidade de excluir alguma regra (ao invés de apenas acrescentar novas)

In front of these cases, we propose here to analyse the performance of the default rules in order to add new rules to improve
hyphenation by correcting some issues and taking in account non-typical
patterns. We describe the methodology used in the section \ref{sec-methodology}.

\section{Methodology}\label{sec-methodology}

In order to apprise the performance of each set of rules, we need a list of
correctly hyphenated words that are used in Portuguese. For that, we use the
frequency of occurrence to avoid selecting really rare words. To build a large
list with such information, we relied on data collected from online
dictionaries and corpora of Portuguese. We also looked up Portuguese grammars
to check on the hyphenation (\emph{translineation}) rules and we made some manual
corrections on the data.


\subsection{Data collection}\label{sec-data-coll} 

Initially we selected \href{https://www.linguateca.pt/cetenfolha/index_info.html}{CETENFolha} 
as the source corpus. Using only this corpus led to two problems: many words
were still missing and it was built on text from 1994, prior to the
Orthographic Agreement. This problem is a result of the Orthographic Agreement
of the Portuguese Language, carried out in 1990. 

The Agreement's adopting transition period started in 2009 and ended as it
turned mandatory in 2016.  Even after the agreement, some words still hold
different spellings in the participant countries. It is desirable to have
hyphenation rules that might work properly, despite those idiosyncrasies.  For
example, the word \emph{reception} might be written as \emph{receção}
(Portugal) or \emph{recepção} (Brazil); the word \emph{action} might be written
as \emph{acção} (Portugal) or \emph{ação} (Brazil); and the word \emph{project}
might be written as \emph{projecto} (Portugal) or \emph{projeto} (Brazil).

% We have used the Portuguese Wikipedia dump as our Portuguese corpus. Texts in Wikipedia might be written by Brazilian or Portuguese contributors, the both spellings may appear.

% cat /ms/downloads/samples/wikipedia/ptwiki-latest-pages-articles-multistream_wordlist.txt | tr -dc '0-9\n' | paste -sd+ | bc
% 302226482
% cat /ms/downloads/samples/wikipedia/ptwiki-latest-pages-articles-multistream_wordlist.txt | awk '{CUMCOUNT+=$1; if(CUMCOUNT < 287115158) print;}' >/tmp/ptwikitop95.txt
% cat /ms/downloads/samples/wikipedia/ptwiki-latest-pages-articles-multistream_wordlist.txt | awk '{CUMCOUNT+=$1; if(CUMCOUNT < 299204217) print;}' >/tmp/ptwikitop99.txt
% cat /tmp/ptwikitop99.txt | tr -d "0-9 " |  while read -r word; do if ! grep -q "^${word}," hyphenations.csv; then echo $word; fi; done > /tmp/ptwikitop99_notinlist.txt

Because of these idiosyncrasies, we decided to incorporate the list of words from
\href{https://www.palavras.net/}{Palavras NET} and afterwards we also included
words from the \href{https://pt.wikipedia.org}{Portuguese Wikipedia} dump.
Although the list of words created was huge, we kept only those words we could
find their hyphenation on at least one of the following online dictionaries: 
\href{https://michaelis.uol.com.br/}{Michaelis},
\href{https://dicionario.priberam.org/}{Priberam},
\href{https://pt.wiktionary.org}{Wikcionário},
\href{https://aulete.com.br/}{Aulete},
\href{http://www.portaldalinguaportuguesa.org/}{Portal da Língua Portuguesa} e
\href{https://www.dicio.com.br/}{Dicio}.
This leads to the creation of a dictionary with \DictionarySize{} words.

Table \ref{tblstats} presents the number of words with a given number of
hyphenation found in the dictionaries. It also presents the number of words
that have the highest agreed hyphenation. For example, the word \emph{como}
was hyphenated by the six dictionaries as \emph{co-mo}. It will add one to the
counts on both lines where the hyphenations number is 6 (first column). The
word \emph{sua} was hyphenated as \emph{su-a} in 4 dictionaries, \emph{sua} in
one dictionary and one dictionary did not provide its hyphenation. Therefore,
it will add one in the first line, second column (since hyphenations were found
in 5 dictionaries). It will also add one on the second line, third column
(since the most frequent hyphenation was found in 4 dictionaries). It is clear
now why we might have the number of same hyphenations (specially for small
values) greater the number of hyphenations found. 


\begin{table}
\centering
\caption{Considering the six dictionaries used, the first line of this table
presents the number words that a given number of hyphenations were found in the
dictionaries. The second line presents the number of words that have a given
number of hyphenation as its most frequent form found in the
dictionaries.}\label{tblstats}
\small
\begin{tabular}{*{8}{l}}
    hyphenations & 6 & 5 & 4 & 3 & 2 & 1 & 0 \\
    \hline
    hyphenations found & \NumberOfSixHyphens{} & \NumberOfFiveHyphens{} &
    \NumberOfFourHyphen{} & \NumberOfThreeHyphens{} & \NumberOfTwoHyphens{} &
    \NumberOfOneHyphens{} & \NumberOfNoHyphens{} \\

    same hyphenations & \NumberOfSixAgrees{} & \NumberOfFiveAgrees{} &
    \NumberOfFourAgrees{} & \NumberOfThreeAgrees{} & \NumberOfTwoAgrees{} &
    \NumberOfOneAgrees{} & n/a
\end{tabular}
\end{table}

In summary, data from \emph{CETENFolha} and \emph{Palavras NET} were compiled.
Their words were hyphenated based on the hyphenation original \TeX{} rules and
contrasted with the dictionary results. The performance was appraised and the
errors systematized, in order to find in the literature possible rules that
could complement and improve it.


\section{Writing systems}\label{sec-writing-systems}

There is a wide range of writing systems. They might be categorized into
logographic, syllabic, and alphabetic writing systems.  Although they are
distinct categories, writing systems might be built on an interplay of these
categories \cite{coulmas2003}. The principles that guide an alphabetic writing
system or a phonemic writing system can vary significantly based on the
specific language and its history. Nevertheless, we might highlight here some
of its key points:
\begin{enumerate}[label=\arabic*)]
    \item the most usual principle consists on representing the sounds of a
	language with written symbols, using one or a combination of symbols to
	represent each sound (e.g. the orthographic system of Portuguese uses
	an alphabet of 27 letters that aim to represent different sounds of the
	language); 
    \item etymology also plays an important role, where the
	spelling of a word reflects its origins and history (e.g. the French
	language often reflects the Latin roots of words in its spelling);
    \item morphology is used as a guide in structuring many languages, where
	they use specific letters or symbols to indicate word endings,
	prefixes, or suffixes (e.g. Russian uses different forms of the
	Cyrillic alphabet to indicate gender and case in its nouns); 
    \item the way a language has evolved over time also contribute to its 
	written form (e.g. the spelling of many English words have changed 
	over time to reflect changes in pronunciation);
\end{enumerate}
The principles that guide orthographic systems for languages with logographic
writing systems like Chinese ideograms, cuneiform writing, and Egyptian
hieroglyphs differ significantly from those of alphabetic writing systems. In
logographic writing systems, individual symbols or characters represent entire
words or ideas rather than phonemes or sounds. As a result, the principles that
guide their orthographic systems are based more on semantic and visual
principles rather than phonemic principles.


\section{Grammar rules for hyphenation in Portuguese}\label{sec-grammar-port}

Portuguese uses an alphabetical writing system, ensuing that its spelling is
guided by phonological principles \cite{cagliari2015}. The fact that spelling
reflects, to some extent, the way words are pronounced, means that the
hyphenation of words is also influenced by the phonemic system, where words are
divided into syllables. It is noteworthy that different languages divide words
based on different principles, for example, English tends to be guided by
morphological principles, such as \emph{walk-ing}, \emph{un-happy},
\emph{work-s}, and \emph{ear-ly}. Also, other factors affect word hyphenation in
English, such as the presence of doubled consonants and digraphs\footnote{One
should not confuse consonant clusters with digraphs, since the latter represent
only one sound, while the former represent two or more sounds.}. There are also
a myriad of exceptions that might be grasped into rules.
% https://english.stackexchange.com/questions/385/what-are-the-rules-for-splitting-words-at-the-end-of-a-line

Merely stating that an orthographic system is guided by phonological issues
does not necessarily mean that its hyphenation rules directly reflect the way a
word is pronounced -- this is because Portuguese does not have a direct
letter-sound correspondence, and vice versa.  The orthographic system operates
according to its own specific rules. For instance, although consonant clusters that
produce a single sound (i.e., digraphs) are pronounced within a single syllable
in words like \emph{chave}, \emph{ilha}, \emph{shampoo}, \emph{carro}, and
\emph{pássaro}, their representation in writing determines how they are
separated.  Specifically, different consonants within a digraph must remain
together, while identical consonants are separated. As a result, we have
\emph{cha-ve}, \emph{i-lha}, and \emph{sham-poo}, but \emph{car-ro}.

%When there is no space left in line for a whole word, it might be split in two
%using a hyphen as an indication of such procedure. 

In Portuguese, hyphenations are allowed on syllables boundaries and, in
general, follow phonological principles.  According to the Grammar
\cite{cunha2016,bergstrom2011,cegalla2008}, some rules might still apply:
\begin{enumerate}
    \item\label{rule-di-triphthong} diphthong or triphthong should not be split
	(e.g. \emph{m\underline{\emph{ui}}-to}, \emph{Pa-ra-g\underline{\emph{uai}}});
    \item\label{rule-unstressed} the sequences \emph{ia}, \emph{ie}, \emph{io}, \emph{oa}, \emph{ua},
	\emph{ue} and \emph{uo}, when in final unstressed position, should not be split 
	(e.g. \emph{gló-r\underline{\emph{ia}}}, \emph{vi-tó-r\underline{\emph{ia}}}, 
	\emph{cá-r\underline{\emph{ie}}}, \emph{es-pé-c\underline{\emph{ie}}}, 
	\emph{Má-r\underline{\emph{io}}}, \emph{má-g\underline{\emph{oa}}}, 
	\emph{ré-g\underline{\emph{ua}}}, \emph{tê-n\underline{\emph{ue}}}, 
	\emph{con-tí-g\underline{\emph{uo}}}, \emph{am-bí-g\underline{\emph{uo}}});
    \item\label{rule-c-clusters} consonant clusters starting a syllable should not be
	split (e.g. \emph{\underline{\emph{pn}}eu-má-ti-co}, \emph{\underline{\emph{ps}}i-có-lo-go}, 
	\emph{\underline{\emph{mn}}e-mô-ni-co});
    \item\label{rules-digraphs-ns} the digraphs	\emph{ch}, \emph{lh}, \emph{nh}
	should not be split (e.g. \emph{ra-\underline{\emph{ch}}ar},
	\emph{a-bro-\underline{\emph{lh}}os}, \emph{ma-\underline{\emph{nh}}ã};
    \item\label{rule-guqu} bigrams like \emph{gu} and \emph{qu} whose vowel \emph{u} is not 
	pronounced are never separated from the vowel or diphthong that follows it 
	(e.g. \emph{ne-\underline{\emph{gu}e}}, \emph{pe-\underline{\emph{qu}e}});
    \item\label{rule-nasalization} since they are digraphs, a vowel and its following 
	nasalization marker (a graphic nasal consonant) should not be split (e.g. \emph{\underline{\emph{am}}-bição},
	\emph{m\underline{\emph{an}}-cha});
    \item\label{rule-decreasing} decreasing diphthongs should not be split (e.g. 
	\emph{\underline{\emph{ai}}-ro-so}, \emph{ca-d\underline{\emph{ei}}-ra}, 
	\emph{o-ra-ç\underline{\emph{ão}}});
    \item\label{rule-rising} rising diphthong should not be split (e.g.
	\emph{a-b\underline{\emph{ai}}-xo}; \emph{c\underline{\emph{au}}-te-la}, 
	\emph{pa-p\underline{\emph{éi}}s}, \emph{cha-p\underline{\emph{éu}}}, 
	\emph{pre-f\underline{\emph{ei}}-to}, \emph{r\underline{\emph{eu}}-nir},
	\emph{n\underline{\emph{oi}}-te}; \emph{ca-la-b\underline{\emph{ou}}-ço}, 
	\emph{as-te-r\underline{\emph{ói}}-de}; 
	\emph{re-tri-b\underline{\emph{ui}}});
    \item\label{rule-singlev} disyllables whose syllable has a single vowel should 
	not be split (e.g. \emph{\underline{\emph{a}}to}, \emph{ru\underline{\emph{a}}}, 
	\emph{\underline{\emph{ó}}dio}, \emph{\underline{\emph{u}}nha});
    \item\label{rule-orphan} words with more than two syllables, when divided, cannot 
	isolate a syllable composed of a single vowel (e.g. \emph{\underline{\emph{a}}gos-to}, 
	\emph{la-go\underline{\emph{a}}}, \emph{\underline{\emph{i}}da-de});


    \hspace{10em} \hbox to 5cm{\leaders\hbox to 10pt{\hss . \hss}\hfil} 

    \item\label{rule-hiatus} hiatus vowels and those vowel sequences where each vowel
	belongs to a different syllable should be split (e.g.
	\emph{sa\underline{\emph{-ú-}}de}, \emph{ra\underline{\emph{-i-}}nha}, \emph{d\underline{\emph{o-e}}r},
	\emph{v\underline{\emph{o-o}}s}), the same procedure is used splitting diphthongs
	in different syllables (e.g. \emph{c\underline{\emph{ai-ai}}s}) or diphthong and
	vowel in different syllables (e.g. \emph{en-s\underline{\emph{ai-o}}s});
    \item\label{rule-consonants} consonant sequences, when in different syllables, should
	be split (e.g. \emph{a\underline{\emph{f-t}}a}, \emph{a\underline{\emph{b-d}}i-car},
	\emph{re\underline{\emph{s-c}}i-são}, \emph{a\underline{\emph{b-s}}o-lu-to});
    \item\label{rule-digraphs} the following consonant digraphs should be split:
	\emph{rr}, \emph{ss}, \emph{mm}, \emph{nn}, \emph{sc}, \emph{sç} and
	\emph{xc} (e.g. \emph{te\underline{\emph{r-r}}a}, \emph{pro-fe\underline{\emph{s-s}}or},
	\emph{co-mu\underline{\emph{m-m}}en-te}, \emph{co\underline{\emph{n-n}}os-co},
	\emph{de\underline{\emph{s-c}}er}, \emph{cre\underline{\emph{s-ç}}a}, \emph{e\underline{\emph{x-c}}e-der}).
\end{enumerate}

Note that rules \ref{rule-singlev} and \ref{rule-orphan} are fiddling rules in \TeX{} hyphenation, 
since \TeX{} already has directives to avoid hyphenated widows and orphans. 
As mentioned in \Cref{sec-intro}, the variables \verb|lefthyphenmin| and \verb|righthyphenmin|
are used to control the minimum length for fragments of hyphenated words.
This is the ground why we will not consider rules \ref{rule-singlev} and \ref{rule-orphan}
when analysing and elaborating the hyphenation rules.


In some situations a hyphen should be repeated in the start of the following
line: 
\begin{enumerate}
\item when a compound word using a hyphen is split across lines (e.g.
    \emph{couve-/-flor}, \emph{ex-/-presidente});
\item when a different meaning could be conveyed when splitting a pronoun (e.g.
    \emph{prazer de ver-/-me}\footnote{Possible conveyed meanings: 
    \emph{pleasure in seeing me} or \emph{worm's pleasure}.}). 
\end{enumerate}

It should be avoided to leave a single vowel in the end or start of a line (e.g.
the word \emph{san-guí-nea} should be split only on the hyphen gave in the
example, avoiding the split that would leave the final vowel \emph{a}
left alone in the start of a new line: \emph{sanguíne-a}). It is also considered a better approach
to avoid splitting disyllables made of four letters (e.g. \emph{para},
\emph{como}, \emph{cede}).


Systematizing the rules that guide syllable boundaries and, consequently,
hyphenation in Portuguese proved to be fundamental for the improvement of the
\TeX{} hyphenator rules, since they indicate the possible or prohibited contexts
of hyphenation. Expomos, a seguir, uma análise comparativa...

... to continue
\vspace{10ex}



\section{Creating the dictionary}
A \emph{bash} script was created to crawl on six online dictionaries and extract a word hyphenation, when found. 



\section{Updating the rules}
Let us remember that the standard \TeX{} hyphenation rules, proposed by
\cite{rezende1987,hyphpt}, when tested against a set of \num{\DictionarySize}
words, present 1,368 (????) errors, of which 1,273 (????) correspond to places
where the hyphenation was not carried out and 125 (????) to erroneous markings.
Faced with the errors made by the standard hyphenator, a set of \num{\NumberOfNewRules} rules were
established, which reduced the errors to 132 (???), with 118 (????) places
where the hyphenation was not marked and 35 (????) where there was erroneous
marking. Such rules are presented and exemplified below, as well as the
necessary exception rules and an example in which the application of the rule
is demonstrated.

\subsection{Results from the new set of rules}
Although the XXX erroneous words have some common characteristics, which would
allow, to some extent, to reduce the amount of hyphenation errors, there is a
limitation arising from the way \TeX{}'s rules are conceived. We present,
below, the systematists found in these data that could be predicted by another
structure of rules:

\begin{description} 
    \item[Morphological determination:] Prefixes such as \emph{re-},
	\emph{sub-}, \emph{cyber-}, and \emph{auto-} require separating the
	prefix from its stem, what runs into phonological issues. For example,
	\emph{reiniciar} and \emph{sublinhar}, as they contain prefixes, should
	be hyphenated as \emph{re-i-ni-ci-ar} and \emph{sub-li-nhar},
	respecting the morphological formation of the word. However,
	considering that the Portuguese language hyphenates its words from
	syllabic phonological correlates, words that require morphological
	information, such as those, do not have their hyphenation performed
	properly. They add up, in our corpus, to twelve words (eight words with
	\emph{re-} and one case with \emph{auto-}, \emph{meta-}, \emph{sub-},
	\emph{cyber-}, \emph{geo-} or \emph{neuro-}).

    \item[Foreignness:] There is a group of 20 words in the corpus that are
	terminologies or words that were incorporated into the Portuguese
	language without full phonological adaptation, such as
	\emph{darwinismo}, \emph{quilowatt}, and \emph{esfiha}. The lack of
	adaptation makes the phonological pattern very specific to the word,
	making it impossible to incorporate in \TeX{}'s rules alongside the
	other rules. The solution is to add them to the exception word list.

    \item[Word-initial consonant clusters:] Portuguese has few cases of
	consonant clusters at the beginning of a word. They, in general, are
	etymological remnants and are currently unproductive in the language,
	since there are no neologisms with this pattern. Encounters like
	\emph{ps-} and \emph{pn-} are more frequent, as they are present in
	words like \emph{psicologia} and \emph{pneu}, which have an average
	frequency in the language. They might be predicted by rules, which
	comprise XXX words. However, there are consonant clusters that are
	found in very specific and low frequency words. Although possible, it
	is not worth adding very specific rules using the consonant clusters
	contained in the words \emph{dzeta}, \emph{gnu}, \emph{cnidário},
	\emph{ftálico}, and \emph{gnaisse} -- that is, just five words.

    \item[Abbreviations, Acronyms, or Initialisms:] Whether for efficiency, 
	convenience, clarity or jargon, it is usual the usage of a shortening 
	version of a word or a phrase. 
	Abbreviation a way to shorten a word of phrase. Some examples in our
	Portuguese corpus are: 
	\emph{etc.}\footnote{Latin expression \emph{et cetera}, meaning 
	`and other similar things'.}, 
	\emph{Dr.}\footnote{\emph{doutor} (doctor, person with PhD title, but popularly used to designate an erudite individual)}, 
	\emph{Exmo.}\footnote{\emph{Excelentíssimo} (honourable)},
	\emph{cap.}\footnote{\emph{capítulo} (chapter)}, \emph{Univ.}\footnote{Universidade (university)}, 
	\emph{ed.}\footnote{\emph{edição} (edition)}, 
	\emph{s.n.}\footnote{sine nomine, Latin expression meaning `without a name', 
	mostly used in the context of publishing.}.
	Initialism consists of a set of initial 
	letters of words that form an shortened version. They are not always 
	understood by the set of hyphenation rules described in this work, 
	since they do not follow orthographic (or phonotactic) standards of 
	the language. In the corpus, it is possible to find some abbreviations, 
	such as \emph{SESC}\footnote{Serviço Social do Comércio}, 
	\emph{INSS}\footnote{\emph{Instituto Nacional do Seguro Social} (National Institute of Social Security)}, 
	\emph{PCdoB}\footnote{\emph{Partido Comunista do Brasil} (Communist Party of Brazil)},
	\emph{PM}\footnote{\emph{Polícia Militar} (military police)}, and 
	\emph{UFRJ}\footnote{Universidade Federal do Rio de Janeiro}.
	Acronym is a type of abbreviation that shortens a phrase by combining
	the first letter (or letters) of each word, creating a new pronounceable word.
	Some example from the corpus are: 
	\emph{Anatel}\footnote{\emph{Agência Nacional de Telecomunicações} (National Telecommunications Agency)},
	\emph{Ovni}\footnote{\emph{Objeto voador não identificado} (unidentified flying object - UFO)},
	\emph{Sida}\footnote{\emph{Síndrome da Imunodeficiência Adquirida} (acquired immunodeficiency syndrome - AIDS)} (in Portugal) and
	\emph{Mercosul}\footnote{\emph{Mercado Comum do Sul} (Southern Common Market).}.
\end{description}


\section{Creating a new set of rules using Patgen}



\section{Comparing the set of rules} 
Let us remember that the standard \TeX{} hyphenation rules, proposed by
\cite{hyphpt}, amount to XXX errors, in a set of XXXX words.  Among those, XXX
correspond to places where the hyphenation was not carried out and XXX to
mismarkings. Faced with the errors made by the standard hyphenator, a set of
\NumberOfNewRules{} rules were established, which reduced the errors to XXXX,
with XXXX unmarked hyphenation places and XXX mismarkings.  Such rules are
presented and exemplified below, as well as the necessary exception rules and
an example in which the application of the rule is demonstrated.

\section{Old Text}


Foi criado um conjunto de \NumberOfNewRules{} regras para aprimorar as hifenizações no
português. Analisando um conjunto de 37798 palavras, as regras padrão apresentam
1368 erros (1273 pontos de hifenização não marcados e 125 pontos marcados
erroneamente)\footnote{A soma não é 1368 pois temos 30 casos em que os dois
tipos de erros ocorrem.}, enquanto as novas regras apresentaram 132 erros (118
pontos de hifenização não marcados e 35 pontos marcados
erroneamente)\footnote{35 palavras apresentam ambos tipos de erro.}.
%%% <<< fazer correção após correção manual das hifenizações tidas como corretas.

Foi criado um conjunto de 37798 palavras obtidas do corpus
\href{https://www.linguateca.pt/cetenfolha/index_info.html}{CETENFolha},
juntamente com palavras do \href{https://www.palavras.net/}{Palavras NET} e
algumas palavras do corpus baseado na
\href{https://pt.wikipedia.org}{Wikipedia}. 
Foram mantidas apenas as palavras
para as quais encontrou-se a hifenização nos seguintes dicionários online:
\href{https://michaelis.uol.com.br/}{Michaelis},
\href{https://dicionario.priberam.org/}{Priberam},
\href{https://pt.wiktionary.org}{Wikcionário},
\href{https://aulete.com.br/}{Aulete},
\href{http://www.portaldalinguaportuguesa.org/}{Portal da Língua Portuguesa} e
\href{https://www.dicio.com.br/}{Dicio}.
Adotamos como hifenização `correta' aquela que ao menos 4 dicionários
concordavam. Algumas hifenizações foram realizadas manualmente pelo autor.
Considerando a frequência de ocorrência das palavras nos corpus, o conjuto de
palavras escolhido representa 55\% do total das ocorrências relativo ao
Wikipedia e 58\% em relação ao CETENFolha.
% Corpus Wikipedia
%
% 166341424 / 302226482 = 0.55
% 
% soma da freq. de occ. das palavras utilizadas
% cat portuguese_4_utf8.dic portuguese_extra_utf8.dic | tr -d '-' | while read -r word; do grep "[0-9]\+ ${word}$" /ms/downloads/samples/wikipedia/ptwiki-latest-pages-articles-multistream_wordlist.txt | head -n 1 | tr -dc '0-9\n'; done | paste -s -d+ - | bc 
% = 166341424
%
% soma da freq. de occ. de todas palavras no corpus
% cat /ms/downloads/samples/wikipedia/ptwiki-latest-pages-articles-multistream_wordlist.txt | tr -dc '0-9\n' | paste -s -d+ - | bc
% = 302226482
%
% CETENFolha
% 13314084 / 22728804 = 0.5857
Este percentual é relativamente baixo, mas deve-se ao fato de que não obtivemos
sucesso em encontrar a hifenização de muitas palavras nos dicionários
analisados e também por não observarmos um consenso de ao menos 4 dicionários
com uma mesma hifenização.

Analisando os erros resultantes da hifenização utilizando as regras padrão do
TeX, foram estabelecidas algumas regras como forma de abarcar um conjunto maior
de palavras. Tais regras são elencadas a seguir, expondo também as regras de
exceção necessárias e um exemplo onde a regra será utilizada.

\begin{enumerate}
\item\label{rulegrp_psi} regra: \texttt{p2si}, \texttt{p2sí} -- psicologia, psíquico\\
exceção: \texttt{p3sia}, \texttt{p3sin} -- epilepsia, rodopsina

\item\label{rulegrp_tc} regra: \texttt{t2c} -- tchau%, tcheco

\item\label{rulegrp_pneu} regra: \texttt{p2neu} -- pneumonia
    
\item\label{rulegrp_gno} regra: \texttt{.g2no}, \texttt{.g2nó} -- gnomo, gnóstico

\item\label{rulegrp_pt} regra: \texttt{.p2t} -- ptose%, pterossauro

\item\label{rulegrp_cza} regra: \texttt{c2za} -- czar

\item\label{rulegrp_s2} regra: \texttt{.s2} -- stalinismo

\item\label{rulegrp_t2} regra: \texttt{.t2} -- tsunami

\item\label{rulegrp_mn} regra: \texttt{.m2n} -- mnemônico

\item\label{rulegrp_zz} regra: \texttt{z1z}, \texttt{p1p}, \texttt{z2z.} -- pizza, shopping, jazz

\item\label{rulegrp_sub} regra: \texttt{su2b3r}, \texttt{su2b3l} -- sublunar, subrotina
exceção: \texttt{su3b4li1nh}, \texttt{su3b4li1ma}, \texttt{su3b4li1me}, \texttt{su3b4li1mid} -- sublinhar, sublimar, sublime, sublimidade

\item\label{rulegrp_so} regra: \texttt{1sô} -- subsônico

\item\label{rulegrp_pseu} regra: \texttt{1p2seu1d} -- pseudônimo

\item\label{rulegrp_ein} regra: \texttt{e1in1c}, \texttt{e1in1f}, \texttt{e1in1g}, \texttt{e1in1s}, \texttt{e1in1t}, \texttt{e1in1v} -- reincidência, reinfecção, reingressa, reinserção, reintegração, reinventar
exceção: \texttt{e1in2st}, \texttt{e1in2sc} -- reinstalado, reinscrever 

\item\label{rulegrp_oin} regra: \texttt{o1in} -- coincidência, agroindustrial, bioinformática, amendoins, gastrointestinal
exceção: \texttt{o1ina} -- boina

\item\label{rulegrp_eim} regra: \texttt{e1im} -- reimpresso, coimbra, coimperador
exceção: \texttt{que2i}, \texttt{te2im} -- queimar, queimadura, teimosia, tirateima

\item\label{rulegrp_aa} regra: \texttt{a1â}, \texttt{a1ã}, \texttt{a1é}, \texttt{a1í}, \texttt{a1ó}, \texttt{a1ô}, \texttt{a1ú}, \texttt{e1á}, \texttt{e1â}, \texttt{e1ã}, \texttt{e1ê}, \texttt{e1í}, \texttt{e1ó}, \texttt{e1ô}, \texttt{e1ú}, \texttt{i1á}, \texttt{i1ã}, \texttt{i1ã}, \texttt{i1é}, \texttt{i1ó}, \texttt{i1ú}, \texttt{o1á}, \texttt{o1ã}, \texttt{o1é}, \texttt{o1ê}, \texttt{o1í}, \texttt{o1ó}, \texttt{u1á}, \texttt{u1ã}, \texttt{u1é}, \texttt{u1ê}, \texttt{u1í} --  abraâmico, abraão, aéreo, país, caótico, faraônico, saúde, balneário, oceânico, campeã, veêm, veículo, teórico, napoleônico, conteúdo, diário, região, soviético, periódico, viúva, razoável, joão, poético, boêmia, heroísmo, alcoólico, usuário, itapuã, suécia, cauê, suíça \\
exceção: \texttt{1gu2é}, \texttt{1gu2ê}, \texttt{1gu2í}, \texttt{1qu2á}, \texttt{1qu2é}, \texttt{1qu2ê}, \texttt{1qu2í} -- alguém, português, linguística, aquático, inquérito, sequência, química

\item\label{rulegrp_air} regra: \texttt{a1ir.}, \texttt{u1ir.} -- sair, diminuir

\item\label{rulegrp_ia} regra: \texttt{í1a} -- baía

\item\label{rulegrp_ain} regra: \texttt{a1ind}, \texttt{a1i1nh} -- ainda, rainha

\item\label{rulegrp_oin} regra: \texttt{o1i1nh} -- moinho

\item\label{rulegrp_aui} regra: \texttt{au1i1c}, \texttt{du1i1c}, \texttt{u1i1ç}, \texttt{u1i1d}, \texttt{cu2i}, \texttt{dru2i}, \texttt{flu2id}, \texttt{bu1i1n}, \texttt{cu1i1n}, \texttt{fu1i1n}, \texttt{nu1i1n}, \texttt{ru1i1na}, \texttt{ru1i1no}, \texttt{su1i1ti}, \texttt{tu1i1ti}, \texttt{u1iz} -- cacauicultor, sanduicheira, constituição, continuidade, cuidador, druida, fluido, tabuinha, picuinha, fuinha, genuinamente, arruinado, arruinou, jesuitismo, intuitivo, juizado 
exceção: \texttt{cu3i1da1de}, \texttt{bu1i2n1d}, \texttt{bu1i2n1t}  -- acuidade, contribuindo, contribuinte

\item\label{rulegrp_gua} regra: \texttt{1gu4á}, \texttt{1gu4ã}, \texttt{1qu4ã} -- jaraguá, saguão, quão

\item\label{rulegrp_iur} regra: \texttt{i1ur} -- diurno

\item\label{rulegrp_io} regra: \texttt{í1o} -- íon

\item\label{rulegrp_proi} regra: \texttt{pro1i1b} -- proibição

\item\label{rulegrp_iun} regra: \texttt{i1un} -- triunfar

\item\label{rulegrp_quo} regra: \texttt{1qu4ó}, \texttt{1qu4â} -- quórum, equânime

%\item\label{rulegrp_uo} regra: \texttt{u3ó} -- fluór

\item\label{rulegrp_eo} regra: \texttt{é3o} -- alvéolo

\item\label{rulegrp_ia} regra: \texttt{i4a.}, \texttt{i4e.}, \texttt{i4o.}, \texttt{o4a.}, \texttt{u4a.} -- economia, espécie, vazio, destoa, institua 

\item\label{rulegrp_lo} regra: \texttt{1lô} -- camelô

\item\label{rulegrp_co} regra: \texttt{1cô} -- recôncavo

\item\label{rulegrp_bo} regra: \texttt{1bô}, \texttt{1dô}, \texttt{1fô}, \texttt{1gô}, \texttt{1pô}, \texttt{1mô}, \texttt{1nô}, \texttt{1rô}, \texttt{1tô}, \texttt{1vô}, \texttt{1xô}, \texttt{1zô} -- robô, judô,  telefônica, xangô, capô, sumô, econômico, tarô, chatô, vovô, saxônia, amazônia

\item\label{rulegrp_cco} regra: \texttt{1çô} -- maçônico

% As sequências gráficas ia, ie, io, oa, ua quando em posição
% final átona, que são normalmente pronunciadas como ditongos
% crescentes, mas que podem corresponder foneticamente a duas
% vogais (hiato), sobretudo quando pronunciadas pausadamente:
% vitó-ria, espé-cie, exercí-cio, nó-doa, lé-gua, tê-nue, ambí-guo.
    \setcounter{numberRulesGroups}{\value{enumi}}
\end{enumerate}

The \NumberOfNewRules{} rules were grouped above in a list of
\arabic{numberRulesGroups} types of rules. They may be further organized into four
large groups. The first, which comprises rules \ref{rulegrp_psi} to
\ref{rulegrp_zz}, includes consonant clusters such as \emph{czar}, \emph{ptose}
and \emph{gnomo}. They, unlike the examples that will be exposed in section
\ref{secxxx} (exceptions - item 9 currently), present a set of derived words,
which makes their marking advantageous in view of the number of cases that are
included in this marking rule. The second group, comprising rules
\ref{rulegrp_sub} to \ref{rulegrp_eim}, delimits the morphological boundary
between prefixes and radicals. As noted, although phonological issues guide the
separation of numerous words in Portuguese, there are also those that are
guided by morphology. This is the case of words that have the prefixes
\emph{sub-} and \emph{re-}, such as \emph{sublunar} and \emph{reinserção}. The
third group, comprising rules \ref{rulegrp_aa} to \ref{rulegrp_ia}, seeks to
understand a set of words that have vowel combinations that do not follow the
general rules. This is because the Portuguese language has vowel encounters
with the second vowel graphically marked that can be separated, forming
hiatuses, such as \emph{caótico}, \emph{balneário} and \emph{razoável}, while
there are also words with a similar structure that constitute a diphthong, such
as português, \emph{alguém} and \emph{linguística}. It is remarkable, of
course, that the latter are formed by the digraphs \emph{qu-} and \emph{gu-},
while the former by vowels other than \emph{i} and \emph{u}. The fourth and
last group, in turn, which comprises rules \ref{rulegrp_lo} to
\ref{rulegrp_cco}, which are counterparts of rules that were already in the
default rules, but did not contemplate the cases with certain accents. They
were then added to encompass words such as \emph{camelô}, \emph{recôncavo},
\emph{amazônia}, and \emph{maçônico}.

It is important to highlight that the words included in these rules are, in general, of low frequency and were incorporated into the Portuguese language without a phonotactic adaptation, which causes these idiosyncrasies and exceptions to the language. They are characterized as marked cases, since it is not possible to defend that they reproduce a phonological pattern of the language because they are not productive, that is, they are not taken as an example and/or derive new words.


O conjunto de \NumberOfNewRules{} regras é apresentado a seguir:
\begin{multicols}{4}
    \verbatiminput{patch.TeX.pt-br.patterns}
\end{multicols}

A Tabela \ref{tab-resultados} apresenta os erros cometidos no conjunto das
37798 palavras analisadas. As palavras omitidas são aquelas que foram
hifenizadas corretamente pelos dois conjuntos de regra. Utilizamos aqui a
marcação adotada pelo \emph{Patgen}: \texttt{*} para indicar pontos de
hifenização corretamente marcados, \texttt{.} para indicar pontos de hifenização
marcados erroneamente e \texttt{-} para indicar pontos de hifenização não
encontrados. Os pontos marcados erroneamente e aqueles não encontrados são
considerados erros. Quando ocorre algum erro de hifenização, marcaremos a
palavra com \xmark. Quando não houver erro algum, marcaremos a palavra com
\cmark. Ao final, podemos verificar que as regras padrão apresentaram 1368 erros
(3,62\%) e as novas regras 132 erros (0,35\%).

\begin{longtable}{l l l}
\caption{Comparativo dos erros cometidos pelos dois conjuntos de regras de
hifenização.}\label{tab-resultados}\\
    \hline
    palavra & regras padrão & regras adicionais \\
    \hline
    \endhead
    aarônico & a*a-rô*ni*co \xmark & a*a*rô*ni*co \cmark \\
abadia & a*ba*di*a \cmark & a*ba*di-a \xmark \\
abdômen & ab-dô*men \xmark & ab*dô*men \cmark \\
abdominoplastia & ab*do*mi*no*plas*ti*a \cmark & ab*do*mi*no*plas*ti-a \xmark \\
abecedário & a*be*ce*dá*ri.o \xmark & a*be*ce*dá*rio \cmark \\
abegoaria & a*be*go*a*ri*a \cmark & a*be*go*a*ri-a \xmark \\
abiótica & a*bi-ó*ti*ca \xmark & a*bi*ó*ti*ca \cmark \\
abiótico & a*bi-ó*ti*co \xmark & a*bi*ó*ti*co \cmark \\
abissínio & a*bis*sí*ni.o \xmark & a*bis*sí*nio \cmark \\
abiu & a*bi-u \xmark & a*bi*u \cmark \\
abluir & a*blu-ir \xmark & a*blu*ir \cmark \\
aboio & a*boi*o \cmark & a*boi-o \xmark \\
abortício & a*bor*tí*ci.o \xmark & a*bor*tí*cio \cmark \\
abraâmico & a*bra-â*mi*co \xmark & a*bra*â*mi*co \cmark \\
abrangência & a*bran*gên*ci.a \xmark & a*bran*gên*cia \cmark \\
abreugrafia & a*breu*gra*fi*a \cmark & a*breu*gra*fi-a \xmark \\
abrótea & a*bró*te.a \xmark & a*bró*tea \cmark \\
abruptamente & a.b-rup*ta*men*te \xmark & a.b-rup*ta*men*te \xmark \\
abrupto & a.b-rup*to \xmark & a.b-rup*to \xmark \\
absenteísmo & ab*sen*te-ís*mo \xmark & ab*sen*te*ís*mo \cmark \\
absenteísta & ab*sen*te-ís*ta \xmark & ab*sen*te*ís*ta \cmark \\
absorvância & ab*sor*vân*ci.a \xmark & ab*sor*vân*cia \cmark \\
absorvência & ab*sor*vên*ci.a \xmark & ab*sor*vên*cia \cmark \\
abstêmio & abs*tê*mi.o \xmark & abs*tê*mio \cmark \\
abstinência & abs*ti*nên*ci.a \xmark & abs*ti*nên*cia \cmark \\
abstraído & abs*tra-í*do \xmark & abs*tra*í*do \cmark \\
abstrair & abs*tra-ir \xmark & abs*tra*ir \cmark \\
abulia & a*bu*li*a \cmark & a*bu*li-a \xmark \\
abundância & a*bun*dân*ci*a \cmark & a*bun*dân*ci-a \xmark \\
acácia & a*cá*ci.a \xmark & a*cá*cia \cmark \\
acácio & a*cá*ci.o \xmark & a*cá*cio \cmark \\
academia & a*ca*de*mi*a \cmark & a*ca*de*mi-a \xmark \\
acadêmia & a*ca*dê*mi.a \xmark & a*ca*dê*mia \cmark \\
acádio & a*cá*di.o \xmark & a*cá*dio \cmark \\
açaí & a*ça-í \xmark & a*ça*í \cmark \\
acalmia & a*cal*mi*a \cmark & a*cal*mi-a \xmark \\
acatisia & a*ca*ti*si*a \cmark & a*ca*ti*si-a \xmark \\
acauã & a*cau-ã \xmark & a*cau*ã \cmark \\
acefalia & a*ce*fa*li*a \cmark & a*ce*fa*li-a \xmark \\
acelerômetro & a*ce*le-rô*me*tro \xmark & a*ce*le*rô*me*tro \cmark \\
acéquia & a*cé*qui.a \xmark & a*cé*quia \cmark \\
acessório & a*ces*só*ri.o \xmark & a*ces*só*rio \cmark \\
acetaldeído & a*ce*tal*de-í*do \xmark & a*ce*tal*de*í*do \cmark \\
acidemia & a*ci*de*mi*a \cmark & a*ci*de*mi-a \xmark \\
acidúria & a*ci*dú*ri.a \xmark & a*ci*dú*ria \cmark \\
acônito & a-cô*ni*to \xmark & a*cô*ni*to \cmark \\
acordeão & a*cor*de-ão \xmark & a*cor*de*ão \cmark \\
açoteia & a*ço*tei*a \cmark & a*ço*tei-a \xmark \\
acrobacia & a*cro*ba*ci*a \cmark & a*cro*ba*ci-a \xmark \\
acrofobia & a*cro*fo*bi*a \cmark & a*cro*fo*bi-a \xmark \\
acroleína & a*cro*le-í*na \xmark & a*cro*le*í*na \cmark \\
acromatopsia & a*cro*ma*top*si*a \cmark & a*cro*ma*top*si-a \xmark \\
acromegalia & a*cro*me*ga*li*a \cmark & a*cro*me*ga*li-a \xmark \\
acrotério & a*cro*té*ri.o \xmark & a*cro*té*rio \cmark \\
actínio & ac*tí*ni.o \xmark & ac*tí*nio \cmark \\
acurácia & a*cu*rá*ci.a \xmark & a*cu*rá*cia \cmark \\
acusatório & a*cu*sa*tó*ri.o \xmark & a*cu*sa*tó*rio \cmark \\
adágio & a*dá*gi.o \xmark & a*dá*gio \cmark \\
adail & a*da-il \xmark & a*da-il \xmark \\
adenia & a*de*ni*a \cmark & a*de*ni-a \xmark \\
aderência & a*de*rên*ci.a \xmark & a*de*rên*cia \cmark \\
adimplência & a*dim*plên*ci.a \xmark & a*dim*plên*cia \cmark \\
adjacência & ad*ja*cên*ci.a \xmark & ad*ja*cên*cia \cmark \\
adjutório & ad*ju*tó*ri.o \xmark & ad*ju*tó*rio \cmark \\
adolescência & a*do*les*cên*ci.a \xmark & a*do*les*cên*cia \cmark \\
adônis & a-dô*nis \xmark & a*dô*nis \cmark \\
adriático & a*dri-á*ti*co \xmark & a*dri*á*ti*co \cmark \\
adstringência & ads*trin*gên*ci.a \xmark & ads*trin*gên*cia \cmark \\
adua & a*du*a \cmark & a*du-a \xmark \\
adulária & a*du*lá*ri.a \xmark & a*du*lá*ria \cmark \\
adultério & a*dul*té*ri.o \xmark & a*dul*té*rio \cmark \\
adventícia & ad*ven*tí*ci.a \xmark & ad*ven*tí*cia \cmark \\
adventício & ad*ven*tí*ci.o \xmark & ad*ven*tí*cio \cmark \\
advérbio & ad*vér*bi.o \xmark & ad*vér*bio \cmark \\
adversário & ad*ver*sá*ri.o \xmark & ad*ver*sá*rio \cmark \\
advertência & ad*ver*tên*ci.a \xmark & ad*ver*tên*cia \cmark \\
advocacia & ad*vo*ca*ci*a \cmark & ad*vo*ca*ci-a \xmark \\
advogacia & ad*vo*ga*ci*a \cmark & ad*vo*ga*ci-a \xmark \\
aéreo & a-é*re.o \xmark & a*é*reo \cmark \\
aeróbio & a*e*ró*bi.o \xmark & a*e*ró*bio \cmark \\
aerofobia & a*e*ro*fo*bi*a \cmark & a*e*ro*fo*bi-a \xmark \\
aerofólio & a*e*ro*fó*li.o \xmark & a*e*ro*fó*lio \cmark \\
aerofotografia & a*e*ro*fo*to*gra*fi*a \cmark & a*e*ro*fo*to*gra*fi-a \xmark \\
aerografia & a*e*ro*gra*fi*a \cmark & a*e*ro*gra*fi-a \xmark \\
aeromancia & a*e*ro*man*ci*a \cmark & a*e*ro*man*ci-a \xmark \\
aeronomia & a*e*ro*no*mi*a \cmark & a*e*ro*no*mi-a \xmark \\
aeroportuário & a*e*ro*por*tu-á*ri.o \xmark & a*e*ro*por*tu*á*rio \cmark \\
aerovia & a*e*ro*vi*a \cmark & a*e*ro*vi-a \xmark \\
aeroviário & a*e*ro*vi-á*ri.o \xmark & a*e*ro*vi*á*rio \cmark \\
aético & a-é*ti*co \xmark & a*é*ti*co \cmark \\
afasia & a*fa*si*a \cmark & a*fa*si-a \xmark \\
afélio & a*fé*li.o \xmark & a*fé*lio \cmark \\
afluência & a*flu-ên*ci.a \xmark & a*flu*ên*cia \cmark \\
afluir & a*flu-ir \xmark & a*flu*ir \cmark \\
aforia & a*fo*ri*a \cmark & a*fo*ri-a \xmark \\
afrodisíaco & a*fro*di*sí-a*co \xmark & a*fro*di*sí*a*co \cmark \\
agamia & a*ga*mi*a \cmark & a*ga*mi-a \xmark \\
agência & a*gên*ci.a \xmark & a*gên*cia \cmark \\
agenesia & a*ge*ne*si*a \cmark & a*ge*ne*si-a \xmark \\
ágio & á*gi*o \cmark & á*gi-o \xmark \\
aglossia & a*glos*si*a \cmark & a*glos*si-a \xmark \\
agnosia & ag*no*si*a \cmark & ag*no*si-a \xmark \\
agogô & a*go-gô \xmark & a*go*gô \cmark \\
agonia & a*go*ni*a \cmark & a*go*ni-a \xmark \\
agônico & a-gô*ni*co \xmark & a*gô*ni*co \cmark \\
agorafobia & a*go*ra*fo*bi*a \cmark & a*go*ra*fo*bi-a \xmark \\
agrário & a*grá*ri.o \xmark & a*grá*rio \cmark \\
agrião & a*gri-ão \xmark & a*gri*ão \cmark \\
agroindústria & a*gro-in*dús*tri.a \xmark & a*gro*in*dús*tria \cmark \\
agronomia & a*gro*no*mi*a \cmark & a*gro*no*mi-a \xmark \\
agronômico & a*gro-nô*mi*co \xmark & a*gro*nô*mi*co \cmark \\
agropecuária & a*gro*pe*cu-á*ri.a \xmark & a*gro*pe*cu*á*ria \cmark \\
agroquímica & a*gro-quí*mi*ca \xmark & a*gro*quí*mi*ca \cmark \\
agroquímico & a*gro-quí*mi*co \xmark & a*gro*quí*mi*co \cmark \\
aguaí & a*gua-í \xmark & a*gua*í \cmark \\
águia & á*gui.a \xmark & á*guia \cmark \\
aia & ai*a \cmark & ai-a \xmark \\
aí & a-í \xmark & a*í \cmark \\
ainda & a-in*da \xmark & a*in*da \cmark \\
ainsa & a-in*sa \xmark & a-in*sa \xmark \\
aio & ai*o \cmark & ai-o \xmark \\
aiquidô & ai*qui-dô \xmark & ai*qui*dô \cmark \\
ajuizado & a*ju-i*za*do \xmark & a*ju*i*za*do \cmark \\
ajuizar & a*ju-i*zar \xmark & a*ju*i*zar \cmark \\
alagoa & a*la*go*a \cmark & a*la*go-a \xmark \\
alagoinha & a*la*go-i*nha \xmark & a*la*go*i*nha \cmark \\
alaúde & a*la-ú*de \xmark & a*la*ú*de \cmark \\
albergaria & al*ber*ga*ri*a \cmark & al*ber*ga*ri-a \xmark \\
albugínea & al*bu*gí*ne*a \cmark & al*bu*gí*ne-a \xmark \\
albuminúria & al*bu*mi*nú*ri.a \xmark & al*bu*mi*nú*ria \cmark \\
alcaidaria & al*cai*da*ri*a \cmark & al*cai*da*ri-a \xmark \\
alcaptonúria & al*cap*to*nú*ri.a \xmark & al*cap*to*nú*ria \cmark \\
alcaravia & al*ca*ra*vi*a \cmark & al*ca*ra*vi-a \xmark \\
alcaria & al*ca*ri*a \cmark & al*ca*ri-a \xmark \\
alcateia & al*ca*tei*a \cmark & al*ca*tei-a \xmark \\
alcíone & al*cí-o*ne \xmark & al*cí*o*ne \cmark \\
alcoólatra & al*co-ó*la*tra \xmark & al*co*ó*la*tra \cmark \\
alcoolemia & al*co*o*le*mi*a \cmark & al*co*o*le*mi-a \xmark \\
alcoólico & al*co-ó*li*co \xmark & al*co*ó*li*co \cmark \\
aldeão & al*de-ão \xmark & al*de*ão \cmark \\
aldeia & al*dei*a \cmark & al*dei-a \xmark \\
aldeído & al*de-í*do \xmark & al*de*í*do \cmark \\
álea & á*le.a \xmark & á*lea \cmark \\
aleatório & a*le*a*tó*ri.o \xmark & a*le*a*tó*rio \cmark \\
alegoria & a*le*go*ri*a \cmark & a*le*go*ri-a \xmark \\
alegria & a*le*gri*a \cmark & a*le*gri-a \xmark \\
aleia & a*lei*a \cmark & a*lei-a \xmark \\
aleivosia & a*lei*vo*si*a \cmark & a*lei*vo*si-a \xmark \\
alelopatia & a*le*lo*pa*ti*a \cmark & a*le*lo*pa*ti-a \xmark \\
aleluia & a*le*lui*a \cmark & a*le*lui-a \xmark \\
alergia & a*ler*gi*a \cmark & a*ler*gi-a \xmark \\
alergologia & a*ler*go*lo*gi*a \cmark & a*ler*go*lo*gi-a \xmark \\
aletria & a*le*tri*a \cmark & a*le*tri-a \xmark \\
aleúte & a*le-ú*te \xmark & a*le*ú*te \cmark \\
alexia & a*le*xi*a \cmark & a*le*xi-a \xmark \\
alfaia & al*fai*a \cmark & al*fai-a \xmark \\
alfaiataria & al*fai*a*ta*ri*a \cmark & al*fai*a*ta*ri-a \xmark \\
alfandegário & al*fan*de*gá*ri.o \xmark & al*fan*de*gá*rio \cmark \\
alfarrábio & al*far*rá*bi.o \xmark & al*far*rá*bio \cmark \\
alforria & al*for*ri*a \cmark & al*for*ri-a \xmark \\
algaravia & al*ga*ra*vi*a \cmark & al*ga*ra*vi-a \xmark \\
algarvio & al*gar*vi*o \cmark & al*gar*vi-o \xmark \\
algofobia & al*go*fo*bi*a \cmark & al*go*fo*bi-a \xmark \\
alheio & a*lhei*o \cmark & a*lhei-o \xmark \\
aliá & a*li-á \xmark & a*li*á \cmark \\
aliáceo & a*li-á*ce.o \xmark & a*li*á*ceo \cmark \\
aliás & a*li-ás \xmark & a*li*ás \cmark \\
alimária & a*li*má*ri.a \xmark & a*li*má*ria \cmark \\
alimentício & a*li*men*tí*ci.o \xmark & a*li*men*tí*cio \cmark \\
alínea & a*lí*ne.a \xmark & a*lí*nea \cmark \\
alívio & a*lí*vi.o \xmark & a*lí*vio \cmark \\
almóada & al*mó-a*da \xmark & al*mó-a*da \xmark \\
almôndega & al-môn*de*ga \xmark & al*môn*de*ga \cmark \\
alô & a-lô \xmark & a*lô \cmark \\
alódio & a*ló*di.o \xmark & a*ló*dio \cmark \\
aloé & a*lo-é \xmark & a*lo*é \cmark \\
aloés & a*lo-és \xmark & a*lo*és \cmark \\
alometria & a*lo*me*tri*a \cmark & a*lo*me*tri-a \xmark \\
alopatia & a*lo*pa*ti*a \cmark & a*lo*pa*ti-a \xmark \\
alopecia & a*lo*pe*ci*a \cmark & a*lo*pe*ci-a \xmark \\
alquimia & al*qui*mi*a \cmark & al*qui*mi-a \xmark \\
alquímico & al-quí*mi*co \xmark & al*quí*mi*co \cmark \\
alteia & al*tei*a \cmark & al*tei-a \xmark \\
alternância & al*ter*nân*ci.a \xmark & al*ter*nân*cia \cmark \\
altimetria & al*ti*me*tri*a \cmark & al*ti*me*tri-a \xmark \\
altruísmo & al*tru-ís*mo \xmark & al*tru*ís*mo \cmark \\
altruísta & al*tru-ís*ta \xmark & al*tru*ís*ta \cmark \\
aluá & a*lu-á \xmark & a*lu*á \cmark \\
alucinatório & a*lu*ci*na*tó*ri.o \xmark & a*lu*ci*na*tó*rio \cmark \\
aluir & a*lu-ir \xmark & a*lu*ir \cmark \\
alumínio & a*lu*mí*ni.o \xmark & a*lu*mí*nio \cmark \\
aluvião & a*lu*vi-ão \xmark & a*lu*vi*ão \cmark \\
alvedrio & al*ve*dri*o \cmark & al*ve*dri-o \xmark \\
alvenaria & al*ve*na*ri*a \cmark & al*ve*na*ri-a \xmark \\
alvéolo & al*vé-o*lo \xmark & al*vé*o*lo \cmark \\
amalteia & a*mal*tei*a \cmark & a*mal*tei-a \xmark \\
amásia & a*má*si.a \xmark & a*má*sia \cmark \\
amásio & a*má*si.o \xmark & a*má*sio \cmark \\
amazia & a*ma*zi*a \cmark & a*ma*zi-a \xmark \\
amazônico & a*ma-zô*ni*co \xmark & a*ma*zô*ni*co \cmark \\
ambidestria & am*bi*des*tri*a \cmark & am*bi*des*tri-a \xmark \\
ambiência & am*bi*ên*ci.a \xmark & am*bi*ên*cia \cmark \\
ambivalência & am*bi*va*lên*ci.a \xmark & am*bi*va*lên*cia \cmark \\
ambliopia & am*bli*o*pi*a \cmark & am*bli*o*pi-a \xmark \\
ambrosia & am*bro*si*a \cmark & am*bro*si-a \xmark \\
ambrósia & am*bró*si.a \xmark & am*bró*sia \cmark \\
ambulacrário & am*bu*la*crá*ri.o \xmark & am*bu*la*crá*rio \cmark \\
ambulância & am*bu*lân*ci.a \xmark & am*bu*lân*cia \cmark \\
ambulatório & am*bu*la*tó*ri.o \xmark & am*bu*la*tó*rio \cmark \\
amebíase & a*me*bí-a*se \xmark & a*me*bí*a*se \cmark \\
ameia & a*mei*a \cmark & a*mei-a \xmark \\
amêijoa & a*mêi*jo.a \xmark & a*mêi*joa \cmark \\
amelia & a*me*li*a \cmark & a*me*li-a \xmark \\
amélia & a*mé*li.a \xmark & a*mé*lia \cmark \\
amêndoa & a*mên*do.a \xmark & a*mên*doa \cmark \\
amendoim & a*men*do-im \xmark & a*men*do*im \cmark \\
amenorreia & a*me*nor*rei*a \cmark & a*me*nor*rei-a \xmark \\
americanofilia & a*me*ri*ca*no*fi*li*a \cmark & a*me*ri*ca*no*fi*li-a \xmark \\
americanofobia & a*me*ri*ca*no*fo*bi*a \cmark & a*me*ri*ca*no*fo*bi-a \xmark \\
amerício & a*me*rí*ci.o \xmark & a*me*rí*cio \cmark \\
ameríndio & a*me*rín*di.o \xmark & a*me*rín*dio \cmark \\
amigdalectomia & a*mig*da*lec*to*mi*a \cmark & a*mig*da*lec*to*mi-a \xmark \\
aminoácido & a*mi*no-á*ci*do \xmark & a*mi*no*á*ci*do \cmark \\
amiúde & a*mi-ú*de \xmark & a*mi*ú*de \cmark \\
amnesia & am*ne*si*a \cmark & am*ne*si-a \xmark \\
âmnio & âm*ni.o \xmark & âm*nio \cmark \\
amônia & a-mô*ni.a \xmark & a*mô*nia \cmark \\
amoníaco & a*mo*ní-a*co \xmark & a*mo*ní*a*co \cmark \\
amontoa & a*mon*to*a \cmark & a*mon*to-a \xmark \\
ampelografia & am*pe*lo*gra*fi*a \cmark & am*pe*lo*gra*fi-a \xmark \\
anaeróbio & a*na*e*ró*bi.o \xmark & a*na*e*ró*bio \cmark \\
anafilaxia & a*na*fi*la*xi*a \cmark & a*na*fi*la*xi-a \xmark \\
anafrodisíaco & a*na*fro*di*sí-a*co \xmark & a*na*fro*di*sí*a*co \cmark \\
analgesia & a*nal*ge*si*a \cmark & a*nal*ge*si-a \xmark \\
analogia & a*na*lo*gi*a \cmark & a*na*lo*gi-a \xmark \\
anarquia & a*nar*qui*a \cmark & a*nar*qui-a \xmark \\
anastácio & a*nas*tá*ci.o \xmark & a*nas*tá*cio \cmark \\
anatólio & a*na*tó*li.o \xmark & a*na*tó*lio \cmark \\
anatomia & a*na*to*mi*a \cmark & a*na*to*mi-a \xmark \\
anatômico & a*na-tô*mi*co \xmark & a*na*tô*mi*co \cmark \\
anauê & a*nau-ê \xmark & a*nau*ê \cmark \\
ancião & an*ci-ão \xmark & an*ci*ão \cmark \\
ancilostomíase & an*ci*los*to*mí-a*se \xmark & an*ci*los*to*mí*a*se \cmark \\
androfilia & an*dro*fi*li*a \cmark & an*dro*fi*li-a \xmark \\
androfobia & an*dro*fo*bi*a \cmark & an*dro*fo*bi-a \xmark \\
androginia & an*dro*gi*ni*a \cmark & an*dro*gi*ni-a \xmark \\
andrologia & an*dro*lo*gi*a \cmark & an*dro*lo*gi-a \xmark \\
anedonia & a*ne*do*ni*a \cmark & a*ne*do*ni-a \xmark \\
anedotário & a*ne*do*tá*ri.o \xmark & a*ne*do*tá*rio \cmark \\
anelídeo & a*ne*lí*de.o \xmark & a*ne*lí*deo \cmark \\
anemia & a*ne*mi*a \cmark & a*ne*mi-a \xmark \\
anemofilia & a*ne*mo*fi*li*a \cmark & a*ne*mo*fi*li-a \xmark \\
anemômetro & a*ne-mô*me*tro \xmark & a*ne*mô*me*tro \cmark \\
anencefalia & a*nen*ce*fa*li*a \cmark & a*nen*ce*fa*li-a \xmark \\
anestesia & a*nes*te*si*a \cmark & a*nes*te*si-a \xmark \\
anestesiologia & a*nes*te*si*o*lo*gi*a \cmark & a*nes*te*si*o*lo*gi-a \xmark \\
aneuploidia & a*neu*ploi*di*a \cmark & a*neu*ploi*di-a \xmark \\
anfião & an*fi-ão \xmark & an*fi*ão \cmark \\
anfíbio & an*fí*bi.o \xmark & an*fí*bio \cmark \\
anfibólio & an*fi*bó*li.o \xmark & an*fi*bó*lio \cmark \\
anfibologia & an*fi*bo*lo*gi*a \cmark & an*fi*bo*lo*gi-a \xmark \\
anfictião & an*fic*ti-ão \xmark & an*fic*ti*ão \cmark \\
anfictionia & an*fic*ti*o*ni*a \cmark & an*fic*ti*o*ni-a \xmark \\
anfitrião & an*fi*tri-ão \xmark & an*fi*tri*ão \cmark \\
angelologia & an*ge*lo*lo*gi*a \cmark & an*ge*lo*lo*gi-a \xmark \\
angiografia & an*gi*o*gra*fi*a \cmark & an*gi*o*gra*fi-a \xmark \\
angiologia & an*gi*o*lo*gi*a \cmark & an*gi*o*lo*gi-a \xmark \\
angioplastia & an*gi*o*plas*ti*a \cmark & an*gi*o*plas*ti-a \xmark \\
angioscopia & an*gi*os*co*pi*a \cmark & an*gi*os*co*pi-a \xmark \\
anglofilia & an*glo*fi*li*a \cmark & an*glo*fi*li-a \xmark \\
anglofonia & an*glo*fo*ni*a \cmark & an*glo*fo*ni-a \xmark \\
angústia & an*gús*ti.a \xmark & an*gús*tia \cmark \\
anião & a*ni-ão \xmark & a*ni*ão \cmark \\
aniridia & a*ni*ri*di*a \cmark & a*ni*ri*di-a \xmark \\
anisogamia & a*ni*so*ga*mi*a \cmark & a*ni*so*ga*mi-a \xmark \\
anisotropia & a*ni*so*tro*pi*a \cmark & a*ni*so*tro*pi-a \xmark \\
anistia & a*nis*ti*a \cmark & a*nis*ti-a \xmark \\
aniversário & a*ni*ver*sá*ri.o \xmark & a*ni*ver*sá*rio \cmark \\
anodinia & a*no*di*ni*a \cmark & a*no*di*ni-a \xmark \\
anomalia & a*no*ma*li*a \cmark & a*no*ma*li-a \xmark \\
anômalo & a-nô*ma*lo \xmark & a*nô*ma*lo \cmark \\
anomia & a*no*mi*a \cmark & a*no*mi-a \xmark \\
anônimo & a-nô*ni*mo \xmark & a*nô*ni*mo \cmark \\
anorexia & a*no*re*xi*a \cmark & a*no*re*xi-a \xmark \\
anosmia & a*nos*mi*a \cmark & a*nos*mi-a \xmark \\
anoxia & a*no*xi*a \cmark & a*no*xi-a \xmark \\
anseio & an*sei*o \cmark & an*sei-o \xmark \\
ânsia & ân*si.a \xmark & ân*sia \cmark \\
antagônico & an*ta-gô*ni*co \xmark & an*ta*gô*ni*co \cmark \\
antecedência & an*te*ce*dên*ci.a \xmark & an*te*ce*dên*cia \cmark \\
anterídio & an*te*rí*di.o \xmark & an*te*rí*dio \cmark \\
antestreia & an*tes*trei*a \cmark & an*tes*trei-a \xmark \\
antiácido & an*ti-á*ci*do \xmark & an*ti*á*ci*do \cmark \\
antiaéreo & an*ti*a-é*re.o \xmark & an*ti*a*é*reo \cmark \\
antialcoólico & an*ti*al*co-ó*li*co \xmark & an*ti*al*co*ó*li*co \cmark \\
antibiótico & an*ti*bi-ó*ti*co \xmark & an*ti*bi*ó*ti*co \cmark \\
antidemocracia & an*ti*de*mo*cra*ci*a \cmark & an*ti*de*mo*cra*ci-a \xmark \\
antiético & an*ti-é*ti*co \xmark & an*ti*é*ti*co \cmark \\
antifonário & an*ti*fo*ná*ri.o \xmark & an*ti*fo*ná*rio \cmark \\
antimatéria & an*ti*ma*té*ri.a \xmark & an*ti*ma*té*ria \cmark \\
antimônio & an*ti-mô*ni.o \xmark & an*ti*mô*nio \cmark \\
antinomia & an*ti*no*mi*a \cmark & an*ti*no*mi-a \xmark \\
antinômico & an*ti-nô*mi*co \xmark & an*ti*nô*mi*co \cmark \\
antipatia & an*ti*pa*ti*a \cmark & an*ti*pa*ti-a \xmark \\
antipatriótico & an*ti*pa*tri-ó*ti*co \xmark & an*ti*pa*tri*ó*ti*co \cmark \\
antipoluição & an*ti*po*lu-i*ção \xmark & an*ti*po*lu*i*ção \cmark \\
antiquário & an*ti-quá*ri.o \xmark & an*ti*quá*rio \cmark \\
antiquíssimo & an*ti-quís*si*mo \xmark & an*ti*quís*si*mo \cmark \\
antissepsia & an*tis*sep*si*a \cmark & an*tis*sep*si-a \xmark \\
antologia & an*to*lo*gi*a \cmark & an*to*lo*gi-a \xmark \\
antônimo & an-tô*ni*mo \xmark & an*tô*ni*mo \cmark \\
antonomásia & an*to*no*má*si.a \xmark & an*to*no*má*sia \cmark \\
antropofagia & an*tro*po*fa*gi*a \cmark & an*tro*po*fa*gi-a \xmark \\
antropologia & an*tro*po*lo*gi*a \cmark & an*tro*po*lo*gi-a \xmark \\
antropometria & an*tro*po*me*tri*a \cmark & an*tro*po*me*tri-a \xmark \\
antroponímia & an*tro*po*ní*mi.a \xmark & an*tro*po*ní*mia \cmark \\
antropônimo & an*tro-pô*ni*mo \xmark & an*tro*pô*ni*mo \cmark \\
antroposofia & an*tro*po*so*fi*a \cmark & an*tro*po*so*fi-a \xmark \\
antúrio & an*tú*ri.o \xmark & an*tú*rio \cmark \\
anuário & a*nu-á*ri.o \xmark & a*nu*á*rio \cmark \\
anuência & a*nu-ên*ci.a \xmark & a*nu*ên*cia \cmark \\
anuidade & a*nu-i*da*de \xmark & a*nu*i*da*de \cmark \\
anuir & a*nu-ir \xmark & a*nu*ir \cmark \\
anúncio & a*nún*ci.o \xmark & a*nún*cio \cmark \\
anúria & a*nú*ri.a \xmark & a*nú*ria \cmark \\
aórtico & a-ór*ti*co \xmark & a*ór*ti*co \cmark \\
apanágio & a*pa*ná*gi.o \xmark & a*pa*ná*gio \cmark \\
aparência & a*pa*rên*ci.a \xmark & a*pa*rên*cia \cmark \\
apartidário & a*par*ti*dá*ri.o \xmark & a*par*ti*dá*rio \cmark \\
apatia & a*pa*ti*a \cmark & a*pa*ti-a \xmark \\
apendicectomia & a*pen*di*cec*to*mi*a \cmark & a*pen*di*cec*to*mi-a \xmark \\
aperiódico & a*pe*ri-ó*di*co \xmark & a*pe*ri*ó*di*co \cmark \\
apetência & a*pe*tên*ci.a \xmark & a*pe*tên*cia \cmark \\
aplasia & a*pla*si*a \cmark & a*pla*si-a \xmark \\
apneia & ap*nei*a \cmark & ap*nei-a \xmark \\
apofonia & a*po*fo*ni*a \cmark & a*po*fo*ni-a \xmark \\
apoio & a*poi*o \cmark & a*poi-o \xmark \\
apologia & a*po*lo*gi*a \cmark & a*po*lo*gi-a \xmark \\
apomixia & a*po*mi*xi*a \cmark & a*po*mi*xi-a \xmark \\
apoplexia & a*po*ple*xi*a \cmark & a*po*ple*xi-a \xmark \\
aporia & a*po*ri*a \cmark & a*po*ri-a \xmark \\
aposentadoria & a*po*sen*ta*do*ri*a \cmark & a*po*sen*ta*do*ri-a \xmark \\
apostasia & a*pos*ta*si*a \cmark & a*pos*ta*si-a \xmark \\
apoteótico & a*po*te-ó*ti*co \xmark & a*po*te*ó*ti*co \cmark \\
apraxia & a*pra*xi*a \cmark & a*pra*xi-a \xmark \\
apreciável & a*pre*ci-á*vel \xmark & a*pre*ci*á*vel \cmark \\
apuí & a*pu-í \xmark & a*pu*í \cmark \\
aquário & a-quá*ri.o \xmark & a*quá*rio \cmark \\
aquariofilia & a*qua*ri*o*fi*li*a \cmark & a*qua*ri*o*fi*li-a \xmark \\
aquático & a-quá*ti*co \xmark & a*quá*ti*co \cmark \\
aquém & a-quém \xmark & a*quém \cmark \\
aquênio & a-quê*ni.o \xmark & a*quê*nio \cmark \\
aquícola & a-quí*co*la \xmark & a*quí*co*la \cmark \\
aquiescência & a*qui*es*cên*ci.a \xmark & a*qui*es*cên*cia \cmark \\
aquífero & a-quí*fe*ro \xmark & a*quí*fe*ro \cmark \\
aquileia & a*qui*lei*a \cmark & a*qui*lei-a \xmark \\
arábias & a*rá*bi.as \xmark & a*rá*bi.as \xmark \\
aracnídeo & a*rac*ní*de.o \xmark & a*rac*ní*deo \cmark \\
aracnofobia & a*rac*no*fo*bi*a \cmark & a*rac*no*fo*bi-a \xmark \\
aracnologia & a*rac*no*lo*gi*a \cmark & a*rac*no*lo*gi-a \xmark \\
arapuá & a*ra*pu-á \xmark & a*ra*pu*á \cmark \\
araucária & a*rau*cá*ri.a \xmark & a*rau*cá*ria \cmark \\
araúna & a*ra-ú*na \xmark & a*ra*ú*na \cmark \\
arbitrário & ar*bi*trá*ri.o \xmark & ar*bi*trá*rio \cmark \\
arbítrio & ar*bí*tri.o \xmark & ar*bí*trio \cmark \\
arbóreo & ar*bó*re.o \xmark & ar*bó*reo \cmark \\
arcádia & ar*cá*di.a \xmark & ar*cá*dia \cmark \\
arcaísmo & ar*ca-ís*mo \xmark & ar*ca*ís*mo \cmark \\
arcaria & ar*ca*ri*a \cmark & ar*ca*ri-a \xmark \\
ardência & ar*dên*ci.a \xmark & ar*dên*cia \cmark \\
ardósia & ar*dó*si.a \xmark & ar*dó*sia \cmark \\
árduo & ár*du.o \xmark & ár*duo \cmark \\
área & á*re.a \xmark & á*rea \cmark \\
areão & a*re-ão \xmark & a*re*ão \cmark \\
areia & a*rei*a \cmark & a*rei-a \xmark \\
areópago & a*re-ó*pa*go \xmark & a*re*ó*pa*go \cmark \\
argênteo & ar*gên*te*o \cmark & ar*gên*te-o \xmark \\
argônio & ar-gô*ni.o \xmark & ar*gô*nio \cmark \\
argúcia & ar*gú*ci.a \xmark & ar*gú*cia \cmark \\
ária & á*ri.a \xmark & á*ria \cmark \\
aríete & a*rí-e*te \xmark & a*rí*e*te \cmark \\
aristocracia & a*ris*to*cra*ci*a \cmark & a*ris*to*cra*ci-a \xmark \\
armário & ar*má*ri.o \xmark & ar*má*rio \cmark \\
armênio & ar*mê*ni.o \xmark & ar*mê*nio \cmark \\
armistício & ar*mis*tí*ci.o \xmark & ar*mis*tí*cio \cmark \\
aromaterapia & a*ro*ma*te*ra*pi*a \cmark & a*ro*ma*te*ra*pi-a \xmark \\
arquegônio & ar*que-gô*ni.o \xmark & ar*que*gô*nio \cmark \\
arqueologia & ar*que*o*lo*gi*a \cmark & ar*que*o*lo*gi-a \xmark \\
arqueólogo & ar*que-ó*lo*go \xmark & ar*que*ó*lo*go \cmark \\
arquétipo & ar-qué*ti*po \xmark & ar*qué*ti*po \cmark \\
arquiconfraria & ar*qui*con*fra*ri*a \cmark & ar*qui*con*fra*ri-a \xmark \\
arquidiácono & ar*qui*di-á*co*no \xmark & ar*qui*di*á*co*no \cmark \\
arquimilionário & ar*qui*mi*li*o*ná*ri.o \xmark & ar*qui*mi*li*o*ná*rio \cmark \\
arquitetônica & ar*qui*te-tô*ni*ca \xmark & ar*qui*te*tô*ni*ca \cmark \\
arquitetônico & ar*qui*te-tô*ni*co \xmark & ar*qui*te*tô*ni*co \cmark \\
arraia & ar*rai*a \cmark & ar*rai-a \xmark \\
arredio & ar*re*di*o \cmark & ar*re*di-o \xmark \\
arreio & ar*rei*o \cmark & ar*rei-o \xmark \\
arrelia & ar*re*li*a \cmark & ar*re*li-a \xmark \\
arrepio & ar*re*pi*o \cmark & ar*re*pi-o \xmark \\
arritmia & ar*rit*mi*a \cmark & ar*rit*mi-a \xmark \\
arrogância & ar*ro*gân*ci.a \xmark & ar*ro*gân*cia \cmark \\
arroio & ar*roi*o \cmark & ar*roi-o \xmark \\
arroteia & ar*ro*tei*a \cmark & ar*ro*tei-a \xmark \\
arruinar & ar*ru-i*nar \xmark & ar*ru*i*nar \cmark \\
arsênio & ar*sê*ni.o \xmark & ar*sê*nio \cmark \\
artemísia & ar*te*mí*si.a \xmark & ar*te*mí*sia \cmark \\
artéria & ar*té*ri.a \xmark & ar*té*ria \cmark \\
arteriografia & ar*te*ri*o*gra*fi*a \cmark & ar*te*ri*o*gra*fi-a \xmark \\
arteríola & ar*te*rí-o*la \xmark & ar*te*rí*o*la \cmark \\
articulatório & ar*ti*cu*la*tó*ri*o \cmark & ar*ti*cu*la*tó*ri-o \xmark \\
artifício & ar*ti*fí*ci.o \xmark & ar*ti*fí*cio \cmark \\
artilharia & ar*ti*lha*ri*a \cmark & ar*ti*lha*ri-a \xmark \\
artralgia & ar*tral*gi*a \cmark & ar*tral*gi-a \xmark \\
artroscopia & ar*tros*co*pi*a \cmark & ar*tros*co*pi-a \xmark \\
aruá & a*ru-á \xmark & a*ru*á \cmark \\
ascaridíase & as*ca*ri*dí-a*se \xmark & as*ca*ri*dí*a*se \cmark \\
ascendência & as*cen*dên*ci.a \xmark & as*cen*dên*cia \cmark \\
asfixia & as*fi*xi*a \cmark & as*fi*xi-a \xmark \\
asiática & a*si-á*ti*ca \xmark & a*si*á*ti*ca \cmark \\
asiático & a*si-á*ti*co \xmark & a*si*á*ti*co \cmark \\
assassínio & as*sas*sí*ni.o \xmark & as*sas*sí*nio \cmark \\
assédio & as*sé*di.o \xmark & as*sé*dio \cmark \\
assembleia & as*sem*blei*a \cmark & as*sem*blei-a \xmark \\
assepsia & as*sep*si*a \cmark & as*sep*si-a \xmark \\
assessoria & as*ses*so*ri*a \cmark & as*ses*so*ri-a \xmark \\
assiduidade & as*si*du-i*da*de \xmark & as*si*du*i*da*de \cmark \\
assíduo & as*sí*du.o \xmark & as*sí*duo \cmark \\
assimetria & as*si*me*tri*a \cmark & as*si*me*tri-a \xmark \\
assincronia & as*sin*cro*ni*a \cmark & as*sin*cro*ni-a \xmark \\
assiriologia & as*si*ri*o*lo*gi*a \cmark & as*si*ri*o*lo*gi-a \xmark \\
assiriólogo & as*si*ri-ó*lo*go \xmark & as*si*ri*ó*lo*go \cmark \\
assistência & as*sis*tên*ci.a \xmark & as*sis*tên*cia \cmark \\
assobio & as*so*bi*o \cmark & as*so*bi-o \xmark \\
associável & as*so*ci-á*vel \xmark & as*so*ci*á*vel \cmark \\
assonância & as*so*nân*ci.a \xmark & as*so*nân*cia \cmark \\
astenia & as*te*ni*a \cmark & as*te*ni-a \xmark \\
astéria & as*té*ri.a \xmark & as*té*ria \cmark \\
astreia & as*trei*a \cmark & as*trei-a \xmark \\
astrobiologia & as*tro*bi*o*lo*gi*a \cmark & as*tro*bi*o*lo*gi-a \xmark \\
astrofotografia & as*tro*fo*to*gra*fi*a \cmark & as*tro*fo*to*gra*fi-a \xmark \\
astrolábio & as*tro*lá*bi.o \xmark & as*tro*lá*bio \cmark \\
astrologia & as*tro*lo*gi*a \cmark & as*tro*lo*gi-a \xmark \\
astrometria & as*tro*me*tri*a \cmark & as*tro*me*tri-a \xmark \\
astronomia & as*tro*no*mi*a \cmark & as*tro*no*mi-a \xmark \\
astronômico & as*tro-nô*mi*co \xmark & as*tro*nô*mi*co \cmark \\
astroquímica & as*tro-quí*mi*ca \xmark & as*tro*quí*mi*ca \cmark \\
astúcia & as*tú*ci.a \xmark & as*tú*cia \cmark \\
atalaia & a*ta*lai*a \cmark & a*ta*lai-a \xmark \\
ataraxia & a*ta*ra*xi*a \cmark & a*ta*ra*xi-a \xmark \\
ataúde & a*ta-ú*de \xmark & a*ta*ú*de \cmark \\
ataxia & a*ta*xi*a \cmark & a*ta*xi-a \xmark \\
ateísmo & a*te-ís*mo \xmark & a*te*ís*mo \cmark \\
ateísta & a*te-ís*ta \xmark & a*te*ís*ta \cmark \\
atelectasia & a*te*lec*ta*si*a \cmark & a*te*lec*ta*si-a \xmark \\
atentatório & a*ten*ta*tó*ri.o \xmark & a*ten*ta*tó*rio \cmark \\
atômico & a-tô*mi*co \xmark & a*tô*mi*co \cmark \\
atonia & a*to*ni*a \cmark & a*to*ni-a \xmark \\
atônito & a-tô*ni*to \xmark & a*tô*ni*to \cmark \\
atopia & a*to*pi*a \cmark & a*to*pi-a \xmark \\
atrabiliário & a*tra*bi*li-á*ri.o \xmark & a*tra*bi*li*á*rio \cmark \\
atraído & a*tra-í*do \xmark & a*tra*í*do \cmark \\
atrair & a*tra-ir \xmark & a*tra*ir \cmark \\
atresia & a*tre*si*a \cmark & a*tre*si-a \xmark \\
atribuição & a*tri*bu-i*ção \xmark & a*tri*bu*i*ção \cmark \\
atribuir & a*tri*bu-ir \xmark & a*tri*bu*ir \cmark \\
atribuível & a*tri*bu-í*vel \xmark & a*tri*bu*í*vel \cmark \\
átrio & á*tri.o \xmark & á*trio \cmark \\
atrofia & a*tro*fi*a \cmark & a*tro*fi-a \xmark \\
atuário & a*tu-á*ri.o \xmark & a*tu*á*rio \cmark \\
audácia & au*dá*ci.a \xmark & au*dá*cia \cmark \\
audiência & au*di*ên*ci*a \cmark & au*di*ên*ci-a \xmark \\
áudio & áu*di.o \xmark & áu*dio \cmark \\
audiologia & au*di*o*lo*gi*a \cmark & au*di*o*lo*gi-a \xmark \\
audiometria & au*di*o*me*tri*a \cmark & au*di*o*me*tri-a \xmark \\
auditoria & au*di*to*ri*a \cmark & au*di*to*ri-a \xmark \\
auditório & au*di*tó*ri.o \xmark & au*di*tó*rio \cmark \\
auê & au-ê \xmark & au*ê \cmark \\
augúrio & au*gú*ri.o \xmark & au*gú*rio \cmark \\
aurélia & au*ré*li.a \xmark & au*ré*lia \cmark \\
áureo & áu*re.o \xmark & áu*reo \cmark \\
auréola & au*ré-o*la \xmark & au*ré*o*la \cmark \\
ausência & au*sên*ci.a \xmark & au*sên*cia \cmark \\
auspício & aus*pí*ci.o \xmark & aus*pí*cio \cmark \\
austríaco & aus*trí-a*co \xmark & aus*trí*a*co \cmark \\
autarquia & au*tar*qui*a \cmark & au*tar*qui-a \xmark \\
autobiografia & au*to*bi*o*gra*fi*a \cmark & au*to*bi*o*gra*fi-a \xmark \\
autobiógrafo & au*to*bi-ó*gra*fo \xmark & au*to*bi*ó*gra*fo \cmark \\
autocomplacência & au*to*com*pla*cên*ci.a \xmark & au*to*com*pla*cên*cia \cmark \\
autoconsciência & au*to*cons*ci*ên*ci.a \xmark & au*to*cons*ci*ên*cia \cmark \\
autocracia & au*to*cra*ci*a \cmark & au*to*cra*ci-a \xmark \\
autodestruição & au*to*des*tru-i*ção \xmark & au*to*des*tru*i*ção \cmark \\
autodomínio & au*to*do*mí*ni.o \xmark & au*to*do*mí*nio \cmark \\
autofagia & au*to*fa*gi*a \cmark & au*to*fa*gi-a \xmark \\
autogamia & au*to*ga*mi*a \cmark & au*to*ga*mi-a \xmark \\
autogestionário & au*to*ges*ti*o*ná*ri.o \xmark & au*to*ges*ti*o*ná*rio \cmark \\
autografia & au*to*gra*fi*a \cmark & au*to*gra*fi-a \xmark \\
autoimunidade & au*to-i*mu*ni*da*de \xmark & au*to*i*mu*ni*da*de \cmark \\
autônimo & au-tô*ni*mo \xmark & au*tô*ni*mo \cmark \\
autonomia & au*to*no*mi*a \cmark & au*to*no*mi-a \xmark \\
autonômico & au*to-nô*mi*co \xmark & au*to*nô*mi*co \cmark \\
autônomo & au-tô*no*mo \xmark & au*tô*no*mo \cmark \\
autópsia & au*tóp*si.a \xmark & au*tóp*sia \cmark \\
autoria & au*to*ri*a \cmark & au*to*ri-a \xmark \\
autoritário & au*to*ri*tá*ri.o \xmark & au*to*ri*tá*rio \cmark \\
autotomia & au*to*to*mi*a \cmark & au*to*to*mi-a \xmark \\
auxílio & au*xí*li.o \xmark & au*xí*lio \cmark \\
avaria & a*va*ri*a \cmark & a*va*ri-a \xmark \\
aveia & a*vei*a \cmark & a*vei-a \xmark \\
averroísmo & a*ver*ro-ís*mo \xmark & a*ver*ro*ís*mo \cmark \\
avessia & a*ves*si*a \cmark & a*ves*si-a \xmark \\
avião & a*vi-ão \xmark & a*vi*ão \cmark \\
aviário & a*vi-á*ri.o \xmark & a*vi*á*rio \cmark \\
avio & a*vi*o \cmark & a*vi-o \xmark \\
avô & a-vô \xmark & a*vô \cmark \\
axiologia & a*xi*o*lo*gi*a \cmark & a*xi*o*lo*gi-a \xmark \\
azagaia & a*za*gai*a \cmark & a*za*gai-a \xmark \\
azálea & a*zá*le*a \cmark & a*zá*le-a \xmark \\
azaleia & a*za*lei*a \cmark & a*za*lei-a \xmark \\
azia & a*zi*a \cmark & a*zi-a \xmark \\
azotemia & a*zo*te*mi*a \cmark & a*zo*te*mi-a \xmark \\
azulejaria & a*zu*le*ja*ri*a \cmark & a*zu*le*ja*ri-a \xmark \\
babalaô & ba*ba*la-ô \xmark & ba*ba*la*ô \cmark \\
babilônico & ba*bi-lô*ni*co \xmark & ba*bi*lô*ni*co \cmark \\
babuíno & ba*bu-í*no \xmark & ba*bu*í*no \cmark \\
bacia & ba*ci*a \cmark & ba*ci-a \xmark \\
bacio & ba*ci*o \cmark & ba*ci-o \xmark \\
bactéria & bac*té*ri.a \xmark & bac*té*ria \cmark \\
bacteriófago & bac*te*ri-ó*fa*go \xmark & bac*te*ri*ó*fa*go \cmark \\
bacteriologia & bac*te*ri*o*lo*gi*a \cmark & bac*te*ri*o*lo*gi-a \xmark \\
bacterioscopia & bac*te*ri*os*co*pi*a \cmark & bac*te*ri*os*co*pi-a \xmark \\
bafio & ba*fi*o \cmark & ba*fi-o \xmark \\
bafômetro & ba-fô*me*tro \xmark & ba*fô*me*tro \cmark \\
bahamense & ba-ha*men*se \xmark & ba-ha*men*se \xmark \\
baia & bai*a \cmark & bai-a \xmark \\
baía & ba-í-a \xmark & ba*í-a \xmark \\
baião & bai-ão \xmark & bai*ão \cmark \\
bailio & bai*li*o \cmark & bai*li-o \xmark \\
bainha & ba-i*nha \xmark & ba*i*nha \cmark \\
baio & bai*o \cmark & bai-o \xmark \\
baiuca & bai-u*ca \xmark & bai*u*ca \cmark \\
baixaria & bai*xa*ri*a \cmark & bai*xa*ri-a \xmark \\
baixio & bai*xi*o \cmark & bai*xi-o \xmark \\
balaio & ba*lai*o \cmark & ba*lai-o \xmark \\
balaustrada & ba*la-us*tra*da \xmark & ba*la-us*tra*da \xmark \\
balaústre & ba*la-ús*tre \xmark & ba*la*ús*tre \cmark \\
balboa & bal*bo*a \cmark & bal*bo-a \xmark \\
balbuciência & bal*bu*ci*ên*ci*a \cmark & bal*bu*ci*ên*ci-a \xmark \\
balbucio & bal*bu*ci*o \cmark & bal*bu*ci-o \xmark \\
balbúrdia & bal*búr*di.a \xmark & bal*búr*dia \cmark \\
baldio & bal*di*o \cmark & bal*di-o \xmark \\
baleárico & ba*le-á*ri*co \xmark & ba*le*á*ri*co \cmark \\
baleia & ba*lei*a \cmark & ba*lei-a \xmark \\
baleio & ba*lei*o \cmark & ba*lei-o \xmark \\
balio & ba*li*o \cmark & ba*li-o \xmark \\
balneário & bal*ne-á*ri*o \xmark & bal*ne*á*ri-o \xmark \\
bálteo & bál*te*o \cmark & bál*te-o \xmark \\
bamboleio & bam*bo*lei*o \cmark & bam*bo*lei-o \xmark \\
bancaria & ban*ca*ri*a \cmark & ban*ca*ri-a \xmark \\
bancário & ban*cá*ri.o \xmark & ban*cá*rio \cmark \\
bangalô & ban*ga-lô \xmark & ban*ga*lô \cmark \\
banjoísta & ban*jo-ís*ta \xmark & ban*jo*ís*ta \cmark \\
barataria & ba*ra*ta*ri*a \cmark & ba*ra*ta*ri-a \xmark \\
barateio & ba*ra*tei*o \cmark & ba*ra*tei-o \xmark \\
baraúna & ba*ra-ú*na \xmark & ba*ra*ú*na \cmark \\
barbaria & bar*ba*ri*a \cmark & bar*ba*ri-a \xmark \\
barbárie & bar*bá*ri.e \xmark & bar*bá*rie \cmark \\
barbearia & bar*be*a*ri*a \cmark & bar*be*a*ri-a \xmark \\
bário & bá*ri.o \xmark & bá*rio \cmark \\
barômetro & ba-rô*me*tro \xmark & ba*rô*me*tro \cmark \\
baronia & ba*ro*ni*a \cmark & ba*ro*ni-a \xmark \\
bastião & bas*ti-ão \xmark & bas*ti*ão \cmark \\
bateia & ba*tei*a \cmark & ba*tei-a \xmark \\
bateria & ba*te*ri*a \cmark & ba*te*ri-a \xmark \\
batimetria & ba*ti*me*tri*a \cmark & ba*ti*me*tri-a \xmark \\
batistério & ba*tis*té*ri.o \xmark & ba*tis*té*rio \cmark \\
batráquio & ba*trá*qui.o \xmark & ba*trá*quio \cmark \\
batuíra & ba*tu-í*ra \xmark & ba*tu*í*ra \cmark \\
baú & ba-ú \xmark & ba*ú \cmark \\
bazófia & ba*zó*fi.a \xmark & ba*zó*fia \cmark \\
beduíno & be*du-í*no \xmark & be*du*í*no \cmark \\
behaviorismo & be-ha*vi*o*ris*mo \xmark & be-ha*vi*o*ris*mo \xmark \\
beligerância & be*li*ge*rân*ci.a \xmark & be*li*ge*rân*cia \cmark \\
belisário & be*li*sá*ri.o \xmark & be*li*sá*rio \cmark \\
beneficência & be*ne*fi*cên*ci.a \xmark & be*ne*fi*cên*cia \cmark \\
beneficiário & be*ne*fi*ci-á*ri*o \xmark & be*ne*fi*ci*á*ri-o \xmark \\
benefício & be*ne*fí*ci.o \xmark & be*ne*fí*cio \cmark \\
benemerência & be*ne*me*rên*ci.a \xmark & be*ne*me*rên*cia \cmark \\
benevolência & be*ne*vo*lên*ci.a \xmark & be*ne*vo*lên*cia \cmark \\
benfeitoria & ben*fei*to*ri*a \cmark & ben*fei*to*ri-a \xmark \\
benjoim & ben*jo-im \xmark & ben*jo*im \cmark \\
benzocaína & ben*zo*ca-í*na \xmark & ben*zo*ca*í*na \cmark \\
beócio & be-ó*ci.o \xmark & be*ó*cio \cmark \\
berçário & ber*çá*ri.o \xmark & ber*çá*rio \cmark \\
berílio & be*rí*li.o \xmark & be*rí*lio \cmark \\
berquélio & ber-qué*li.o \xmark & ber*qué*lio \cmark \\
béstia & bés*ti*a \cmark & bés*ti-a \xmark \\
bestiário & bes*ti-á*ri.o \xmark & bes*ti*á*rio \cmark \\
bia & bi*a \cmark & bi-a \xmark \\
bibelô & bi*be-lô \xmark & bi*be*lô \cmark \\
bíblia & bí*bli.a \xmark & bí*blia \cmark \\
bibliofilia & bi*bli*o*fi*li*a \cmark & bi*bli*o*fi*li-a \xmark \\
bibliófilo & bi*bli-ó*fi*lo \xmark & bi*bli*ó*fi*lo \cmark \\
bibliofobia & bi*bli*o*fo*bi*a \cmark & bi*bli*o*fo*bi-a \xmark \\
bibliografia & bi*bli*o*gra*fi*a \cmark & bi*bli*o*gra*fi-a \xmark \\
bibliógrafo & bi*bli-ó*gra*fo \xmark & bi*bli*ó*gra*fo \cmark \\
bibliologia & bi*bli*o*lo*gi*a \cmark & bi*bli*o*lo*gi-a \xmark \\
bibliotecário & bi*bli*o*te*cá*ri.o \xmark & bi*bli*o*te*cá*rio \cmark \\
biblioteconomia & bi*bli*o*te*co*no*mi*a \cmark & bi*bli*o*te*co*no*mi-a \xmark \\
bicampeão & bi*cam*pe-ão \xmark & bi*cam*pe*ão \cmark \\
bicentenário & bi*cen*te*ná*ri.o \xmark & bi*cen*te*ná*rio \cmark \\
bicicletário & bi*ci*cle*tá*ri.o \xmark & bi*ci*cle*tá*rio \cmark \\
bicromia & bi*cro*mi*a \cmark & bi*cro*mi-a \xmark \\
bicuíba & bi*cu-í*ba \xmark & bi*cu*í*ba \cmark \\
biênio & bi*ê*ni.o \xmark & bi*ê*nio \cmark \\
bigamia & bi*ga*mi*a \cmark & bi*ga*mi-a \xmark \\
bijuteria & bi*ju*te*ri*a \cmark & bi*ju*te*ri-a \xmark \\
bilbaíno & bil*ba-í*no \xmark & bil*ba*í*no \cmark \\
bilheteria & bi*lhe*te*ri*a \cmark & bi*lhe*te*ri-a \xmark \\
bilião & bi*li-ão \xmark & bi*li*ão \cmark \\
bilionário & bi*li*o*ná*ri.o \xmark & bi*li*o*ná*rio \cmark \\
binário & bi*ná*ri.o \xmark & bi*ná*rio \cmark \\
binômio & bi-nô*mi.o \xmark & bi*nô*mio \cmark \\
biociência & bi*o*ci*ên*ci.a \xmark & bi*o*ci*ên*cia \cmark \\
bioenergia & bi*o*e*ner*gi*a \cmark & bi*o*e*ner*gi-a \xmark \\
bioética & bi*o-é*ti*ca \xmark & bi*o*é*ti*ca \cmark \\
biofilia & bi*o*fi*li*a \cmark & bi*o*fi*li-a \xmark \\
biófilo & bi-ó*fi*lo \xmark & bi*ó*fi*lo \cmark \\
biogeografia & bi*o*ge*o*gra*fi*a \cmark & bi*o*ge*o*gra*fi-a \xmark \\
biografia & bi*o*gra*fi*a \cmark & bi*o*gra*fi-a \xmark \\
biógrafo & bi-ó*gra*fo \xmark & bi*ó*gra*fo \cmark \\
biologia & bi*o*lo*gi*a \cmark & bi*o*lo*gi-a \xmark \\
biólogo & bi-ó*lo*go \xmark & bi*ó*lo*go \cmark \\
bioluminescência & bi*o*lu*mi*nes*cên*ci.a \xmark & bi*o*lu*mi*nes*cên*cia \cmark \\
bioquímica & bi*o-quí*mi*ca \xmark & bi*o*quí*mi*ca \cmark \\
bioquímico & bi*o-quí*mi*co \xmark & bi*o*quí*mi*co \cmark \\
biotecnologia & bi*o*tec*no*lo*gi*a \cmark & bi*o*tec*no*lo*gi-a \xmark \\
biotério & bi*o*té*ri.o \xmark & bi*o*té*rio \cmark \\
biótico & bi-ó*ti*co \xmark & bi*ó*ti*co \cmark \\
biótipo & bi-ó*ti*po \xmark & bi*ó*ti*po \cmark \\
biótopo & bi-ó*to*po \xmark & bi*ó*to*po \cmark \\
bipartidário & bi*par*ti*dá*ri.o \xmark & bi*par*ti*dá*rio \cmark \\
biquíni & bi-quí*ni \xmark & bi*quí*ni \cmark \\
birô & bi-rô \xmark & bi*rô \cmark \\
bisavô & bi*sa-vô \xmark & bi*sa*vô \cmark \\
biscaia & bis*cai*a \cmark & bis*cai-a \xmark \\
biscainho & bis*ca-i*nho \xmark & bis*ca*i*nho \cmark \\
bissemanário & bis*se*ma*ná*ri.o \xmark & bis*se*ma*ná*rio \cmark \\
biunívoco & bi-u*ní*vo*co \xmark & bi*u*ní*vo*co \cmark \\
bizâncio & bi*zân*ci.o \xmark & bi*zân*cio \cmark \\
bizarria & bi*zar*ri*a \cmark & bi*zar*ri-a \xmark \\
blasfêmia & blas*fê*mi.a \xmark & blas*fê*mia \cmark \\
blastômero & blas-tô*me*ro \xmark & blas*tô*me*ro \cmark \\
bloqueio & blo*quei*o \cmark & blo*quei-o \xmark \\
boa & bo*a \cmark & bo-a \xmark \\
boá & bo-á \xmark & bo*á \cmark \\
boataria & bo*a*ta*ri*a \cmark & bo*a*ta*ri-a \xmark \\
bocaiuva & bo*cai-u*va \xmark & bo*cai*u*va \cmark \\
bócio & bó*ci.o \xmark & bó*cio \cmark \\
boé & bo-é \xmark & bo*é \cmark \\
boemia & bo*e*mi*a \cmark & bo*e*mi-a \xmark \\
bôer & bô-er \xmark & bô-er \xmark \\
boia & boi*a \cmark & boi-a \xmark \\
boião & boi-ão \xmark & boi*ão \cmark \\
boiuna & boi-u*na \xmark & boi*u*na \cmark \\
boleia & bo*lei*a \cmark & bo*lei-a \xmark \\
bombardeio & bom*bar*dei*o \cmark & bom*bar*dei-o \xmark \\
bongô & bon-gô \xmark & bon*gô \cmark \\
bonomia & bo*no*mi*a \cmark & bo*no*mi-a \xmark \\
borderô & bor*de-rô \xmark & bor*de*rô \cmark \\
bordô & bor-dô \xmark & bor*dô \cmark \\
borracharia & bor*ra*cha*ri*a \cmark & bor*ra*cha*ri-a \xmark \\
bosníaco & bos*ní-a*co \xmark & bos*ní*a*co \cmark \\
bósnio & bós*ni.o \xmark & bós*nio \cmark \\
boticário & bo*ti*cá*ri.o \xmark & bo*ti*cá*rio \cmark \\
bovídeo & bo*ví*de.o \xmark & bo*ví*deo \cmark \\
bradicardia & bra*di*car*di*a \cmark & bra*di*car*di-a \xmark \\
brânquia & brân*qui.a \xmark & brân*quia \cmark \\
braquicefalia & bra*qui*ce*fa*li*a \cmark & bra*qui*ce*fa*li-a \xmark \\
braquidactilia & bra*qui*dac*ti*li*a \cmark & bra*qui*dac*ti*li-a \xmark \\
braquiópode & bra*qui-ó*po*de \xmark & bra*qui*ó*po*de \cmark \\
brasiguaio & bra*si*guai*o \cmark & bra*si*guai-o \xmark \\
brasílio & bra*sí*li.o \xmark & bra*sí*lio \cmark \\
braúna & bra-ú*na \xmark & bra*ú*na \cmark \\
bravio & bra*vi*o \cmark & bra*vi-o \xmark \\
brechtiano & bre.ch*ti*a*no \xmark & bre.ch*ti*a*no \xmark \\
breviário & bre*vi-á*ri.o \xmark & bre*vi*á*rio \cmark \\
brio & bri*o \cmark & bri-o \xmark \\
broa & bro*a \cmark & bro-a \xmark \\
bromatologia & bro*ma*to*lo*gi*a \cmark & bro*ma*to*lo*gi-a \xmark \\
bromélia & bro*mé*li.a \xmark & bro*mé*lia \cmark \\
broncoscopia & bron*cos*co*pi*a \cmark & bron*cos*co*pi-a \xmark \\
brônquio & brôn*qui.o \xmark & brôn*quio \cmark \\
bronquíolo & bron-quí-o*lo \xmark & bron*quí*o*lo \cmark \\
bruxaria & bru*xa*ri*a \cmark & bru*xa*ri-a \xmark \\
bucelário & bu*ce*lá*ri.o \xmark & bu*ce*lá*rio \cmark \\
bué & bu-é \xmark & bu*é \cmark \\
bugia & bu*gi*a \cmark & bu*gi-a \xmark \\
bugio & bu*gi*o \cmark & bu*gi-o \xmark \\
buir & bu-ir \xmark & bu*ir \cmark \\
bulício & bu*lí*ci.o \xmark & bu*lí*cio \cmark \\
bulimia & bu*li*mi*a \cmark & bu*li*mi-a \xmark \\
buquê & bu-quê \xmark & bu*quê \cmark \\
burguesia & bur*gue*si*a \cmark & bur*gue*si-a \xmark \\
burocracia & bu*ro*cra*ci*a \cmark & bu*ro*cra*ci-a \xmark \\
búteo & bú*te*o \cmark & bú*te-o \xmark \\
butiá & bu*ti-á \xmark & bu*ti*á \cmark \\
bútio & bú*ti.o \xmark & bú*tio \cmark \\
búzio & bú*zi.o \xmark & bú*zio \cmark \\
cabeceio & ca*be*cei*o \cmark & ca*be*cei-o \xmark \\
cabrião & ca*bri-ão \xmark & ca*bri*ão \cmark \\
cabriúva & ca*bri-ú*va \xmark & ca*bri*ú*va \cmark \\
cacaio & ca*cai*o \cmark & ca*cai-o \xmark \\
cacatua & ca*ca*tu*a \cmark & ca*ca*tu-a \xmark \\
cacauicultura & ca*cau-i*cul*tu*ra \xmark & ca*cau*i*cul*tu*ra \cmark \\
cacofonia & ca*co*fo*ni*a \cmark & ca*co*fo*ni-a \xmark \\
cactácea & cac*tá*ce.a \xmark & cac*tá*cea \cmark \\
cadeia & ca*dei*a \cmark & ca*dei-a \xmark \\
cadência & ca*dên*ci.a \xmark & ca*dên*cia \cmark \\
cádmio & cád*mi.o \xmark & cád*mio \cmark \\
cafarnaum & ca*far*na-um \xmark & ca*far*na-um \xmark \\
cafeína & ca*fe-í*na \xmark & ca*fe*í*na \cmark \\
cafeteria & ca*fe*te*ri*a \cmark & ca*fe*te*ri-a \xmark \\
cafraria & ca*fra*ri*a \cmark & ca*fra*ri-a \xmark \\
caíco & ca-í*co \xmark & ca*í*co \cmark \\
caída & ca-í*da \xmark & ca*í*da \cmark \\
caído & ca-í*do \xmark & ca*í*do \cmark \\
caim & ca-im \xmark & ca-im \xmark \\
caimento & ca-i*men*to \xmark & ca-i*men*to \xmark \\
caingangue & ca-in*gan*gue \xmark & ca-in*gan*gue \xmark \\
caíque & ca-í*que \xmark & ca*í*que \cmark \\
cair & ca-ir \xmark & ca*ir \cmark \\
caixilharia & cai*xi*lha*ri*a \cmark & cai*xi*lha*ri-a \xmark \\
caixotaria & cai*xo*ta*ri*a \cmark & cai*xo*ta*ri-a \xmark \\
cajuína & ca*ju-í*na \xmark & ca*ju*í*na \cmark \\
calafrio & ca*la*fri*o \cmark & ca*la*fri-o \xmark \\
calcâneo & cal*câ*ne.o \xmark & cal*câ*neo \cmark \\
calcário & cal*cá*ri*o \cmark & cal*cá*ri-o \xmark \\
calcedônia & cal*ce-dô*ni.a \xmark & cal*ce*dô*nia \cmark \\
calcemia & cal*ce*mi*a \cmark & cal*ce*mi-a \xmark \\
cálcio & cál*ci.o \xmark & cál*cio \cmark \\
calcografia & cal*co*gra*fi*a \cmark & cal*co*gra*fi-a \xmark \\
caldário & cal*dá*ri.o \xmark & cal*dá*rio \cmark \\
caldeiraria & cal*dei*ra*ri*a \cmark & cal*dei*ra*ri-a \xmark \\
caleidoscópio & ca*lei*dos*có*pi.o \xmark & ca*lei*dos*có*pio \cmark \\
calendário & ca*len*dá*ri.o \xmark & ca*len*dá*rio \cmark \\
califasia & ca*li*fa*si*a \cmark & ca*li*fa*si-a \xmark \\
califórnia & ca*li*fór*ni*a \cmark & ca*li*fór*ni-a \xmark \\
califórnio & ca*li*fór*ni.o \xmark & ca*li*fór*nio \cmark \\
caligrafia & ca*li*gra*fi*a \cmark & ca*li*gra*fi-a \xmark \\
calistenia & ca*lis*te*ni*a \cmark & ca*lis*te*ni-a \xmark \\
calmaria & cal*ma*ri*a \cmark & cal*ma*ri-a \xmark \\
caloria & ca*lo*ri*a \cmark & ca*lo*ri-a \xmark \\
calorimetria & ca*lo*ri*me*tri*a \cmark & ca*lo*ri*me*tri-a \xmark \\
calúnia & ca*lú*ni.a \xmark & ca*lú*nia \cmark \\
calvário & cal*vá*ri.o \xmark & cal*vá*rio \cmark \\
calvície & cal*ví*ci.e \xmark & cal*ví*cie \cmark \\
camaleão & ca*ma*le-ão \xmark & ca*ma*le*ão \cmark \\
camarário & ca*ma*rá*ri.o \xmark & ca*ma*rá*rio \cmark \\
cambaio & cam*bai*o \cmark & cam*bai-o \xmark \\
cambaleio & cam*ba*lei*o \cmark & cam*ba*lei-o \xmark \\
câmbio & câm*bi.o \xmark & câm*bio \cmark \\
camboa & cam*bo*a \cmark & cam*bo-a \xmark \\
cambraia & cam*brai*a \cmark & cam*brai-a \xmark \\
cambuí & cam*bu-í \xmark & cam*bu*í \cmark \\
camélia & ca*mé*li*a \cmark & ca*mé*li-a \xmark \\
camelídeo & ca*me*lí*de.o \xmark & ca*me*lí*deo \cmark \\
camelô & ca*me-lô \xmark & ca*me*lô \cmark \\
camião & ca*mi-ão \xmark & ca*mi*ão \cmark \\
camisaria & ca*mi*sa*ri*a \cmark & ca*mi*sa*ri-a \xmark \\
campainha & cam*pa-i*nha \xmark & cam*pa*i*nha \cmark \\
campanário & cam*pa*ná*ri.o \xmark & cam*pa*ná*rio \cmark \\
campeão & cam*pe-ão \xmark & cam*pe*ão \cmark \\
canaria & ca*na*ri*a \cmark & ca*na*ri-a \xmark \\
canária & ca*ná*ri.a \xmark & ca*ná*ria \cmark \\
canário & ca*ná*ri.o \xmark & ca*ná*rio \cmark \\
cancelário & can*ce*lá*ri.o \xmark & can*ce*lá*rio \cmark \\
cancerologia & can*ce*ro*lo*gi*a \cmark & can*ce*ro*lo*gi-a \xmark \\
candeia & can*dei*a \cmark & can*dei-a \xmark \\
candelária & can*de*lá*ri*a \cmark & can*de*lá*ri-a \xmark \\
candidíase & can*di*dí-a*se \xmark & can*di*dí*a*se \cmark \\
canhoneio & ca*nho*nei*o \cmark & ca*nho*nei-o \xmark \\
canídeo & ca*ní*de.o \xmark & ca*ní*deo \cmark \\
canoa & ca*no*a \cmark & ca*no-a \xmark \\
canoísta & ca*no-ís*ta \xmark & ca*no*ís*ta \cmark \\
canônico & ca-nô*ni*co \xmark & ca*nô*ni*co \cmark \\
cantaria & can*ta*ri*a \cmark & can*ta*ri-a \xmark \\
cantoria & can*to*ri*a \cmark & can*to*ri-a \xmark \\
caótico & ca-ó*ti*co \xmark & ca*ó*ti*co \cmark \\
capacitância & ca*pa*ci*tân*ci.a \xmark & ca*pa*ci*tân*cia \cmark \\
capadócio & ca*pa*dó*ci.o \xmark & ca*pa*dó*cio \cmark \\
capatazia & ca*pa*ta*zi*a \cmark & ca*pa*ta*zi-a \xmark \\
capeia & ca*pei*a \cmark & ca*pei-a \xmark \\
capelania & ca*pe*la*ni*a \cmark & ca*pe*la*ni-a \xmark \\
capicua & ca*pi*cu*a \cmark & ca*pi*cu-a \xmark \\
capitania & ca*pi*ta*ni*a \cmark & ca*pi*ta*ni-a \xmark \\
capitânia & ca*pi*tâ*ni.a \xmark & ca*pi*tâ*nia \cmark \\
capnomancia & cap*no*man*ci*a \cmark & cap*no*man*ci-a \xmark \\
capô & ca-pô \xmark & ca*pô \cmark \\
capricórnio & ca*pri*cór*ni.o \xmark & ca*pri*cór*nio \cmark \\
caquexia & ca*que*xi*a \cmark & ca*que*xi-a \xmark \\
carabídeo & ca*ra*bí*de.o \xmark & ca*ra*bí*deo \cmark \\
caracterologia & ca*rac*te*ro*lo*gi*a \cmark & ca*rac*te*ro*lo*gi-a \xmark \\
caraíba & ca*ra-í*ba \xmark & ca*ra*í*ba \cmark \\
carbonária & car*bo*ná*ri.a \xmark & car*bo*ná*ria \cmark \\
carbonário & car*bo*ná*ri.o \xmark & car*bo*ná*rio \cmark \\
carbônico & car-bô*ni*co \xmark & car*bô*ni*co \cmark \\
carcerário & car*ce*rá*ri.o \xmark & car*ce*rá*rio \cmark \\
cardápio & car*dá*pi.o \xmark & car*dá*pio \cmark \\
cárdia & cár*di.a \xmark & cár*dia \cmark \\
cardíaco & car*dí-a*co \xmark & car*dí*a*co \cmark \\
cardinalício & car*di*na*lí*ci.o \xmark & car*di*na*lí*cio \cmark \\
cardiologia & car*di*o*lo*gi*a \cmark & car*di*o*lo*gi-a \xmark \\
cardiomiopatia & car*di*o*mi*o*pa*ti*a \cmark & car*di*o*mi*o*pa*ti-a \xmark \\
cardiopatia & car*di*o*pa*ti*a \cmark & car*di*o*pa*ti-a \xmark \\
cardiorrespiratório & car*di*or*res*pi*ra*tó*ri.o \xmark & car*di*or*res*pi*ra*tó*rio \cmark \\
carélio & ca*ré*li.o \xmark & ca*ré*lio \cmark \\
carência & ca*rên*ci.a \xmark & ca*rên*cia \cmark \\
carestia & ca*res*ti*a \cmark & ca*res*ti-a \xmark \\
carícia & ca*rí*ci.a \xmark & ca*rí*cia \cmark \\
cárie & cá*ri.e \xmark & cá*rie \cmark \\
cariogamia & ca*ri*o*ga*mi*a \cmark & ca*ri*o*ga*mi-a \xmark \\
cariótipo & ca*ri-ó*ti*po \xmark & ca*ri*ó*ti*po \cmark \\
carnaíba & car*na-í*ba \xmark & car*na*í*ba \cmark \\
carnaúba & car*na-ú*ba \xmark & car*na*ú*ba \cmark \\
carnaubal & car*na-u*bal \xmark & car*na-u*bal \xmark \\
caroá & ca*ro-á \xmark & ca*ro*á \cmark \\
carolíngio & ca*ro*lín*gi.o \xmark & ca*ro*lín*gio \cmark \\
carotídeo & ca*ro*tí*de.o \xmark & ca*ro*tí*deo \cmark \\
carpintaria & car*pin*ta*ri*a \cmark & car*pin*ta*ri-a \xmark \\
carroçaria & car*ro*ça*ri*a \cmark & car*ro*ça*ri-a \xmark \\
carroceria & car*ro*ce*ri*a \cmark & car*ro*ce*ri-a \xmark \\
cartapácio & car*ta*pá*ci.o \xmark & car*ta*pá*cio \cmark \\
cartografia & car*to*gra*fi*a \cmark & car*to*gra*fi-a \xmark \\
cartomancia & car*to*man*ci*a \cmark & car*to*man*ci-a \xmark \\
cartorário & car*to*rá*ri.o \xmark & car*to*rá*rio \cmark \\
cartório & car*tó*ri.o \xmark & car*tó*rio \cmark \\
cartulário & car*tu*lá*ri.o \xmark & car*tu*lá*rio \cmark \\
carvoaria & car*vo*a*ri*a \cmark & car*vo*a*ri-a \xmark \\
casaria & ca*sa*ri*a \cmark & ca*sa*ri-a \xmark \\
casario & ca*sa*ri*o \cmark & ca*sa*ri-o \xmark \\
caseína & ca*se-í*na \xmark & ca*se*í*na \cmark \\
casório & ca*só*ri.o \xmark & ca*só*rio \cmark \\
cássia & cás*si.a \xmark & cás*sia \cmark \\
castelania & cas*te*la*ni*a \cmark & cas*te*la*ni-a \xmark \\
casuísmo & ca*su-ís*mo \xmark & ca*su*ís*mo \cmark \\
casuísta & ca*su-ís*ta \xmark & ca*su*ís*ta \cmark \\
casuística & ca*su-ís*ti*ca \xmark & ca*su*ís*ti*ca \cmark \\
casuístico & ca*su-ís*ti*co \xmark & ca*su*ís*ti*co \cmark \\
catalepsia & ca*ta*lep*si*a \cmark & ca*ta*lep*si-a \xmark \\
cataplexia & ca*ta*ple*xi*a \cmark & ca*ta*ple*xi-a \xmark \\
catatonia & ca*ta*to*ni*a \cmark & ca*ta*to*ni-a \xmark \\
catatônico & ca*ta-tô*ni*co \xmark & ca*ta*tô*ni*co \cmark \\
catenária & ca*te*ná*ri.a \xmark & ca*te*ná*ria \cmark \\
catequético & ca*te-qué*ti*co \xmark & ca*te*qué*ti*co \cmark \\
catexia & ca*te*xi*a \cmark & ca*te*xi-a \xmark \\
catião & ca*ti-ão \xmark & ca*ti*ão \cmark \\
catilinária & ca*ti*li*ná*ri.a \xmark & ca*ti*li*ná*ria \cmark \\
catoptromancia & ca*top*tro*man*ci*a \cmark & ca*top*tro*man*ci-a \xmark \\
catraia & ca*trai*a \cmark & ca*trai-a \xmark \\
cauã & cau-ã \xmark & cau*ã \cmark \\
cauim & cau-im \xmark & cau*im \cmark \\
caúna & ca-ú*na \xmark & ca*ú*na \cmark \\
cavalaria & ca*va*la*ri*a \cmark & ca*va*la*ri-a \xmark \\
cávea & cá*ve*a \cmark & cá*ve-a \xmark \\
ceceio & ce*cei*o \cmark & ce*cei-o \xmark \\
cedência & ce*dên*ci.a \xmark & ce*dên*cia \cmark \\
cefaleia & ce*fa*lei*a \cmark & ce*fa*lei-a \xmark \\
ceia & cei*a \cmark & cei-a \xmark \\
celíaco & ce*lí-a*co \xmark & ce*lí*a*co \cmark \\
celibatário & ce*li*ba*tá*ri.o \xmark & ce*li*ba*tá*rio \cmark \\
cemitério & ce*mi*té*ri.o \xmark & ce*mi*té*rio \cmark \\
cenário & ce*ná*ri.o \xmark & ce*ná*rio \cmark \\
cenóbio & ce*nó*bi.o \xmark & ce*nó*bio \cmark \\
cenografia & ce*no*gra*fi*a \cmark & ce*no*gra*fi-a \xmark \\
cenotáfio & ce*no*tá*fi.o \xmark & ce*no*tá*fio \cmark \\
censitário & cen*si*tá*ri.o \xmark & cen*si*tá*rio \cmark \\
censório & cen*só*ri.o \xmark & cen*só*rio \cmark \\
centáurea & cen*táu*re.a \xmark & cen*táu*rea \cmark \\
centeio & cen*tei*o \cmark & cen*tei-o \xmark \\
centenário & cen*te*ná*ri.o \xmark & cen*te*ná*rio \cmark \\
centopeia & cen*to*pei*a \cmark & cen*to*pei-a \xmark \\
centúria & cen*tú*ri.a \xmark & cen*tú*ria \cmark \\
centurião & cen*tu*ri-ão \xmark & cen*tu*ri*ão \cmark \\
cerambicídeo & ce*ram*bi*cí*de.o \xmark & ce*ram*bi*cí*deo \cmark \\
ceratoplastia & ce*ra*to*plas*ti*a \cmark & ce*ra*to*plas*ti-a \xmark \\
cercania & cer*ca*ni*a \cmark & cer*ca*ni-a \xmark \\
cércea & cér*ce.a \xmark & cér*cea \cmark \\
cerefólio & ce*re*fó*li.o \xmark & ce*re*fó*lio \cmark \\
cerimônia & ce*ri-mô*ni.a \xmark & ce*ri*mô*nia \cmark \\
cério & cé*ri.o \xmark & cé*rio \cmark \\
cerúleo & ce*rú*le.o \xmark & ce*rú*leo \cmark \\
cervejaria & cer*ve*ja*ri*a \cmark & cer*ve*ja*ri-a \xmark \\
cervídeo & cer*ví*de.o \xmark & cer*ví*deo \cmark \\
cesárea & ce*sá*re.a \xmark & ce*sá*rea \cmark \\
cesáreo & ce*sá*re.o \xmark & ce*sá*reo \cmark \\
césio & cé*si.o \xmark & cé*sio \cmark \\
cessionário & ces*si*o*ná*ri.o \xmark & ces*si*o*ná*rio \cmark \\
cestaria & ces*ta*ri*a \cmark & ces*ta*ri-a \xmark \\
cetáceo & ce*tá*ce.o \xmark & ce*tá*ceo \cmark \\
champô & cham-pô \xmark & cham*pô \cmark \\
chancelaria & chan*ce*la*ri*a \cmark & chan*ce*la*ri-a \xmark \\
chapelaria & cha*pe*la*ri*a \cmark & cha*pe*la*ri-a \xmark \\
charcutaria & char*cu*ta*ri*a \cmark & char*cu*ta*ri-a \xmark \\
charrua & char*ru*a \cmark & char*ru-a \xmark \\
charutaria & cha*ru*ta*ri*a \cmark & cha*ru*ta*ri-a \xmark \\
chatô & cha-tô \xmark & cha*tô \cmark \\
chefia & che*fi*a \cmark & che*fi-a \xmark \\
cheia & chei*a \cmark & chei-a \xmark \\
cheio & chei*o \cmark & chei-o \xmark \\
chicória & chi*có*ri.a \xmark & chi*có*ria \cmark \\
chio & chi*o \cmark & chi-o \xmark \\
choperia & cho*pe*ri*a \cmark & cho*pe*ri-a \xmark \\
chororô & cho*ro-rô \xmark & cho*ro*rô \cmark \\
chuleio & chu*lei*o \cmark & chu*lei-o \xmark \\
churrascaria & chur*ras*ca*ri*a \cmark & chur*ras*ca*ri-a \xmark \\
ciática & ci-á*ti*ca \xmark & ci*á*ti*ca \cmark \\
ciático & ci-á*ti*co \xmark & ci*á*ti*co \cmark \\
ciberespaço & ci*be.r-es*pa*ço \xmark & ci*be.r-es*pa*ço \xmark \\
cibório & ci*bó*ri.o \xmark & ci*bó*rio \cmark \\
cicio & ci*ci*o \cmark & ci*ci-o \xmark \\
ciclídeo & ci*clí*de.o \xmark & ci*clí*deo \cmark \\
ciclotimia & ci*clo*ti*mi*a \cmark & ci*clo*ti*mi-a \xmark \\
ciclovia & ci*clo*vi*a \cmark & ci*clo*vi-a \xmark \\
cicloviário & ci*clo*vi-á*ri.o \xmark & ci*clo*vi*á*rio \cmark \\
cidadania & ci*da*da*ni*a \cmark & ci*da*da*ni-a \xmark \\
ciência & ci*ên*ci.a \xmark & ci*ên*cia \cmark \\
cientologia & ci*en*to*lo*gi*a \cmark & ci*en*to*lo*gi-a \xmark \\
cilício & ci*lí*ci.o \xmark & ci*lí*cio \cmark \\
cílio & cí*li.o \xmark & cí*lio \cmark \\
cimério & ci*mé*ri.o \xmark & ci*mé*rio \cmark \\
cinábrio & ci*ná*bri.o \xmark & ci*ná*brio \cmark \\
cinefilia & ci*ne*fi*li*a \cmark & ci*ne*fi*li-a \xmark \\
cinematografia & ci*ne*ma*to*gra*fi*a \cmark & ci*ne*ma*to*gra*fi-a \xmark \\
cinérea & ci*né*re.a \xmark & ci*né*rea \cmark \\
cinescópio & ci*nes*có*pi.o \xmark & ci*nes*có*pio \cmark \\
cinesiologia & ci*ne*si*o*lo*gi*a \cmark & ci*ne*si*o*lo*gi-a \xmark \\
cinesioterapia & ci*ne*si*o*te*ra*pi*a \cmark & ci*ne*si*o*te*ra*pi-a \xmark \\
cinestesia & ci*nes*te*si*a \cmark & ci*nes*te*si-a \xmark \\
cinofilia & ci*no*fi*li*a \cmark & ci*no*fi*li-a \xmark \\
cinologia & ci*no*lo*gi*a \cmark & ci*no*lo*gi-a \xmark \\
cinquentenário & cin*quen*te*ná*ri.o \xmark & cin*quen*te*ná*rio \cmark \\
cintilografia & cin*ti*lo*gra*fi*a \cmark & cin*ti*lo*gra*fi-a \xmark \\
cio & ci*o \cmark & ci-o \xmark \\
ciprinídeo & ci*pri*ní*de.o \xmark & ci*pri*ní*deo \cmark \\
circulatório & cir*cu*la*tó*ri.o \xmark & cir*cu*la*tó*rio \cmark \\
circunferência & cir*cun*fe*rên*ci.a \xmark & cir*cun*fe*rên*cia \cmark \\
circunstância & cir*cuns*tân*ci.a \xmark & cir*cuns*tân*cia \cmark \\
círio & cí*ri.o \xmark & cí*rio \cmark \\
cirurgia & ci*rur*gi*a \cmark & ci*rur*gi-a \xmark \\
cirurgião & ci*rur*gi-ão \xmark & ci*rur*gi*ão \cmark \\
cistoscopia & cis*tos*co*pi*a \cmark & cis*tos*co*pi-a \xmark \\
cistoscópio & cis*tos*có*pi.o \xmark & cis*tos*có*pio \cmark \\
citânia & ci*tâ*ni.a \xmark & ci*tâ*nia \cmark \\
citologia & ci*to*lo*gi*a \cmark & ci*to*lo*gi-a \xmark \\
citopatologia & ci*to*pa*to*lo*gi*a \cmark & ci*to*pa*to*lo*gi-a \xmark \\
ciúme & ci-ú*me \xmark & ci*ú*me \cmark \\
cizânia & ci*zâ*ni.a \xmark & ci*zâ*nia \cmark \\
clamídia & cla*mí*di.a \xmark & cla*mí*dia \cmark \\
claraboia & cla*ra*boi*a \cmark & cla*ra*boi-a \xmark \\
clarividência & cla*ri*vi*dên*ci.a \xmark & cla*ri*vi*dên*cia \cmark \\
claustrofobia & claus*tro*fo*bi*a \cmark & claus*tro*fo*bi-a \xmark \\
clavicórdio & cla*vi*cór*di.o \xmark & cla*vi*cór*dio \cmark \\
clemência & cle*mên*ci.a \xmark & cle*mên*cia \cmark \\
cleptocracia & clep*to*cra*ci*a \cmark & clep*to*cra*ci-a \xmark \\
cleptomania & clep*to*ma*ni*a \cmark & clep*to*ma*ni-a \xmark \\
climatério & cli*ma*té*ri.o \xmark & cli*ma*té*rio \cmark \\
climatologia & cli*ma*to*lo*gi*a \cmark & cli*ma*to*lo*gi-a \xmark \\
clípeo & clí*pe.o \xmark & clí*peo \cmark \\
clorofórmio & clo*ro*fór*mi.o \xmark & clo*ro*fór*mio \cmark \\
cnidário & c.ni*dá*ri.o \xmark & c.ni*dá*rio \xmark \\
coa & co*a \cmark & co-a \xmark \\
coágulo & co-á*gu*lo \xmark & co*á*gu*lo \cmark \\
coalescência & co*a*les*cên*ci.a \xmark & co*a*les*cên*cia \cmark \\
coautoria & co*au*to*ri*a \cmark & co*au*to*ri-a \xmark \\
cobaia & co*bai*a \cmark & co*bai-a \xmark \\
cobardia & co*bar*di*a \cmark & co*bar*di-a \xmark \\
cocaína & co*ca-í*na \xmark & co*ca*í*na \cmark \\
cóclea & có*cle.a \xmark & có*clea \cmark \\
cocô & co-cô \xmark & co*cô \cmark \\
côdea & cô*de.a \xmark & cô*dea \cmark \\
codeína & co*de-í*na \xmark & co*de*í*na \cmark \\
coerência & co*e*rên*ci.a \xmark & co*e*rên*cia \cmark \\
coetâneo & co*e*tâ*ne*o \cmark & co*e*tâ*ne-o \xmark \\
coexistência & co*e*xis*tên*ci.a \xmark & co*e*xis*tên*cia \cmark \\
coibir & co-i*bir \xmark & co-i*bir \xmark \\
coimbrão & co-im*brão \xmark & co*im*brão \cmark \\
coincidência & co-in*ci*dên*ci.a \xmark & co*in*ci*dên*cia \cmark \\
coincidente & co-in*ci*den*te \xmark & co*in*ci*den*te \cmark \\
coincidir & co-in*ci*dir \xmark & co*in*ci*dir \cmark \\
coirmão & co-ir*mão \xmark & co-ir*mão \xmark \\
colcheia & col*chei*a \cmark & col*chei-a \xmark \\
colecistectomia & co*le*cis*tec*to*mi*a \cmark & co*le*cis*tec*to*mi-a \xmark \\
colégio & co*lé*gi.o \xmark & co*lé*gio \cmark \\
coleóptero & co*le-óp*te*ro \xmark & co*le*óp*te*ro \cmark \\
coletânea & co*le*tâ*ne.a \xmark & co*le*tâ*nea \cmark \\
coletoria & co*le*to*ri*a \cmark & co*le*to*ri-a \xmark \\
colírio & co*lí*ri.o \xmark & co*lí*rio \cmark \\
colmeia & col*mei*a \cmark & col*mei-a \xmark \\
colódio & co*ló*di.o \xmark & co*ló*dio \cmark \\
colonia & co*lo*ni*a \cmark & co*lo*ni-a \xmark \\
colônia & co-lô*ni.a \xmark & co*lô*nia \cmark \\
colonoscopia & co*lo*nos*co*pi*a \cmark & co*lo*nos*co*pi-a \xmark \\
colóquio & co*ló*qui.o \xmark & co*ló*quio \cmark \\
colorimetria & co*lo*ri*me*tri*a \cmark & co*lo*ri*me*tri-a \xmark \\
colostomia & co*los*to*mi*a \cmark & co*los*to*mi-a \xmark \\
colposcopia & col*pos*co*pi*a \cmark & col*pos*co*pi-a \xmark \\
colposcópio & col*pos*có*pi.o \xmark & col*pos*có*pio \cmark \\
columbário & co*lum*bá*ri.o \xmark & co*lum*bá*rio \cmark \\
colúmbio & co*lúm*bi.o \xmark & co*lúm*bio \cmark \\
columbofilia & co*lum*bo*fi*li*a \cmark & co*lum*bo*fi*li-a \xmark \\
colutório & co*lu*tó*ri.o \xmark & co*lu*tó*rio \cmark \\
comandância & co*man*dân*ci.a \xmark & co*man*dân*cia \cmark \\
combinatório & com*bi*na*tó*ri.o \xmark & com*bi*na*tó*rio \cmark \\
comboio & com*boi*o \cmark & com*boi-o \xmark \\
comedia & co*me*di*a \cmark & co*me*di-a \xmark \\
comédia & co*mé*di.a \xmark & co*mé*dia \cmark \\
comediógrafo & co*me*di-ó*gra*fo \xmark & co*me*di*ó*gra*fo \cmark \\
comedoria & co*me*do*ri*a \cmark & co*me*do*ri-a \xmark \\
comendatário & co*men*da*tá*ri.o \xmark & co*men*da*tá*rio \cmark \\
comentário & co*men*tá*ri.o \xmark & co*men*tá*rio \cmark \\
comerciário & co*mer*ci-á*ri.o \xmark & co*mer*ci*á*rio \cmark \\
comércio & co*mér*ci.o \xmark & co*mér*cio \cmark \\
cometário & co*me*tá*ri.o \xmark & co*me*tá*rio \cmark \\
comício & co*mí*ci.o \xmark & co*mí*cio \cmark \\
comissário & co*mis*sá*ri.o \xmark & co*mis*sá*rio \cmark \\
compadrio & com*pa*dri*o \cmark & com*pa*dri-o \xmark \\
companhia & com*pa*nhi*a \cmark & com*pa*nhi-a \xmark \\
comparência & com*pa*rên*ci.a \xmark & com*pa*rên*cia \cmark \\
comparsaria & com*par*sa*ri*a \cmark & com*par*sa*ri-a \xmark \\
compêndio & com*pên*di.o \xmark & com*pên*dio \cmark \\
compensatório & com*pen*sa*tó*ri.o \xmark & com*pen*sa*tó*rio \cmark \\
competência & com*pe*tên*ci.a \xmark & com*pe*tên*cia \cmark \\
compilatório & com*pi*la*tó*ri.o \xmark & com*pi*la*tó*rio \cmark \\
complacência & com*pla*cên*ci.a \xmark & com*pla*cên*cia \cmark \\
compulsória & com*pul*só*ri.a \xmark & com*pul*só*ria \cmark \\
compulsório & com*pul*só*ri.o \xmark & com*pul*só*rio \cmark \\
comunitário & co*mu*ni*tá*ri.o \xmark & co*mu*ni*tá*rio \cmark \\
concelhio & con*ce*lhi*o \cmark & con*ce*lhi-o \xmark \\
concessionária & con*ces*si*o*ná*ri.a \xmark & con*ces*si*o*ná*ria \cmark \\
concessionário & con*ces*si*o*ná*ri.o \xmark & con*ces*si*o*ná*rio \cmark \\
conciliábulo & con*ci*li-á*bu*lo \xmark & con*ci*li*á*bu*lo \cmark \\
conciliatório & con*ci*li*a*tó*ri.o \xmark & con*ci*li*a*tó*rio \cmark \\
conciliável & con*ci*li-á*vel \xmark & con*ci*li*á*vel \cmark \\
concílio & con*cí*li.o \xmark & con*cí*lio \cmark \\
concluído & con*clu-í*do \xmark & con*clu*í*do \cmark \\
concluinte & con*clu-in*te \xmark & con*clu*in*te \cmark \\
concluir & con*clu-ir \xmark & con*clu*ir \cmark \\
concomitância & con*co*mi*tân*ci.a \xmark & con*co*mi*tân*cia \cmark \\
concordância & con*cor*dân*ci.a \xmark & con*cor*dân*cia \cmark \\
concordatário & con*cor*da*tá*ri.o \xmark & con*cor*da*tá*rio \cmark \\
concórdia & con*cór*di.a \xmark & con*cór*dia \cmark \\
concorrência & con*cor*rên*ci.a \xmark & con*cor*rên*cia \cmark \\
concupiscência & con*cu*pis*cên*ci.a \xmark & con*cu*pis*cên*cia \cmark \\
condenatório & con*de*na*tó*ri.o \xmark & con*de*na*tó*rio \cmark \\
condescendência & con*des*cen*dên*ci.a \xmark & con*des*cen*dên*cia \cmark \\
condoído & con*do-í*do \xmark & con*do*í*do \cmark \\
condolência & con*do*lên*ci.a \xmark & con*do*lên*cia \cmark \\
condomínio & con*do*mí*ni.o \xmark & con*do*mí*nio \cmark \\
condômino & con-dô*mi*no \xmark & con*dô*mi*no \cmark \\
conduíte & con*du-í*te \xmark & con*du*í*te \cmark \\
condutância & con*du*tân*ci.a \xmark & con*du*tân*cia \cmark \\
confeitaria & con*fei*ta*ri*a \cmark & con*fei*ta*ri-a \xmark \\
conferência & con*fe*rên*ci.a \xmark & con*fe*rên*cia \cmark \\
confessionário & con*fes*si*o*ná*ri.o \xmark & con*fes*si*o*ná*rio \cmark \\
confiável & con*fi-á*vel \xmark & con*fi*á*vel \cmark \\
confidência & con*fi*dên*ci.a \xmark & con*fi*dên*cia \cmark \\
confirmatório & con*fir*ma*tó*ri.o \xmark & con*fir*ma*tó*rio \cmark \\
confluência & con*flu-ên*ci.a \xmark & con*flu*ên*cia \cmark \\
confluir & con*flu-ir \xmark & con*flu*ir \cmark \\
confraria & con*fra*ri*a \cmark & con*fra*ri-a \xmark \\
côngio & côn*gi.o \xmark & côn*gio \cmark \\
congruência & con*gru-ên*ci.a \xmark & con*gru*ên*cia \cmark \\
conivência & co*ni*vên*ci.a \xmark & co*ni*vên*cia \cmark \\
conluio & con*lui*o \cmark & con*lui-o \xmark \\
consanguíneo & con*san*guí*ne.o \xmark & con*san*guí*neo \cmark \\
consciência & cons*ci*ên*ci.a \xmark & cons*ci*ên*cia \cmark \\
cônscio & côns*ci*o \cmark & côns*ci-o \xmark \\
consentâneo & con*sen*tâ*ne.o \xmark & con*sen*tâ*neo \cmark \\
consequência & con*se-quên*ci.a \xmark & con*se*quên*cia \cmark \\
conservatório & con*ser*va*tó*ri.o \xmark & con*ser*va*tó*rio \cmark \\
consílio & con*sí*li.o \xmark & con*sí*lio \cmark \\
consistência & con*sis*tên*ci.a \xmark & con*sis*tên*cia \cmark \\
consistório & con*sis*tó*ri*o \cmark & con*sis*tó*ri-o \xmark \\
consócio & con*só*ci.o \xmark & con*só*cio \cmark \\
consonância & con*so*nân*ci*a \cmark & con*so*nân*ci-a \xmark \\
consórcio & con*sór*ci.o \xmark & con*sór*cio \cmark \\
conspícuo & cons*pí*cu.o \xmark & cons*pí*cuo \cmark \\
conspiratório & cons*pi*ra*tó*ri.o \xmark & cons*pi*ra*tó*rio \cmark \\
constância & cons*tân*ci.a \xmark & cons*tân*cia \cmark \\
constituição & cons*ti*tu-i*ção \xmark & cons*ti*tu*i*ção \cmark \\
constituído & cons*ti*tu-í*do \xmark & cons*ti*tu*í*do \cmark \\
constituinte & cons*ti*tu-in*te \xmark & cons*ti*tu*in*te \cmark \\
constituir & cons*ti*tu-ir \xmark & cons*ti*tu*ir \cmark \\
construído & cons*tru-í*do \xmark & cons*tru*í*do \cmark \\
construir & cons*tru-ir \xmark & cons*tru*ir \cmark \\
consuetudinário & con*su*e*tu*di*ná*ri.o \xmark & con*su*e*tu*di*ná*rio \cmark \\
consultoria & con*sul*to*ri*a \cmark & con*sul*to*ri-a \xmark \\
consultório & con*sul*tó*ri.o \xmark & con*sul*tó*rio \cmark \\
contadoria & con*ta*do*ri*a \cmark & con*ta*do*ri-a \xmark \\
contágio & con*tá*gi.o \xmark & con*tá*gio \cmark \\
contaria & con*ta*ri*a \cmark & con*ta*ri-a \xmark \\
contemporâneo & con*tem*po*râ*ne.o \xmark & con*tem*po*râ*neo \cmark \\
conterrâneo & con*ter*râ*ne.o \xmark & con*ter*râ*neo \cmark \\
conteudista & con*te-u*dis*ta \xmark & con*te-u*dis*ta \xmark \\
conteúdo & con*te-ú*do \xmark & con*te*ú*do \cmark \\
continência & con*ti*nên*ci.a \xmark & con*ti*nên*cia \cmark \\
contingência & con*tin*gên*ci.a \xmark & con*tin*gên*cia \cmark \\
contínua & con*tí*nu.a \xmark & con*tí*nua \cmark \\
continuidade & con*ti*nu-i*da*de \xmark & con*ti*nu*i*da*de \cmark \\
continuísta & con*ti*nu-ís*ta \xmark & con*ti*nu*ís*ta \cmark \\
contínuo & con*tí*nu.o \xmark & con*tí*nuo \cmark \\
contrabateria & con*tra*ba*te*ri*a \cmark & con*tra*ba*te*ri-a \xmark \\
contraditória & con*tra*di*tó*ri.a \xmark & con*tra*di*tó*ria \cmark \\
contraditório & con*tra*di*tó*ri.o \xmark & con*tra*di*tó*rio \cmark \\
contradomínio & con*tra*do*mí*ni.o \xmark & con*tra*do*mí*nio \cmark \\
contraído & con*tra-í*do \xmark & con*tra*í*do \cmark \\
contraindicação & con*tra-in*di*ca*ção \xmark & con*tra*in*di*ca*ção \cmark \\
contrair & con*tra-ir \xmark & con*tra*ir \cmark \\
contrário & con*trá*ri.o \xmark & con*trá*rio \cmark \\
contrastaria & con*tras*ta*ri*a \cmark & con*tras*ta*ri-a \xmark \\
contribuição & con*tri*bu-i*ção \xmark & con*tri*bu*i*ção \cmark \\
contribuidor & con*tri*bu-i*dor \xmark & con*tri*bu*i*dor \cmark \\
contribuinte & con*tri*bu-in*te \xmark & con*tri*bu*in*te \cmark \\
contribuir & con*tri*bu-ir \xmark & con*tri*bu*ir \cmark \\
controladoria & con*tro*la*do*ri*a \cmark & con*tro*la*do*ri-a \xmark \\
controvérsia & con*tro*vér*si.a \xmark & con*tro*vér*sia \cmark \\
contumácia & con*tu*má*ci.a \xmark & con*tu*má*cia \cmark \\
contundência & con*tun*dên*ci*a \cmark & con*tun*dên*ci-a \xmark \\
conúbio & co*nú*bi.o \xmark & co*nú*bio \cmark \\
conveniência & con*ve*ni*ên*ci.a \xmark & con*ve*ni*ên*cia \cmark \\
convênio & con*vê*ni.o \xmark & con*vê*nio \cmark \\
convergência & con*ver*gên*ci.a \xmark & con*ver*gên*cia \cmark \\
convivência & con*vi*vên*ci.a \xmark & con*vi*vên*cia \cmark \\
convívio & con*ví*vi*o \cmark & con*ví*vi-o \xmark \\
convocatória & con*vo*ca*tó*ri.a \xmark & con*vo*ca*tó*ria \cmark \\
convocatório & con*vo*ca*tó*ri.o \xmark & con*vo*ca*tó*rio \cmark \\
coordenadoria & co*or*de*na*do*ri*a \cmark & co*or*de*na*do*ri-a \xmark \\
copaíba & co*pa-í*ba \xmark & co*pa*í*ba \cmark \\
cópia & có*pi.a \xmark & có*pia \cmark \\
copião & co*pi-ão \xmark & co*pi*ão \cmark \\
copio & co*pi*o \cmark & co*pi-o \xmark \\
coprofagia & co*pro*fa*gi*a \cmark & co*pro*fa*gi-a \xmark \\
coprofilia & co*pro*fi*li*a \cmark & co*pro*fi*li-a \xmark \\
coprolalia & co*pro*la*li*a \cmark & co*pro*la*li-a \xmark \\
coqueria & co*que*ri*a \cmark & co*que*ri-a \xmark \\
coquetelaria & co*que*te*la*ri*a \cmark & co*que*te*la*ri-a \xmark \\
coqueteria & co*que*te*ri*a \cmark & co*que*te*ri-a \xmark \\
cordoaria & cor*do*a*ri*a \cmark & cor*do*a*ri-a \xmark \\
coreia & co*rei*a \cmark & co*rei-a \xmark \\
coreografia & co*re*o*gra*fi*a \cmark & co*re*o*gra*fi-a \xmark \\
coreógrafo & co*re-ó*gra*fo \xmark & co*re*ó*gra*fo \cmark \\
coreomania & co*re*o*ma*ni*a \cmark & co*re*o*ma*ni-a \xmark \\
coriáceo & co*ri-á*ce.o \xmark & co*ri*á*ceo \cmark \\
coríntio & co*rín*ti.o \xmark & co*rín*tio \cmark \\
cório & có*ri.o \xmark & có*rio \cmark \\
córnea & cór*ne.a \xmark & cór*nea \cmark \\
córneo & cór*ne.o \xmark & cór*neo \cmark \\
cornucópia & cor*nu*có*pi.a \xmark & cor*nu*có*pia \cmark \\
coroa & co*ro*a \cmark & co*ro-a \xmark \\
corografia & co*ro*gra*fi*a \cmark & co*ro*gra*fi-a \xmark \\
coroideia & co*roi*dei*a \cmark & co*roi*dei-a \xmark \\
coroinha & co*ro-i*nha \xmark & co*ro*i*nha \cmark \\
corolário & co*ro*lá*ri.o \xmark & co*ro*lá*rio \cmark \\
coronária & co*ro*ná*ri.a \xmark & co*ro*ná*ria \cmark \\
coronário & co*ro*ná*ri.o \xmark & co*ro*ná*rio \cmark \\
corpóreo & cor*pó*re.o \xmark & cor*pó*reo \cmark \\
corpulência & cor*pu*lên*ci.a \xmark & cor*pu*lên*cia \cmark \\
corregedoria & cor*re*ge*do*ri*a \cmark & cor*re*ge*do*ri-a \xmark \\
correia & cor*rei*a \cmark & cor*rei-a \xmark \\
correio & cor*rei*o \cmark & cor*rei-o \xmark \\
correligionário & cor*re*li*gi*o*ná*ri.o \xmark & cor*re*li*gi*o*ná*rio \cmark \\
correria & cor*re*ri*a \cmark & cor*re*ri-a \xmark \\
correspondência & cor*res*pon*dên*ci.a \xmark & cor*res*pon*dên*cia \cmark \\
corroído & cor*ro-í*do \xmark & cor*ro*í*do \cmark \\
corruíra & cor*ru-í*ra \xmark & cor*ru*í*ra \cmark \\
corrupião & cor*ru*pi-ão \xmark & cor*ru*pi*ão \cmark \\
corrupio & cor*ru*pi*o \cmark & cor*ru*pi-o \xmark \\
corsário & cor*sá*ri.o \xmark & cor*sá*rio \cmark \\
cortesia & cor*te*si*a \cmark & cor*te*si-a \xmark \\
corveia & cor*vei*a \cmark & cor*vei-a \xmark \\
cosmetologia & cos*me*to*lo*gi*a \cmark & cos*me*to*lo*gi-a \xmark \\
cosmogonia & cos*mo*go*ni*a \cmark & cos*mo*go*ni-a \xmark \\
cosmografia & cos*mo*gra*fi*a \cmark & cos*mo*gra*fi-a \xmark \\
cosmologia & cos*mo*lo*gi*a \cmark & cos*mo*lo*gi-a \xmark \\
cotia & co*ti*a \cmark & co*ti-a \xmark \\
cotonifício & co*to*ni*fí*ci.o \xmark & co*to*ni*fí*cio \cmark \\
cotovia & co*to*vi*a \cmark & co*to*vi-a \xmark \\
coudelaria & cou*de*la*ri*a \cmark & cou*de*la*ri-a \xmark \\
covardia & co*var*di*a \cmark & co*var*di-a \xmark \\
coxia & co*xi*a \cmark & co*xi-a \xmark \\
crânio & crâ*ni.o \xmark & crâ*nio \cmark \\
craniometria & cra*ni*o*me*tri*a \cmark & cra*ni*o*me*tri-a \xmark \\
craniotomia & cra*ni*o*to*mi*a \cmark & cra*ni*o*to*mi-a \xmark \\
crapô & cra-pô \xmark & cra*pô \cmark \\
credência & cre*dên*ci.a \xmark & cre*dên*cia \cmark \\
crediário & cre*di-á*ri.o \xmark & cre*di*á*rio \cmark \\
creditício & cre*di*tí*ci.o \xmark & cre*di*tí*cio \cmark \\
creperia & cre*pe*ri*a \cmark & cre*pe*ri-a \xmark \\
cretáceo & cre*tá*ce.o \xmark & cre*tá*ceo \cmark \\
cria & cri*a \cmark & cri-a \xmark \\
criatório & cri*a*tó*ri.o \xmark & cri*a*tó*rio \cmark \\
criciúma & cri*ci-ú*ma \xmark & cri*ci*ú*ma \cmark \\
criminologia & cri*mi*no*lo*gi*a \cmark & cri*mi*no*lo*gi-a \xmark \\
criobiologia & cri*o*bi*o*lo*gi*a \cmark & cri*o*bi*o*lo*gi-a \xmark \\
criocirurgia & cri*o*ci*rur*gi*a \cmark & cri*o*ci*rur*gi-a \xmark \\
criogenia & cri*o*ge*ni*a \cmark & cri*o*ge*ni-a \xmark \\
crioscopia & cri*os*co*pi*a \cmark & cri*os*co*pi-a \xmark \\
crioterapia & cri*o*te*ra*pi*a \cmark & cri*o*te*ra*pi-a \xmark \\
criptologia & crip*to*lo*gi*a \cmark & crip*to*lo*gi-a \xmark \\
criptônimo & crip-tô*ni*mo \xmark & crip*tô*ni*mo \cmark \\
criptônio & crip-tô*ni.o \xmark & crip*tô*nio \cmark \\
criptorquidia & crip*tor*qui*di*a \cmark & crip*tor*qui*di-a \xmark \\
cristalografia & cris*ta*lo*gra*fi*a \cmark & cris*ta*lo*gra*fi-a \xmark \\
cristologia & cris*to*lo*gi*a \cmark & cris*to*lo*gi-a \xmark \\
critério & cri*té*ri.o \xmark & cri*té*rio \cmark \\
criticaria & cri*ti*ca*ri*a \cmark & cri*ti*ca*ri-a \xmark \\
cróceo & cró*ce.o \xmark & cró*ceo \cmark \\
cromatografia & cro*ma*to*gra*fi*a \cmark & cro*ma*to*gra*fi-a \xmark \\
crômio & crô*mi.o \xmark & crô*mio \cmark \\
cromolitografia & cro*mo*li*to*gra*fi*a \cmark & cro*mo*li*to*gra*fi-a \xmark \\
cromoterapia & cro*mo*te*ra*pi*a \cmark & cro*mo*te*ra*pi-a \xmark \\
cronofotografia & cro*no*fo*to*gra*fi*a \cmark & cro*no*fo*to*gra*fi-a \xmark \\
cronografia & cro*no*gra*fi*a \cmark & cro*no*gra*fi-a \xmark \\
cronologia & cro*no*lo*gi*a \cmark & cro*no*lo*gi-a \xmark \\
cronometria & cro*no*me*tri*a \cmark & cro*no*me*tri-a \xmark \\
cronômetro & cro-nô*me*tro \xmark & cro*nô*me*tro \cmark \\
crotalária & cro*ta*lá*ri.a \xmark & cro*ta*lá*ria \cmark \\
crustáceo & crus*tá*ce.o \xmark & crus*tá*ceo \cmark \\
crúzio & crú*zi.o \xmark & crú*zio \cmark \\
cubiculário & cu*bi*cu*lá*ri.o \xmark & cu*bi*cu*lá*rio \cmark \\
cucurbitácea & cu*cur*bi*tá*ce.a \xmark & cu*cur*bi*tá*cea \cmark \\
cuia & cui*a \cmark & cui-a \xmark \\
cuíca & cu-í*ca \xmark & cu*í*ca \cmark \\
culinária & cu*li*ná*ri.a \xmark & cu*li*ná*ria \cmark \\
culinário & cu*li*ná*ri.o \xmark & cu*li*ná*rio \cmark \\
culminância & cul*mi*nân*ci.a \xmark & cul*mi*nân*cia \cmark \\
cúneo & cú*ne.o \xmark & cú*neo \cmark \\
curadoria & cu*ra*do*ri*a \cmark & cu*ra*do*ri-a \xmark \\
cúria & cú*ri.a \xmark & cú*ria \cmark \\
curião & cu*ri-ão \xmark & cu*ri*ão \cmark \\
curie & cu*ri*e \cmark & cu*ri-e \xmark \\
curió & cu*ri-ó \xmark & cu*ri*ó \cmark \\
cúrio & cú*ri.o \xmark & cú*rio \cmark \\
curiúva & cu*ri-ú*va \xmark & cu*ri*ú*va \cmark \\
curvilíneo & cur*vi*lí*ne.o \xmark & cur*vi*lí*neo \cmark \\
custeio & cus*tei*o \cmark & cus*tei-o \xmark \\
custódia & cus*tó*di.a \xmark & cus*tó*dia \cmark \\
custódio & cus*tó*di.o \xmark & cus*tó*dio \cmark \\
cutâneo & cu*tâ*ne.o \xmark & cu*tâ*neo \cmark \\
cutelaria & cu*te*la*ri*a \cmark & cu*te*la*ri-a \xmark \\
cutia & cu*ti*a \cmark & cu*ti-a \xmark \\
cuxiú & cu*xi-ú \xmark & cu*xi*ú \cmark \\
czar & c.zar \xmark & czar \cmark \\
czarina & c.za*ri*na \xmark & cza*ri*na \cmark \\
czarismo & c.za*ris*mo \xmark & cza*ris*mo \cmark \\
czarista & c.za*ris*ta \xmark & cza*ris*ta \cmark \\
dáblio & dá*bli.o \xmark & dá*blio \cmark \\
dácio & dá*ci.o \xmark & dá*cio \cmark \\
dactiloscopia & dac*ti*los*co*pi*a \cmark & dac*ti*los*co*pi-a \xmark \\
dadaísmo & da*da-ís*mo \xmark & da*da*ís*mo \cmark \\
dadaísta & da*da-ís*ta \xmark & da*da*ís*ta \cmark \\
daguerreótipo & da*guer*re-ó*ti*po \xmark & da*guer*re*ó*ti*po \cmark \\
daí & da-í \xmark & da*í \cmark \\
dália & dá*li.a \xmark & dá*lia \cmark \\
daltônico & dal-tô*ni*co \xmark & dal*tô*ni*co \cmark \\
danceteria & dan*ce*te*ri*a \cmark & dan*ce*te*ri-a \xmark \\
darwinismo & dar-wi*nis*mo \xmark & dar-wi*nis*mo \xmark \\
darwinista & dar-wi*nis*ta \xmark & dar-wi*nis*ta \xmark \\
datário & da*tá*ri.o \xmark & da*tá*rio \cmark \\
datilografia & da*ti*lo*gra*fi*a \cmark & da*ti*lo*gra*fi-a \xmark \\
datiloscopia & da*ti*los*co*pi*a \cmark & da*ti*los*co*pi-a \xmark \\
deambulatório & de*am*bu*la*tó*ri.o \xmark & de*am*bu*la*tó*rio \cmark \\
deão & de-ão \xmark & de*ão \cmark \\
debelatório & de*be*la*tó*ri.o \xmark & de*be*la*tó*rio \cmark \\
decacampeão & de*ca*cam*pe-ão \xmark & de*ca*cam*pe*ão \cmark \\
decadência & de*ca*dên*ci.a \xmark & de*ca*dên*cia \cmark \\
decaída & de*ca-í*da \xmark & de*ca*í*da \cmark \\
decaído & de*ca-í*do \xmark & de*ca*í*do \cmark \\
decair & de*ca-ir \xmark & de*ca*ir \cmark \\
decalcomania & de*cal*co*ma*ni*a \cmark & de*cal*co*ma*ni-a \xmark \\
decania & de*ca*ni*a \cmark & de*ca*ni-a \xmark \\
decência & de*cên*ci.a \xmark & de*cên*cia \cmark \\
decêndio & de*cên*di.o \xmark & de*cên*dio \cmark \\
decênio & de*cê*ni.o \xmark & de*cê*nio \cmark \\
decídua & de*cí*du.a \xmark & de*cí*dua \cmark \\
decíduo & de*cí*du.o \xmark & de*cí*duo \cmark \\
decisório & de*ci*só*ri.o \xmark & de*ci*só*rio \cmark \\
declamatório & de*cla*ma*tó*ri.o \xmark & de*cla*ma*tó*rio \cmark \\
declaratório & de*cla*ra*tó*ri.o \xmark & de*cla*ra*tó*rio \cmark \\
declínio & de*clí*ni.o \xmark & de*clí*nio \cmark \\
decorrência & de*cor*rên*ci.a \xmark & de*cor*rên*cia \cmark \\
decurião & de*cu*ri-ão \xmark & de*cu*ri*ão \cmark \\
dedicatória & de*di*ca*tó*ri.a \xmark & de*di*ca*tó*ria \cmark \\
defensoria & de*fen*so*ri*a \cmark & de*fen*so*ri-a \xmark \\
deferência & de*fe*rên*ci.a \xmark & de*fe*rên*cia \cmark \\
deficiência & de*fi*ci*ên*ci.a \xmark & de*fi*ci*ên*cia \cmark \\
deficitário & de*fi*ci*tá*ri.o \xmark & de*fi*ci*tá*rio \cmark \\
definitório & de*fi*ni*tó*ri.o \xmark & de*fi*ni*tó*rio \cmark \\
deflacionário & de*fla*ci*o*ná*ri.o \xmark & de*fla*ci*o*ná*rio \cmark \\
deflúvio & de*flú*vi.o \xmark & de*flú*vio \cmark \\
degenerescência & de*ge*ne*res*cên*ci.a \xmark & de*ge*ne*res*cên*cia \cmark \\
deia & dei*a \cmark & dei-a \xmark \\
deificação & de-i*fi*ca*ção \xmark & de-i*fi*ca*ção \xmark \\
deiscente & de-is*cen*te \xmark & de-is*cen*te \xmark \\
deísmo & de-ís*mo \xmark & de*ís*mo \cmark \\
deísta & de-ís*ta \xmark & de*ís*ta \cmark \\
delegacia & de*le*ga*ci*a \cmark & de*le*ga*ci-a \xmark \\
deletério & de*le*té*ri.o \xmark & de*le*té*rio \cmark \\
delfínio & del*fí*ni.o \xmark & del*fí*nio \cmark \\
delícia & de*lí*ci.a \xmark & de*lí*cia \cmark \\
delinquência & de*lin-quên*ci.a \xmark & de*lin*quên*cia \cmark \\
délio & dé*li.o \xmark & dé*lio \cmark \\
delírio & de*lí*ri.o \xmark & de*lí*rio \cmark \\
demagogia & de*ma*go*gi*a \cmark & de*ma*go*gi-a \xmark \\
demasia & de*ma*si*a \cmark & de*ma*si-a \xmark \\
demência & de*mên*ci.a \xmark & de*mên*cia \cmark \\
demissionário & de*mis*si*o*ná*ri.o \xmark & de*mis*si*o*ná*rio \cmark \\
demiúrgico & de*mi-úr*gi*co \xmark & de*mi*úr*gi*co \cmark \\
demiurgo & de*mi-ur*go \xmark & de*mi*ur*go \cmark \\
democracia & de*mo*cra*ci*a \cmark & de*mo*cra*ci-a \xmark \\
demografia & de*mo*gra*fi*a \cmark & de*mo*gra*fi-a \xmark \\
demoníaco & de*mo*ní-a*co \xmark & de*mo*ní*a*co \cmark \\
demônio & de-mô*ni.o \xmark & de*mô*nio \cmark \\
demonologia & de*mo*no*lo*gi*a \cmark & de*mo*no*lo*gi-a \xmark \\
denário & de*ná*ri.o \xmark & de*ná*rio \cmark \\
dendrocronologia & den*dro*cro*no*lo*gi*a \cmark & den*dro*cro*no*lo*gi-a \xmark \\
dendrologia & den*dro*lo*gi*a \cmark & den*dro*lo*gi-a \xmark \\
densitometria & den*si*to*me*tri*a \cmark & den*si*to*me*tri-a \xmark \\
dentária & den*tá*ri.a \xmark & den*tá*ria \cmark \\
dentário & den*tá*ri.o \xmark & den*tá*rio \cmark \\
dentifrício & den*ti*frí*ci.o \xmark & den*ti*frí*cio \cmark \\
denúncia & de*nún*ci.a \xmark & de*nún*cia \cmark \\
deontologia & de*on*to*lo*gi*a \cmark & de*on*to*lo*gi-a \xmark \\
dependência & de*pen*dên*ci.a \xmark & de*pen*dên*cia \cmark \\
depoimento & de*po-i*men*to \xmark & de*po*i*men*to \cmark \\
depositário & de*po*si*tá*ri.o \xmark & de*po*si*tá*rio \cmark \\
depreciável & de*pre*ci-á*vel \xmark & de*pre*ci*á*vel \cmark \\
dermatologia & der*ma*to*lo*gi*a \cmark & der*ma*to*lo*gi-a \xmark \\
derrisório & der*ri*só*ri.o \xmark & der*ri*só*rio \cmark \\
desafio & de*sa*fi*o \cmark & de*sa*fi-o \xmark \\
deságio & de*sá*gi.o \xmark & de*sá*gio \cmark \\
desarmonia & de*sar*mo*ni*a \cmark & de*sar*mo*ni-a \xmark \\
desassistência & de*sas*sis*tên*ci.a \xmark & de*sas*sis*tên*cia \cmark \\
desbloqueio & des*blo*quei*o \cmark & des*blo*quei-o \xmark \\
descaída & des*ca-í*da \xmark & des*ca*í*da \cmark \\
descair & des*ca-ir \xmark & des*ca*ir \cmark \\
descendência & des*cen*dên*ci.a \xmark & des*cen*dên*cia \cmark \\
desconstruir & des*cons*tru-ir \xmark & des*cons*tru*ir \cmark \\
descontinuidade & des*con*ti*nu-i*da*de \xmark & des*con*ti*nu*i*da*de \cmark \\
descontraído & des*con*tra-í*do \xmark & des*con*tra*í*do \cmark \\
descontrair & des*con*tra-ir \xmark & des*con*tra*ir \cmark \\
descortesia & des*cor*te*si*a \cmark & des*cor*te*si-a \xmark \\
deselegância & de*se*le*gân*ci.a \xmark & de*se*le*gân*cia \cmark \\
desembainhar & de*sem*ba-i*nhar \xmark & de*sem*ba*i*nhar \cmark \\
desequilíbrio & de*se*qui*lí*bri.o \xmark & de*se*qui*lí*brio \cmark \\
desfastio & des*fas*ti*o \cmark & des*fas*ti-o \xmark \\
desídia & de*sí*di.a \xmark & de*sí*dia \cmark \\
desígnio & de*síg*ni.o \xmark & de*síg*nio \cmark \\
desinência & de*si*nên*ci.a \xmark & de*si*nên*cia \cmark \\
desistência & de*sis*tên*ci.a \xmark & de*sis*tên*cia \cmark \\
desmaio & des*mai*o \cmark & des*mai-o \xmark \\
desmemória & des*me*mó*ri.a \xmark & des*me*mó*ria \cmark \\
desnecessário & des*ne*ces*sá*ri.o \xmark & des*ne*ces*sá*rio \cmark \\
desnorteio & des*nor*tei*o \cmark & des*nor*tei-o \xmark \\
desobediência & de*so*be*di*ên*ci.a \xmark & de*so*be*di*ên*cia \cmark \\
desobstruir & de*sobs*tru-ir \xmark & de*sobs*tru*ir \cmark \\
despautério & des*pau*té*ri.o \xmark & des*pau*té*rio \cmark \\
desperdício & des*per*dí*ci.o \xmark & des*per*dí*cio \cmark \\
despoluição & des*po*lu-i*ção \xmark & des*po*lu*i*ção \cmark \\
despoluir & des*po*lu-ir \xmark & des*po*lu*ir \cmark \\
desprestígio & des*pres*tí*gi*o \cmark & des*pres*tí*gi-o \xmark \\
destampatório & des*tam*pa*tó*ri.o \xmark & des*tam*pa*tó*rio \cmark \\
destilaria & des*ti*la*ri*a \cmark & des*ti*la*ri-a \xmark \\
destinatário & des*ti*na*tá*ri.o \xmark & des*ti*na*tá*rio \cmark \\
destituição & des*ti*tu-i*ção \xmark & des*ti*tu*i*ção \cmark \\
destituído & des*ti*tu-í*do \xmark & des*ti*tu*í*do \cmark \\
destituir & des*ti*tu-ir \xmark & des*ti*tu*ir \cmark \\
destruição & des*tru-i*ção \xmark & des*tru*i*ção \cmark \\
destruído & des*tru-í*do \xmark & des*tru*í*do \cmark \\
destruidor & des*tru-i*dor \xmark & des*tru-i*dor \xmark \\
destruir & des*tru-ir \xmark & des*tru*ir \cmark \\
desunião & de*su*ni-ão \xmark & de*su*ni*ão \cmark \\
desvario & des*va*ri*o \cmark & des*va*ri-o \xmark \\
desvio & des*vi*o \cmark & des*vi-o \xmark \\
deutério & deu*té*ri.o \xmark & deu*té*rio \cmark \\
devaneio & de*va*nei*o \cmark & de*va*nei-o \xmark \\
diábase & di-á*ba*se \xmark & di*á*ba*se \cmark \\
diabásio & di*a*bá*si.o \xmark & di*a*bá*sio \cmark \\
diaconia & di*a*co*ni*a \cmark & di*a*co*ni-a \xmark \\
diácono & di-á*co*no \xmark & di*á*co*no \cmark \\
diacronia & di*a*cro*ni*a \cmark & di*a*cro*ni-a \xmark \\
díada & dí-a*da \xmark & dí*a*da \cmark \\
díade & dí-a*de \xmark & dí*a*de \cmark \\
dia & di*a \cmark & di-a \xmark \\
diádico & di-á*di*co \xmark & di*á*di*co \cmark \\
diádoco & di-á*do*co \xmark & di*á*do*co \cmark \\
diáfano & di-á*fa*no \xmark & di*á*fa*no \cmark \\
diáfise & di-á*fi*se \xmark & di*á*fi*se \cmark \\
dialetologia & di*a*le*to*lo*gi*a \cmark & di*a*le*to*lo*gi-a \xmark \\
diálise & di-á*li*se \xmark & di*á*li*se \cmark \\
diálogo & di-á*lo*go \xmark & di*á*lo*go \cmark \\
diária & di-á*ri.a \xmark & di*á*ria \cmark \\
diário & di-á*ri.o \xmark & di*á*rio \cmark \\
diarquia & di*ar*qui*a \cmark & di*ar*qui-a \xmark \\
diarreia & di*ar*rei*a \cmark & di*ar*rei-a \xmark \\
diáspora & di-ás*po*ra \xmark & di*ás*po*ra \cmark \\
diásporo & di-ás*po*ro \xmark & di*ás*po*ro \cmark \\
diástole & di-ás*to*le \xmark & di*ás*to*le \cmark \\
diatômico & di*a-tô*mi*co \xmark & di*a*tô*mi*co \cmark \\
dicastério & di*cas*té*ri.o \xmark & di*cas*té*rio \cmark \\
dicionário & di*ci*o*ná*ri.o \xmark & di*ci*o*ná*rio \cmark \\
dicogamia & di*co*ga*mi*a \cmark & di*co*ga*mi-a \xmark \\
dicotomia & di*co*to*mi*a \cmark & di*co*to*mi-a \xmark \\
dicotômico & di*co-tô*mi*co \xmark & di*co*tô*mi*co \cmark \\
dicroísmo & di*cro-ís*mo \xmark & di*cro*ís*mo \cmark \\
didímio & di*dí*mi.o \xmark & di*dí*mio \cmark \\
diédrico & di-é*dri*co \xmark & di*é*dri*co \cmark \\
diérese & di-é*re*se \xmark & di*é*re*se \cmark \\
difamatório & di*fa*ma*tó*ri.o \xmark & di*fa*ma*tó*rio \cmark \\
diferenciável & di*fe*ren*ci-á*vel \xmark & di*fe*ren*ci*á*vel \cmark \\
difluência & di*flu-ên*ci.a \xmark & di*flu*ên*cia \cmark \\
difteria & dif*te*ri*a \cmark & dif*te*ri-a \xmark \\
digestório & di*ges*tó*ri.o \xmark & di*ges*tó*rio \cmark \\
diglossia & di*glos*si*a \cmark & di*glos*si-a \xmark \\
dignitário & dig*ni*tá*ri.o \xmark & dig*ni*tá*rio \cmark \\
diligência & di*li*gên*ci.a \xmark & di*li*gên*cia \cmark \\
diluição & di*lu-i*ção \xmark & di*lu*i*ção \cmark \\
diluído & di*lu-í*do \xmark & di*lu*í*do \cmark \\
diluir & di*lu-ir \xmark & di*lu*ir \cmark \\
dilúvio & di*lú*vi.o \xmark & di*lú*vio \cmark \\
diminuição & di*mi*nu-i*ção \xmark & di*mi*nu*i*ção \cmark \\
diminuído & di*mi*nu-í*do \xmark & di*mi*nu*í*do \cmark \\
diminuir & di*mi*nu-ir \xmark & di*mi*nu*ir \cmark \\
dinamarquês & di*na*mar-quês \xmark & di*na*mar*quês \cmark \\
dinamômetro & di*na-mô*me*tro \xmark & di*na*mô*me*tro \cmark \\
dinastia & di*nas*ti*a \cmark & di*nas*ti-a \xmark \\
díodo & dí-o*do \xmark & dí*o*do \cmark \\
dioneia & di*o*nei*a \cmark & di*o*nei-a \xmark \\
dionisíaco & di*o*ni*sí-a*co \xmark & di*o*ni*sí*a*co \cmark \\
dioptria & di*op*tri*a \cmark & di*op*tri-a \xmark \\
dióptrica & di-óp*tri*ca \xmark & di*óp*tri*ca \cmark \\
dióxido & di-ó*xi*do \xmark & di*ó*xi*do \cmark \\
diplegia & di*ple*gi*a \cmark & di*ple*gi-a \xmark \\
diplomacia & di*plo*ma*ci*a \cmark & di*plo*ma*ci-a \xmark \\
diplopia & di*plo*pi*a \cmark & di*plo*pi-a \xmark \\
diretoria & di*re*to*ri*a \cmark & di*re*to*ri-a \xmark \\
diretório & di*re*tó*ri.o \xmark & di*re*tó*rio \cmark \\
disartria & di*sar*tri*a \cmark & di*sar*tri-a \xmark \\
discalculia & dis*cal*cu*li*a \cmark & dis*cal*cu*li-a \xmark \\
discinesia & dis*ci*ne*si*a \cmark & dis*ci*ne*si-a \xmark \\
discografia & dis*co*gra*fi*a \cmark & dis*co*gra*fi-a \xmark \\
discordância & dis*cor*dân*ci.a \xmark & dis*cor*dân*cia \cmark \\
discórdia & dis*cór*di.a \xmark & dis*cór*dia \cmark \\
discrepância & dis*cre*pân*ci.a \xmark & dis*cre*pân*cia \cmark \\
discricionário & dis*cri*ci*o*ná*ri.o \xmark & dis*cri*ci*o*ná*rio \cmark \\
discriminatório & dis*cri*mi*na*tó*ri.o \xmark & dis*cri*mi*na*tó*rio \cmark \\
disenteria & di*sen*te*ri*a \cmark & di*sen*te*ri-a \xmark \\
disfagia & dis*fa*gi*a \cmark & dis*fa*gi-a \xmark \\
disfemia & dis*fe*mi*a \cmark & dis*fe*mi-a \xmark \\
disfonia & dis*fo*ni*a \cmark & dis*fo*ni-a \xmark \\
disforia & dis*fo*ri*a \cmark & dis*fo*ri-a \xmark \\
disgenesia & dis*ge*ne*si*a \cmark & dis*ge*ne*si-a \xmark \\
disgrafia & dis*gra*fi*a \cmark & dis*gra*fi-a \xmark \\
dislalia & dis*la*li*a \cmark & dis*la*li-a \xmark \\
dislexia & dis*le*xi*a \cmark & dis*le*xi-a \xmark \\
dismenorreia & dis*me*nor*rei*a \cmark & dis*me*nor*rei-a \xmark \\
dismorfia & dis*mor*fi*a \cmark & dis*mor*fi-a \xmark \\
dispareunia & dis*pa*reu*ni*a \cmark & dis*pa*reu*ni-a \xmark \\
dispêndio & dis*pên*di.o \xmark & dis*pên*dio \cmark \\
dispensário & dis*pen*sá*ri.o \xmark & dis*pen*sá*rio \cmark \\
dispepsia & dis*pep*si*a \cmark & dis*pep*si-a \xmark \\
displasia & dis*pla*si*a \cmark & dis*pla*si-a \xmark \\
dispneia & disp*nei*a \cmark & disp*nei-a \xmark \\
dispraxia & dis*pra*xi*a \cmark & dis*pra*xi-a \xmark \\
disprósio & dis*pró*si.o \xmark & dis*pró*sio \cmark \\
disritmia & dis*rit*mi*a \cmark & dis*rit*mi-a \xmark \\
dissacarídeo & dis*sa*ca*rí*de.o \xmark & dis*sa*ca*rí*deo \cmark \\
dissidência & dis*si*dên*ci.a \xmark & dis*si*dên*cia \cmark \\
dissídio & dis*sí*di.o \xmark & dis*sí*dio \cmark \\
dissimetria & dis*si*me*tri*a \cmark & dis*si*me*tri-a \xmark \\
dissonância & dis*so*nân*ci.a \xmark & dis*so*nân*cia \cmark \\
distanásia & dis*ta*ná*si.a \xmark & dis*ta*ná*sia \cmark \\
distância & dis*tân*ci.a \xmark & dis*tân*cia \cmark \\
distimia & dis*ti*mi*a \cmark & dis*ti*mi-a \xmark \\
distocia & dis*to*ci*a \cmark & dis*to*ci-a \xmark \\
distonia & dis*to*ni*a \cmark & dis*to*ni-a \xmark \\
distopia & dis*to*pi*a \cmark & dis*to*pi-a \xmark \\
distraído & dis*tra-í*do \xmark & dis*tra*í*do \cmark \\
distrair & dis*tra-ir \xmark & dis*tra*ir \cmark \\
distribuição & dis*tri*bu-i*ção \xmark & dis*tri*bu*i*ção \cmark \\
distribuído & dis*tri*bu-í*do \xmark & dis*tri*bu*í*do \cmark \\
distribuidora & dis*tri*bu-i*do*ra \xmark & dis*tri*bu*i*do*ra \cmark \\
distribuidor & dis*tri*bu-i*dor \xmark & dis*tri*bu*i*dor \cmark \\
distribuir & dis*tri*bu-ir \xmark & dis*tri*bu*ir \cmark \\
distrofia & dis*tro*fi*a \cmark & dis*tro*fi-a \xmark \\
distúrbio & dis*túr*bi.o \xmark & dis*túr*bio \cmark \\
disúria & di*sú*ri.a \xmark & di*sú*ria \cmark \\
diurno & di-ur*no \xmark & di*ur*no \cmark \\
divergência & di*ver*gên*ci.a \xmark & di*ver*gên*cia \cmark \\
divinatório & di*vi*na*tó*ri.o \xmark & di*vi*na*tó*rio \cmark \\
divisionário & di*vi*si*o*ná*ri.o \xmark & di*vi*si*o*ná*rio \cmark \\
divisória & di*vi*só*ri.a \xmark & di*vi*só*ria \cmark \\
divisório & di*vi*só*ri.o \xmark & di*vi*só*rio \cmark \\
divórcio & di*vór*ci.o \xmark & di*vór*cio \cmark \\
doçaria & do*ça*ri*a \cmark & do*ça*ri-a \xmark \\
docência & do*cên*ci.a \xmark & do*cên*cia \cmark \\
doceria & do*ce*ri*a \cmark & do*ce*ri-a \xmark \\
documentário & do*cu*men*tá*ri.o \xmark & do*cu*men*tá*rio \cmark \\
dodecafonia & do*de*ca*fo*ni*a \cmark & do*de*ca*fo*ni-a \xmark \\
dodecafônico & do*de*ca-fô*ni*co \xmark & do*de*ca*fô*ni*co \cmark \\
doentio & do*en*ti*o \cmark & do*en*ti-o \xmark \\
doído & do-í*do \xmark & do*í*do \cmark \\
dolência & do*lên*ci.a \xmark & do*lên*cia \cmark \\
domiciliário & do*mi*ci*li-á*ri.o \xmark & do*mi*ci*li*á*rio \cmark \\
domicílio & do*mi*cí*li.o \xmark & do*mi*cí*lio \cmark \\
dominância & do*mi*nân*ci.a \xmark & do*mi*nân*cia \cmark \\
domínio & do*mí*ni.o \xmark & do*mí*nio \cmark \\
donataria & do*na*ta*ri*a \cmark & do*na*ta*ri-a \xmark \\
donatário & do*na*tá*ri.o \xmark & do*na*tá*rio \cmark \\
dório & dó*ri.o \xmark & dó*rio \cmark \\
dormência & dor*mên*ci.a \xmark & dor*mên*cia \cmark \\
dormitório & dor*mi*tó*ri*o \cmark & dor*mi*tó*ri-o \xmark \\
dosimetria & do*si*me*tri*a \cmark & do*si*me*tri-a \xmark \\
doutrinário & dou*tri*ná*ri.o \xmark & dou*tri*ná*rio \cmark \\
doxografia & do*xo*gra*fi*a \cmark & do*xo*gra*fi-a \xmark \\
doxologia & do*xo*lo*gi*a \cmark & do*xo*lo*gi-a \xmark \\
drágea & drá*ge.a \xmark & drá*gea \cmark \\
dramatologia & dra*ma*to*lo*gi*a \cmark & dra*ma*to*lo*gi-a \xmark \\
dramaturgia & dra*ma*tur*gi*a \cmark & dra*ma*tur*gi-a \xmark \\
dríade & drí-a*de \xmark & drí*a*de \cmark \\
drogaria & dro*ga*ri*a \cmark & dro*ga*ri-a \xmark \\
dromedário & dro*me*dá*ri.o \xmark & dro*me*dá*rio \cmark \\
drupáceo & dru*pá*ce.o \xmark & dru*pá*ceo \cmark \\
dúbio & dú*bi.o \xmark & dú*bio \cmark \\
dulcineia & dul*ci*nei*a \cmark & dul*ci*nei-a \xmark \\
dulia & du*li*a \cmark & du*li-a \xmark \\
duo & du*o \cmark & du-o \xmark \\
duralumínio & du*ra*lu*mí*ni.o \xmark & du*ra*lu*mí*nio \cmark \\
durião & du*ri-ão \xmark & du*ri*ão \cmark \\
duúnviro & du-ún*vi*ro \xmark & du-ún*vi*ro \xmark \\
dúzia & dú*zi.a \xmark & dú*zia \cmark \\
dzeta & d.ze*ta \xmark & d.ze*ta \xmark \\
ébrio & é*bri*o \cmark & é*bri-o \xmark \\
ebúrneo & e*búr*ne.o \xmark & e*búr*neo \cmark \\
eclampsia & e*clamp*si*a \cmark & e*clamp*si-a \xmark \\
eclâmpsia & e*clâmp*si*a \cmark & e*clâmp*si-a \xmark \\
eclesiástico & e*cle*si-ás*ti*co \xmark & e*cle*si*ás*ti*co \cmark \\
eclesiologia & e*cle*si*o*lo*gi*a \cmark & e*cle*si*o*lo*gi-a \xmark \\
ecocídio & e*co*cí*di.o \xmark & e*co*cí*dio \cmark \\
ecografia & e*co*gra*fi*a \cmark & e*co*gra*fi-a \xmark \\
ecolalia & e*co*la*li*a \cmark & e*co*la*li-a \xmark \\
ecologia & e*co*lo*gi*a \cmark & e*co*lo*gi-a \xmark \\
econometria & e*co*no*me*tri*a \cmark & e*co*no*me*tri-a \xmark \\
economia & e*co*no*mi*a \cmark & e*co*no*mi-a \xmark \\
econômico & e*co-nô*mi*co \xmark & e*co*nô*mi*co \cmark \\
ecônomo & e-cô*no*mo \xmark & e*cô*no*mo \cmark \\
ecotoxicologia & e*co*to*xi*co*lo*gi*a \cmark & e*co*to*xi*co*lo*gi-a \xmark \\
ectasia & ec*ta*si*a \cmark & ec*ta*si-a \xmark \\
ectopia & ec*to*pi*a \cmark & ec*to*pi-a \xmark \\
edafologia & e*da*fo*lo*gi*a \cmark & e*da*fo*lo*gi-a \xmark \\
edifício & e*di*fí*ci.o \xmark & e*di*fí*cio \cmark \\
editoria & e*di*to*ri*a \cmark & e*di*to*ri-a \xmark \\
educandário & e*du*can*dá*ri.o \xmark & e*du*can*dá*rio \cmark \\
efervescência & e*fer*ves*cên*ci.a \xmark & e*fer*ves*cên*cia \cmark \\
eficácia & e*fi*cá*ci.a \xmark & e*fi*cá*cia \cmark \\
eficiência & e*fi*ci*ên*ci.a \xmark & e*fi*ci*ên*cia \cmark \\
efígie & e*fí*gi.e \xmark & e*fí*gie \cmark \\
eflorescência & e*flo*res*cên*ci.a \xmark & e*flo*res*cên*cia \cmark \\
eflúvio & e*flú*vi.o \xmark & e*flú*vio \cmark \\
egéria & e*gé*ri.a \xmark & e*gé*ria \cmark \\
egípcio & e*gíp*ci.o \xmark & e*gíp*cio \cmark \\
egiptologia & e*gip*to*lo*gi*a \cmark & e*gip*to*lo*gi-a \xmark \\
egoísmo & e*go-ís*mo \xmark & e*go*ís*mo \cmark \\
egoísta & e*go-ís*ta \xmark & e*go*ís*ta \cmark \\
egoístico & e*go-ís*ti*co \xmark & e*go*ís*ti*co \cmark \\
egolatria & e*go*la*tri*a \cmark & e*go*la*tri-a \xmark \\
egrégio & e*gré*gi.o \xmark & e*gré*gio \cmark \\
eia & ei*a \cmark & ei-a \xmark \\
ejaculatório & e*ja*cu*la*tó*ri.o \xmark & e*ja*cu*la*tó*rio \cmark \\
elastério & e*las*té*ri.o \xmark & e*las*té*rio \cmark \\
elastômero & e*las-tô*me*ro \xmark & e*las*tô*me*ro \cmark \\
elatério & e*la*té*ri.o \xmark & e*la*té*rio \cmark \\
elefantíase & e*le*fan*tí-a*se \xmark & e*le*fan*tí*a*se \cmark \\
elegância & e*le*gân*ci.a \xmark & e*le*gân*cia \cmark \\
elegíaco & e*le*gí-a*co \xmark & e*le*gí*a*co \cmark \\
elegia & e*le*gi*a \cmark & e*le*gi-a \xmark \\
eletrocardiógrafo & e*le*tro*car*di-ó*gra*fo \xmark & e*le*tro*car*di*ó*gra*fo \cmark \\
eletroímã & e*le*tro-í*mã \xmark & e*le*tro*í*mã \cmark \\
eletroquímica & e*le*tro-quí*mi*ca \xmark & e*le*tro*quí*mi*ca \cmark \\
eletroquímico & e*le*tro-quí*mi*co \xmark & e*le*tro*quí*mi*co \cmark \\
eletrotecnia & e*le*tro*tec*ni*a \cmark & e*le*tro*tec*ni-a \xmark \\
eletroterapia & e*le*tro*te*ra*pi*a \cmark & e*le*tro*te*ra*pi-a \xmark \\
elevatório & e*le*va*tó*ri.o \xmark & e*le*va*tó*rio \cmark \\
eliminatória & e*li*mi*na*tó*ri.a \xmark & e*li*mi*na*tó*ria \cmark \\
eliminatório & e*li*mi*na*tó*ri.o \xmark & e*li*mi*na*tó*rio \cmark \\
elogiável & e*lo*gi-á*vel \xmark & e*lo*gi*á*vel \cmark \\
elogio & e*lo*gi*o \cmark & e*lo*gi-o \xmark \\
eloquência & e*lo-quên*ci.a \xmark & e*lo*quên*cia \cmark \\
elucidário & e*lu*ci*dá*ri.o \xmark & e*lu*ci*dá*rio \cmark \\
eluição & e*lu-i*ção \xmark & e*lu*i*ção \cmark \\
embainhar & em*ba-i*nhar \xmark & em*ba*i*nhar \cmark \\
embolia & em*bo*li*a \cmark & em*bo*li-a \xmark \\
embrião & em*bri-ão \xmark & em*bri*ão \cmark \\
embriologia & em*bri*o*lo*gi*a \cmark & em*bri*o*lo*gi-a \xmark \\
embrionário & em*bri*o*ná*ri.o \xmark & em*bri*o*ná*rio \cmark \\
ementário & e*men*tá*ri*o \cmark & e*men*tá*ri-o \xmark \\
emergência & e*mer*gên*ci.a \xmark & e*mer*gên*cia \cmark \\
eminência & e*mi*nên*ci.a \xmark & e*mi*nên*cia \cmark \\
emissário & e*mis*sá*ri.o \xmark & e*mis*sá*rio \cmark \\
empáfia & em*pá*fi.a \xmark & em*pá*fia \cmark \\
empatia & em*pa*ti*a \cmark & em*pa*ti-a \xmark \\
empório & em*pó*ri.o \xmark & em*pó*rio \cmark \\
empregatício & em*pre*ga*tí*ci.o \xmark & em*pre*ga*tí*cio \cmark \\
empresário & em*pre*sá*ri.o \xmark & em*pre*sá*rio \cmark \\
encantatório & en*can*ta*tó*ri.o \xmark & en*can*ta*tó*rio \cmark \\
encefalopatia & en*ce*fa*lo*pa*ti*a \cmark & en*ce*fa*lo*pa*ti-a \xmark \\
enchia & en*chi*a \cmark & en*chi-a \xmark \\
enciclopédia & en*ci*clo*pé*di.a \xmark & en*ci*clo*pé*dia \cmark \\
endemia & en*de*mi*a \cmark & en*de*mi-a \xmark \\
endocárdio & en*do*cár*di.o \xmark & en*do*cár*dio \cmark \\
endocrinologia & en*do*cri*no*lo*gi*a \cmark & en*do*cri*no*lo*gi-a \xmark \\
endodontia & en*do*don*ti*a \cmark & en*do*don*ti-a \xmark \\
endodôntico & en*do-dôn*ti*co \xmark & en*do*dôn*ti*co \cmark \\
endogamia & en*do*ga*mi*a \cmark & en*do*ga*mi-a \xmark \\
endométrio & en*do*mé*tri.o \xmark & en*do*mé*trio \cmark \\
endoscopia & en*dos*co*pi*a \cmark & en*dos*co*pi-a \xmark \\
endoscópio & en*dos*có*pi.o \xmark & en*dos*có*pio \cmark \\
endotélio & en*do*té*li.o \xmark & en*do*té*lio \cmark \\
energia & e*ner*gi*a \cmark & e*ner*gi-a \xmark \\
enfermaria & en*fer*ma*ri*a \cmark & en*fer*ma*ri-a \xmark \\
engenharia & en*ge*nha*ri*a \cmark & en*ge*nha*ri-a \xmark \\
enguia & en*gui*a \cmark & en*gui-a \xmark \\
enjoo & en*jo*o \cmark & en*jo-o \xmark \\
enologia & e*no*lo*gi*a \cmark & e*no*lo*gi-a \xmark \\
ensaio & en*sai*o \cmark & en*sai-o \xmark \\
ensaísmo & en*sa-ís*mo \xmark & en*sa*ís*mo \cmark \\
ensaísta & en*sa-ís*ta \xmark & en*sa*ís*ta \cmark \\
ensaística & en*sa-ís*ti*ca \xmark & en*sa*ís*ti*ca \cmark \\
ensaístico & en*sa-ís*ti*co \xmark & en*sa*ís*ti*co \cmark \\
entalpia & en*tal*pi*a \cmark & en*tal*pi-a \xmark \\
enteléquia & en*te*lé*qui.a \xmark & en*te*lé*quia \cmark \\
enteropatia & en*te*ro*pa*ti*a \cmark & en*te*ro*pa*ti-a \xmark \\
entomofagia & en*to*mo*fa*gi*a \cmark & en*to*mo*fa*gi-a \xmark \\
entomofilia & en*to*mo*fi*li*a \cmark & en*to*mo*fi*li-a \xmark \\
entomologia & en*to*mo*lo*gi*a \cmark & en*to*mo*lo*gi-a \xmark \\
entrância & en*trân*ci.a \xmark & en*trân*cia \cmark \\
entremeio & en*tre*mei*o \cmark & en*tre*mei-o \xmark \\
entropia & en*tro*pi*a \cmark & en*tro*pi-a \xmark \\
entusiástico & en*tu*si-ás*ti*co \xmark & en*tu*si*ás*ti*co \cmark \\
envio & en*vi*o \cmark & en*vi-o \xmark \\
envoltório & en*vol*tó*ri.o \xmark & en*vol*tó*rio \cmark \\
envolvência & en*vol*vên*ci.a \xmark & en*vol*vên*cia \cmark \\
enxertia & en*xer*ti*a \cmark & en*xer*ti-a \xmark \\
enxovia & en*xo*vi*a \cmark & en*xo*vi-a \xmark \\
enzimologia & en*zi*mo*lo*gi*a \cmark & en*zi*mo*lo*gi-a \xmark \\
eólico & e-ó*li*co \xmark & e*ó*li*co \cmark \\
eólio & e-ó*li.o \xmark & e*ó*lio \cmark \\
éon & é-on \xmark & é*on \cmark \\
eoo & e*o*o \cmark & e*o-o \xmark \\
eosinofilia & e*o*si*no*fi*li*a \cmark & e*o*si*no*fi*li-a \xmark \\
eparquia & e*par*qui*a \cmark & e*par*qui-a \xmark \\
epicôndilo & e*pi-côn*di*lo \xmark & e*pi*côn*di*lo \cmark \\
epidemia & e*pi*de*mi*a \cmark & e*pi*de*mi-a \xmark \\
epidemiologia & e*pi*de*mi*o*lo*gi*a \cmark & e*pi*de*mi*o*lo*gi-a \xmark \\
epifenômeno & e*pi*fe-nô*me*no \xmark & e*pi*fe*nô*me*no \cmark \\
epigrafia & e*pi*gra*fi*a \cmark & e*pi*gra*fi-a \xmark \\
epilepsia & e*pi*lep*si*a \cmark & e*pi*lep*si-a \xmark \\
epipódio & e*pi*pó*di.o \xmark & e*pi*pó*dio \cmark \\
episcópio & e*pis*có*pi.o \xmark & e*pis*có*pio \cmark \\
episiotomia & e*pi*si*o*to*mi*a \cmark & e*pi*si*o*to*mi-a \xmark \\
episódio & e*pi*só*di.o \xmark & e*pi*só*dio \cmark \\
epistemologia & e*pis*te*mo*lo*gi*a \cmark & e*pis*te*mo*lo*gi-a \xmark \\
epistolário & e*pis*to*lá*ri.o \xmark & e*pis*to*lá*rio \cmark \\
epistolografia & e*pis*to*lo*gra*fi*a \cmark & e*pis*to*lo*gra*fi-a \xmark \\
epitáfio & e*pi*tá*fi.o \xmark & e*pi*tá*fio \cmark \\
epitalâmio & e*pi*ta*lâ*mi.o \xmark & e*pi*ta*lâ*mio \cmark \\
epitélio & e*pi*té*li.o \xmark & e*pi*té*lio \cmark \\
epizootia & e*pi*zo*o*ti*a \cmark & e*pi*zo*o*ti-a \xmark \\
epônimo & e-pô*ni*mo \xmark & e*pô*ni*mo \cmark \\
epopeia & e*po*pei*a \cmark & e*po*pei-a \xmark \\
equânime & e-quâ*ni*me \xmark & e*quâ*ni*me \cmark \\
equídeo & e-quí*de.o \xmark & e*quí*deo \cmark \\
equidistância & e*qui*dis*tân*ci.a \xmark & e*qui*dis*tân*cia \cmark \\
equilíbrio & e*qui*lí*bri.o \xmark & e*qui*lí*brio \cmark \\
equinócio & e*qui*nó*ci.o \xmark & e*qui*nó*cio \cmark \\
equivalência & e*qui*va*lên*ci.a \xmark & e*qui*va*lên*cia \cmark \\
equívoco & e-quí*vo*co \xmark & e*quí*vo*co \cmark \\
equoterapia & e*quo*te*ra*pi*a \cmark & e*quo*te*ra*pi-a \xmark \\
erário & e*rá*ri.o \xmark & e*rá*rio \cmark \\
érbio & ér*bi.o \xmark & ér*bio \cmark \\
erétria & e*ré*tri.a \xmark & e*ré*tria \cmark \\
ergometria & er*go*me*tri*a \cmark & er*go*me*tri-a \xmark \\
ergômetro & er-gô*me*tro \xmark & er*gô*me*tro \cmark \\
ergonomia & er*go*no*mi*a \cmark & er*go*no*mi-a \xmark \\
ergonômico & er*go-nô*mi*co \xmark & er*go*nô*mi*co \cmark \\
erotomania & e*ro*to*ma*ni*a \cmark & e*ro*to*ma*ni-a \xmark \\
errância & er*rân*ci.a \xmark & er*rân*cia \cmark \\
esbórnia & es*bór*ni.a \xmark & es*bór*nia \cmark \\
escadaria & es*ca*da*ri*a \cmark & es*ca*da*ri-a \xmark \\
escadório & es*ca*dó*ri.o \xmark & es*ca*dó*rio \cmark \\
escândio & es*cân*di.o \xmark & es*cân*dio \cmark \\
escanteio & es*can*tei*o \cmark & es*can*tei-o \xmark \\
escapulário & es*ca*pu*lá*ri.o \xmark & es*ca*pu*lá*rio \cmark \\
escarabeídeo & es*ca*ra*be-í*de.o \xmark & es*ca*ra*be*í*deo \cmark \\
escárnio & es*cár*ni.o \xmark & es*cár*nio \cmark \\
escatologia & es*ca*to*lo*gi*a \cmark & es*ca*to*lo*gi-a \xmark \\
esclerodermia & es*cle*ro*der*mi*a \cmark & es*cle*ro*der*mi-a \xmark \\
escleroterapia & es*cle*ro*te*ra*pi*a \cmark & es*cle*ro*te*ra*pi-a \xmark \\
escoa & es*co*a \cmark & es*co-a \xmark \\
escócia & es*có*ci.a \xmark & es*có*cia \cmark \\
escólio & es*có*li.o \xmark & es*có*lio \cmark \\
escória & es*có*ri.a \xmark & es*có*ria \cmark \\
escorpião & es*cor*pi-ão \xmark & es*cor*pi*ão \cmark \\
escorregadio & es*cor*re*ga*di*o \cmark & es*cor*re*ga*di-o \xmark \\
escorrência & es*cor*rên*ci.a \xmark & es*cor*rên*cia \cmark \\
escravaria & es*cra*va*ri*a \cmark & es*cra*va*ri-a \xmark \\
escritório & es*cri*tó*ri.o \xmark & es*cri*tó*rio \cmark \\
escriturário & es*cri*tu*rá*ri.o \xmark & es*cri*tu*rá*rio \cmark \\
escrofulária & es*cro*fu*lá*ri.a \xmark & es*cro*fu*lá*ria \cmark \\
escrutínio & es*cru*tí*ni.o \xmark & es*cru*tí*nio \cmark \\
escuderia & es*cu*de*ri*a \cmark & es*cu*de*ri-a \xmark \\
esculápio & es*cu*lá*pi.o \xmark & es*cu*lá*pio \cmark \\
esfigmomanômetro & es*fig*mo*ma-nô*me*tro \xmark & es*fig*mo*ma*nô*me*tro \cmark \\
esfiha & es*fi-ha \xmark & es*fi-ha \xmark \\
esguio & es*gui*o \cmark & es*gui-o \xmark \\
esofagectomia & e*so*fa*gec*to*mi*a \cmark & e*so*fa*gec*to*mi-a \xmark \\
esôfago & e-sô*fa*go \xmark & e*sô*fa*go \cmark \\
espadaúdo & es*pa*da-ú*do \xmark & es*pa*da*ú*do \cmark \\
especiaria & es*pe*ci*a*ri*a \cmark & es*pe*ci*a*ri-a \xmark \\
espécie & es*pé*ci.e \xmark & es*pé*cie \cmark \\
espectrofotômetro & es*pec*tro*fo-tô*me*tro \xmark & es*pec*tro*fo*tô*me*tro \cmark \\
espectrografia & es*pec*tro*gra*fi*a \cmark & es*pec*tro*gra*fi-a \xmark \\
espectrometria & es*pec*tro*me*tri*a \cmark & es*pec*tro*me*tri-a \xmark \\
espectroscopia & es*pec*tros*co*pi*a \cmark & es*pec*tros*co*pi-a \xmark \\
espeleologia & es*pe*le*o*lo*gi*a \cmark & es*pe*le*o*lo*gi-a \xmark \\
espeleólogo & es*pe*le-ó*lo*go \xmark & es*pe*le*ó*lo*go \cmark \\
espia & es*pi*a \cmark & es*pi-a \xmark \\
espião & es*pi-ão \xmark & es*pi*ão \cmark \\
espirometria & es*pi*ro*me*tri*a \cmark & es*pi*ro*me*tri-a \xmark \\
esplenectomia & es*ple*nec*to*mi*a \cmark & es*ple*nec*to*mi-a \xmark \\
esplenomegalia & es*ple*no*me*ga*li*a \cmark & es*ple*no*me*ga*li-a \xmark \\
espólio & es*pó*li.o \xmark & es*pó*lio \cmark \\
espontâneo & es*pon*tâ*ne.o \xmark & es*pon*tâ*neo \cmark \\
esporângio & es*po*rân*gi.o \xmark & es*po*rân*gio \cmark \\
espúrio & es*pú*ri.o \xmark & es*pú*rio \cmark \\
esquadria & es*qua*dri*a \cmark & es*qua*dri-a \xmark \\
esquálido & es-quá*li*do \xmark & es*quá*li*do \cmark \\
esquizofrenia & es*qui*zo*fre*ni*a \cmark & es*qui*zo*fre*ni-a \xmark \\
esquizogonia & es*qui*zo*go*ni*a \cmark & es*qui*zo*go*ni-a \xmark \\
essência & es*sên*ci.a \xmark & es*sên*cia \cmark \\
estacaria & es*ta*ca*ri*a \cmark & es*ta*ca*ri-a \xmark \\
estacionário & es*ta*ci*o*ná*ri.o \xmark & es*ta*ci*o*ná*rio \cmark \\
estadia & es*ta*di*a \cmark & es*ta*di-a \xmark \\
estádio & es*tá*di.o \xmark & es*tá*dio \cmark \\
estagiário & es*ta*gi-á*ri.o \xmark & es*ta*gi*á*rio \cmark \\
estágio & es*tá*gi.o \xmark & es*tá*gio \cmark \\
estaminódio & es*ta*mi*nó*di.o \xmark & es*ta*mi*nó*dio \cmark \\
estamparia & es*tam*pa*ri*a \cmark & es*tam*pa*ri-a \xmark \\
estância & es*tân*ci.a \xmark & es*tân*cia \cmark \\
estapafúrdio & es*ta*pa*fúr*di.o \xmark & es*ta*pa*fúr*dio \cmark \\
estátua & es*tá*tu.a \xmark & es*tá*tua \cmark \\
estatuária & es*ta*tu-á*ri.a \xmark & es*ta*tu*á*ria \cmark \\
estatuário & es*ta*tu-á*ri.o \xmark & es*ta*tu*á*rio \cmark \\
estatutário & es*ta*tu*tá*ri.o \xmark & es*ta*tu*tá*rio \cmark \\
esteárico & es*te-á*ri*co \xmark & es*te*á*ri*co \cmark \\
esteatorreia & es*te*a*tor*rei*a \cmark & es*te*a*tor*rei-a \xmark \\
esteganografia & es*te*ga*no*gra*fi*a \cmark & es*te*ga*no*gra*fi-a \xmark \\
esteio & es*tei*o \cmark & es*tei-o \xmark \\
estelionatário & es*te*li*o*na*tá*ri.o \xmark & es*te*li*o*na*tá*rio \cmark \\
estenografia & es*te*no*gra*fi*a \cmark & es*te*no*gra*fi-a \xmark \\
estequiometria & es*te*qui*o*me*tri*a \cmark & es*te*qui*o*me*tri-a \xmark \\
estereóbata & es*te*re-ó*ba*ta \xmark & es*te*re*ó*ba*ta \cmark \\
estéreo & es*té*re.o \xmark & es*té*reo \cmark \\
estereofonia & es*te*re*o*fo*ni*a \cmark & es*te*re*o*fo*ni-a \xmark \\
estereologia & es*te*re*o*lo*gi*a \cmark & es*te*re*o*lo*gi-a \xmark \\
estereoquímica & es*te*re*o-quí*mi*ca \xmark & es*te*re*o*quí*mi*ca \cmark \\
estereoquímico & es*te*re*o-quí*mi*co \xmark & es*te*re*o*quí*mi*co \cmark \\
estereoscopia & es*te*re*os*co*pi*a \cmark & es*te*re*os*co*pi-a \xmark \\
estereoscópio & es*te*re*os*có*pi.o \xmark & es*te*re*os*có*pio \cmark \\
estereotipia & es*te*re*o*ti*pi*a \cmark & es*te*re*o*ti*pi-a \xmark \\
estereótipo & es*te*re-ó*ti*po \xmark & es*te*re*ó*ti*po \cmark \\
estereotomia & es*te*re*o*to*mi*a \cmark & es*te*re*o*to*mi-a \xmark \\
estetoscópio & es*te*tos*có*pi.o \xmark & es*te*tos*có*pio \cmark \\
estígio & es*tí*gi.o \xmark & es*tí*gio \cmark \\
estio & es*ti*o \cmark & es*ti-o \xmark \\
estipêndio & es*ti*pên*di.o \xmark & es*ti*pên*dio \cmark \\
estômago & es-tô*ma*go \xmark & es*tô*ma*go \cmark \\
estomatologia & es*to*ma*to*lo*gi*a \cmark & es*to*ma*to*lo*gi-a \xmark \\
estória & es*tó*ri.a \xmark & es*tó*ria \cmark \\
estratégia & es*tra*té*gi.a \xmark & es*tra*té*gia \cmark \\
estratigrafia & es*tra*ti*gra*fi*a \cmark & es*tra*ti*gra*fi-a \xmark \\
estrebaria & es*tre*ba*ri*a \cmark & es*tre*ba*ri-a \xmark \\
estreia & es*trei*a \cmark & es*trei-a \xmark \\
estria & es*tri*a \cmark & es*tri-a \xmark \\
estroboscópio & es*tro*bos*có*pi.o \xmark & es*tro*bos*có*pio \cmark \\
estrogênio & es*tro*gê*ni.o \xmark & es*tro*gê*nio \cmark \\
estrôncio & es*trôn*ci.o \xmark & es*trôn*cio \cmark \\
estuário & es*tu-á*ri.o \xmark & es*tu*á*rio \cmark \\
estúdio & es*tú*di.o \xmark & es*tú*dio \cmark \\
estultícia & es*tul*tí*ci.a \xmark & es*tul*tí*cia \cmark \\
estúrdia & es*túr*di.a \xmark & es*túr*dia \cmark \\
esvair & es*va-ir \xmark & es*va*ir \cmark \\
etário & e*tá*ri.o \xmark & e*tá*rio \cmark \\
etéreo & e*té*re.o \xmark & e*té*reo \cmark \\
etimologia & e*ti*mo*lo*gi*a \cmark & e*ti*mo*lo*gi-a \xmark \\
etiologia & e*ti*o*lo*gi*a \cmark & e*ti*o*lo*gi-a \xmark \\
etíope & e*tí-o*pe \xmark & e*tí*o*pe \cmark \\
etiópico & e*ti-ó*pi*co \xmark & e*ti*ó*pi*co \cmark \\
etnia & et*ni*a \cmark & et*ni-a \xmark \\
etnocídio & et*no*cí*di.o \xmark & et*no*cí*dio \cmark \\
etnografia & et*no*gra*fi*a \cmark & et*no*gra*fi-a \xmark \\
etnologia & et*no*lo*gi*a \cmark & et*no*lo*gi-a \xmark \\
etnomusicologia & et*no*mu*si*co*lo*gi*a \cmark & et*no*mu*si*co*lo*gi-a \xmark \\
etnônimo & et-nô*ni*mo \xmark & et*nô*ni*mo \cmark \\
etólio & e*tó*li.o \xmark & e*tó*lio \cmark \\
etologia & e*to*lo*gi*a \cmark & e*to*lo*gi-a \xmark \\
eucariótico & eu*ca*ri-ó*ti*co \xmark & eu*ca*ri*ó*ti*co \cmark \\
eucaristia & eu*ca*ris*ti*a \cmark & eu*ca*ris*ti-a \xmark \\
eufonia & eu*fo*ni*a \cmark & eu*fo*ni-a \xmark \\
eufônico & eu-fô*ni*co \xmark & eu*fô*ni*co \cmark \\
euforia & eu*fo*ri*a \cmark & eu*fo*ri-a \xmark \\
eufrásia & eu*frá*si.a \xmark & eu*frá*sia \cmark \\
eugenia & eu*ge*ni*a \cmark & eu*ge*ni-a \xmark \\
eulalia & eu*la*li*a \cmark & eu*la*li-a \xmark \\
eurasiático & eu*ra*si-á*ti*co \xmark & eu*ra*si*á*ti*co \cmark \\
europeísmo & eu*ro*pe-ís*mo \xmark & eu*ro*pe*ís*mo \cmark \\
europeísta & eu*ro*pe-ís*ta \xmark & eu*ro*pe*ís*ta \cmark \\
európio & eu*ró*pi.o \xmark & eu*ró*pio \cmark \\
eustasia & eus*ta*si*a \cmark & eus*ta*si-a \xmark \\
eutanásia & eu*ta*ná*si.a \xmark & eu*ta*ná*sia \cmark \\
eutério & eu*té*ri.o \xmark & eu*té*rio \cmark \\
eutonia & eu*to*ni*a \cmark & eu*to*ni-a \xmark \\
evangeliário & e*van*ge*li-á*ri.o \xmark & e*van*ge*li*á*rio \cmark \\
evidência & e*vi*dên*ci.a \xmark & e*vi*dên*cia \cmark \\
evoé & e*vo-é \xmark & e*vo*é \cmark \\
evolucionário & e*vo*lu*ci*o*ná*ri.o \xmark & e*vo*lu*ci*o*ná*rio \cmark \\
evoluído & e*vo*lu-í*do \xmark & e*vo*lu*í*do \cmark \\
evoluir & e*vo*lu-ir \xmark & e*vo*lu*ir \cmark \\
excelência & ex*ce*lên*ci.a \xmark & ex*ce*lên*cia \cmark \\
excitatório & ex*ci*ta*tó*ri.o \xmark & ex*ci*ta*tó*rio \cmark \\
excluído & ex*clu-í*do \xmark & ex*clu*í*do \cmark \\
excluir & ex*clu-ir \xmark & ex*clu*ir \cmark \\
excrescência & ex*cres*cên*ci.a \xmark & ex*cres*cên*cia \cmark \\
excretório & ex*cre*tó*ri.o \xmark & ex*cre*tó*rio \cmark \\
executória & e*xe*cu*tó*ri.a \xmark & e*xe*cu*tó*ria \cmark \\
executório & e*xe*cu*tó*ri.o \xmark & e*xe*cu*tó*rio \cmark \\
exéquias & e*xé*qui.as \xmark & e*xé*qui.as \xmark \\
exequível & e*xe-quí*vel \xmark & e*xe*quí*vel \cmark \\
exercício & e*xer*cí*ci.o \xmark & e*xer*cí*cio \cmark \\
exigência & e*xi*gên*ci.a \xmark & e*xi*gên*cia \cmark \\
exílio & e*xí*li*o \cmark & e*xí*li-o \xmark \\
exímio & e*xí*mi.o \xmark & e*xí*mio \cmark \\
existência & e*xis*tên*ci.a \xmark & e*xis*tên*cia \cmark \\
exobiologia & e*xo*bi*o*lo*gi*a \cmark & e*xo*bi*o*lo*gi-a \xmark \\
exodontia & e*xo*don*ti*a \cmark & e*xo*don*ti-a \xmark \\
exoftalmia & e*xof*tal*mi*a \cmark & e*xof*tal*mi-a \xmark \\
exogamia & e*xo*ga*mi*a \cmark & e*xo*ga*mi-a \xmark \\
exorbitância & e*xor*bi*tân*ci.a \xmark & e*xor*bi*tân*cia \cmark \\
exórdio & e*xór*di.o \xmark & e*xór*dio \cmark \\
expedicionário & ex*pe*di*ci*o*ná*ri.o \xmark & ex*pe*di*ci*o*ná*rio \cmark \\
experiência & ex*pe*ri*ên*ci.a \xmark & ex*pe*ri*ên*cia \cmark \\
expiatório & ex*pi*a*tó*ri.o \xmark & ex*pi*a*tó*rio \cmark \\
explanatório & ex*pla*na*tó*ri.o \xmark & ex*pla*na*tó*rio \cmark \\
exploratório & ex*plo*ra*tó*ri.o \xmark & ex*plo*ra*tó*rio \cmark \\
extemporâneo & ex*tem*po*râ*ne.o \xmark & ex*tem*po*râ*neo \cmark \\
extermínio & ex*ter*mí*ni.o \xmark & ex*ter*mí*nio \cmark \\
extrair & ex*tra-ir \xmark & ex*tra*ir \cmark \\
extranumerário & ex*tra*nu*me*rá*ri.o \xmark & ex*tra*nu*me*rá*rio \cmark \\
extraordinário & ex*tra*or*di*ná*ri.o \xmark & ex*tra*or*di*ná*rio \cmark \\
extravagância & ex*tra*va*gân*ci.a \xmark & ex*tra*va*gân*cia \cmark \\
extravio & ex*tra*vi*o \cmark & ex*tra*vi-o \xmark \\
exuberância & e*xu*be*rân*ci.a \xmark & e*xu*be*rân*cia \cmark \\
exutório & e*xu*tó*ri.o \xmark & e*xu*tó*rio \cmark \\
fabulário & fa*bu*lá*ri.o \xmark & fa*bu*lá*rio \cmark \\
facaia & fa*cai*a \cmark & fa*cai-a \xmark \\
factício & fac*tí*ci.o \xmark & fac*tí*cio \cmark \\
facúndia & fa*cún*di.a \xmark & fa*cún*dia \cmark \\
faia & fai*a \cmark & fai-a \xmark \\
faim & fa-im \xmark & fa-im \xmark \\
faísca & fa-ís*ca \xmark & fa*ís*ca \cmark \\
faiscante & fa-is*can*te \xmark & fa-is*can*te \xmark \\
faiscar & fa-is*car \xmark & fa-is*car \xmark \\
falácia & fa*lá*ci.a \xmark & fa*lá*cia \cmark \\
falanstério & fa*lans*té*ri.o \xmark & fa*lans*té*rio \cmark \\
falatório & fa*la*tó*ri.o \xmark & fa*la*tó*rio \cmark \\
falcatrua & fal*ca*tru*a \cmark & fal*ca*tru-a \xmark \\
falcoaria & fal*co*a*ri*a \cmark & fal*co*a*ri-a \xmark \\
falência & fa*lên*ci.a \xmark & fa*lên*cia \cmark \\
falésia & fa*lé*si.a \xmark & fa*lé*sia \cmark \\
faloplastia & fa*lo*plas*ti*a \cmark & fa*lo*plas*ti-a \xmark \\
falsário & fal*sá*ri.o \xmark & fal*sá*rio \cmark \\
família & fa*mí*li.a \xmark & fa*mí*lia \cmark \\
familistério & fa*mi*lis*té*ri.o \xmark & fa*mi*lis*té*rio \cmark \\
fancaria & fan*ca*ri*a \cmark & fan*ca*ri-a \xmark \\
fantasia & fan*ta*si*a \cmark & fan*ta*si-a \xmark \\
fantasmagoria & fan*tas*ma*go*ri*a \cmark & fan*tas*ma*go*ri-a \xmark \\
faraó & fa*ra-ó \xmark & fa*ra*ó \cmark \\
faraônico & fa*ra-ô*ni*co \xmark & fa*ra*ô*ni*co \cmark \\
farfalharia & far*fa*lha*ri*a \cmark & far*fa*lha*ri-a \xmark \\
farináceo & fa*ri*ná*ce.o \xmark & fa*ri*ná*ceo \cmark \\
faríngeo & fa*rín*ge.o \xmark & fa*rín*geo \cmark \\
farisaísmo & fa*ri*sa-ís*mo \xmark & fa*ri*sa*ís*mo \cmark \\
farmácia & far*má*ci.a \xmark & far*má*cia \cmark \\
farmacopeia & far*ma*co*pei*a \cmark & far*ma*co*pei-a \xmark \\
farmacoterapia & far*ma*co*te*ra*pi*a \cmark & far*ma*co*te*ra*pi-a \xmark \\
fáscia & fás*ci.a \xmark & fás*cia \cmark \\
fascínio & fas*cí*ni.o \xmark & fas*cí*nio \cmark \\
fascíola & fas*cí-o*la \xmark & fas*cí*o*la \cmark \\
fasquia & fas*qui*a \cmark & fas*qui-a \xmark \\
fastígio & fas*tí*gi.o \xmark & fas*tí*gio \cmark \\
fastio & fas*ti*o \cmark & fas*ti-o \xmark \\
fatia & fa*ti*a \cmark & fa*ti-a \xmark \\
fátuo & fá*tu.o \xmark & fá*tuo \cmark \\
faúlha & fa-ú*lha \xmark & fa*ú*lha \cmark \\
favônio & fa-vô*ni.o \xmark & fa*vô*nio \cmark \\
fazendário & fa*zen*dá*ri.o \xmark & fa*zen*dá*rio \cmark \\
fecharia & fe*cha*ri*a \cmark & fe*cha*ri-a \xmark \\
feérico & fe-é*ri*co \xmark & fe*é*ri*co \cmark \\
feio & fei*o \cmark & fei-o \xmark \\
feitiçaria & fei*ti*ça*ri*a \cmark & fei*ti*ça*ri-a \xmark \\
feitio & fei*ti*o \cmark & fei*ti-o \xmark \\
feitoria & fei*to*ri*a \cmark & fei*to*ri-a \xmark \\
feiura & fei-u*ra \xmark & fei*u*ra \cmark \\
felícia & fe*lí*ci.a \xmark & fe*lí*cia \cmark \\
felídeo & fe*lí*de.o \xmark & fe*lí*deo \cmark \\
felonia & fe*lo*ni*a \cmark & fe*lo*ni-a \xmark \\
fêmea & fê*me.a \xmark & fê*mea \cmark \\
fenício & fe*ní*ci.o \xmark & fe*ní*cio \cmark \\
fenilcetonúria & fe*nil*ce*to*nú*ri.a \xmark & fe*nil*ce*to*nú*ria \cmark \\
fenologia & fe*no*lo*gi*a \cmark & fe*no*lo*gi-a \xmark \\
fenômeno & fe-nô*me*no \xmark & fe*nô*me*no \cmark \\
fenomenologia & fe*no*me*no*lo*gi*a \cmark & fe*no*me*no*lo*gi-a \xmark \\
féria & fé*ri.a \xmark & fé*ria \cmark \\
férmio & fér*mi.o \xmark & fér*mio \cmark \\
feroês & fe*ro-ês \xmark & fe*ro*ês \cmark \\
ferramentaria & fer*ra*men*ta*ri*a \cmark & fer*ra*men*ta*ri-a \xmark \\
ferraria & fer*ra*ri*a \cmark & fer*ra*ri-a \xmark \\
férreo & fér*re.o \xmark & fér*reo \cmark \\
ferrovia & fer*ro*vi*a \cmark & fer*ro*vi-a \xmark \\
ferroviário & fer*ro*vi-á*ri.o \xmark & fer*ro*vi*á*rio \cmark \\
ferrugíneo & fer*ru*gí*ne.o \xmark & fer*ru*gí*neo \cmark \\
feudatário & feu*da*tá*ri.o \xmark & feu*da*tá*rio \cmark \\
fiável & fi-á*vel \xmark & fi*á*vel \cmark \\
fibromialgia & fi*bro*mi*al*gi*a \cmark & fi*bro*mi*al*gi-a \xmark \\
fichário & fi*chá*ri.o \xmark & fi*chá*rio \cmark \\
ficologia & fi*co*lo*gi*a \cmark & fi*co*lo*gi-a \xmark \\
fictício & fic*tí*ci.o \xmark & fic*tí*cio \cmark \\
fidalguia & fi*dal*gui*a \cmark & fi*dal*gui-a \xmark \\
fideísmo & fi*de-ís*mo \xmark & fi*de*ís*mo \cmark \\
fidejussória & fi*de*jus*só*ri.a \xmark & fi*de*jus*só*ria \cmark \\
fidúcia & fi*dú*ci.a \xmark & fi*dú*cia \cmark \\
fiduciário & fi*du*ci-á*ri.o \xmark & fi*du*ci*á*rio \cmark \\
filantropia & fi*lan*tro*pi*a \cmark & fi*lan*tro*pi-a \xmark \\
filarmônica & fi*lar-mô*ni*ca \xmark & fi*lar*mô*ni*ca \cmark \\
filarmônico & fi*lar-mô*ni*co \xmark & fi*lar*mô*ni*co \cmark \\
filatelia & fi*la*te*li*a \cmark & fi*la*te*li-a \xmark \\
filáucia & fi*láu*ci.a \xmark & fi*láu*cia \cmark \\
filmografia & fil*mo*gra*fi*a \cmark & fil*mo*gra*fi-a \xmark \\
filmologia & fil*mo*lo*gi*a \cmark & fil*mo*lo*gi-a \xmark \\
filogenia & fi*lo*ge*ni*a \cmark & fi*lo*ge*ni-a \xmark \\
filologia & fi*lo*lo*gi*a \cmark & fi*lo*lo*gi-a \xmark \\
filosofia & fi*lo*so*fi*a \cmark & fi*lo*so*fi-a \xmark \\
filotaxia & fi*lo*ta*xi*a \cmark & fi*lo*ta*xi-a \xmark \\
fímbria & fím*bri.a \xmark & fím*bria \cmark \\
financiável & fi*nan*ci-á*vel \xmark & fi*nan*ci*á*vel \cmark \\
finório & fi*nó*ri.o \xmark & fi*nó*rio \cmark \\
fio & fi*o \cmark & fi-o \xmark \\
fisiatria & fi*si*a*tri*a \cmark & fi*si*a*tri-a \xmark \\
fisiocracia & fi*si*o*cra*ci*a \cmark & fi*si*o*cra*ci-a \xmark \\
fisiografia & fi*si*o*gra*fi*a \cmark & fi*si*o*gra*fi-a \xmark \\
fisiologia & fi*si*o*lo*gi*a \cmark & fi*si*o*lo*gi-a \xmark \\
fisiólogo & fi*si-ó*lo*go \xmark & fi*si*ó*lo*go \cmark \\
fisionomia & fi*si*o*no*mi*a \cmark & fi*si*o*no*mi-a \xmark \\
fisioterapia & fi*si*o*te*ra*pi*a \cmark & fi*si*o*te*ra*pi-a \xmark \\
fitogeografia & fi*to*ge*o*gra*fi*a \cmark & fi*to*ge*o*gra*fi-a \xmark \\
fitopatologia & fi*to*pa*to*lo*gi*a \cmark & fi*to*pa*to*lo*gi-a \xmark \\
fitoquímica & fi*to-quí*mi*ca \xmark & fi*to*quí*mi*ca \cmark \\
fitotecnia & fi*to*tec*ni*a \cmark & fi*to*tec*ni-a \xmark \\
fitoterapia & fi*to*te*ra*pi*a \cmark & fi*to*te*ra*pi-a \xmark \\
fiúza & fi-ú*za \xmark & fi*ú*za \cmark \\
flagrância & fla*grân*ci.a \xmark & fla*grân*cia \cmark \\
flamínio & fla*mí*ni.o \xmark & fla*mí*nio \cmark \\
flatulência & fla*tu*lên*ci.a \xmark & fla*tu*lên*cia \cmark \\
flebografia & fle*bo*gra*fi*a \cmark & fle*bo*gra*fi-a \xmark \\
flebotomia & fle*bo*to*mi*a \cmark & fle*bo*to*mi-a \xmark \\
flexografia & fle*xo*gra*fi*a \cmark & fle*xo*gra*fi-a \xmark \\
floreio & flo*rei*o \cmark & flo*rei-o \xmark \\
florescência & flo*res*cên*ci.a \xmark & flo*res*cên*cia \cmark \\
florilégio & flo*ri*lé*gi.o \xmark & flo*ri*lé*gio \cmark \\
fluência & flu-ên*ci.a \xmark & flu*ên*cia \cmark \\
fluidez & flu-i*dez \xmark & flu-i*dez \xmark \\
fluídico & flu-í*di*co \xmark & flu*í*di*co \cmark \\
fluir & flu-ir \xmark & flu*ir \cmark \\
fluorescência & flu*o*res*cên*ci.a \xmark & flu*o*res*cên*cia \cmark \\
flúor & flú-or \xmark & flú*or \cmark \\
fluoroscopia & flu*o*ros*co*pi*a \cmark & flu*o*ros*co*pi-a \xmark \\
fobia & fo*bi*a \cmark & fo*bi-a \xmark \\
focídeo & fo*cí*de.o \xmark & fo*cí*deo \cmark \\
foguetório & fo*gue*tó*ri.o \xmark & fo*gue*tó*rio \cmark \\
folheio & fo*lhei*o \cmark & fo*lhei-o \xmark \\
foliáceo & fo*li-á*ce.o \xmark & fo*li*á*ceo \cmark \\
folia & fo*li*a \cmark & fo*li-a \xmark \\
folião & fo*li-ão \xmark & fo*li*ão \cmark \\
folíolo & fo*lí-o*lo \xmark & fo*lí*o*lo \cmark \\
fonoaudiólogo & fo*no*au*di-ó*lo*go \xmark & fo*no*au*di*ó*lo*go \cmark \\
fonografia & fo*no*gra*fi*a \cmark & fo*no*gra*fi-a \xmark \\
fonologia & fo*no*lo*gi*a \cmark & fo*no*lo*gi-a \xmark \\
fontanário & fon*ta*ná*ri.o \xmark & fon*ta*ná*rio \cmark \\
forâneo & fo*râ*ne.o \xmark & fo*râ*neo \cmark \\
formaldeído & for*mal*de-í*do \xmark & for*mal*de*í*do \cmark \\
formaria & for*ma*ri*a \cmark & for*ma*ri-a \xmark \\
formulário & for*mu*lá*ri.o \xmark & for*mu*lá*rio \cmark \\
fosforescência & fos*fo*res*cên*ci.a \xmark & fos*fo*res*cên*cia \cmark \\
fotocópia & fo*to*có*pi.a \xmark & fo*to*có*pia \cmark \\
fotofobia & fo*to*fo*bi*a \cmark & fo*to*fo*bi-a \xmark \\
fotogenia & fo*to*ge*ni*a \cmark & fo*to*ge*ni-a \xmark \\
fotografia & fo*to*gra*fi*a \cmark & fo*to*gra*fi-a \xmark \\
fotogrametria & fo*to*gra*me*tri*a \cmark & fo*to*gra*me*tri-a \xmark \\
fotolitografia & fo*to*li*to*gra*fi*a \cmark & fo*to*li*to*gra*fi-a \xmark \\
fotoluminescência & fo*to*lu*mi*nes*cên*ci.a \xmark & fo*to*lu*mi*nes*cên*cia \cmark \\
fotometria & fo*to*me*tri*a \cmark & fo*to*me*tri-a \xmark \\
fotômetro & fo-tô*me*tro \xmark & fo*tô*me*tro \cmark \\
fotoperíodo & fo*to*pe*rí-o*do \xmark & fo*to*pe*rí*o*do \cmark \\
fotoquímica & fo*to-quí*mi*ca \xmark & fo*to*quí*mi*ca \cmark \\
fotoquímico & fo*to-quí*mi*co \xmark & fo*to*quí*mi*co \cmark \\
fototaxia & fo*to*ta*xi*a \cmark & fo*to*ta*xi-a \xmark \\
fototerapia & fo*to*te*ra*pi*a \cmark & fo*to*te*ra*pi-a \xmark \\
fototipia & fo*to*ti*pi*a \cmark & fo*to*ti*pi-a \xmark \\
fototipografia & fo*to*ti*po*gra*fi*a \cmark & fo*to*ti*po*gra*fi-a \xmark \\
fóvea & fó*ve.a \xmark & fó*vea \cmark \\
fracionário & fra*ci*o*ná*ri.o \xmark & fra*ci*o*ná*rio \cmark \\
fragmentário & frag*men*tá*ri.o \xmark & frag*men*tá*rio \cmark \\
fragrância & fra*grân*ci.a \xmark & fra*grân*cia \cmark \\
fraldário & fral*dá*ri.o \xmark & fral*dá*rio \cmark \\
frâncio & frân*ci.o \xmark & frân*cio \cmark \\
francofilia & fran*co*fi*li*a \cmark & fran*co*fi*li-a \xmark \\
francofonia & fran*co*fo*ni*a \cmark & fran*co*fo*ni-a \xmark \\
franqueável & fran*que-á*vel \xmark & fran*que*á*vel \cmark \\
franquia & fran*qui*a \cmark & fran*qui-a \xmark \\
fraseologia & fra*se*o*lo*gi*a \cmark & fra*se*o*lo*gi-a \xmark \\
fratria & fra*tri*a \cmark & fra*tri-a \xmark \\
fratricídio & fra*tri*cí*di.o \xmark & fra*tri*cí*dio \cmark \\
freático & fre-á*ti*co \xmark & fre*á*ti*co \cmark \\
freguesia & fre*gue*si*a \cmark & fre*gue*si-a \xmark \\
freio & frei*o \cmark & frei-o \xmark \\
freiria & frei*ri*a \cmark & frei*ri-a \xmark \\
frenologia & fre*no*lo*gi*a \cmark & fre*no*lo*gi-a \xmark \\
frequência & fre-quên*ci.a \xmark & fre*quên*cia \cmark \\
fria & fri*a \cmark & fri-a \xmark \\
friável & fri-á*vel \xmark & fri*á*vel \cmark \\
frígio & frí*gi.o \xmark & frí*gio \cmark \\
frigoria & fri*go*ri*a \cmark & fri*go*ri-a \xmark \\
frio & fri*o \cmark & fri-o \xmark \\
frísio & frí*si.o \xmark & frí*sio \cmark \\
frontaria & fron*ta*ri*a \cmark & fron*ta*ri-a \xmark \\
frontispício & fron*tis*pí*ci.o \xmark & fron*tis*pí*cio \cmark \\
fruição & fru-i*ção \xmark & fru*i*ção \cmark \\
fruir & fru-ir \xmark & fru*ir \cmark \\
frutaria & fru*ta*ri*a \cmark & fru*ta*ri-a \xmark \\
ftálico & f.tá*li*co \xmark & f.tá*li*co \xmark \\
fuá & fu-á \xmark & fu*á \cmark \\
fúcsia & fúc*si.a \xmark & fúc*sia \cmark \\
fúfio & fú*fi.o \xmark & fú*fio \cmark \\
fugidio & fu*gi*di*o \cmark & fu*gi*di-o \xmark \\
fuinha & fu-i*nha \xmark & fu*i*nha \cmark \\
fulgurância & ful*gu*rân*ci.a \xmark & ful*gu*rân*cia \cmark \\
funcionário & fun*ci*o*ná*ri.o \xmark & fun*ci*o*ná*rio \cmark \\
fundiário & fun*di-á*ri.o \xmark & fun*di*á*rio \cmark \\
funerária & fu*ne*rá*ri.a \xmark & fu*ne*rá*ria \cmark \\
funerário & fu*ne*rá*ri.o \xmark & fu*ne*rá*rio \cmark \\
funéreo & fu*né*re.o \xmark & fu*né*reo \cmark \\
funilaria & fu*ni*la*ri*a \cmark & fu*ni*la*ri-a \xmark \\
fúria & fú*ri.a \xmark & fú*ria \cmark \\
futevôlei & fu*te-vô*lei \xmark & fu*te*vô*lei \cmark \\
futurologia & fu*tu*ro*lo*gi*a \cmark & fu*tu*ro*lo*gi-a \xmark \\
fuzilaria & fu*zi*la*ri*a \cmark & fu*zi*la*ri-a \xmark \\
fuzuê & fu*zu-ê \xmark & fu*zu*ê \cmark \\
gabião & ga*bi-ão \xmark & ga*bi*ão \cmark \\
gadolínio & ga*do*lí*ni.o \xmark & ga*do*lí*nio \cmark \\
gaélico & ga-é*li*co \xmark & ga*é*li*co \cmark \\
gafaria & ga*fa*ri*a \cmark & ga*fa*ri-a \xmark \\
gaia & gai*a \cmark & gai-a \xmark \\
gaio & gai*o \cmark & gai-o \xmark \\
galanteio & ga*lan*tei*o \cmark & ga*lan*tei-o \xmark \\
galanteria & ga*lan*te*ri*a \cmark & ga*lan*te*ri-a \xmark \\
galateia & ga*la*tei*a \cmark & ga*la*tei-a \xmark \\
galáxia & ga*lá*xi.a \xmark & ga*lá*xia \cmark \\
galeão & ga*le-ão \xmark & ga*le*ão \cmark \\
galeria & ga*le*ri*a \cmark & ga*le*ri-a \xmark \\
galhardia & ga*lhar*di*a \cmark & ga*lhar*di-a \xmark \\
galináceo & ga*li*ná*ce.o \xmark & ga*li*ná*ceo \cmark \\
gálio & gá*li.o \xmark & gá*lio \cmark \\
galizia & ga*li*zi*a \cmark & ga*li*zi-a \xmark \\
galvanômetro & gal*va-nô*me*tro \xmark & gal*va*nô*me*tro \cmark \\
galvanoplastia & gal*va*no*plas*ti*a \cmark & gal*va*no*plas*ti-a \xmark \\
gamagrafia & ga*ma*gra*fi*a \cmark & ga*ma*gra*fi-a \xmark \\
gâmbia & gâm*bi.a \xmark & gâm*bia \cmark \\
gamboa & gam*bo*a \cmark & gam*bo-a \xmark \\
gametângio & ga*me*tân*gi.o \xmark & ga*me*tân*gio \cmark \\
ganância & ga*nân*ci.a \xmark & ga*nân*cia \cmark \\
gandaia & gan*dai*a \cmark & gan*dai-a \xmark \\
gandaio & gan*dai*o \cmark & gan*dai-o \xmark \\
gânglio & gân*gli.o \xmark & gân*glio \cmark \\
garantia & ga*ran*ti*a \cmark & ga*ran*ti-a \xmark \\
gardênia & gar*dê*ni.a \xmark & gar*dê*nia \cmark \\
garoa & ga*ro*a \cmark & ga*ro-a \xmark \\
gasóleo & ga*só*le.o \xmark & ga*só*leo \cmark \\
gasometria & ga*so*me*tri*a \cmark & ga*so*me*tri-a \xmark \\
gasômetro & ga-sô*me*tro \xmark & ga*sô*me*tro \cmark \\
gastrectomia & gas*trec*to*mi*a \cmark & gas*trec*to*mi-a \xmark \\
gastroenterologia & gas*tro*en*te*ro*lo*gi*a \cmark & gas*tro*en*te*ro*lo*gi-a \xmark \\
gastrointestinal & gas*tro-in*tes*ti*nal \xmark & gas*tro*in*tes*ti*nal \cmark \\
gastronomia & gas*tro*no*mi*a \cmark & gas*tro*no*mi-a \xmark \\
gastronômico & gas*tro-nô*mi*co \xmark & gas*tro*nô*mi*co \cmark \\
gastroplastia & gas*tro*plas*ti*a \cmark & gas*tro*plas*ti-a \xmark \\
gastrostomia & gas*tros*to*mi*a \cmark & gas*tros*to*mi-a \xmark \\
gauchada & ga-u*cha*da \xmark & ga-u*cha*da \xmark \\
gauchar & ga-u*char \xmark & ga-u*char \xmark \\
gauchismo & ga-u*chis*mo \xmark & ga-u*chis*mo \xmark \\
gaúcho & ga-ú*cho \xmark & ga*ú*cho \cmark \\
gaudério & gau*dé*ri.o \xmark & gau*dé*rio \cmark \\
gáudio & gáu*di.o \xmark & gáu*dio \cmark \\
gávea & gá*ve.a \xmark & gá*vea \cmark \\
gavião & ga*vi-ão \xmark & ga*vi*ão \cmark \\
gazua & ga*zu*a \cmark & ga*zu-a \xmark \\
geio & gei*o \cmark & gei-o \xmark \\
geleia & ge*lei*a \cmark & ge*lei-a \xmark \\
gelosia & ge*lo*si*a \cmark & ge*lo*si-a \xmark \\
gêmeo & gê*me.o \xmark & gê*meo \cmark \\
gemologia & ge*mo*lo*gi*a \cmark & ge*mo*lo*gi-a \xmark \\
gendarmaria & gen*dar*ma*ri*a \cmark & gen*dar*ma*ri-a \xmark \\
genealogia & ge*ne*a*lo*gi*a \cmark & ge*ne*a*lo*gi-a \xmark \\
generalício & ge*ne*ra*lí*ci.o \xmark & ge*ne*ra*lí*cio \cmark \\
geneterapia & ge*ne*te*ra*pi*a \cmark & ge*ne*te*ra*pi-a \xmark \\
gênio & gê*ni.o \xmark & gê*nio \cmark \\
genitália & ge*ni*tá*li.a \xmark & ge*ni*tá*lia \cmark \\
geniturinário & ge*ni*tu*ri*ná*ri.o \xmark & ge*ni*tu*ri*ná*rio \cmark \\
genocídio & ge*no*cí*di.o \xmark & ge*no*cí*dio \cmark \\
genômico & ge-nô*mi*co \xmark & ge*nô*mi*co \cmark \\
gentilício & gen*ti*lí*ci.o \xmark & gen*ti*lí*cio \cmark \\
gentio & gen*ti*o \cmark & gen*ti-o \xmark \\
genuflexório & ge*nu*fle*xó*ri.o \xmark & ge*nu*fle*xó*rio \cmark \\
genuinidade & ge*nu-i*ni*da*de \xmark & ge*nu*i*ni*da*de \cmark \\
genuíno & ge*nu-í*no \xmark & ge*nu*í*no \cmark \\
geocronologia & ge*o*cro*no*lo*gi*a \cmark & ge*o*cro*no*lo*gi-a \xmark \\
geodesia & ge*o*de*si*a \cmark & ge*o*de*si-a \xmark \\
geodésia & ge*o*dé*si.a \xmark & ge*o*dé*sia \cmark \\
geoestacionário & ge*o*es*ta*ci*o*ná*ri.o \xmark & ge*o*es*ta*ci*o*ná*rio \cmark \\
geofagia & ge*o*fa*gi*a \cmark & ge*o*fa*gi-a \xmark \\
geófito & ge-ó*fi*to \xmark & ge*ó*fi*to \cmark \\
geognosia & ge*og*no*si*a \cmark & ge*og*no*si-a \xmark \\
geografia & ge*o*gra*fi*a \cmark & ge*o*gra*fi-a \xmark \\
geógrafo & ge-ó*gra*fo \xmark & ge*ó*gra*fo \cmark \\
geologia & ge*o*lo*gi*a \cmark & ge*o*lo*gi-a \xmark \\
geólogo & ge-ó*lo*go \xmark & ge*ó*lo*go \cmark \\
geomancia & ge*o*man*ci*a \cmark & ge*o*man*ci-a \xmark \\
geômetra & ge-ô*me*tra \xmark & ge*ô*me*tra \cmark \\
geometria & ge*o*me*tri*a \cmark & ge*o*me*tri-a \xmark \\
geomorfologia & ge*o*mor*fo*lo*gi*a \cmark & ge*o*mor*fo*lo*gi-a \xmark \\
geoquímica & ge*o-quí*mi*ca \xmark & ge*o*quí*mi*ca \cmark \\
geoquímico & ge*o-quí*mi*co \xmark & ge*o*quí*mi*co \cmark \\
geotectônica & ge*o*tec-tô*ni*ca \xmark & ge*o*tec*tô*ni*ca \cmark \\
gerânio & ge*râ*ni.o \xmark & ge*râ*nio \cmark \\
gerência & ge*rên*ci.a \xmark & ge*rên*cia \cmark \\
geriatria & ge*ri*a*tri*a \cmark & ge*ri*a*tri-a \xmark \\
geriátrico & ge*ri-á*tri*co \xmark & ge*ri*á*tri*co \cmark \\
germania & ger*ma*ni*a \cmark & ger*ma*ni-a \xmark \\
germânio & ger*mâ*ni.o \xmark & ger*mâ*nio \cmark \\
gerontocracia & ge*ron*to*cra*ci*a \cmark & ge*ron*to*cra*ci-a \xmark \\
gerontologia & ge*ron*to*lo*gi*a \cmark & ge*ron*to*lo*gi-a \xmark \\
gerúndio & ge*rún*di.o \xmark & ge*rún*dio \cmark \\
gerúsia & ge*rú*si.a \xmark & ge*rú*sia \cmark \\
giardíase & gi*ar*dí-a*se \xmark & gi*ar*dí*a*se \cmark \\
gigantomaquia & gi*gan*to*ma*qui*a \cmark & gi*gan*to*ma*qui-a \xmark \\
gigolô & gi*go-lô \xmark & gi*go*lô \cmark \\
ginásio & gi*ná*si.o \xmark & gi*ná*sio \cmark \\
ginecologia & gi*ne*co*lo*gi*a \cmark & gi*ne*co*lo*gi-a \xmark \\
ginecomastia & gi*ne*co*mas*ti*a \cmark & gi*ne*co*mas*ti-a \xmark \\
gio & gi*o \cmark & gi-o \xmark \\
giratório & gi*ra*tó*ri.o \xmark & gi*ra*tó*rio \cmark \\
gíria & gí*ri.a \xmark & gí*ria \cmark \\
giroscópio & gi*ros*có*pi.o \xmark & gi*ros*có*pio \cmark \\
glaciário & gla*ci-á*ri.o \xmark & gla*ci*á*rio \cmark \\
glaciologia & gla*ci*o*lo*gi*a \cmark & gla*ci*o*lo*gi-a \xmark \\
gládio & glá*di.o \xmark & glá*dio \cmark \\
gladíolo & gla*dí-o*lo \xmark & gla*dí*o*lo \cmark \\
gláucia & gláu*ci.a \xmark & gláu*cia \cmark \\
gláucio & gláu*ci.o \xmark & gláu*cio \cmark \\
glia & gli*a \cmark & gli-a \xmark \\
glicemia & gli*ce*mi*a \cmark & gli*ce*mi-a \xmark \\
glicéria & gli*cé*ri.a \xmark & gli*cé*ria \cmark \\
glicínia & gli*cí*ni.a \xmark & gli*cí*nia \cmark \\
glicogênio & gli*co*gê*ni.o \xmark & gli*co*gê*nio \cmark \\
glicoproteína & gli*co*pro*te-í*na \xmark & gli*co*pro*te*í*na \cmark \\
glicosúria & gli*co*sú*ri.a \xmark & gli*co*sú*ria \cmark \\
glória & gló*ri.a \xmark & gló*ria \cmark \\
glossário & glos*sá*ri.o \xmark & glos*sá*rio \cmark \\
glossofaríngeo & glos*so*fa*rín*ge.o \xmark & glos*so*fa*rín*geo \cmark \\
glossolalia & glos*so*la*li*a \cmark & glos*so*la*li-a \xmark \\
glúteo & glú*te.o \xmark & glú*teo \cmark \\
glutonaria & glu*to*na*ri*a \cmark & glu*to*na*ri-a \xmark \\
glutoneria & glu*to*ne*ri*a \cmark & glu*to*ne*ri-a \xmark \\
gnaisse & g.nais*se \xmark & g.nais*se \xmark \\
gnomo & g.no*mo \xmark & gno*mo \cmark \\
gnose & g.no*se \xmark & gno*se \cmark \\
gnosiologia & g.no*si*o*lo*gi*a \xmark & gno*si*o*lo*gi-a \xmark \\
gnosticismo & g.nos*ti*cis*mo \xmark & gnos*ti*cis*mo \cmark \\
gnóstico & g.nós*ti*co \xmark & gnós*ti*co \cmark \\
gnu & g.nu \xmark & g.nu \xmark \\
goês & go-ês \xmark & go*ês \cmark \\
gonorreia & go*nor*rei*a \cmark & go*nor*rei-a \xmark \\
górdio & gór*di.o \xmark & gór*dio \cmark \\
gorjeio & gor*jei*o \cmark & gor*jei-o \xmark \\
grafia & gra*fi*a \cmark & gra*fi-a \xmark \\
grafologia & gra*fo*lo*gi*a \cmark & gra*fo*lo*gi-a \xmark \\
grafomaníaco & gra*fo*ma*ní-a*co \xmark & gra*fo*ma*ní*a*co \cmark \\
grafômano & gra-fô*ma*no \xmark & gra*fô*ma*no \cmark \\
gramatologia & gra*ma*to*lo*gi*a \cmark & gra*ma*to*lo*gi-a \xmark \\
gramínea & gra*mí*ne.a \xmark & gra*mí*nea \cmark \\
gramíneo & gra*mí*ne.o \xmark & gra*mí*neo \cmark \\
grandiloquência & gran*di*lo-quên*ci.a \xmark & gran*di*lo*quên*cia \cmark \\
granulometria & gra*nu*lo*me*tri*a \cmark & gra*nu*lo*me*tri-a \xmark \\
grapiúna & gra*pi-ú*na \xmark & gra*pi*ú*na \cmark \\
gratuidade & gra*tu-i*da*de \xmark & gra*tu-i*da*de \xmark \\
graúdo & gra-ú*do \xmark & gra*ú*do \cmark \\
graúna & gra-ú*na \xmark & gra*ú*na \cmark \\
gravimetria & gra*vi*me*tri*a \cmark & gra*vi*me*tri-a \xmark \\
graxaim & gra*xa-im \xmark & gra*xa-im \xmark \\
gregário & gre*gá*ri.o \xmark & gre*gá*rio \cmark \\
gregório & gre*gó*ri.o \xmark & gre*gó*rio \cmark \\
gritaria & gri*ta*ri*a \cmark & gri*ta*ri-a \xmark \\
grosseria & gros*se*ri*a \cmark & gros*se*ri-a \xmark \\
grua & gru*a \cmark & gru-a \xmark \\
gruiforme & gru-i*for*me \xmark & gru-i*for*me \xmark \\
guaíba & gua-í*ba \xmark & gua*í*ba \cmark \\
guarânia & gua*râ*ni.a \xmark & gua*râ*nia \cmark \\
guardiania & guar*di*a*ni*a \cmark & guar*di*a*ni-a \xmark \\
guardião & guar*di-ão \xmark & guar*di*ão \cmark \\
guia & gui*a \cmark & gui-a \xmark \\
guião & gui-ão \xmark & gui*ão \cmark \\
guria & gu*ri*a \cmark & gu*ri-a \xmark \\
háfnio & háf*ni.o \xmark & háf*nio \cmark \\
hagiografia & ha*gi*o*gra*fi*a \cmark & ha*gi*o*gra*fi-a \xmark \\
hagiógrafo & ha*gi-ó*gra*fo \xmark & ha*gi*ó*gra*fo \cmark \\
hagiologia & ha*gi*o*lo*gi*a \cmark & ha*gi*o*lo*gi-a \xmark \\
hagiológio & ha*gi*o*ló*gi.o \xmark & ha*gi*o*ló*gio \cmark \\
halogêneo & ha*lo*gê*ne.o \xmark & ha*lo*gê*neo \cmark \\
halogênio & ha*lo*gê*ni.o \xmark & ha*lo*gê*nio \cmark \\
halterofilia & hal*te*ro*fi*li*a \cmark & hal*te*ro*fi*li-a \xmark \\
hamburgueria & ham*bur*gue*ri*a \cmark & ham*bur*gue*ri-a \xmark \\
hanseático & han*se-á*ti*co \xmark & han*se*á*ti*co \cmark \\
hanseníase & han*se*ní-a*se \xmark & han*se*ní*a*se \cmark \\
harmonia & har*mo*ni*a \cmark & har*mo*ni-a \xmark \\
harmônica & har-mô*ni*ca \xmark & har*mô*ni*ca \cmark \\
harmônico & har-mô*ni*co \xmark & har*mô*ni*co \cmark \\
harmônio & har-mô*ni.o \xmark & har*mô*nio \cmark \\
harpia & har*pi*a \cmark & har*pi-a \xmark \\
haustório & haus*tó*ri.o \xmark & haus*tó*rio \cmark \\
hebdomadário & heb*do*ma*dá*ri.o \xmark & heb*do*ma*dá*rio \cmark \\
hebraísmo & he*bra-ís*mo \xmark & he*bra*ís*mo \cmark \\
hebraísta & he*bra-ís*ta \xmark & he*bra*ís*ta \cmark \\
hedônico & he-dô*ni*co \xmark & he*dô*ni*co \cmark \\
hegemonia & he*ge*mo*ni*a \cmark & he*ge*mo*ni-a \xmark \\
hegemônico & he*ge-mô*ni*co \xmark & he*ge*mô*ni*co \cmark \\
helíaco & he*lí-a*co \xmark & he*lí*a*co \cmark \\
heliografia & he*li*o*gra*fi*a \cmark & he*li*o*gra*fi-a \xmark \\
heliógrafo & he*li-ó*gra*fo \xmark & he*li*ó*gra*fo \cmark \\
hélio & hé*li.o \xmark & hé*lio \cmark \\
helmintologia & hel*min*to*lo*gi*a \cmark & hel*min*to*lo*gi-a \xmark \\
helvécio & hel*vé*ci*o \cmark & hel*vé*ci-o \xmark \\
hemácia & he*má*ci.a \xmark & he*má*cia \cmark \\
hematologia & he*ma*to*lo*gi*a \cmark & he*ma*to*lo*gi-a \xmark \\
hematopoiético & he*ma*to*poi-é*ti*co \xmark & he*ma*to*poi*é*ti*co \cmark \\
hematúria & he*ma*tú*ri.a \xmark & he*ma*tú*ria \cmark \\
hemeralopia & he*me*ra*lo*pi*a \cmark & he*me*ra*lo*pi-a \xmark \\
hemianopsia & he*mi*a*nop*si*a \cmark & he*mi*a*nop*si-a \xmark \\
hemiparesia & he*mi*pa*re*si*a \cmark & he*mi*pa*re*si-a \xmark \\
hemiplegia & he*mi*ple*gi*a \cmark & he*mi*ple*gi-a \xmark \\
hemisfério & he*mis*fé*ri.o \xmark & he*mis*fé*rio \cmark \\
hemistíquio & he*mis*tí*qui.o \xmark & he*mis*tí*quio \cmark \\
hemodiálise & he*mo*di-á*li*se \xmark & he*mo*di*á*li*se \cmark \\
hemofilia & he*mo*fi*li*a \cmark & he*mo*fi*li-a \xmark \\
hemoglobinemia & he*mo*glo*bi*ne*mi*a \cmark & he*mo*glo*bi*ne*mi-a \xmark \\
hemoglobinúria & he*mo*glo*bi*nú*ri.a \xmark & he*mo*glo*bi*nú*ria \cmark \\
hemorragia & he*mor*ra*gi*a \cmark & he*mor*ra*gi-a \xmark \\
hemostasia & he*mos*ta*si*a \cmark & he*mos*ta*si-a \xmark \\
hemoterapia & he*mo*te*ra*pi*a \cmark & he*mo*te*ra*pi-a \xmark \\
henoteísmo & he*no*te-ís*mo \xmark & he*no*te*ís*mo \cmark \\
henoteísta & he*no*te-ís*ta \xmark & he*no*te*ís*ta \cmark \\
hepatologia & he*pa*to*lo*gi*a \cmark & he*pa*to*lo*gi-a \xmark \\
hepatomegalia & he*pa*to*me*ga*li*a \cmark & he*pa*to*me*ga*li-a \xmark \\
hepatopatia & he*pa*to*pa*ti*a \cmark & he*pa*to*pa*ti-a \xmark \\
heptacampeão & hep*ta*cam*pe-ão \xmark & hep*ta*cam*pe*ão \cmark \\
heptarquia & hep*tar*qui*a \cmark & hep*tar*qui-a \xmark \\
herbáceo & her*bá*ce.o \xmark & her*bá*ceo \cmark \\
hercúleo & her*cú*le.o \xmark & her*cú*leo \cmark \\
hereditário & he*re*di*tá*ri.o \xmark & he*re*di*tá*rio \cmark \\
heresia & he*re*si*a \cmark & he*re*si-a \xmark \\
hermínio & her*mí*ni.o \xmark & her*mí*nio \cmark \\
hérnia & hér*ni.a \xmark & hér*nia \cmark \\
heroína & he*ro-í*na \xmark & he*ro*í*na \cmark \\
heroísmo & he*ro-ís*mo \xmark & he*ro*ís*mo \cmark \\
herpetologia & her*pe*to*lo*gi*a \cmark & her*pe*to*lo*gi-a \xmark \\
heteria & he*te*ri*a \cmark & he*te*ri-a \xmark \\
heteroátomo & he*te*ro-á*to*mo \xmark & he*te*ro*á*to*mo \cmark \\
heterocromia & he*te*ro*cro*mi*a \cmark & he*te*ro*cro*mi-a \xmark \\
heterocronia & he*te*ro*cro*ni*a \cmark & he*te*ro*cro*ni-a \xmark \\
heterodoxia & he*te*ro*do*xi*a \cmark & he*te*ro*do*xi-a \xmark \\
heterofilia & he*te*ro*fi*li*a \cmark & he*te*ro*fi*li-a \xmark \\
heterogêneo & he*te*ro*gê*ne.o \xmark & he*te*ro*gê*neo \cmark \\
heterologia & he*te*ro*lo*gi*a \cmark & he*te*ro*lo*gi-a \xmark \\
heteronímia & he*te*ro*ní*mi.a \xmark & he*te*ro*ní*mia \cmark \\
heterônimo & he*te-rô*ni*mo \xmark & he*te*rô*ni*mo \cmark \\
heteronomia & he*te*ro*no*mi*a \cmark & he*te*ro*no*mi-a \xmark \\
heterotopia & he*te*ro*to*pi*a \cmark & he*te*ro*to*pi-a \xmark \\
hévea & hé*ve.a \xmark & hé*vea \cmark \\
hexacampeão & he*xa*cam*pe-ão \xmark & he*xa*cam*pe*ão \cmark \\
hidroavião & hi*dro*a*vi-ão \xmark & hi*dro*a*vi*ão \cmark \\
hidrocefalia & hi*dro*ce*fa*li*a \cmark & hi*dro*ce*fa*li-a \xmark \\
hidrofobia & hi*dro*fo*bi*a \cmark & hi*dro*fo*bi-a \xmark \\
hidrogênio & hi*dro*gê*ni.o \xmark & hi*dro*gê*nio \cmark \\
hidrogeologia & hi*dro*ge*o*lo*gi*a \cmark & hi*dro*ge*o*lo*gi-a \xmark \\
hidrologia & hi*dro*lo*gi*a \cmark & hi*dro*lo*gi-a \xmark \\
hidromancia & hi*dro*man*ci*a \cmark & hi*dro*man*ci-a \xmark \\
hidrometria & hi*dro*me*tri*a \cmark & hi*dro*me*tri-a \xmark \\
hidropisia & hi*dro*pi*si*a \cmark & hi*dro*pi*si-a \xmark \\
hidroponia & hi*dro*po*ni*a \cmark & hi*dro*po*ni-a \xmark \\
hidropônico & hi*dro-pô*ni*co \xmark & hi*dro*pô*ni*co \cmark \\
hidroterapia & hi*dro*te*ra*pi*a \cmark & hi*dro*te*ra*pi-a \xmark \\
hidrovia & hi*dro*vi*a \cmark & hi*dro*vi-a \xmark \\
hidroviário & hi*dro*vi-á*ri.o \xmark & hi*dro*vi*á*rio \cmark \\
hidrozoário & hi*dro*zo-á*ri.o \xmark & hi*dro*zo*á*rio \cmark \\
hierarquia & hi*e*rar*qui*a \cmark & hi*e*rar*qui-a \xmark \\
hierofania & hi*e*ro*fa*ni*a \cmark & hi*e*ro*fa*ni-a \xmark \\
higiénico & hi*gi-é*ni*co \xmark & hi*gi*é*ni*co \cmark \\
hilária & hi*lá*ri.a \xmark & hi*lá*ria \cmark \\
hilário & hi*lá*ri.o \xmark & hi*lá*rio \cmark \\
hileia & hi*lei*a \cmark & hi*lei-a \xmark \\
hilozoísmo & hi*lo*zo-ís*mo \xmark & hi*lo*zo*ís*mo \cmark \\
hinário & hi*ná*ri.o \xmark & hi*ná*rio \cmark \\
hinduísmo & hin*du-ís*mo \xmark & hin*du*ís*mo \cmark \\
hinduísta & hin*du-ís*ta \xmark & hin*du*ís*ta \cmark \\
hinologia & hi*no*lo*gi*a \cmark & hi*no*lo*gi-a \xmark \\
hinterlândia & hin*ter*lân*di.a \xmark & hin*ter*lân*dia \cmark \\
hióideo & hi-ói*de.o \xmark & hi*ói*deo \cmark \\
hiperacusia & hi*pe*ra*cu*si*a \cmark & hi*pe*ra*cu*si-a \xmark \\
hiperalgesia & hi*pe*ral*ge*si*a \cmark & hi*pe*ral*ge*si-a \xmark \\
hiperbóreo & hi*per*bó*re.o \xmark & hi*per*bó*reo \cmark \\
hipercalcemia & hi*per*cal*ce*mi*a \cmark & hi*per*cal*ce*mi-a \xmark \\
hipercapnia & hi*per*cap*ni*a \cmark & hi*per*cap*ni-a \xmark \\
hiperdulia & hi*per*du*li*a \cmark & hi*per*du*li-a \xmark \\
hiperemia & hi*pe*re*mi*a \cmark & hi*pe*re*mi-a \xmark \\
hiperestesia & hi*pe*res*te*si*a \cmark & hi*pe*res*te*si-a \xmark \\
hiperlipidemia & hi*per*li*pi*de*mi*a \cmark & hi*per*li*pi*de*mi-a \xmark \\
hipernatremia & hi*per*na*tre*mi*a \cmark & hi*per*na*tre*mi-a \xmark \\
hiperplasia & hi*per*pla*si*a \cmark & hi*per*pla*si-a \xmark \\
hipersônico & hi*per-sô*ni*co \xmark & hi*per*sô*ni*co \cmark \\
hipertermia & hi*per*ter*mi*a \cmark & hi*per*ter*mi-a \xmark \\
hipertonia & hi*per*to*ni*a \cmark & hi*per*to*ni-a \xmark \\
hipertônico & hi*per-tô*ni*co \xmark & hi*per*tô*ni*co \cmark \\
hipertrofia & hi*per*tro*fi*a \cmark & hi*per*tro*fi-a \xmark \\
hipnoterapia & hip*no*te*ra*pi*a \cmark & hip*no*te*ra*pi-a \xmark \\
hipocalcemia & hi*po*cal*ce*mi*a \cmark & hi*po*cal*ce*mi-a \xmark \\
hipocalemia & hi*po*ca*le*mi*a \cmark & hi*po*ca*le*mi-a \xmark \\
hipocapnia & hi*po*cap*ni*a \cmark & hi*po*cap*ni-a \xmark \\
hipocondríaco & hi*po*con*drí-a*co \xmark & hi*po*con*drí*a*co \cmark \\
hipocondria & hi*po*con*dri*a \cmark & hi*po*con*dri-a \xmark \\
hipocôndrio & hi*po-côn*dri.o \xmark & hi*po*côn*drio \cmark \\
hipocrisia & hi*po*cri*si*a \cmark & hi*po*cri*si-a \xmark \\
hipofisário & hi*po*fi*sá*ri.o \xmark & hi*po*fi*sá*rio \cmark \\
hipoglicemia & hi*po*gli*ce*mi*a \cmark & hi*po*gli*ce*mi-a \xmark \\
hipomaníaco & hi*po*ma*ní-a*co \xmark & hi*po*ma*ní*a*co \cmark \\
hipomania & hi*po*ma*ni*a \cmark & hi*po*ma*ni-a \xmark \\
hiponatremia & hi*po*na*tre*mi*a \cmark & hi*po*na*tre*mi-a \xmark \\
hipônimo & hi-pô*ni*mo \xmark & hi*pô*ni*mo \cmark \\
hipoplasia & hi*po*pla*si*a \cmark & hi*po*pla*si-a \xmark \\
hipotecário & hi*po*te*cá*ri.o \xmark & hi*po*te*cá*rio \cmark \\
hipoterapia & hi*po*te*ra*pi*a \cmark & hi*po*te*ra*pi-a \xmark \\
hipotermia & hi*po*ter*mi*a \cmark & hi*po*ter*mi-a \xmark \\
hipotonia & hi*po*to*ni*a \cmark & hi*po*to*ni-a \xmark \\
hipotônico & hi*po-tô*ni*co \xmark & hi*po*tô*ni*co \cmark \\
hipotrofia & hi*po*tro*fi*a \cmark & hi*po*tro*fi-a \xmark \\
hipovolemia & hi*po*vo*le*mi*a \cmark & hi*po*vo*le*mi-a \xmark \\
hipoxemia & hi*po*xe*mi*a \cmark & hi*po*xe*mi-a \xmark \\
histerectomia & his*te*rec*to*mi*a \cmark & his*te*rec*to*mi-a \xmark \\
histeria & his*te*ri*a \cmark & his*te*ri-a \xmark \\
histeroscopia & his*te*ros*co*pi*a \cmark & his*te*ros*co*pi-a \xmark \\
histeroscópio & his*te*ros*có*pi.o \xmark & his*te*ros*có*pio \cmark \\
histologia & his*to*lo*gi*a \cmark & his*to*lo*gi-a \xmark \\
histopatologia & his*to*pa*to*lo*gi*a \cmark & his*to*pa*to*lo*gi-a \xmark \\
histoquímica & his*to-quí*mi*ca \xmark & his*to*quí*mi*ca \cmark \\
história & his*tó*ri.a \xmark & his*tó*ria \cmark \\
historiografia & his*to*ri*o*gra*fi*a \cmark & his*to*ri*o*gra*fi-a \xmark \\
historiógrafo & his*to*ri-ó*gra*fo \xmark & his*to*ri*ó*gra*fo \cmark \\
historiologia & his*to*ri*o*lo*gi*a \cmark & his*to*ri*o*lo*gi-a \xmark \\
histrião & his*tri-ão \xmark & his*tri*ão \cmark \\
hodômetro & ho-dô*me*tro \xmark & ho*dô*me*tro \cmark \\
hólmio & hól*mi.o \xmark & hól*mio \cmark \\
holografia & ho*lo*gra*fi*a \cmark & ho*lo*gra*fi-a \xmark \\
holonomia & ho*lo*no*mi*a \cmark & ho*lo*no*mi-a \xmark \\
holonômico & ho*lo-nô*mi*co \xmark & ho*lo*nô*mi*co \cmark \\
homeopatia & ho*me*o*pa*ti*a \cmark & ho*me*o*pa*ti-a \xmark \\
homeostasia & ho*me*os*ta*si*a \cmark & ho*me*os*ta*si-a \xmark \\
homeotermia & ho*me*o*ter*mi*a \cmark & ho*me*o*ter*mi-a \xmark \\
homicídio & ho*mi*cí*di.o \xmark & ho*mi*cí*dio \cmark \\
homilia & ho*mi*li*a \cmark & ho*mi*li-a \xmark \\
hominídeo & ho*mi*ní*de.o \xmark & ho*mi*ní*deo \cmark \\
homizio & ho*mi*zi*o \cmark & ho*mi*zi-o \xmark \\
homofilia & ho*mo*fi*li*a \cmark & ho*mo*fi*li-a \xmark \\
homofobia & ho*mo*fo*bi*a \cmark & ho*mo*fo*bi-a \xmark \\
homofonia & ho*mo*fo*ni*a \cmark & ho*mo*fo*ni-a \xmark \\
homofônico & ho*mo-fô*ni*co \xmark & ho*mo*fô*ni*co \cmark \\
homogêneo & ho*mo*gê*ne.o \xmark & ho*mo*gê*neo \cmark \\
homogenia & ho*mo*ge*ni*a \cmark & ho*mo*ge*ni-a \xmark \\
homografia & ho*mo*gra*fi*a \cmark & ho*mo*gra*fi-a \xmark \\
homologatório & ho*mo*lo*ga*tó*ri.o \xmark & ho*mo*lo*ga*tó*rio \cmark \\
homologia & ho*mo*lo*gi*a \cmark & ho*mo*lo*gi-a \xmark \\
homonímia & ho*mo*ní*mi.a \xmark & ho*mo*ní*mia \cmark \\
homônimo & ho-mô*ni*mo \xmark & ho*mô*ni*mo \cmark \\
homoplasia & ho*mo*pla*si*a \cmark & ho*mo*pla*si-a \xmark \\
homotetia & ho*mo*te*ti*a \cmark & ho*mo*te*ti-a \xmark \\
honorário & ho*no*rá*ri.o \xmark & ho*no*rá*rio \cmark \\
honraria & hon*ra*ri*a \cmark & hon*ra*ri-a \xmark \\
horário & ho*rá*ri.o \xmark & ho*rá*rio \cmark \\
hormônio & hor-mô*ni.o \xmark & hor*mô*nio \cmark \\
hormonoterapia & hor*mo*no*te*ra*pi*a \cmark & hor*mo*no*te*ra*pi-a \xmark \\
hortênsia & hor*tên*si.a \xmark & hor*tên*sia \cmark \\
hospedaria & hos*pe*da*ri*a \cmark & hos*pe*da*ri-a \xmark \\
hospício & hos*pí*ci.o \xmark & hos*pí*cio \cmark \\
hospitalário & hos*pi*ta*lá*ri.o \xmark & hos*pi*ta*lá*rio \cmark \\
hóstia & hós*ti.a \xmark & hós*tia \cmark \\
hotelaria & ho*te*la*ri*a \cmark & ho*te*la*ri-a \xmark \\
humanitário & hu*ma*ni*tá*ri.o \xmark & hu*ma*ni*tá*rio \cmark \\
iá & i-á \xmark & i*á \cmark \\
ião & i-ão \xmark & i*ão \cmark \\
iatrogenia & i*a*tro*ge*ni*a \cmark & i*a*tro*ge*ni-a \xmark \\
iatroquímica & i*a*tro-quí*mi*ca \xmark & i*a*tro*quí*mi*ca \cmark \\
ibijaú & i*bi*ja-ú \xmark & i*bi*ja*ú \cmark \\
icônico & i-cô*ni*co \xmark & i*cô*ni*co \cmark \\
iconoclastia & i*co*no*clas*ti*a \cmark & i*co*no*clas*ti-a \xmark \\
iconografia & i*co*no*gra*fi*a \cmark & i*co*no*gra*fi-a \xmark \\
iconolatria & i*co*no*la*tri*a \cmark & i*co*no*la*tri-a \xmark \\
iconologia & i*co*no*lo*gi*a \cmark & i*co*no*lo*gi-a \xmark \\
icterícia & ic*te*rí*ci.a \xmark & ic*te*rí*cia \cmark \\
ictiologia & ic*ti*o*lo*gi*a \cmark & ic*ti*o*lo*gi-a \xmark \\
ictiólogo & ic*ti-ó*lo*go \xmark & ic*ti*ó*lo*go \cmark \\
ideário & i*de-á*ri.o \xmark & i*de*á*rio \cmark \\
ideia & i*dei*a \cmark & i*dei-a \xmark \\
identitário & i*den*ti*tá*ri.o \xmark & i*den*ti*tá*rio \cmark \\
ideografia & i*de*o*gra*fi*a \cmark & i*de*o*gra*fi-a \xmark \\
ideologia & i*de*o*lo*gi*a \cmark & i*de*o*lo*gi-a \xmark \\
ideólogo & i*de-ó*lo*go \xmark & i*de*ó*lo*go \cmark \\
idílio & i*dí*li.o \xmark & i*dí*lio \cmark \\
idiossincrasia & i*di*os*sin*cra*si*a \cmark & i*di*os*sin*cra*si-a \xmark \\
idiotia & i*di*o*ti*a \cmark & i*di*o*ti-a \xmark \\
idiótico & i*di-ó*ti*co \xmark & i*di*ó*ti*co \cmark \\
idolatria & i*do*la*tri*a \cmark & i*do*la*tri-a \xmark \\
idôneo & i-dô*ne.o \xmark & i*dô*neo \cmark \\
ígneo & íg*ne.o \xmark & íg*neo \cmark \\
ignomínia & ig*no*mí*ni*a \cmark & ig*no*mí*ni-a \xmark \\
ignorância & ig*no*rân*ci.a \xmark & ig*no*rân*cia \cmark \\
igualitário & i*gua*li*tá*ri.o \xmark & i*gua*li*tá*rio \cmark \\
iguaria & i*gua*ri*a \cmark & i*gua*ri-a \xmark \\
iídiche & i-í*di*che \xmark & i*í*di*che \cmark \\
ileíte & i*le-í*te \xmark & i*le*í*te \cmark \\
íleo & í*le.o \xmark & í*leo \cmark \\
ileostomia & i*le*os*to*mi*a \cmark & i*le*os*to*mi-a \xmark \\
ilhoa & i*lho*a \cmark & i*lho-a \xmark \\
ilíaco & i*lí-a*co \xmark & i*lí*a*co \cmark \\
ilíada & i*lí-a*da \xmark & i*lí*a*da \cmark \\
ilício & i*lí*ci.o \xmark & i*lí*cio \cmark \\
ílio & í*li.o \xmark & í*lio \cmark \\
ilírio & i*lí*ri.o \xmark & i*lí*rio \cmark \\
iliteracia & i*li*te*ra*ci*a \cmark & i*li*te*ra*ci-a \xmark \\
ilocutório & i*lo*cu*tó*ri.o \xmark & i*lo*cu*tó*rio \cmark \\
iluminância & i*lu*mi*nân*ci.a \xmark & i*lu*mi*nân*cia \cmark \\
ilusório & i*lu*só*ri.o \xmark & i*lu*só*rio \cmark \\
imaginária & i*ma*gi*ná*ri.a \xmark & i*ma*gi*ná*ria \cmark \\
imaginário & i*ma*gi*ná*ri.o \xmark & i*ma*gi*ná*rio \cmark \\
imbróglio & im*bró*gli.o \xmark & im*bró*glio \cmark \\
imbuído & im*bu-í*do \xmark & im*bu*í*do \cmark \\
imbuir & im*bu-ir \xmark & im*bu*ir \cmark \\
imigratório & i*mi*gra*tó*ri.o \xmark & i*mi*gra*tó*rio \cmark \\
iminência & i*mi*nên*ci.a \xmark & i*mi*nên*cia \cmark \\
imiscuir & i*mis*cu-ir \xmark & i*mis*cu*ir \cmark \\
imobiliária & i*mo*bi*li-á*ri.a \xmark & i*mo*bi*li*á*ria \cmark \\
imobiliário & i*mo*bi*li-á*ri.o \xmark & i*mo*bi*li*á*rio \cmark \\
imodéstia & i*mo*dés*ti.a \xmark & i*mo*dés*tia \cmark \\
impaciência & im*pa*ci*ên*ci.a \xmark & im*pa*ci*ên*cia \cmark \\
impatriótico & im*pa*tri-ó*ti*co \xmark & im*pa*tri*ó*ti*co \cmark \\
impedância & im*pe*dân*ci.a \xmark & im*pe*dân*cia \cmark \\
impenitência & im*pe*ni*tên*ci.a \xmark & im*pe*ni*tên*cia \cmark \\
imperdoável & im*per*do-á*vel \xmark & im*per*do*á*vel \cmark \\
imperícia & im*pe*rí*ci.a \xmark & im*pe*rí*cia \cmark \\
império & im*pé*ri.o \xmark & im*pé*rio \cmark \\
impermanência & im*per*ma*nên*ci.a \xmark & im*per*ma*nên*cia \cmark \\
impermeável & im*per*me-á*vel \xmark & im*per*me*á*vel \cmark \\
impertinência & im*per*ti*nên*ci.a \xmark & im*per*ti*nên*cia \cmark \\
ímpio & ím*pi.o \xmark & ím*pio \cmark \\
implantodontia & im*plan*to*don*ti*a \cmark & im*plan*to*don*ti-a \xmark \\
implicância & im*pli*cân*ci.a \xmark & im*pli*cân*cia \cmark \\
implúvio & im*plú*vi.o \xmark & im*plú*vio \cmark \\
imponência & im*po*nên*ci.a \xmark & im*po*nên*cia \cmark \\
importância & im*por*tân*ci.a \xmark & im*por*tân*cia \cmark \\
impotência & im*po*tên*ci.a \xmark & im*po*tên*cia \cmark \\
imprevidência & im*pre*vi*dên*ci.a \xmark & im*pre*vi*dên*cia \cmark \\
improcedência & im*pro*ce*dên*ci.a \xmark & im*pro*ce*dên*cia \cmark \\
impronunciável & im*pro*nun*ci-á*vel \xmark & im*pro*nun*ci*á*vel \cmark \\
impropério & im*pro*pé*ri.o \xmark & im*pro*pé*rio \cmark \\
impróprio & im*pró*pri.o \xmark & im*pró*prio \cmark \\
imprudência & im*pru*dên*ci.a \xmark & im*pru*dên*cia \cmark \\
imunitário & i*mu*ni*tá*ri.o \xmark & i*mu*ni*tá*rio \cmark \\
imunodeficiência & i*mu*no*de*fi*ci*ên*ci.a \xmark & i*mu*no*de*fi*ci*ên*cia \cmark \\
imunologia & i*mu*no*lo*gi*a \cmark & i*mu*no*lo*gi-a \xmark \\
imunopatologia & i*mu*no*pa*to*lo*gi*a \cmark & i*mu*no*pa*to*lo*gi-a \xmark \\
imunoterapia & i*mu*no*te*ra*pi*a \cmark & i*mu*no*te*ra*pi-a \xmark \\
inácia & i*ná*ci.a \xmark & i*ná*cia \cmark \\
inadiável & i*na*di-á*vel \xmark & i*na*di*á*vel \cmark \\
inadimplência & i*na*dim*plên*ci.a \xmark & i*na*dim*plên*cia \cmark \\
inapetência & i*na*pe*tên*ci.a \xmark & i*na*pe*tên*cia \cmark \\
incandescência & in*can*des*cên*ci.a \xmark & in*can*des*cên*cia \cmark \\
incendiário & in*cen*di-á*ri.o \xmark & in*cen*di*á*rio \cmark \\
incêndio & in*cên*di.o \xmark & in*cên*dio \cmark \\
incensário & in*cen*sá*ri.o \xmark & in*cen*sá*rio \cmark \\
incidência & in*ci*dên*ci.a \xmark & in*ci*dên*cia \cmark \\
incipiência & in*ci*pi*ên*ci.a \xmark & in*ci*pi*ên*cia \cmark \\
inclemência & in*cle*mên*ci.a \xmark & in*cle*mên*cia \cmark \\
inclinômetro & in*cli-nô*me*tro \xmark & in*cli*nô*me*tro \cmark \\
incluir & in*clu-ir \xmark & in*clu*ir \cmark \\
incoerência & in*co*e*rên*ci.a \xmark & in*co*e*rên*cia \cmark \\
incômodo & in-cô*mo*do \xmark & in*cô*mo*do \cmark \\
incompetência & in*com*pe*tên*ci.a \xmark & in*com*pe*tên*cia \cmark \\
inconciliável & in*con*ci*li-á*vel \xmark & in*con*ci*li*á*vel \cmark \\
inconfidência & in*con*fi*dên*ci.a \xmark & in*con*fi*dên*cia \cmark \\
incongruência & in*con*gru-ên*ci.a \xmark & in*con*gru*ên*cia \cmark \\
inconsciência & in*cons*ci*ên*ci.a \xmark & in*cons*ci*ên*cia \cmark \\
inconsequência & in*con*se-quên*ci.a \xmark & in*con*se*quên*cia \cmark \\
inconsistência & in*con*sis*tên*ci.a \xmark & in*con*sis*tên*cia \cmark \\
inconspícuo & in*cons*pí*cu.o \xmark & in*cons*pí*cuo \cmark \\
inconstância & in*cons*tân*ci.a \xmark & in*cons*tân*cia \cmark \\
incontinência & in*con*ti*nên*ci.a \xmark & in*con*ti*nên*cia \cmark \\
inconveniência & in*con*ve*ni*ên*ci.a \xmark & in*con*ve*ni*ên*cia \cmark \\
incorpóreo & in*cor*pó*re.o \xmark & in*cor*pó*reo \cmark \\
incumbência & in*cum*bên*ci.a \xmark & in*cum*bên*cia \cmark \\
incúria & in*cú*ri.a \xmark & in*cú*ria \cmark \\
indaiá & in*dai-á \xmark & in*dai*á \cmark \\
indecência & in*de*cên*ci.a \xmark & in*de*cên*cia \cmark \\
indeiscente & in*de-is*cen*te \xmark & in*de-is*cen*te \xmark \\
independência & in*de*pen*dên*ci.a \xmark & in*de*pen*dên*cia \cmark \\
indício & in*dí*ci.o \xmark & in*dí*cio \cmark \\
indigência & in*di*gên*ci.a \xmark & in*di*gên*cia \cmark \\
índio & ín*di.o \xmark & ín*dio \cmark \\
indissociável & in*dis*so*ci-á*vel \xmark & in*dis*so*ci*á*vel \cmark \\
indolência & in*do*lên*ci.a \xmark & in*do*lên*cia \cmark \\
indologia & in*do*lo*gi*a \cmark & in*do*lo*gi-a \xmark \\
indonésio & in*do*né*si.o \xmark & in*do*né*sio \cmark \\
indulgência & in*dul*gên*ci.a \xmark & in*dul*gên*cia \cmark \\
indumentária & in*du*men*tá*ri.a \xmark & in*du*men*tá*ria \cmark \\
indúsio & in*dú*si.o \xmark & in*dú*sio \cmark \\
indústria & in*dús*tri.a \xmark & in*dús*tria \cmark \\
indutância & in*du*tân*ci.a \xmark & in*du*tân*cia \cmark \\
inédia & i*né*di.a \xmark & i*né*dia \cmark \\
ineficácia & i*ne*fi*cá*ci.a \xmark & i*ne*fi*cá*cia \cmark \\
ineficiência & i*ne*fi*ci*ên*ci.a \xmark & i*ne*fi*ci*ên*cia \cmark \\
inegociável & i*ne*go*ci-á*vel \xmark & i*ne*go*ci*á*vel \cmark \\
inépcia & i*nép*ci.a \xmark & i*nép*cia \cmark \\
inequívoco & i*ne-quí*vo*co \xmark & i*ne*quí*vo*co \cmark \\
inércia & i*nér*ci.a \xmark & i*nér*cia \cmark \\
inerência & i*ne*rên*ci.a \xmark & i*ne*rên*cia \cmark \\
inerrância & i*ner*rân*ci.a \xmark & i*ner*rân*cia \cmark \\
inexequível & i*ne*xe-quí*vel \xmark & i*ne*xe*quí*vel \cmark \\
inexistência & i*ne*xis*tên*ci.a \xmark & i*ne*xis*tên*cia \cmark \\
inexperiência & i*nex*pe*ri*ên*ci.a \xmark & i*nex*pe*ri*ên*cia \cmark \\
infâmia & in*fâ*mi.a \xmark & in*fâ*mia \cmark \\
infância & in*fân*ci.a \xmark & in*fân*cia \cmark \\
infantaria & in*fan*ta*ri*a \cmark & in*fan*ta*ri-a \xmark \\
infanteria & in*fan*te*ri*a \cmark & in*fan*te*ri-a \xmark \\
infanticídio & in*fan*ti*cí*di.o \xmark & in*fan*ti*cí*dio \cmark \\
infectologia & in*fec*to*lo*gi*a \cmark & in*fec*to*lo*gi-a \xmark \\
inferência & in*fe*rên*ci.a \xmark & in*fe*rên*cia \cmark \\
inflacionário & in*fla*ci*o*ná*ri.o \xmark & in*fla*ci*o*ná*rio \cmark \\
inflamatório & in*fla*ma*tó*ri.o \xmark & in*fla*ma*tó*rio \cmark \\
inflorescência & in*flo*res*cên*ci.a \xmark & in*flo*res*cên*cia \cmark \\
influência & in*flu-ên*ci.a \xmark & in*flu*ên*cia \cmark \\
influenciável & in*flu*en*ci-á*vel \xmark & in*flu*en*ci*á*vel \cmark \\
influído & in*flu-í*do \xmark & in*flu*í*do \cmark \\
influir & in*flu-ir \xmark & in*flu*ir \cmark \\
infografia & in*fo*gra*fi*a \cmark & in*fo*gra*fi-a \xmark \\
infortúnio & in*for*tú*ni.o \xmark & in*for*tú*nio \cmark \\
infrutescência & in*fru*tes*cên*ci.a \xmark & in*fru*tes*cên*cia \cmark \\
ingenuidade & in*ge*nu-i*da*de \xmark & in*ge*nu*i*da*de \cmark \\
ingênuo & in*gê*nu.o \xmark & in*gê*nuo \cmark \\
ingerência & in*ge*rên*ci.a \xmark & in*ge*rên*cia \cmark \\
inglório & in*gló*ri.o \xmark & in*gló*rio \cmark \\
inibitória & i*ni*bi*tó*ri.a \xmark & i*ni*bi*tó*ria \cmark \\
inibitório & i*ni*bi*tó*ri.o \xmark & i*ni*bi*tó*rio \cmark \\
iniciático & i*ni*ci-á*ti*co \xmark & i*ni*ci*á*ti*co \cmark \\
início & i*ní*ci.o \xmark & i*ní*cio \cmark \\
injúria & in*jú*ri.a \xmark & in*jú*ria \cmark \\
inobservância & i*nob*ser*vân*ci.a \xmark & i*nob*ser*vân*cia \cmark \\
inocência & i*no*cên*ci.a \xmark & i*no*cên*cia \cmark \\
inócuo & i*nó*cu.o \xmark & i*nó*cuo \cmark \\
inoperância & i*no*pe*rân*ci.a \xmark & i*no*pe*rân*cia \cmark \\
inópia & i*nó*pi.a \xmark & i*nó*pia \cmark \\
inquisitório & in*qui*si*tó*ri.o \xmark & in*qui*si*tó*rio \cmark \\
insaciável & in*sa*ci-á*vel \xmark & in*sa*ci*á*vel \cmark \\
insatisfatório & in*sa*tis*fa*tó*ri.o \xmark & in*sa*tis*fa*tó*rio \cmark \\
insídia & in*sí*di*a \cmark & in*sí*di-a \xmark \\
insígnia & in*síg*ni.a \xmark & in*síg*nia \cmark \\
insignificância & in*sig*ni*fi*cân*ci.a \xmark & in*sig*ni*fi*cân*cia \cmark \\
insistência & in*sis*tên*ci.a \xmark & in*sis*tên*cia \cmark \\
insolência & in*so*lên*ci.a \xmark & in*so*lên*cia \cmark \\
insolvência & in*sol*vên*ci.a \xmark & in*sol*vên*cia \cmark \\
insônia & in-sô*ni.a \xmark & in*sô*nia \cmark \\
inspiratório & ins*pi*ra*tó*ri*o \cmark & ins*pi*ra*tó*ri-o \xmark \\
instância & ins*tân*ci.a \xmark & ins*tân*cia \cmark \\
instantâneo & ins*tan*tâ*ne*o \cmark & ins*tan*tâ*ne-o \xmark \\
instituição & ins*ti*tu-i*ção \xmark & ins*ti*tu*i*ção \cmark \\
instituído & ins*ti*tu-í*do \xmark & ins*ti*tu*í*do \cmark \\
instituidor & ins*ti*tu-i*dor \xmark & ins*ti*tu-i*dor \xmark \\
instituir & ins*ti*tu-ir \xmark & ins*ti*tu*ir \cmark \\
instruído & ins*tru-í*do \xmark & ins*tru*í*do \cmark \\
instruir & ins*tru-ir \xmark & ins*tru*ir \cmark \\
insubsistência & in*sub*sis*tên*ci.a \xmark & in*sub*sis*tên*cia \cmark \\
insubstituível & in*subs*ti*tu-í*vel \xmark & in*subs*ti*tu*í*vel \cmark \\
insuficiência & in*su*fi*ci*ên*ci.a \xmark & in*su*fi*ci*ên*cia \cmark \\
insulinoterapia & in*su*li*no*te*ra*pi*a \cmark & in*su*li*no*te*ra*pi-a \xmark \\
inteligência & in*te*li*gên*ci.a \xmark & in*te*li*gên*cia \cmark \\
intempérie & in*tem*pé*ri.e \xmark & in*tem*pé*rie \cmark \\
intendência & in*ten*dên*ci.a \xmark & in*ten*dên*cia \cmark \\
interbancário & in*ter*ban*cá*ri.o \xmark & in*ter*ban*cá*rio \cmark \\
intercambiável & in*ter*cam*bi-á*vel \xmark & in*ter*cam*bi*á*vel \cmark \\
intercâmbio & in*ter*câm*bi.o \xmark & in*ter*câm*bio \cmark \\
intercorrência & in*ter*cor*rên*ci.a \xmark & in*ter*cor*rên*cia \cmark \\
interdependência & in*ter*de*pen*dên*ci.a \xmark & in*ter*de*pen*dên*cia \cmark \\
interferência & in*ter*fe*rên*ci.a \xmark & in*ter*fe*rên*cia \cmark \\
interferometria & in*ter*fe*ro*me*tri*a \cmark & in*ter*fe*ro*me*tri-a \xmark \\
interferômetro & in*ter*fe-rô*me*tro \xmark & in*ter*fe*rô*me*tro \cmark \\
interflúvio & in*ter*flú*vi.o \xmark & in*ter*flú*vio \cmark \\
interlocutória & in*ter*lo*cu*tó*ri.a \xmark & in*ter*lo*cu*tó*ria \cmark \\
interlúdio & in*ter*lú*di.o \xmark & in*ter*lú*dio \cmark \\
intermediário & in*ter*me*di-á*ri.o \xmark & in*ter*me*di*á*rio \cmark \\
intermédio & in*ter*mé*di.o \xmark & in*ter*mé*dio \cmark \\
intermitência & in*ter*mi*tên*ci.a \xmark & in*ter*mi*tên*cia \cmark \\
internúncio & in*ter*nún*ci.o \xmark & in*ter*nún*cio \cmark \\
interoceânico & in*te*ro*ce-â*ni*co \xmark & in*te*ro*ce*â*ni*co \cmark \\
interplanetário & in*ter*pla*ne*tá*ri.o \xmark & in*ter*pla*ne*tá*rio \cmark \\
interrogatório & in*ter*ro*ga*tó*ri.o \xmark & in*ter*ro*ga*tó*rio \cmark \\
interstício & in*ters*tí*ci.o \xmark & in*ters*tí*cio \cmark \\
interuniversitário & in*te*ru*ni*ver*si*tá*ri.o \xmark & in*te*ru*ni*ver*si*tá*rio \cmark \\
interventoria & in*ter*ven*to*ri*a \cmark & in*ter*ven*to*ri-a \xmark \\
intolerância & in*to*le*rân*ci.a \xmark & in*to*le*rân*cia \cmark \\
intransigência & in*tran*si*gên*ci.a \xmark & in*tran*si*gên*cia \cmark \\
introdutório & in*tro*du*tó*ri.o \xmark & in*tro*du*tó*rio \cmark \\
intuição & in*tu-i*ção \xmark & in*tu*i*ção \cmark \\
intuir & in*tu-ir \xmark & in*tu*ir \cmark \\
intumescência & in*tu*mes*cên*ci.a \xmark & in*tu*mes*cên*cia \cmark \\
inúbia & i*nú*bi.a \xmark & i*nú*bia \cmark \\
invariância & in*va*ri*ân*ci.a \xmark & in*va*ri*ân*cia \cmark \\
invariável & in*va*ri-á*vel \xmark & in*va*ri*á*vel \cmark \\
inventário & in*ven*tá*ri.o \xmark & in*ven*tá*rio \cmark \\
inviável & in*vi-á*vel \xmark & in*vi*á*vel \cmark \\
involuntário & in*vo*lun*tá*ri.o \xmark & in*vo*lun*tá*rio \cmark \\
iódico & i-ó*di*co \xmark & i*ó*di*co \cmark \\
ioiô & i.o-i.ô \xmark & i.o-i.ô \xmark \\
íon & í-on \xmark & í*on \cmark \\
iraúna & i*ra-ú*na \xmark & i*ra*ú*na \cmark \\
iridescência & i*ri*des*cên*ci.a \xmark & i*ri*des*cên*cia \cmark \\
irídio & i*rí*di.o \xmark & i*rí*dio \cmark \\
ironia & i*ro*ni*a \cmark & i*ro*ni-a \xmark \\
irônico & i-rô*ni*co \xmark & i*rô*ni*co \cmark \\
iroquês & i*ro-quês \xmark & i*ro*quês \cmark \\
irreconciliável & ir*re*con*ci*li-á*vel \xmark & ir*re*con*ci*li*á*vel \cmark \\
irrefreável & ir*re*fre-á*vel \xmark & ir*re*fre*á*vel \cmark \\
irrelevância & ir*re*le*vân*ci.a \xmark & ir*re*le*vân*cia \cmark \\
irreligião & ir*re*li*gi-ão \xmark & ir*re*li*gi*ão \cmark \\
irremediável & ir*re*me*di-á*vel \xmark & ir*re*me*di*á*vel \cmark \\
irrenunciável & ir*re*nun*ci-á*vel \xmark & ir*re*nun*ci*á*vel \cmark \\
irretorquível & ir*re*tor-quí*vel \xmark & ir*re*tor*quí*vel \cmark \\
irreverência & ir*re*ve*rên*ci.a \xmark & ir*re*ve*rên*cia \cmark \\
irrisório & ir*ri*só*ri.o \xmark & ir*ri*só*rio \cmark \\
isogamia & i*so*ga*mi*a \cmark & i*so*ga*mi-a \xmark \\
isomeria & i*so*me*ri*a \cmark & i*so*me*ri-a \xmark \\
isômero & i-sô*me*ro \xmark & i*sô*me*ro \cmark \\
isometria & i*so*me*tri*a \cmark & i*so*me*tri-a \xmark \\
isomorfia & i*so*mor*fi*a \cmark & i*so*mor*fi-a \xmark \\
isonomia & i*so*no*mi*a \cmark & i*so*no*mi-a \xmark \\
isostasia & i*sos*ta*si*a \cmark & i*sos*ta*si-a \xmark \\
isotônico & i*so-tô*ni*co \xmark & i*so*tô*ni*co \cmark \\
isotopia & i*so*to*pi*a \cmark & i*so*to*pi-a \xmark \\
isotropia & i*so*tro*pi*a \cmark & i*so*tro*pi-a \xmark \\
isquemia & is*que*mi*a \cmark & is*que*mi-a \xmark \\
isquêmico & is-quê*mi*co \xmark & is*quê*mi*co \cmark \\
isquiático & is*qui-á*ti*co \xmark & is*qui*á*ti*co \cmark \\
ísquio & ís*qui.o \xmark & ís*quio \cmark \\
itaimbé & i*ta-im*bé \xmark & i*ta-im*bé \xmark \\
itapiúna & i*ta*pi-ú*na \xmark & i*ta*pi*ú*na \cmark \\
itapuá & i*ta*pu-á \xmark & i*ta*pu*á \cmark \\
itapuã & i*ta*pu-ã \xmark & i*ta*pu*ã \cmark \\
itaúba & i*ta-ú*ba \xmark & i*ta*ú*ba \cmark \\
itérbio & i*tér*bi.o \xmark & i*tér*bio \cmark \\
itinerário & i*ti*ne*rá*ri.o \xmark & i*ti*ne*rá*rio \cmark \\
ítrio & í*tri.o \xmark & í*trio \cmark \\
jabô & ja-bô \xmark & ja*bô \cmark \\
jacobeia & ja*co*bei*a \cmark & ja*co*bei-a \xmark \\
jactância & jac*tân*ci.a \xmark & jac*tân*cia \cmark \\
jacuí & ja*cu-í \xmark & ja*cu*í \cmark \\
jaculatória & ja*cu*la*tó*ri.a \xmark & ja*cu*la*tó*ria \cmark \\
jambalaia & jam*ba*lai*a \cmark & jam*ba*lai-a \xmark \\
janaúba & ja*na-ú*ba \xmark & ja*na*ú*ba \cmark \\
jandaia & jan*dai*a \cmark & jan*dai-a \xmark \\
jandaíra & jan*da-í*ra \xmark & jan*da*í*ra \cmark \\
januária & ja*nu-á*ri.a \xmark & ja*nu*á*ria \cmark \\
jaó & ja-ó \xmark & ja*ó \cmark \\
japuí & ja*pu-í \xmark & ja*pu*í \cmark \\
jaracatiá & ja*ra*ca*ti-á \xmark & ja*ra*ca*ti*á \cmark \\
jaramataia & ja*ra*ma*tai*a \cmark & ja*ra*ma*tai-a \xmark \\
jataí & ja*ta-í \xmark & ja*ta*í \cmark \\
jataúba & ja*ta-ú*ba \xmark & ja*ta*ú*ba \cmark \\
jauá & jau-á \xmark & jau*á \cmark \\
jaú & ja-ú \xmark & ja*ú \cmark \\
jejunostomia & je*ju*nos*to*mi*a \cmark & je*ju*nos*to*mi-a \xmark \\
jesuíta & je*su-í*ta \xmark & je*su*í*ta \cmark \\
jesuítico & je*su-í*ti*co \xmark & je*su*í*ti*co \cmark \\
jesuitismo & je*su-i*tis*mo \xmark & je*su*i*tis*mo \cmark \\
jia & ji*a \cmark & ji-a \xmark \\
jiboia & ji*boi*a \cmark & ji*boi-a \xmark \\
jihadista & ji-ha*dis*ta \xmark & ji-ha*dis*ta \xmark \\
jiquitaia & ji*qui*tai*a \cmark & ji*qui*tai-a \xmark \\
joalharia & jo*a*lha*ri*a \cmark & jo*a*lha*ri-a \xmark \\
joalheria & jo*a*lhe*ri*a \cmark & jo*a*lhe*ri-a \xmark \\
joia & joi*a \cmark & joi-a \xmark \\
joinvilense & jo-in*vi*len*se \xmark & jo*in*vi*len*se \cmark \\
joio & joi*o \cmark & joi-o \xmark \\
juá & ju-á \xmark & ju*á \cmark \\
judaísmo & ju*da-ís*mo \xmark & ju*da*ís*mo \cmark \\
judia & ju*di*a \cmark & ju*di-a \xmark \\
judiaria & ju*di*a*ri*a \cmark & ju*di*a*ri-a \xmark \\
judiciário & ju*di*ci-á*ri.o \xmark & ju*di*ci*á*rio \cmark \\
judô & ju-dô \xmark & ju*dô \cmark \\
juizado & ju-i*za*do \xmark & ju*i*za*do \cmark \\
juíza & ju-í*za \xmark & ju*í*za \cmark \\
juiz & ju-iz \xmark & ju*iz \cmark \\
juízo & ju-í*zo \xmark & ju*í*zo \cmark \\
júlio & jú*li.o \xmark & jú*lio \cmark \\
jundiá & jun*di-á \xmark & jun*di*á \cmark \\
juó & ju-ó \xmark & ju-ó \xmark \\
jupiá & ju*pi-á \xmark & ju*pi*á \cmark \\
juquiá & ju*qui-á \xmark & ju*qui*á \cmark \\
jurisprudência & ju*ris*pru*dên*ci.a \xmark & ju*ris*pru*dên*cia \cmark \\
jutaí & ju*ta-í \xmark & ju*ta*í \cmark \\
juvenília & ju*ve*ní*li.a \xmark & ju*ve*ní*lia \cmark \\
kuwaitiano & ku-wai*ti*a*no \xmark & ku-wai*ti*a*no \xmark \\
lábia & lá*bi.a \xmark & lá*bia \cmark \\
lábio & lá*bi.o \xmark & lá*bio \cmark \\
laboratório & la*bo*ra*tó*ri.o \xmark & la*bo*ra*tó*rio \cmark \\
laborterapia & la*bor*te*ra*pi*a \cmark & la*bor*te*ra*pi-a \xmark \\
lacaio & la*cai*o \cmark & la*cai-o \xmark \\
laçaria & la*ça*ri*a \cmark & la*ça*ri-a \xmark \\
lacínia & la*cí*ni.a \xmark & la*cí*nia \cmark \\
lacônico & la-cô*ni*co \xmark & la*cô*ni*co \cmark \\
lacraia & la*crai*a \cmark & la*crai-a \xmark \\
lacrimatório & la*cri*ma*tó*ri.o \xmark & la*cri*ma*tó*rio \cmark \\
lacrimogêneo & la*cri*mo*gê*ne.o \xmark & la*cri*mo*gê*neo \cmark \\
lactário & lac*tá*ri.o \xmark & lac*tá*rio \cmark \\
lácteo & lác*te.o \xmark & lác*teo \cmark \\
lactucário & lac*tu*cá*ri.o \xmark & lac*tu*cá*rio \cmark \\
ladainha & la*da-i*nha \xmark & la*da*i*nha \cmark \\
ladraria & la*dra*ri*a \cmark & la*dra*ri-a \xmark \\
lagoa & la*go*a \cmark & la*go-a \xmark \\
laia & lai*a \cmark & lai-a \xmark \\
lamaísmo & la*ma-ís*mo \xmark & la*ma*ís*mo \cmark \\
lamaísta & la*ma-ís*ta \xmark & la*ma*ís*ta \cmark \\
lâmia & lâ*mi.a \xmark & lâ*mia \cmark \\
lampadário & lam*pa*dá*ri.o \xmark & lam*pa*dá*rio \cmark \\
lampião & lam*pi-ão \xmark & lam*pi*ão \cmark \\
lampreia & lam*prei*a \cmark & lam*prei-a \xmark \\
lamúria & la*mú*ri.a \xmark & la*mú*ria \cmark \\
lanifício & la*ni*fí*ci.o \xmark & la*ni*fí*cio \cmark \\
lantânio & lan*tâ*ni.o \xmark & lan*tâ*nio \cmark \\
laparoscopia & la*pa*ros*co*pi*a \cmark & la*pa*ros*co*pi-a \xmark \\
laparotomia & la*pa*ro*to*mi*a \cmark & la*pa*ro*to*mi-a \xmark \\
lapidaria & la*pi*da*ri*a \cmark & la*pi*da*ri-a \xmark \\
lapidário & la*pi*dá*ri.o \xmark & la*pi*dá*rio \cmark \\
laquê & la-quê \xmark & la*quê \cmark \\
larápio & la*rá*pi.o \xmark & la*rá*pio \cmark \\
larário & la*rá*ri.o \xmark & la*rá*rio \cmark \\
larício & la*rí*ci.o \xmark & la*rí*cio \cmark \\
laringectomia & la*rin*gec*to*mi*a \cmark & la*rin*gec*to*mi-a \xmark \\
laríngeo & la*rín*ge.o \xmark & la*rín*geo \cmark \\
lascívia & las*cí*vi.a \xmark & las*cí*via \cmark \\
lataria & la*ta*ri*a \cmark & la*ta*ri-a \xmark \\
latência & la*tên*ci.a \xmark & la*tên*cia \cmark \\
laticínio & la*ti*cí*ni.o \xmark & la*ti*cí*nio \cmark \\
laticlávio & la*ti*clá*vi.o \xmark & la*ti*clá*vio \cmark \\
latifundiário & la*ti*fun*di-á*ri.o \xmark & la*ti*fun*di*á*rio \cmark \\
latifúndio & la*ti*fún*di.o \xmark & la*ti*fún*dio \cmark \\
latinório & la*ti*nó*ri.o \xmark & la*ti*nó*rio \cmark \\
latoaria & la*to*a*ri*a \cmark & la*to*a*ri-a \xmark \\
latrocínio & la*tro*cí*ni.o \xmark & la*tro*cí*nio \cmark \\
laudatório & lau*da*tó*ri.o \xmark & lau*da*tó*rio \cmark \\
láurea & láu*re.a \xmark & láu*rea \cmark \\
laurêncio & lau*rên*ci.o \xmark & lau*rên*cio \cmark \\
lavanderia & la*van*de*ri*a \cmark & la*van*de*ri-a \xmark \\
lavatório & la*va*tó*ri.o \xmark & la*va*tó*rio \cmark \\
lavradio & la*vra*di*o \cmark & la*vra*di-o \xmark \\
leão & le-ão \xmark & le*ão \cmark \\
lecanomancia & le*ca*no*man*ci*a \cmark & le*ca*no*man*ci-a \xmark \\
legatário & le*ga*tá*ri.o \xmark & le*ga*tá*rio \cmark \\
legendário & le*gen*dá*ri.o \xmark & le*gen*dá*rio \cmark \\
legião & le*gi-ão \xmark & le*gi*ão \cmark \\
legionário & le*gi*o*ná*ri.o \xmark & le*gi*o*ná*rio \cmark \\
leguleio & le*gu*lei*o \cmark & le*gu*lei-o \xmark \\
leitaria & lei*ta*ri*a \cmark & lei*ta*ri-a \xmark \\
leiteria & lei*te*ri*a \cmark & lei*te*ri-a \xmark \\
leitoa & lei*to*a \cmark & lei*to-a \xmark \\
lendário & len*dá*ri.o \xmark & len*dá*rio \cmark \\
leniência & le*ni*ên*ci.a \xmark & le*ni*ên*cia \cmark \\
lenocínio & le*no*cí*ni.o \xmark & le*no*cí*nio \cmark \\
leoa & le*o*a \cmark & le*o-a \xmark \\
leônico & le-ô*ni*co \xmark & le*ô*ni*co \cmark \\
lepidopterologia & le*pi*dop*te*ro*lo*gi*a \cmark & le*pi*dop*te*ro*lo*gi-a \xmark \\
leporídeo & le*po*rí*de.o \xmark & le*po*rí*deo \cmark \\
leprologia & le*pro*lo*gi*a \cmark & le*pro*lo*gi-a \xmark \\
leprosaria & le*pro*sa*ri*a \cmark & le*pro*sa*ri-a \xmark \\
leprosário & le*pro*sá*ri.o \xmark & le*pro*sá*rio \cmark \\
lésbia & lés*bi.a \xmark & lés*bia \cmark \\
lésbio & lés*bi.o \xmark & lés*bio \cmark \\
letargia & le*tar*gi*a \cmark & le*tar*gi-a \xmark \\
letícia & le*tí*ci.a \xmark & le*tí*cia \cmark \\
letria & le*tri*a \cmark & le*tri-a \xmark \\
leucemia & leu*ce*mi*a \cmark & leu*ce*mi-a \xmark \\
leucocitário & leu*co*ci*tá*ri.o \xmark & leu*co*ci*tá*rio \cmark \\
leucopenia & leu*co*pe*ni*a \cmark & leu*co*pe*ni-a \xmark \\
leucoplasia & leu*co*pla*si*a \cmark & leu*co*pla*si-a \xmark \\
leucotomia & leu*co*to*mi*a \cmark & leu*co*to*mi-a \xmark \\
levadia & le*va*di*a \cmark & le*va*di-a \xmark \\
lexicografia & le*xi*co*gra*fi*a \cmark & le*xi*co*gra*fi-a \xmark \\
lexicologia & le*xi*co*lo*gi*a \cmark & le*xi*co*lo*gi-a \xmark \\
lezíria & le*zí*ri.a \xmark & le*zí*ria \cmark \\
lia & li*a \cmark & li-a \xmark \\
liberatório & li*be*ra*tó*ri.o \xmark & li*be*ra*tó*rio \cmark \\
libertário & li*ber*tá*ri.o \xmark & li*ber*tá*rio \cmark \\
líbio & lí*bi.o \xmark & lí*bio \cmark \\
licantropia & li*can*tro*pi*a \cmark & li*can*tro*pi-a \xmark \\
lichia & li*chi*a \cmark & li*chi-a \xmark \\
lício & lí*ci.o \xmark & lí*cio \cmark \\
lidocaína & li*do*ca-í*na \xmark & li*do*ca*í*na \cmark \\
liechtensteinense & li*e.ch*tens*tei*nen*se \xmark & li*e.ch*tens*tei*nen*se \xmark \\
lígio & lí*gi.o \xmark & lí*gio \cmark \\
lílio & lí*li.o \xmark & lí*lio \cmark \\
limnologia & lim*no*lo*gi*a \cmark & lim*no*lo*gi-a \xmark \\
linfadenectomia & lin*fa*de*nec*to*mi*a \cmark & lin*fa*de*nec*to*mi-a \xmark \\
linfadenopatia & lin*fa*de*no*pa*ti*a \cmark & lin*fa*de*no*pa*ti-a \xmark \\
linóleo & li*nó*le*o \cmark & li*nó*le-o \xmark \\
lio & li*o \cmark & li-o \xmark \\
lipídeo & li*pí*de.o \xmark & li*pí*deo \cmark \\
lipídio & li*pí*di.o \xmark & li*pí*dio \cmark \\
lipodistrofia & li*po*dis*tro*fi*a \cmark & li*po*dis*tro*fi-a \xmark \\
lírio & lí*ri*o \cmark & lí*ri-o \xmark \\
lisboa & lis*bo*a \cmark & lis*bo-a \xmark \\
lisimáquia & li*si*má*qui*a \cmark & li*si*má*qui-a \xmark \\
litania & li*ta*ni*a \cmark & li*ta*ni-a \xmark \\
litargírio & li*tar*gí*ri.o \xmark & li*tar*gí*rio \cmark \\
literacia & li*te*ra*ci*a \cmark & li*te*ra*ci-a \xmark \\
literário & li*te*rá*ri.o \xmark & li*te*rá*rio \cmark \\
litíase & li*tí-a*se \xmark & li*tí*a*se \cmark \\
litígio & li*tí*gi.o \xmark & li*tí*gio \cmark \\
lítio & lí*ti.o \xmark & lí*tio \cmark \\
litisconsórcio & li*tis*con*sór*ci.o \xmark & li*tis*con*sór*cio \cmark \\
litispendência & li*tis*pen*dên*ci*a \cmark & li*tis*pen*dên*ci-a \xmark \\
litofania & li*to*fa*ni*a \cmark & li*to*fa*ni-a \xmark \\
litografia & li*to*gra*fi*a \cmark & li*to*gra*fi-a \xmark \\
litologia & li*to*lo*gi*a \cmark & li*to*lo*gi-a \xmark \\
litorâneo & li*to*râ*ne.o \xmark & li*to*râ*neo \cmark \\
litotripsia & li*to*trip*si*a \cmark & li*to*trip*si-a \xmark \\
liturgia & li*tur*gi*a \cmark & li*tur*gi-a \xmark \\
livônio & li-vô*ni.o \xmark & li*vô*nio \cmark \\
livraria & li*vra*ri*a \cmark & li*vra*ri-a \xmark \\
lixívia & li*xí*vi.a \xmark & li*xí*via \cmark \\
loa & lo*a \cmark & lo-a \xmark \\
lobectomia & lo*bec*to*mi*a \cmark & lo*bec*to*mi-a \xmark \\
lobélia & lo*bé*li*a \cmark & lo*bé*li-a \xmark \\
locatário & lo*ca*tá*ri.o \xmark & lo*ca*tá*rio \cmark \\
logosofia & lo*go*so*fi*a \cmark & lo*go*so*fi-a \xmark \\
lólio & ló*li.o \xmark & ló*lio \cmark \\
lombalgia & lom*bal*gi*a \cmark & lom*bal*gi-a \xmark \\
longilíneo & lon*gi*lí*ne.o \xmark & lon*gi*lí*neo \cmark \\
lotaria & lo*ta*ri*a \cmark & lo*ta*ri-a \xmark \\
loteria & lo*te*ri*a \cmark & lo*te*ri-a \xmark \\
loxodromia & lo*xo*dro*mi*a \cmark & lo*xo*dro*mi-a \xmark \\
lua & lu*a \cmark & lu-a \xmark \\
lúcio & lú*ci.o \xmark & lú*cio \cmark \\
ludião & lu*di-ão \xmark & lu*di*ão \cmark \\
lúdio & lú*di.o \xmark & lú*dio \cmark \\
ludopédio & lu*do*pé*di.o \xmark & lu*do*pé*dio \cmark \\
ludoterapia & lu*do*te*ra*pi*a \cmark & lu*do*te*ra*pi-a \xmark \\
luís & lu-ís \xmark & lu*ís \cmark \\
luminância & lu*mi*nân*ci.a \xmark & lu*mi*nân*cia \cmark \\
luminária & lu*mi*ná*ri.a \xmark & lu*mi*ná*ria \cmark \\
luminescência & lu*mi*nes*cên*ci.a \xmark & lu*mi*nes*cên*cia \cmark \\
lunário & lu*ná*ri.o \xmark & lu*ná*rio \cmark \\
lusíada & lu*sí-a*da \xmark & lu*sí*a*da \cmark \\
lusofobia & lu*so*fo*bi*a \cmark & lu*so*fo*bi-a \xmark \\
lusofonia & lu*so*fo*ni*a \cmark & lu*so*fo*ni-a \xmark \\
lutécio & lu*té*ci.o \xmark & lu*té*cio \cmark \\
luteína & lu*te-í*na \xmark & lu*te*í*na \cmark \\
lúteo & lú*te.o \xmark & lú*teo \cmark \\
luteria & lu*te*ri*a \cmark & lu*te*ri-a \xmark \\
luxúria & lu*xú*ri.a \xmark & lu*xú*ria \cmark \\
macadâmia & ma*ca*dâ*mi.a \xmark & ma*ca*dâ*mia \cmark \\
macambúzio & ma*cam*bú*zi.o \xmark & ma*cam*bú*zio \cmark \\
macaúba & ma*ca-ú*ba \xmark & ma*ca*ú*ba \cmark \\
macedônia & ma*ce-dô*ni.a \xmark & ma*ce*dô*nia \cmark \\
macedônico & ma*ce-dô*ni*co \xmark & ma*ce*dô*ni*co \cmark \\
macedônio & ma*ce-dô*ni.o \xmark & ma*ce*dô*nio \cmark \\
maceió & ma*cei-ó \xmark & ma*cei*ó \cmark \\
machio & ma*chi*o \cmark & ma*chi-o \xmark \\
macio & ma*ci*o \cmark & ma*ci-o \xmark \\
maçonaria & ma*ço*na*ri*a \cmark & ma*ço*na*ri-a \xmark \\
maçônico & ma-çô*ni*co \xmark & ma*çô*ni*co \cmark \\
macróbio & ma*cró*bi.o \xmark & ma*cró*bio \cmark \\
macrobiótica & ma*cro*bi-ó*ti*ca \xmark & ma*cro*bi*ó*ti*ca \cmark \\
macrobiótico & ma*cro*bi-ó*ti*co \xmark & ma*cro*bi*ó*ti*co \cmark \\
macrocefalia & ma*cro*ce*fa*li*a \cmark & ma*cro*ce*fa*li-a \xmark \\
macroeconomia & ma*cro*e*co*no*mi*a \cmark & ma*cro*e*co*no*mi-a \xmark \\
macroeconômico & ma*cro*e*co-nô*mi*co \xmark & ma*cro*e*co*nô*mi*co \cmark \\
macronúcleo & ma*cro*nú*cle.o \xmark & ma*cro*nú*cleo \cmark \\
macrorregião & ma*cror*re*gi-ão \xmark & ma*cror*re*gi*ão \cmark \\
macrossomia & ma*cros*so*mi*a \cmark & ma*cros*so*mi-a \xmark \\
macua & ma*cu*a \cmark & ma*cu-a \xmark \\
mádia & má*di.a \xmark & má*dia \cmark \\
maestria & ma*es*tri*a \cmark & ma*es*tri-a \xmark \\
máfia & má*fi.a \xmark & má*fia \cmark \\
mafuá & ma*fu-á \xmark & ma*fu*á \cmark \\
magia & ma*gi*a \cmark & ma*gi-a \xmark \\
magistério & ma*gis*té*ri.o \xmark & ma*gis*té*rio \cmark \\
magnésio & mag*né*si.o \xmark & mag*né*sio \cmark \\
magnetômetro & mag*ne-tô*me*tro \xmark & mag*ne*tô*me*tro \cmark \\
magnetoscópio & mag*ne*tos*có*pi.o \xmark & mag*ne*tos*có*pio \cmark \\
magnificência & mag*ni*fi*cên*ci.a \xmark & mag*ni*fi*cên*cia \cmark \\
magnólia & mag*nó*li*a \cmark & mag*nó*li-a \xmark \\
mágoa & má*go.a \xmark & má*goa \cmark \\
maia & mai*a \cmark & mai-a \xmark \\
maiólica & mai-ó*li*ca \xmark & mai*ó*li*ca \cmark \\
maio & mai*o \cmark & mai-o \xmark \\
maioria & mai*o*ri*a \cmark & mai*o*ri-a \xmark \\
maís & ma-ís \xmark & ma*ís \cmark \\
maiúscula & mai-ús*cu*la \xmark & mai*ús*cu*la \cmark \\
maiúsculo & mai-ús*cu*lo \xmark & mai*ús*cu*lo \cmark \\
majoritário & ma*jo*ri*tá*ri.o \xmark & ma*jo*ri*tá*rio \cmark \\
malacologia & ma*la*co*lo*gi*a \cmark & ma*la*co*lo*gi-a \xmark \\
malafaia & ma*la*fai*a \cmark & ma*la*fai-a \xmark \\
malaio & ma*lai*o \cmark & ma*lai-o \xmark \\
malária & ma*lá*ri.a \xmark & ma*lá*ria \cmark \\
malásio & ma*lá*si.o \xmark & ma*lá*sio \cmark \\
malauiano & ma*lau-i*a*no \xmark & ma*lau*i*a*no \cmark \\
maleável & ma*le-á*vel \xmark & ma*le*á*vel \cmark \\
maledicência & ma*le*di*cên*ci.a \xmark & ma*le*di*cên*cia \cmark \\
malefício & ma*le*fí*ci.o \xmark & ma*le*fí*cio \cmark \\
maléolo & ma*lé-o*lo \xmark & ma*lé*o*lo \cmark \\
malevolência & ma*le*vo*lên*ci.a \xmark & ma*le*vo*lên*cia \cmark \\
malharia & ma*lha*ri*a \cmark & ma*lha*ri-a \xmark \\
malícia & ma*lí*ci.a \xmark & ma*lí*cia \cmark \\
malônico & ma-lô*ni*co \xmark & ma*lô*ni*co \cmark \\
malthusianismo & mal-thu*si*a*nis*mo \xmark & mal-thu*si*a*nis*mo \xmark \\
malthusiano & mal-thu*si*a*no \xmark & mal-thu*si*a*no \xmark \\
malvasia & mal*va*si*a \cmark & mal*va*si-a \xmark \\
mamário & ma*má*ri.o \xmark & ma*má*rio \cmark \\
mamoa & ma*mo*a \cmark & ma*mo-a \xmark \\
mamografia & ma*mo*gra*fi*a \cmark & ma*mo*gra*fi-a \xmark \\
manaia & ma*nai*a \cmark & ma*nai-a \xmark \\
mancebia & man*ce*bi*a \cmark & man*ce*bi-a \xmark \\
mancheia & man*chei*a \cmark & man*chei-a \xmark \\
mandaçaia & man*da*çai*a \cmark & man*da*çai-a \xmark \\
mandatário & man*da*tá*ri.o \xmark & man*da*tá*rio \cmark \\
mandrião & man*dri-ão \xmark & man*dri*ão \cmark \\
maneia & ma*nei*a \cmark & ma*nei-a \xmark \\
maneio & ma*nei*o \cmark & ma*nei-o \xmark \\
maníaco & ma*ní-a*co \xmark & ma*ní*a*co \cmark \\
mania & ma*ni*a \cmark & ma*ni-a \xmark \\
manicômio & ma*ni-cô*mi.o \xmark & ma*ni*cô*mio \cmark \\
maniqueísmo & ma*ni*que-ís*mo \xmark & ma*ni*que*ís*mo \cmark \\
maniqueísta & ma*ni*que-ís*ta \xmark & ma*ni*que*ís*ta \cmark \\
manômetro & ma-nô*me*tro \xmark & ma*nô*me*tro \cmark \\
mantô & man-tô \xmark & man*tô \cmark \\
manúbrio & ma*nú*bri.o \xmark & ma*nú*brio \cmark \\
manuseio & ma*nu*sei*o \cmark & ma*nu*sei-o \xmark \\
maquia & ma*qui*a \cmark & ma*qui-a \xmark \\
maquinaria & ma*qui*na*ri*a \cmark & ma*qui*na*ri-a \xmark \\
maquinário & ma*qui*ná*ri.o \xmark & ma*qui*ná*rio \cmark \\
maracutaia & ma*ra*cu*tai*a \cmark & ma*ra*cu*tai-a \xmark \\
marambaia & ma*ram*bai*a \cmark & ma*ram*bai-a \xmark \\
marcenaria & mar*ce*na*ri*a \cmark & mar*ce*na*ri-a \xmark \\
marchetaria & mar*che*ta*ri*a \cmark & mar*che*ta*ri-a \xmark \\
márcio & már*ci.o \xmark & már*cio \cmark \\
maresia & ma*re*si*a \cmark & ma*re*si-a \xmark \\
marginália & mar*gi*ná*li.a \xmark & mar*gi*ná*lia \cmark \\
maria & ma*ri*a \cmark & ma*ri-a \xmark \\
marinharia & ma*ri*nha*ri*a \cmark & ma*ri*nha*ri-a \xmark \\
mariologia & ma*ri*o*lo*gi*a \cmark & ma*ri*o*lo*gi-a \xmark \\
marmoraria & mar*mo*ra*ri*a \cmark & mar*mo*ra*ri-a \xmark \\
marmóreo & mar*mó*re.o \xmark & mar*mó*reo \cmark \\
marquês & mar-quês \xmark & mar*quês \cmark \\
marruá & mar*ru-á \xmark & mar*ru*á \cmark \\
marsúpio & mar*sú*pi.o \xmark & mar*sú*pio \cmark \\
martírio & mar*tí*ri.o \xmark & mar*tí*rio \cmark \\
martirológio & mar*ti*ro*ló*gi.o \xmark & mar*ti*ro*ló*gio \cmark \\
maruí & ma*ru-í \xmark & ma*ru*í \cmark \\
maruim & ma*ru-im \xmark & ma*ru-im \xmark \\
masdeísmo & mas*de-ís*mo \xmark & mas*de*ís*mo \cmark \\
massoterapia & mas*so*te*ra*pi*a \cmark & mas*so*te*ra*pi-a \xmark \\
mastectomia & mas*tec*to*mi*a \cmark & mas*tec*to*mi-a \xmark \\
mastigatório & mas*ti*ga*tó*ri.o \xmark & mas*ti*ga*tó*rio \cmark \\
mastologia & mas*to*lo*gi*a \cmark & mas*to*lo*gi-a \xmark \\
mastozoologia & mas*to*zo*o*lo*gi*a \cmark & mas*to*zo*o*lo*gi-a \xmark \\
matéria & ma*té*ri.a \xmark & ma*té*ria \cmark \\
matricídio & ma*tri*cí*di.o \xmark & ma*tri*cí*dio \cmark \\
matrimônio & ma*tri-mô*ni.o \xmark & ma*tri*mô*nio \cmark \\
maué & mau-é \xmark & mau*é \cmark \\
mavórcio & ma*vór*ci.o \xmark & ma*vór*cio \cmark \\
meã & me-ã \xmark & me*ã \cmark \\
meão & me-ão \xmark & me*ão \cmark \\
média & mé*di.a \xmark & mé*dia \cmark \\
mediático & me*di-á*ti*co \xmark & me*di*á*ti*co \cmark \\
mediocracia & me*di*o*cra*ci*a \cmark & me*di*o*cra*ci-a \xmark \\
medíocre & me*dí-o*cre \xmark & me*dí*o*cre \cmark \\
médio & mé*di.o \xmark & mé*dio \cmark \\
mediterrâneo & me*di*ter*râ*ne.o \xmark & me*di*ter*râ*neo \cmark \\
mediúnico & me*di-ú*ni*co \xmark & me*di*ú*ni*co \cmark \\
mediunidade & me*di-u*ni*da*de \xmark & me*di*u*ni*da*de \cmark \\
megalomaníaco & me*ga*lo*ma*ní-a*co \xmark & me*ga*lo*ma*ní*a*co \cmark \\
megalomania & me*ga*lo*ma*ni*a \cmark & me*ga*lo*ma*ni-a \xmark \\
megalômano & me*ga-lô*ma*no \xmark & me*ga*lô*ma*no \cmark \\
megatério & me*ga*té*ri.o \xmark & me*ga*té*rio \cmark \\
megawatt & me*ga-watt \xmark & me*ga-watt \xmark \\
meia & mei*a \cmark & mei-a \xmark \\
meio & mei*o \cmark & mei-o \xmark \\
meiótico & mei-ó*ti*co \xmark & mei*ó*ti*co \cmark \\
melancia & me*lan*ci*a \cmark & me*lan*ci-a \xmark \\
melancolia & me*lan*co*li*a \cmark & me*lan*co*li-a \xmark \\
melanésio & me*la*né*si.o \xmark & me*la*né*sio \cmark \\
melania & me*la*ni*a \cmark & me*la*ni-a \xmark \\
melânia & me*lâ*ni.a \xmark & me*lâ*nia \cmark \\
melhoria & me*lho*ri*a \cmark & me*lho*ri-a \xmark \\
melífluo & me*lí*flu.o \xmark & me*lí*fluo \cmark \\
meloa & me*lo*a \cmark & me*lo-a \xmark \\
melodia & me*lo*di*a \cmark & me*lo*di-a \xmark \\
melofobia & me*lo*fo*bi*a \cmark & me*lo*fo*bi-a \xmark \\
melomania & me*lo*ma*ni*a \cmark & me*lo*ma*ni-a \xmark \\
melômano & me-lô*ma*no \xmark & me*lô*ma*no \cmark \\
membranáceo & mem*bra*ná*ce.o \xmark & mem*bra*ná*ceo \cmark \\
memória & me*mó*ri.a \xmark & me*mó*ria \cmark \\
mendelévio & men*de*lé*vi.o \xmark & men*de*lé*vio \cmark \\
mendicância & men*di*cân*ci.a \xmark & men*di*cân*cia \cmark \\
meneio & me*nei*o \cmark & me*nei-o \xmark \\
menológio & me*no*ló*gi.o \xmark & me*no*ló*gio \cmark \\
menorragia & me*nor*ra*gi*a \cmark & me*nor*ra*gi-a \xmark \\
mensário & men*sá*ri.o \xmark & men*sá*rio \cmark \\
mênstruo & mêns*tru.o \xmark & mêns*truo \cmark \\
mercadologia & mer*ca*do*lo*gi*a \cmark & mer*ca*do*lo*gi-a \xmark \\
mercadoria & mer*ca*do*ri*a \cmark & mer*ca*do*ri-a \xmark \\
mercancia & mer*can*ci*a \cmark & mer*can*ci-a \xmark \\
mercearia & mer*ce*a*ri*a \cmark & mer*ce*a*ri-a \xmark \\
mercenário & mer*ce*ná*ri.o \xmark & mer*ce*ná*rio \cmark \\
mércia & mér*ci.a \xmark & mér*cia \cmark \\
mercúrio & mer*cú*ri.o \xmark & mer*cú*rio \cmark \\
mergulhia & mer*gu*lhi*a \cmark & mer*gu*lhi-a \xmark \\
meritocracia & me*ri*to*cra*ci*a \cmark & me*ri*to*cra*ci-a \xmark \\
merovíngio & me*ro*vín*gi.o \xmark & me*ro*vín*gio \cmark \\
mesário & me*sá*ri.o \xmark & me*sá*rio \cmark \\
mesentério & me*sen*té*ri.o \xmark & me*sen*té*rio \cmark \\
mesogleia & me*so*glei*a \cmark & me*so*glei-a \xmark \\
mesotélio & me*so*té*li.o \xmark & me*so*té*lio \cmark \\
mesoterapia & me*so*te*ra*pi*a \cmark & me*so*te*ra*pi-a \xmark \\
mesquinharia & mes*qui*nha*ri*a \cmark & mes*qui*nha*ri-a \xmark \\
metafonia & me*ta*fo*ni*a \cmark & me*ta*fo*ni-a \xmark \\
metalografia & me*ta*lo*gra*fi*a \cmark & me*ta*lo*gra*fi-a \xmark \\
metalurgia & me*ta*lur*gi*a \cmark & me*ta*lur*gi-a \xmark \\
metameria & me*ta*me*ri*a \cmark & me*ta*me*ri-a \xmark \\
metaplasia & me*ta*pla*si*a \cmark & me*ta*pla*si-a \xmark \\
metapsicologia & me*tap*si*co*lo*gi*a \cmark & me*tap*si*co*lo*gi-a \xmark \\
meteórico & me*te-ó*ri*co \xmark & me*te*ó*ri*co \cmark \\
meteorologia & me*te*o*ro*lo*gi*a \cmark & me*te*o*ro*lo*gi-a \xmark \\
metodologia & me*to*do*lo*gi*a \cmark & me*to*do*lo*gi-a \xmark \\
metonímia & me*to*ní*mi.a \xmark & me*to*ní*mia \cmark \\
metrologia & me*tro*lo*gi*a \cmark & me*tro*lo*gi-a \xmark \\
metrorragia & me*tror*ra*gi*a \cmark & me*tror*ra*gi-a \xmark \\
mialgia & mi*al*gi*a \cmark & mi*al*gi-a \xmark \\
miastenia & mi*as*te*ni*a \cmark & mi*as*te*ni-a \xmark \\
micélio & mi*cé*li.o \xmark & mi*cé*lio \cmark \\
micobactéria & mi*co*bac*té*ri.a \xmark & mi*co*bac*té*ria \cmark \\
micologia & mi*co*lo*gi*a \cmark & mi*co*lo*gi-a \xmark \\
microbiólogo & mi*cro*bi-ó*lo*go \xmark & mi*cro*bi*ó*lo*go \cmark \\
micróbio & mi*cró*bi.o \xmark & mi*cró*bio \cmark \\
microcefalia & mi*cro*ce*fa*li*a \cmark & mi*cro*ce*fa*li-a \xmark \\
microcirurgia & mi*cro*ci*rur*gi*a \cmark & mi*cro*ci*rur*gi-a \xmark \\
microempresário & mi*cro*em*pre*sá*ri.o \xmark & mi*cro*em*pre*sá*rio \cmark \\
microfonia & mi*cro*fo*ni*a \cmark & mi*cro*fo*ni-a \xmark \\
microfotografia & mi*cro*fo*to*gra*fi*a \cmark & mi*cro*fo*to*gra*fi-a \xmark \\
microftalmia & mi*crof*tal*mi*a \cmark & mi*crof*tal*mi-a \xmark \\
micrografia & mi*cro*gra*fi*a \cmark & mi*cro*gra*fi-a \xmark \\
microrregião & mi*cror*re*gi-ão \xmark & mi*cror*re*gi*ão \cmark \\
microscopia & mi*cros*co*pi*a \cmark & mi*cros*co*pi-a \xmark \\
microscópio & mi*cros*có*pi.o \xmark & mi*cros*có*pio \cmark \\
microsporângio & mi*cros*po*rân*gi.o \xmark & mi*cros*po*rân*gio \cmark \\
mictório & mic*tó*ri.o \xmark & mic*tó*rio \cmark \\
micuim & mi*cu-im \xmark & mi*cu-im \xmark \\
mídia & mí*di.a \xmark & mí*dia \cmark \\
midríase & mi*drí-a*se \xmark & mi*drí*a*se \cmark \\
mielopatia & mi*e*lo*pa*ti*a \cmark & mi*e*lo*pa*ti-a \xmark \\
migratório & mi*gra*tó*ri.o \xmark & mi*gra*tó*rio \cmark \\
miíase & mi-í-a*se \xmark & mi*í*a*se \cmark \\
míldio & míl*di.o \xmark & míl*dio \cmark \\
milefólio & mi*le*fó*li.o \xmark & mi*le*fó*lio \cmark \\
milenário & mi*le*ná*ri.o \xmark & mi*le*ná*rio \cmark \\
milênio & mi*lê*ni.o \xmark & mi*lê*nio \cmark \\
miliário & mi*li-á*ri.o \xmark & mi*li*á*rio \cmark \\
milícia & mi*lí*ci.a \xmark & mi*lí*cia \cmark \\
milionário & mi*li*o*ná*ri*o \cmark & mi*li*o*ná*ri-o \xmark \\
militância & mi*li*tân*ci.a \xmark & mi*li*tân*cia \cmark \\
mimeógrafo & mi*me-ó*gra*fo \xmark & mi*me*ó*gra*fo \cmark \\
mineralogia & mi*ne*ra*lo*gi*a \cmark & mi*ne*ra*lo*gi-a \xmark \\
minério & mi*né*ri.o \xmark & mi*né*rio \cmark \\
minifúndio & mi*ni*fún*di.o \xmark & mi*ni*fún*dio \cmark \\
minissaia & mi*nis*sai*a \cmark & mi*nis*sai-a \xmark \\
minissérie & mi*nis*sé*ri.e \xmark & mi*nis*sé*rie \cmark \\
ministeriável & mi*nis*te*ri-á*vel \xmark & mi*nis*te*ri*á*vel \cmark \\
ministério & mi*nis*té*ri.o \xmark & mi*nis*té*rio \cmark \\
minoria & mi*no*ri*a \cmark & mi*no*ri-a \xmark \\
minoritário & mi*no*ri*tá*ri.o \xmark & mi*no*ri*tá*rio \cmark \\
minúcia & mi*nú*ci.a \xmark & mi*nú*cia \cmark \\
minudência & mi*nu*dên*ci.a \xmark & mi*nu*dên*cia \cmark \\
miocárdio & mi*o*cár*di.o \xmark & mi*o*cár*dio \cmark \\
mioclonia & mi*o*clo*ni*a \cmark & mi*o*clo*ni-a \xmark \\
miologia & mi*o*lo*gi*a \cmark & mi*o*lo*gi-a \xmark \\
mio & mi*o \cmark & mi-o \xmark \\
miopatia & mi*o*pa*ti*a \cmark & mi*o*pa*ti-a \xmark \\
míope & mí-o*pe \xmark & mí*o*pe \cmark \\
miopia & mi*o*pi*a \cmark & mi*o*pi-a \xmark \\
miríade & mi*rí-a*de \xmark & mi*rí*a*de \cmark \\
miringotomia & mi*rin*go*to*mi*a \cmark & mi*rin*go*to*mi-a \xmark \\
misandria & mi*san*dri*a \cmark & mi*san*dri-a \xmark \\
misantropia & mi*san*tro*pi*a \cmark & mi*san*tro*pi-a \xmark \\
miscelânea & mis*ce*lâ*ne.a \xmark & mis*ce*lâ*nea \cmark \\
miséria & mi*sé*ri.a \xmark & mi*sé*ria \cmark \\
misericórdia & mi*se*ri*cór*di.a \xmark & mi*se*ri*cór*dia \cmark \\
misoginia & mi*so*gi*ni*a \cmark & mi*so*gi*ni-a \xmark \\
missionário & mis*si*o*ná*ri.o \xmark & mis*si*o*ná*rio \cmark \\
mistério & mis*té*ri.o \xmark & mis*té*rio \cmark \\
mistifório & mis*ti*fó*ri.o \xmark & mis*ti*fó*rio \cmark \\
mitocôndria & mi*to-côn*dri.a \xmark & mi*to*côn*dria \cmark \\
mitografia & mi*to*gra*fi*a \cmark & mi*to*gra*fi-a \xmark \\
mitologia & mi*to*lo*gi*a \cmark & mi*to*lo*gi-a \xmark \\
mitomania & mi*to*ma*ni*a \cmark & mi*to*ma*ni-a \xmark \\
miúdas & mi-ú*das \xmark & mi*ú*das \cmark \\
miudeza & mi-u*de*za \xmark & mi*u*de*za \cmark \\
miúdo & mi-ú*do \xmark & mi*ú*do \cmark \\
mixaria & mi*xa*ri*a \cmark & mi*xa*ri-a \xmark \\
mixórdia & mi*xór*di.a \xmark & mi*xór*dia \cmark \\
mnemônica & m.ne-mô*ni*ca \xmark & mne*mô*ni*ca \cmark \\
mnemônico & m.ne-mô*ni*co \xmark & mne*mô*ni*co \cmark \\
moa & mo*a \cmark & mo-a \xmark \\
mobília & mo*bí*li.a \xmark & mo*bí*lia \cmark \\
mobiliário & mo*bi*li-á*ri.o \xmark & mo*bi*li*á*rio \cmark \\
mocoa & mo*co*a \cmark & mo*co-a \xmark \\
modéstia & mo*dés*ti.a \xmark & mo*dés*tia \cmark \\
moído & mo-í*do \xmark & mo*í*do \cmark \\
moinha & mo-i*nha \xmark & mo*i*nha \cmark \\
moinho & mo-i*nho \xmark & mo*i*nho \cmark \\
moio & moi*o \cmark & moi-o \xmark \\
moldávio & mol*dá*vi.o \xmark & mol*dá*vio \cmark \\
moléstia & mo*lés*ti.a \xmark & mo*lés*tia \cmark \\
molibdênio & mo*lib*dê*ni.o \xmark & mo*lib*dê*nio \cmark \\
momentâneo & mo*men*tâ*ne.o \xmark & mo*men*tâ*neo \cmark \\
monadologia & mo*na*do*lo*gi*a \cmark & mo*na*do*lo*gi-a \xmark \\
monarquia & mo*nar*qui*a \cmark & mo*nar*qui-a \xmark \\
monastério & mo*nas*té*ri.o \xmark & mo*nas*té*rio \cmark \\
monetário & mo*ne*tá*ri.o \xmark & mo*ne*tá*rio \cmark \\
monitoria & mo*ni*to*ri*a \cmark & mo*ni*to*ri-a \xmark \\
monitória & mo*ni*tó*ri.a \xmark & mo*ni*tó*ria \cmark \\
monocórdio & mo*no*cór*di.o \xmark & mo*no*cór*dio \cmark \\
monocromia & mo*no*cro*mi*a \cmark & mo*no*cro*mi-a \xmark \\
monofonia & mo*no*fo*ni*a \cmark & mo*no*fo*ni-a \xmark \\
monofônico & mo*no-fô*ni*co \xmark & mo*no*fô*ni*co \cmark \\
monografia & mo*no*gra*fi*a \cmark & mo*no*gra*fi-a \xmark \\
monomaníaco & mo*no*ma*ní-a*co \xmark & mo*no*ma*ní*a*co \cmark \\
monomania & mo*no*ma*ni*a \cmark & mo*no*ma*ni-a \xmark \\
monômero & mo-nô*me*ro \xmark & mo*nô*me*ro \cmark \\
monômio & mo-nô*mi.o \xmark & mo*nô*mio \cmark \\
monopólio & mo*no*pó*li.o \xmark & mo*no*pó*lio \cmark \\
monoquíni & mo*no-quí*ni \xmark & mo*no*quí*ni \cmark \\
monossacarídeo & mo*nos*sa*ca*rí*de.o \xmark & mo*nos*sa*ca*rí*deo \cmark \\
monossomia & mo*nos*so*mi*a \cmark & mo*nos*so*mi-a \xmark \\
monoteísmo & mo*no*te-ís*mo \xmark & mo*no*te*ís*mo \cmark \\
monoteísta & mo*no*te-ís*ta \xmark & mo*no*te*ís*ta \cmark \\
monotipia & mo*no*ti*pi*a \cmark & mo*no*ti*pi-a \xmark \\
monotonia & mo*no*to*ni*a \cmark & mo*no*to*ni-a \xmark \\
montaria & mon*ta*ri*a \cmark & mon*ta*ri-a \xmark \\
montepio & mon*te*pi*o \cmark & mon*te*pi-o \xmark \\
moquém & mo-quém \xmark & mo*quém \cmark \\
moradia & mo*ra*di*a \cmark & mo*ra*di-a \xmark \\
moratória & mo*ra*tó*ri.a \xmark & mo*ra*tó*ria \cmark \\
moratório & mo*ra*tó*ri.o \xmark & mo*ra*tó*rio \cmark \\
mordomia & mor*do*mi*a \cmark & mor*do*mi-a \xmark \\
moreia & mo*rei*a \cmark & mo*rei-a \xmark \\
morfologia & mor*fo*lo*gi*a \cmark & mor*fo*lo*gi-a \xmark \\
morgadio & mor*ga*di*o \cmark & mor*ga*di-o \xmark \\
morrião & mor*ri-ão \xmark & mor*ri*ão \cmark \\
mortuárias & mor*tu-á*ri*as \xmark & mor*tu*á*ri*as \cmark \\
moscóvia & mos*có*vi.a \xmark & mos*có*via \cmark \\
mostruário & mos*tru-á*ri.o \xmark & mos*tru*á*rio \cmark \\
mouraria & mou*ra*ri*a \cmark & mou*ra*ri-a \xmark \\
movelaria & mo*ve*la*ri*a \cmark & mo*ve*la*ri-a \xmark \\
mucuim & mu*cu-im \xmark & mu*cu-im \xmark \\
mulataria & mu*la*ta*ri*a \cmark & mu*la*ta*ri-a \xmark \\
mulherio & mu*lhe*ri*o \cmark & mu*lhe*ri-o \xmark \\
multimídia & mul*ti*mí*di.a \xmark & mul*ti*mí*dia \cmark \\
multimilionário & mul*ti*mi*li*o*ná*ri.o \xmark & mul*ti*mi*li*o*ná*rio \cmark \\
multitudinário & mul*ti*tu*di*ná*ri*o \cmark & mul*ti*tu*di*ná*ri-o \xmark \\
múmia & mú*mi.a \xmark & mú*mia \cmark \\
mundaréu & mun*da*ré-u \xmark & mun*da*ré-u \xmark \\
município & mu*ni*cí*pi.o \xmark & mu*ni*cí*pio \cmark \\
muriático & mu*ri-á*ti*co \xmark & mu*ri*á*ti*co \cmark \\
murmúrio & mur*mú*ri.o \xmark & mur*mú*rio \cmark \\
museografia & mu*se*o*gra*fi*a \cmark & mu*se*o*gra*fi-a \xmark \\
museologia & mu*se*o*lo*gi*a \cmark & mu*se*o*lo*gi-a \xmark \\
museólogo & mu*se-ó*lo*go \xmark & mu*se*ó*lo*go \cmark \\
musicografia & mu*si*co*gra*fi*a \cmark & mu*si*co*gra*fi-a \xmark \\
musicologia & mu*si*co*lo*gi*a \cmark & mu*si*co*lo*gi-a \xmark \\
musicoterapia & mu*si*co*te*ra*pi*a \cmark & mu*si*co*te*ra*pi-a \xmark \\
mustelídeo & mus*te*lí*de.o \xmark & mus*te*lí*deo \cmark \\
mútua & mú*tu.a \xmark & mú*tua \cmark \\
mutuário & mu*tu-á*ri.o \xmark & mu*tu*á*rio \cmark \\
mútuo & mú*tu.o \xmark & mú*tuo \cmark \\
nagô & na-gô \xmark & na*gô \cmark \\
nanotecnologia & na*no*tec*no*lo*gi*a \cmark & na*no*tec*no*lo*gi-a \xmark \\
napoleão & na*po*le-ão \xmark & na*po*le*ão \cmark \\
narratologia & nar*ra*to*lo*gi*a \cmark & nar*ra*to*lo*gi-a \xmark \\
natalício & na*ta*lí*ci.o \xmark & na*ta*lí*cio \cmark \\
natatório & na*ta*tó*ri.o \xmark & na*ta*tó*rio \cmark \\
naturopatia & na*tu*ro*pa*ti*a \cmark & na*tu*ro*pa*ti-a \xmark \\
naufrágio & nau*frá*gi.o \xmark & nau*frá*gio \cmark \\
naumaquia & nau*ma*qui*a \cmark & nau*ma*qui-a \xmark \\
náuplio & náu*pli.o \xmark & náu*plio \cmark \\
náusea & náu*se.a \xmark & náu*sea \cmark \\
navio & na*vi*o \cmark & na*vi-o \xmark \\
necessário & ne*ces*sá*ri.o \xmark & ne*ces*sá*rio \cmark \\
necrofagia & ne*cro*fa*gi*a \cmark & ne*cro*fa*gi-a \xmark \\
necrofilia & ne*cro*fi*li*a \cmark & ne*cro*fi*li-a \xmark \\
necrologia & ne*cro*lo*gi*a \cmark & ne*cro*lo*gi-a \xmark \\
necrológio & ne*cro*ló*gi.o \xmark & ne*cro*ló*gio \cmark \\
necromancia & ne*cro*man*ci*a \cmark & ne*cro*man*ci-a \xmark \\
necropsia & ne*crop*si*a \cmark & ne*crop*si-a \xmark \\
necrotério & ne*cro*té*ri.o \xmark & ne*cro*té*rio \cmark \\
nectário & nec*tá*ri.o \xmark & nec*tá*rio \cmark \\
nédio & né*di.o \xmark & né*dio \cmark \\
neerlandês & ne*er*lan*dês \cmark & ne*er*lan*dês \cmark \\
nefrectomia & ne*frec*to*mi*a \cmark & ne*frec*to*mi-a \xmark \\
nefrolitíase & ne*fro*li*tí-a*se \xmark & ne*fro*li*tí*a*se \cmark \\
nefrologia & ne*fro*lo*gi*a \cmark & ne*fro*lo*gi-a \xmark \\
nefropatia & ne*fro*pa*ti*a \cmark & ne*fro*pa*ti-a \xmark \\
negligência & ne*gli*gên*ci.a \xmark & ne*gli*gên*cia \cmark \\
negligenciável & ne*gli*gen*ci-á*vel \xmark & ne*gli*gen*ci*á*vel \cmark \\
negócio & ne*gó*ci.o \xmark & ne*gó*cio \cmark \\
nênia & nê*ni.a \xmark & nê*nia \cmark \\
neocapitalismo & ne*o*ca*pi*ta*lis*mo \cmark & ne-o*ca*pi*ta*lis*mo \xmark \\
neoclassicismo & ne*o*clas*si*cis*mo \cmark & ne-o*clas*si*cis*mo \xmark \\
neoclássico & ne*o*clás*si*co \cmark & ne-o*clás*si*co \xmark \\
neocolonialismo & ne*o*co*lo*ni*a*lis*mo \cmark & ne-o*co*lo*ni*a*lis*mo \xmark \\
neodarwinismo & ne*o*dar-wi*nis*mo \xmark & ne-o*dar-wi*nis*mo \xmark \\
neodímio & ne*o*dí*mi.o \xmark & ne-o*dí*mio \xmark \\
neofascismo & ne*o*fas*cis*mo \cmark & ne-o*fas*cis*mo \xmark \\
neofascista & ne*o*fas*cis*ta \cmark & ne-o*fas*cis*ta \xmark \\
neófito & ne-ó*fi*to \xmark & ne*ó*fi*to \cmark \\
neogótico & ne*o*gó*ti*co \cmark & ne-o*gó*ti*co \xmark \\
neokantismo & ne*o*kan*tis*mo \cmark & ne-o*kan*tis*mo \xmark \\
neolatino & ne*o*la*ti*no \cmark & ne-o*la*ti*no \xmark \\
neoliberalismo & ne*o*li*be*ra*lis*mo \cmark & ne-o*li*be*ra*lis*mo \xmark \\
neoliberal & ne*o*li*be*ral \cmark & ne-o*li*be*ral \xmark \\
neolítico & ne*o*lí*ti*co \cmark & ne-o*lí*ti*co \xmark \\
neologismo & ne*o*lo*gis*mo \cmark & ne-o*lo*gis*mo \xmark \\
neologista & ne*o*lo*gis*ta \cmark & ne-o*lo*gis*ta \xmark \\
neonatal & ne*o*na*tal \cmark & ne-o*na*tal \xmark \\
neonatologia & ne*o*na*to*lo*gi*a \cmark & ne-o*na*to*lo*gi-a \xmark \\
neonatologista & ne*o*na*to*lo*gis*ta \cmark & ne-o*na*to*lo*gis*ta \xmark \\
neonato & ne*o*na*to \cmark & ne-o*na*to \xmark \\
neonazismo & ne*o*na*zis*mo \cmark & ne-o*na*zis*mo \xmark \\
neonazista & ne*o*na*zis*ta \cmark & ne-o*na*zis*ta \xmark \\
neônio & ne-ô*ni.o \xmark & ne*ô*nio \cmark \\
neon & ne*on \cmark & ne-on \xmark \\
néon & né-on \xmark & né*on \cmark \\
neoplasia & ne*o*pla*si*a \cmark & ne-o*pla*si-a \xmark \\
neoplásico & ne*o*plá*si*co \cmark & ne-o*plá*si*co \xmark \\
neoplasma & ne*o*plas*ma \cmark & ne-o*plas*ma \xmark \\
neoplatônico & ne*o*pla-tô*ni*co \xmark & ne-o*pla*tô*ni*co \xmark \\
neoplatonismo & ne*o*pla*to*nis*mo \cmark & ne-o*pla*to*nis*mo \xmark \\
neopositivismo & ne*o*po*si*ti*vis*mo \cmark & ne-o*po*si*ti*vis*mo \xmark \\
neotenia & ne*o*te*ni*a \cmark & ne-o*te*ni-a \xmark \\
neotomismo & ne*o*to*mis*mo \cmark & ne-o*to*mis*mo \xmark \\
neotrópico & ne*o*tró*pi*co \cmark & ne-o*tró*pi*co \xmark \\
neozelandês & ne*o*ze*lan*dês \cmark & ne-o*ze*lan*dês \xmark \\
nério & né*ri.o \xmark & né*rio \cmark \\
néscio & nés*ci.o \xmark & nés*cio \cmark \\
netúnio & ne*tú*ni.o \xmark & ne*tú*nio \cmark \\
neuralgia & neu*ral*gi*a \cmark & neu*ral*gi-a \xmark \\
neurastenia & neu*ras*te*ni*a \cmark & neu*ras*te*ni-a \xmark \\
neurobiologia & neu*ro*bi*o*lo*gi*a \cmark & neu*ro*bi*o*lo*gi-a \xmark \\
neurociência & neu*ro*ci*ên*ci.a \xmark & neu*ro*ci*ên*cia \cmark \\
neurocirurgia & neu*ro*ci*rur*gi*a \cmark & neu*ro*ci*rur*gi-a \xmark \\
neurocirurgião & neu*ro*ci*rur*gi-ão \xmark & neu*ro*ci*rur*gi*ão \cmark \\
neurofisiologia & neu*ro*fi*si*o*lo*gi*a \cmark & neu*ro*fi*si*o*lo*gi-a \xmark \\
neurologia & neu*ro*lo*gi*a \cmark & neu*ro*lo*gi-a \xmark \\
neurônio & neu-rô*ni.o \xmark & neu*rô*nio \cmark \\
neuropatia & neu*ro*pa*ti*a \cmark & neu*ro*pa*ti-a \xmark \\
neuropatologia & neu*ro*pa*to*lo*gi*a \cmark & neu*ro*pa*to*lo*gi-a \xmark \\
neuropsiquiatria & neu*rop*si*qui*a*tri*a \cmark & neu*rop*si*qui*a*tri-a \xmark \\
neutropenia & neu*tro*pe*ni*a \cmark & neu*tro*pe*ni-a \xmark \\
névoa & né*vo.a \xmark & né*voa \cmark \\
nevralgia & ne*vral*gi*a \cmark & ne*vral*gi-a \xmark \\
ninfolepsia & nin*fo*lep*si*a \cmark & nin*fo*lep*si-a \xmark \\
ninfomaníaca & nin*fo*ma*ní-a*ca \xmark & nin*fo*ma*ní*a*ca \cmark \\
ninfomaníaco & nin*fo*ma*ní-a*co \xmark & nin*fo*ma*ní*a*co \cmark \\
ninfomania & nin*fo*ma*ni*a \cmark & nin*fo*ma*ni-a \xmark \\
nióbio & ni-ó*bi.o \xmark & ni*ó*bio \cmark \\
nipônico & ni-pô*ni*co \xmark & ni*pô*ni*co \cmark \\
nitrogênio & ni*tro*gê*ni.o \xmark & ni*tro*gê*nio \cmark \\
nobélio & no*bé*li.o \xmark & no*bé*lio \cmark \\
nobiliário & no*bi*li-á*ri.o \xmark & no*bi*li*á*rio \cmark \\
nobiliarquia & no*bi*li*ar*qui*a \cmark & no*bi*li*ar*qui-a \xmark \\
nobiliárquico & no*bi*li-ár*qui*co \xmark & no*bi*li*ár*qui*co \cmark \\
noético & no-é*ti*co \xmark & no*é*ti*co \cmark \\
nomografia & no*mo*gra*fi*a \cmark & no*mo*gra*fi-a \xmark \\
nonagenário & no*na*ge*ná*ri.o \xmark & no*na*ge*ná*rio \cmark \\
noologia & no*o*lo*gi*a \cmark & no*o*lo*gi-a \xmark \\
nosocômio & no*so-cô*mi.o \xmark & no*so*cô*mio \cmark \\
nosologia & no*so*lo*gi*a \cmark & no*so*lo*gi-a \xmark \\
nostalgia & nos*tal*gi*a \cmark & nos*tal*gi-a \xmark \\
notário & no*tá*ri.o \xmark & no*tá*rio \cmark \\
notícia & no*tí*ci.a \xmark & no*tí*cia \cmark \\
noticiário & no*ti*ci-á*ri.o \xmark & no*ti*ci*á*rio \cmark \\
notocórdio & no*to*cór*di*o \cmark & no*to*cór*di-o \xmark \\
notório & no*tó*ri.o \xmark & no*tó*rio \cmark \\
novenário & no*ve*ná*ri.o \xmark & no*ve*ná*rio \cmark \\
núbio & nú*bi.o \xmark & nú*bio \cmark \\
nucléolo & nu*clé-o*lo \xmark & nu*clé*o*lo \cmark \\
núcleo & nú*cle.o \xmark & nú*cleo \cmark \\
nucleotídeo & nu*cle*o*tí*de.o \xmark & nu*cle*o*tí*deo \cmark \\
nuclídeo & nu*clí*de.o \xmark & nu*clí*deo \cmark \\
nudibrânquio & nu*di*brân*qui.o \xmark & nu*di*brân*quio \cmark \\
núncio & nún*ci.o \xmark & nún*cio \cmark \\
núpcias & núp*ci.as \xmark & núp*ci.as \xmark \\
nutrologia & nu*tro*lo*gi*a \cmark & nu*tro*lo*gi-a \xmark \\
oásis & o-á*sis \xmark & o*á*sis \cmark \\
obediência & o*be*di*ên*ci.a \xmark & o*be*di*ên*cia \cmark \\
obituário & o*bi*tu-á*ri.o \xmark & o*bi*tu*á*rio \cmark \\
oblívio & o*blí*vi.o \xmark & o*blí*vio \cmark \\
oboé & o*bo-é \xmark & o*bo*é \cmark \\
oboísta & o*bo-ís*ta \xmark & o*bo*ís*ta \cmark \\
obrigatório & o*bri*ga*tó*ri.o \xmark & o*bri*ga*tó*rio \cmark \\
obséquio & ob*sé*qui.o \xmark & ob*sé*quio \cmark \\
observância & ob*ser*vân*ci.a \xmark & ob*ser*vân*cia \cmark \\
observatório & ob*ser*va*tó*ri.o \xmark & ob*ser*va*tó*rio \cmark \\
obsolescência & ob*so*les*cên*ci.a \xmark & ob*so*les*cên*cia \cmark \\
obstetrícia & obs*te*trí*ci.a \xmark & obs*te*trí*cia \cmark \\
obstruído & obs*tru-í*do \xmark & obs*tru*í*do \cmark \\
obstruir & obs*tru-ir \xmark & obs*tru*ir \cmark \\
óbvio & ób*vi.o \xmark & ób*vio \cmark \\
ocasião & o*ca*si-ão \xmark & o*ca*si*ão \cmark \\
oceânico & o*ce-â*ni*co \xmark & o*ce*â*ni*co \cmark \\
oceanografia & o*ce*a*no*gra*fi*a \cmark & o*ce*a*no*gra*fi-a \xmark \\
oceanologia & o*ce*a*no*lo*gi*a \cmark & o*ce*a*no*lo*gi-a \xmark \\
ócio & ó*ci.o \xmark & ó*cio \cmark \\
oclocracia & o*clo*cra*ci*a \cmark & o*clo*cra*ci-a \xmark \\
ocluir & o*clu-ir \xmark & o*clu*ir \cmark \\
ocorrência & o*cor*rên*ci.a \xmark & o*cor*rên*cia \cmark \\
ocráceo & o*crá*ce.o \xmark & o*crá*ceo \cmark \\
octaédrico & oc*ta-é*dri*co \xmark & oc*ta*é*dri*co \cmark \\
octogenário & oc*to*ge*ná*ri.o \xmark & oc*to*ge*ná*rio \cmark \\
odeão & o*de-ão \xmark & o*de*ão \cmark \\
ódio & ó*di.o \xmark & ó*dio \cmark \\
odisseia & o*dis*sei*a \cmark & o*dis*sei-a \xmark \\
odometria & o*do*me*tri*a \cmark & o*do*me*tri-a \xmark \\
odômetro & o-dô*me*tro \xmark & o*dô*me*tro \cmark \\
odontologia & o*don*to*lo*gi*a \cmark & o*don*to*lo*gi-a \xmark \\
ofertório & o*fer*tó*ri.o \xmark & o*fer*tó*rio \cmark \\
ofício & o*fí*ci.o \xmark & o*fí*cio \cmark \\
ofidiário & o*fi*di-á*ri.o \xmark & o*fi*di*á*rio \cmark \\
ofídio & o*fí*di.o \xmark & o*fí*dio \cmark \\
oftalmia & of*tal*mi*a \cmark & of*tal*mi-a \xmark \\
oftalmologia & of*tal*mo*lo*gi*a \cmark & of*tal*mo*lo*gi-a \xmark \\
oftalmoplegia & of*tal*mo*ple*gi*a \cmark & of*tal*mo*ple*gi-a \xmark \\
oftalmoscopia & of*tal*mos*co*pi*a \cmark & of*tal*mos*co*pi-a \xmark \\
oftalmoscópio & of*tal*mos*có*pi.o \xmark & of*tal*mos*có*pio \cmark \\
ofurô & o*fu-rô \xmark & o*fu*rô \cmark \\
oídio & o-í*di.o \xmark & o*í*dio \cmark \\
olaia & o*lai*a \cmark & o*lai-a \xmark \\
olaria & o*la*ri*a \cmark & o*la*ri-a \xmark \\
olearia & o*le*a*ri*a \cmark & o*le*a*ri-a \xmark \\
oleícola & o*le-í*co*la \xmark & o*le*í*co*la \cmark \\
oleína & o*le-í*na \xmark & o*le*í*na \cmark \\
óleo & ó*le.o \xmark & ó*leo \cmark \\
olfatometria & ol*fa*to*me*tri*a \cmark & ol*fa*to*me*tri-a \xmark \\
oligarquia & o*li*gar*qui*a \cmark & o*li*gar*qui-a \xmark \\
oligofrenia & o*li*go*fre*ni*a \cmark & o*li*go*fre*ni-a \xmark \\
oligopólio & o*li*go*pó*li.o \xmark & o*li*go*pó*lio \cmark \\
oligúria & o*li*gú*ri.a \xmark & o*li*gú*ria \cmark \\
olimpíada & o*lim*pí-a*da \xmark & o*lim*pí*a*da \cmark \\
oliváceo & o*li*vá*ce.o \xmark & o*li*vá*ceo \cmark \\
omíada & o*mí-a*da \xmark & o*mí*a*da \cmark \\
oncologia & on*co*lo*gi*a \cmark & on*co*lo*gi-a \xmark \\
ondulatório & on*du*la*tó*ri.o \xmark & on*du*la*tó*rio \cmark \\
onomatopeia & o*no*ma*to*pei*a \cmark & o*no*ma*to*pei-a \xmark \\
ontogenia & on*to*ge*ni*a \cmark & on*to*ge*ni-a \xmark \\
ontologia & on*to*lo*gi*a \cmark & on*to*lo*gi-a \xmark \\
oócito & o-ó*ci*to \xmark & o*ó*ci*to \cmark \\
ooforectomia & o*o*fo*rec*to*mi*a \cmark & o*o*fo*rec*to*mi-a \xmark \\
oogamia & o*o*ga*mi*a \cmark & o*o*ga*mi-a \xmark \\
oólito & o-ó*li*to \xmark & o*ó*li*to \cmark \\
operário & o*pe*rá*ri.o \xmark & o*pe*rá*rio \cmark \\
operatório & o*pe*ra*tó*ri.o \xmark & o*pe*ra*tó*rio \cmark \\
opiáceo & o*pi-á*ce.o \xmark & o*pi*á*ceo \cmark \\
opinião & o*pi*ni-ão \xmark & o*pi*ni*ão \cmark \\
ópio & ó*pi.o \xmark & ó*pio \cmark \\
opróbrio & o*pró*bri.o \xmark & o*pró*brio \cmark \\
optometria & op*to*me*tri*a \cmark & op*to*me*tri-a \xmark \\
opulência & o*pu*lên*ci.a \xmark & o*pu*lên*cia \cmark \\
oratória & o*ra*tó*ri.a \xmark & o*ra*tó*ria \cmark \\
oratório & o*ra*tó*ri.o \xmark & o*ra*tó*rio \cmark \\
orçamentário & or*ça*men*tá*ri.o \xmark & or*ça*men*tá*rio \cmark \\
ordálio & or*dá*li.o \xmark & or*dá*lio \cmark \\
ordinária & or*di*ná*ri.a \xmark & or*di*ná*ria \cmark \\
ordinário & or*di*ná*ri.o \xmark & or*di*ná*rio \cmark \\
orfeão & or*fe-ão \xmark & or*fe*ão \cmark \\
organologia & or*ga*no*lo*gi*a \cmark & or*ga*no*lo*gi-a \xmark \\
orgia & or*gi*a \cmark & or*gi-a \xmark \\
orgiástico & or*gi-ás*ti*co \xmark & or*gi*ás*ti*co \cmark \\
oriá & o*ri-á \xmark & o*ri*á \cmark \\
orifício & o*ri*fí*ci.o \xmark & o*ri*fí*cio \cmark \\
originário & o*ri*gi*ná*ri.o \xmark & o*ri*gi*ná*rio \cmark \\
oriundo & o*ri-un*do \xmark & o*ri*un*do \cmark \\
ornitofilia & or*ni*to*fi*li*a \cmark & or*ni*to*fi*li-a \xmark \\
ornitologia & or*ni*to*lo*gi*a \cmark & or*ni*to*lo*gi-a \xmark \\
orogenia & o*ro*ge*ni*a \cmark & o*ro*ge*ni-a \xmark \\
orografia & o*ro*gra*fi*a \cmark & o*ro*gra*fi-a \xmark \\
orquidário & or*qui*dá*ri.o \xmark & or*qui*dá*rio \cmark \\
orquídea & or-quí*de.a \xmark & or*quí*dea \cmark \\
orquiectomia & or*qui*ec*to*mi*a \cmark & or*qui*ec*to*mi-a \xmark \\
ortoclásio & or*to*clá*si.o \xmark & or*to*clá*sio \cmark \\
ortodontia & or*to*don*ti*a \cmark & or*to*don*ti-a \xmark \\
ortodôntico & or*to-dôn*ti*co \xmark & or*to*dôn*ti*co \cmark \\
ortodoxia & or*to*do*xi*a \cmark & or*to*do*xi-a \xmark \\
ortodromia & or*to*dro*mi*a \cmark & or*to*dro*mi-a \xmark \\
ortografia & or*to*gra*fi*a \cmark & or*to*gra*fi-a \xmark \\
ortopedia & or*to*pe*di*a \cmark & or*to*pe*di-a \xmark \\
ortopraxia & or*to*pra*xi*a \cmark & or*to*pra*xi-a \xmark \\
oscilatório & os*ci*la*tó*ri.o \xmark & os*ci*la*tó*rio \cmark \\
osciloscópio & os*ci*los*có*pi.o \xmark & os*ci*los*có*pio \cmark \\
ósmio & ós*mi.o \xmark & ós*mio \cmark \\
ossário & os*sá*ri.o \xmark & os*sá*rio \cmark \\
ósseo & ós*se.o \xmark & ós*seo \cmark \\
ossuário & os*su-á*ri.o \xmark & os*su*á*rio \cmark \\
osteíte & os*te-í*te \xmark & os*te*í*te \cmark \\
ostensório & os*ten*só*ri.o \xmark & os*ten*só*rio \cmark \\
osteologia & os*te*o*lo*gi*a \cmark & os*te*o*lo*gi-a \xmark \\
osteomalacia & os*te*o*ma*la*ci*a \cmark & os*te*o*ma*la*ci-a \xmark \\
osteopatia & os*te*o*pa*ti*a \cmark & os*te*o*pa*ti-a \xmark \\
osteotomia & os*te*o*to*mi*a \cmark & os*te*o*to*mi-a \xmark \\
ostiário & os*ti-á*ri.o \xmark & os*ti*á*rio \cmark \\
ostíolo & os*tí-o*lo \xmark & os*tí*o*lo \cmark \\
ostomia & os*to*mi*a \cmark & os*to*mi-a \xmark \\
otalgia & o*tal*gi*a \cmark & o*tal*gi-a \xmark \\
otária & o*tá*ri.a \xmark & o*tá*ria \cmark \\
otário & o*tá*ri.o \xmark & o*tá*rio \cmark \\
otorrinolaringologia & o*tor*ri*no*la*rin*go*lo*gi*a \cmark & o*tor*ri*no*la*rin*go*lo*gi-a \xmark \\
ourivesaria & ou*ri*ve*sa*ri*a \cmark & ou*ri*ve*sa*ri-a \xmark \\
ousadia & ou*sa*di*a \cmark & ou*sa*di-a \xmark \\
ouvidoria & ou*vi*do*ri*a \cmark & ou*vi*do*ri-a \xmark \\
ovídeo & o*ví*de.o \xmark & o*ví*deo \cmark \\
oxiácido & o*xi-á*ci*do \xmark & o*xi*á*ci*do \cmark \\
oxigênio & o*xi*gê*ni.o \xmark & o*xi*gê*nio \cmark \\
ozônio & o-zô*ni.o \xmark & o*zô*nio \cmark \\
paciência & pa*ci*ên*ci.a \xmark & pa*ci*ên*cia \cmark \\
padaria & pa*da*ri*a \cmark & pa*da*ri-a \xmark \\
pagadoria & pa*ga*do*ri*a \cmark & pa*ga*do*ri-a \xmark \\
painço & pa-in*ço \xmark & pa-in*ço \xmark \\
painho & pa-i*nho \xmark & pa*i*nho \cmark \\
paio & pai*o \cmark & pai-o \xmark \\
país & pa-ís \xmark & pa*ís \cmark \\
pajeú & pa*je-ú \xmark & pa*je*ú \cmark \\
palácio & pa*lá*ci.o \xmark & pa*lá*cio \cmark \\
paládio & pa*lá*di.o \xmark & pa*lá*dio \cmark \\
palavrório & pa*la*vró*ri.o \xmark & pa*la*vró*rio \cmark \\
paleantropologia & pa*le*an*tro*po*lo*gi*a \cmark & pa*le*an*tro*po*lo*gi-a \xmark \\
paleoclimatologia & pa*le*o*cli*ma*to*lo*gi*a \cmark & pa*le*o*cli*ma*to*lo*gi-a \xmark \\
paleogeografia & pa*le*o*ge*o*gra*fi*a \cmark & pa*le*o*ge*o*gra*fi-a \xmark \\
paleografia & pa*le*o*gra*fi*a \cmark & pa*le*o*gra*fi-a \xmark \\
paleógrafo & pa*le-ó*gra*fo \xmark & pa*le*ó*gra*fo \cmark \\
paleólogo & pa*le-ó*lo*go \xmark & pa*le*ó*lo*go \cmark \\
paleontologia & pa*le*on*to*lo*gi*a \cmark & pa*le*on*to*lo*gi-a \xmark \\
paleozoologia & pa*le*o*zo*o*lo*gi*a \cmark & pa*le*o*zo*o*lo*gi-a \xmark \\
palinologia & pa*li*no*lo*gi*a \cmark & pa*li*no*lo*gi-a \xmark \\
pálio & pá*li.o \xmark & pá*lio \cmark \\
palmatória & pal*ma*tó*ri.a \xmark & pal*ma*tó*ria \cmark \\
panaceia & pa*na*cei*a \cmark & pa*na*cei-a \xmark \\
panaria & pa*na*ri*a \cmark & pa*na*ri-a \xmark \\
pancadaria & pan*ca*da*ri*a \cmark & pan*ca*da*ri-a \xmark \\
pancrácio & pan*crá*ci.o \xmark & pan*crá*cio \cmark \\
pancreático & pan*cre-á*ti*co \xmark & pan*cre*á*ti*co \cmark \\
pandemia & pan*de*mi*a \cmark & pan*de*mi-a \xmark \\
pandemônio & pan*de-mô*ni.o \xmark & pan*de*mô*nio \cmark \\
panenteísmo & pa*nen*te-ís*mo \xmark & pa*nen*te*ís*mo \cmark \\
panfletário & pan*fle*tá*ri*o \cmark & pan*fle*tá*ri-o \xmark \\
pangeia & pan*gei*a \cmark & pan*gei-a \xmark \\
panô & pa-nô \xmark & pa*nô \cmark \\
panóplia & pa*nó*pli.a \xmark & pa*nó*plia \cmark \\
panspermia & pans*per*mi*a \cmark & pans*per*mi-a \xmark \\
pantagruélico & pan*ta*gru-é*li*co \xmark & pan*ta*gru*é*li*co \cmark \\
pantaleão & pan*ta*le-ão \xmark & pan*ta*le*ão \cmark \\
panteão & pan*te-ão \xmark & pan*te*ão \cmark \\
panteísmo & pan*te-ís*mo \xmark & pan*te*ís*mo \cmark \\
panteísta & pan*te-ís*ta \xmark & pan*te*ís*ta \cmark \\
papagaio & pa*pa*gai*o \cmark & pa*pa*gai-o \xmark \\
papagueio & pa*pa*guei*o \cmark & pa*pa*guei-o \xmark \\
papaia & pa*pai*a \cmark & pa*pai-a \xmark \\
papaína & pa*pa-í*na \xmark & pa*pa*í*na \cmark \\
papelaria & pa*pe*la*ri*a \cmark & pa*pe*la*ri-a \xmark \\
papelório & pa*pe*ló*ri.o \xmark & pa*pe*ló*rio \cmark \\
papua & pa*pu*a \cmark & pa*pu-a \xmark \\
papuásio & pa*pu-á*si.o \xmark & pa*pu*á*sio \cmark \\
paquímetro & pa-quí*me*tro \xmark & pa*quí*me*tro \cmark \\
paradisíaco & pa*ra*di*sí-a*co \xmark & pa*ra*di*sí*a*co \cmark \\
parafernália & pa*ra*fer*ná*li.a \xmark & pa*ra*fer*ná*lia \cmark \\
parafilia & pa*ra*fi*li*a \cmark & pa*ra*fi*li-a \xmark \\
paraguaia & pa*ra*guai*a \cmark & pa*ra*guai-a \xmark \\
paraguaio & pa*ra*guai*o \cmark & pa*ra*guai-o \xmark \\
paraibano & pa*ra-i*ba*no \xmark & pa*ra-i*ba*no \xmark \\
paraíba & pa*ra-í*ba \xmark & pa*ra*í*ba \cmark \\
paraíso & pa*ra-í*so \xmark & pa*ra*í*so \cmark \\
paralipômenos & pa*ra*li-pô*me*nos \xmark & pa*ra*li*pô*me*nos \cmark \\
paralisia & pa*ra*li*si*a \cmark & pa*ra*li*si-a \xmark \\
paranoia & pa*ra*noi*a \cmark & pa*ra*noi-a \xmark \\
paranomásia & pa*ra*no*má*si.a \xmark & pa*ra*no*má*sia \cmark \\
paraplegia & pa*ra*ple*gi*a \cmark & pa*ra*ple*gi-a \xmark \\
parasitário & pa*ra*si*tá*ri.o \xmark & pa*ra*si*tá*rio \cmark \\
parasitologia & pa*ra*si*to*lo*gi*a \cmark & pa*ra*si*to*lo*gi-a \xmark \\
parceria & par*ce*ri*a \cmark & par*ce*ri-a \xmark \\
parcimônia & par*ci-mô*ni.a \xmark & par*ci*mô*nia \cmark \\
pareia & pa*rei*a \cmark & pa*rei-a \xmark \\
parélio & pa*ré*li.o \xmark & pa*ré*lio \cmark \\
páreo & pá*re.o \xmark & pá*reo \cmark \\
paresia & pa*re*si*a \cmark & pa*re*si-a \xmark \\
parestesia & pa*res*te*si*a \cmark & pa*res*te*si-a \xmark \\
pária & pá*ri.a \xmark & pá*ria \cmark \\
paritário & pa*ri*tá*ri*o \cmark & pa*ri*tá*ri-o \xmark \\
parlamentário & par*la*men*tá*ri.o \xmark & par*la*men*tá*rio \cmark \\
parlapatório & par*la*pa*tó*ri.o \xmark & par*la*pa*tó*rio \cmark \\
parlatório & par*la*tó*ri.o \xmark & par*la*tó*rio \cmark \\
parnaíba & par*na-í*ba \xmark & par*na*í*ba \cmark \\
paródia & pa*ró*di.a \xmark & pa*ró*dia \cmark \\
paroníquia & pa*ro*ní*qui*a \cmark & pa*ro*ní*qui-a \xmark \\
paronomásia & pa*ro*no*má*si.a \xmark & pa*ro*no*má*sia \cmark \\
paróquia & pa*ró*qui.a \xmark & pa*ró*quia \cmark \\
parquê & par-quê \xmark & par*quê \cmark \\
parquímetro & par-quí*me*tro \xmark & par*quí*me*tro \cmark \\
parricídio & par*ri*cí*di.o \xmark & par*ri*cí*dio \cmark \\
particípio & par*ti*cí*pi.o \xmark & par*ti*cí*pio \cmark \\
partidário & par*ti*dá*ri.o \xmark & par*ti*dá*rio \cmark \\
partidocracia & par*ti*do*cra*ci*a \cmark & par*ti*do*cra*ci-a \xmark \\
parusia & pa*ru*si*a \cmark & pa*ru*si-a \xmark \\
parvoíce & par*vo-í*ce \xmark & par*vo*í*ce \cmark \\
pascácio & pas*cá*ci.o \xmark & pas*cá*cio \cmark \\
passamanaria & pas*sa*ma*na*ri*a \cmark & pas*sa*ma*na*ri-a \xmark \\
passeio & pas*sei*o \cmark & pas*sei-o \xmark \\
pastelaria & pas*te*la*ri*a \cmark & pas*te*la*ri-a \xmark \\
pastifício & pas*ti*fí*ci.o \xmark & pas*ti*fí*cio \cmark \\
pastoreio & pas*to*rei*o \cmark & pas*to*rei-o \xmark \\
pastorícia & pas*to*rí*ci.a \xmark & pas*to*rí*cia \cmark \\
patágio & pa*tá*gi.o \xmark & pa*tá*gio \cmark \\
patauá & pa*tau-á \xmark & pa*tau*á \cmark \\
patchuli & pat*chu*li \cmark & pat-chu*li \xmark \\
patenteável & pa*ten*te-á*vel \xmark & pa*ten*te*á*vel \cmark \\
pátio & pá*ti.o \xmark & pá*tio \cmark \\
patoá & pa*to-á \xmark & pa*to*á \cmark \\
patogenia & pa*to*ge*ni*a \cmark & pa*to*ge*ni-a \xmark \\
patologia & pa*to*lo*gi*a \cmark & pa*to*lo*gi-a \xmark \\
pátria & pá*tri.a \xmark & pá*tria \cmark \\
patrícia & pa*trí*ci.a \xmark & pa*trí*cia \cmark \\
patrício & pa*trí*ci.o \xmark & pa*trí*cio \cmark \\
patrimônio & pa*tri-mô*ni.o \xmark & pa*tri*mô*nio \cmark \\
pátrio & pá*tri.o \xmark & pá*trio \cmark \\
patriótico & pa*tri-ó*ti*co \xmark & pa*tri*ó*ti*co \cmark \\
patroa & pa*tro*a \cmark & pa*tro-a \xmark \\
patrocínio & pa*tro*cí*ni*o \cmark & pa*tro*cí*ni-o \xmark \\
patrologia & pa*tro*lo*gi*a \cmark & pa*tro*lo*gi-a \xmark \\
patuá & pa*tu-á \xmark & pa*tu*á \cmark \\
patuleia & pa*tu*lei*a \cmark & pa*tu*lei-a \xmark \\
paulínia & pau*lí*ni.a \xmark & pau*lí*nia \cmark \\
paul & pa-ul \xmark & pa-ul \xmark \\
pavia & pa*vi*a \cmark & pa*vi-a \xmark \\
pavio & pa*vi*o \cmark & pa*vi-o \xmark \\
pavoa & pa*vo*a \cmark & pa*vo-a \xmark \\
paxiúba & pa*xi-ú*ba \xmark & pa*xi*ú*ba \cmark \\
peão & pe-ão \xmark & pe*ão \cmark \\
peã & pe-ã \xmark & pe*ã \cmark \\
pecíolo & pe*cí-o*lo \xmark & pe*cí*o*lo \cmark \\
pecuária & pe*cu-á*ri.a \xmark & pe*cu*á*ria \cmark \\
pecuário & pe*cu-á*ri.o \xmark & pe*cu*á*rio \cmark \\
pecúlio & pe*cú*li.o \xmark & pe*cú*lio \cmark \\
pecúnia & pe*cú*ni.a \xmark & pe*cú*nia \cmark \\
pecuniário & pe*cu*ni-á*ri.o \xmark & pe*cu*ni*á*rio \cmark \\
pedágio & pe*dá*gi.o \xmark & pe*dá*gio \cmark \\
pedagogia & pe*da*go*gi*a \cmark & pe*da*go*gi-a \xmark \\
pederastia & pe*de*ras*ti*a \cmark & pe*de*ras*ti-a \xmark \\
pediatria & pe*di*a*tri*a \cmark & pe*di*a*tri-a \xmark \\
pediátrico & pe*di-á*tri*co \xmark & pe*di*á*tri*co \cmark \\
peditório & pe*di*tó*ri.o \xmark & pe*di*tó*rio \cmark \\
pedofilia & pe*do*fi*li*a \cmark & pe*do*fi*li-a \xmark \\
pedofobia & pe*do*fo*bi*a \cmark & pe*do*fo*bi-a \xmark \\
pedologia & pe*do*lo*gi*a \cmark & pe*do*lo*gi-a \xmark \\
pedraria & pe*dra*ri*a \cmark & pe*dra*ri-a \xmark \\
pegomancia & pe*go*man*ci*a \cmark & pe*go*man*ci-a \xmark \\
peia & pei*a \cmark & pei-a \xmark \\
peixaria & pei*xa*ri*a \cmark & pei*xa*ri-a \xmark \\
pelágio & pe*lá*gi.o \xmark & pe*lá*gio \cmark \\
pelúcia & pe*lú*ci.a \xmark & pe*lú*cia \cmark \\
pendência & pen*dên*ci.a \xmark & pen*dên*cia \cmark \\
penedia & pe*ne*di*a \cmark & pe*ne*di-a \xmark \\
peneplanície & pe*ne*pla*ní*ci.e \xmark & pe*ne*pla*ní*cie \cmark \\
penitência & pe*ni*tên*ci.a \xmark & pe*ni*tên*cia \cmark \\
penitenciaria & pe*ni*ten*ci*a*ri*a \cmark & pe*ni*ten*ci*a*ri-a \xmark \\
penitenciária & pe*ni*ten*ci-á*ri.a \xmark & pe*ni*ten*ci*á*ria \cmark \\
penitenciário & pe*ni*ten*ci-á*ri.o \xmark & pe*ni*ten*ci*á*rio \cmark \\
pensionário & pen*si*o*ná*ri.o \xmark & pen*si*o*ná*rio \cmark \\
pentacampeão & pen*ta*cam*pe-ão \xmark & pen*ta*cam*pe*ão \cmark \\
pentarquia & pen*tar*qui*a \cmark & pen*tar*qui-a \xmark \\
penúria & pe*nú*ri.a \xmark & pe*nú*ria \cmark \\
peoa & pe*o*a \cmark & pe*o-a \xmark \\
peônia & pe-ô*ni.a \xmark & pe*ô*nia \cmark \\
péon & pé-on \xmark & pé*on \cmark \\
peptídeo & pep*tí*de.o \xmark & pep*tí*deo \cmark \\
pequiá & pe*qui-á \xmark & pe*qui*á \cmark \\
percuciência & per*cu*ci*ên*ci.a \xmark & per*cu*ci*ên*cia \cmark \\
percutâneo & per*cu*tâ*ne.o \xmark & per*cu*tâ*neo \cmark \\
perdoável & per*do-á*vel \xmark & per*do*á*vel \cmark \\
perdulário & per*du*lá*ri.o \xmark & per*du*lá*rio \cmark \\
peremptório & pe*remp*tó*ri.o \xmark & pe*remp*tó*rio \cmark \\
perenifólio & pe*re*ni*fó*li.o \xmark & pe*re*ni*fó*lio \cmark \\
perequê & pe*re-quê \xmark & pe*re*quê \cmark \\
perfídia & per*fí*di.a \xmark & per*fí*dia \cmark \\
perfumaria & per*fu*ma*ri*a \cmark & per*fu*ma*ri-a \xmark \\
pericárdio & pe*ri*cár*di.o \xmark & pe*ri*cár*dio \cmark \\
perícia & pe*rí*ci.a \xmark & pe*rí*cia \cmark \\
pericôndrio & pe*ri-côn*dri.o \xmark & pe*ri*côn*drio \cmark \\
periélio & pe*ri-é*li.o \xmark & pe*ri*é*lio \cmark \\
periferia & pe*ri*fe*ri*a \cmark & pe*ri*fe*ri-a \xmark \\
perigônio & pe*ri-gô*ni.o \xmark & pe*ri*gô*nio \cmark \\
períneo & pe*rí*ne.o \xmark & pe*rí*neo \cmark \\
periódico & pe*ri-ó*di*co \xmark & pe*ri*ó*di*co \cmark \\
período & pe*rí-o*do \xmark & pe*rí*o*do \cmark \\
periósteo & pe*ri-ós*te.o \xmark & pe*ri*ós*teo \cmark \\
peripécia & pe*ri*pé*ci.a \xmark & pe*ri*pé*cia \cmark \\
periscópio & pe*ris*có*pi.o \xmark & pe*ris*có*pio \cmark \\
periurbano & pe*ri-ur*ba*no \xmark & pe*ri*ur*ba*no \cmark \\
perlocucionário & per*lo*cu*ci*o*ná*ri.o \xmark & per*lo*cu*ci*o*ná*rio \cmark \\
permanência & per*ma*nên*ci.a \xmark & per*ma*nên*cia \cmark \\
permeável & per*me-á*vel \xmark & per*me*á*vel \cmark \\
permeio & per*mei*o \cmark & per*mei-o \xmark \\
permissionário & per*mis*si*o*ná*ri.o \xmark & per*mis*si*o*ná*rio \cmark \\
perpétua & per*pé*tu.a \xmark & per*pé*tua \cmark \\
persecutório & per*se*cu*tó*ri.o \xmark & per*se*cu*tó*rio \cmark \\
persicária & per*si*cá*ri.a \xmark & per*si*cá*ria \cmark \\
pérsio & pér*si.o \xmark & pér*sio \cmark \\
persistência & per*sis*tên*ci.a \xmark & per*sis*tên*cia \cmark \\
perspicácia & pers*pi*cá*ci.a \xmark & pers*pi*cá*cia \cmark \\
pertinácia & per*ti*ná*ci.a \xmark & per*ti*ná*cia \cmark \\
pertinência & per*ti*nên*ci.a \xmark & per*ti*nên*cia \cmark \\
perua & pe*ru*a \cmark & pe*ru-a \xmark \\
pescaria & pes*ca*ri*a \cmark & pes*ca*ri-a \xmark \\
pessário & pes*sá*ri.o \xmark & pes*sá*rio \cmark \\
pessoa & pes*so*a \cmark & pes*so-a \xmark \\
pestilência & pes*ti*lên*ci.a \xmark & pes*ti*lên*cia \cmark \\
peticionário & pe*ti*ci*o*ná*ri.o \xmark & pe*ti*ci*o*ná*rio \cmark \\
pétreo & pé*tre.o \xmark & pé*treo \cmark \\
petrografia & pe*tro*gra*fi*a \cmark & pe*tro*gra*fi-a \xmark \\
petróleo & pe*tró*le.o \xmark & pe*tró*leo \cmark \\
petrologia & pe*tro*lo*gi*a \cmark & pe*tro*lo*gi-a \xmark \\
petroquímica & pe*tro-quí*mi*ca \xmark & pe*tro*quí*mi*ca \cmark \\
petroquímico & pe*tro-quí*mi*co \xmark & pe*tro*quí*mi*co \cmark \\
petulância & pe*tu*lân*ci.a \xmark & pe*tu*lân*cia \cmark \\
petúnia & pe*tú*ni*a \cmark & pe*tú*ni-a \xmark \\
pia & pi*a \cmark & pi-a \xmark \\
piá & pi-á \xmark & pi*á \cmark \\
piauiense & pi*au-i*en*se \xmark & pi*au*i*en*se \cmark \\
picardia & pi*car*di*a \cmark & pi*car*di-a \xmark \\
picaria & pi*ca*ri*a \cmark & pi*ca*ri-a \xmark \\
pictografia & pic*to*gra*fi*a \cmark & pic*to*gra*fi-a \xmark \\
picuá & pi*cu-á \xmark & pi*cu*á \cmark \\
picuinha & pi*cu-i*nha \xmark & pi*cu*i*nha \cmark \\
piério & pi-é*ri.o \xmark & pi*é*rio \cmark \\
píer & pí-er \xmark & pí*er \cmark \\
pífio & pí*fi*o \cmark & pí*fi-o \xmark \\
pigmeia & pig*mei*a \cmark & pig*mei-a \xmark \\
píleo & pí*le*o \cmark & pí*le-o \xmark \\
pilhéria & pi*lhé*ri*a \cmark & pi*lhé*ri-a \xmark \\
pindaíba & pin*da-í*ba \xmark & pin*da*í*ba \cmark \\
pintainho & pin*ta-i*nho \xmark & pin*ta*i*nho \cmark \\
píon & pí-on \xmark & pí*on \cmark \\
pio & pi*o \cmark & pi-o \xmark \\
piquê & pi-quê \xmark & pi*quê \cmark \\
piraí & pi*ra-í \xmark & pi*ra*í \cmark \\
pirataria & pi*ra*ta*ri*a \cmark & pi*ra*ta*ri-a \xmark \\
pirelióforo & pi*re*li-ó*fo*ro \xmark & pi*re*li*ó*fo*ro \cmark \\
piromancia & pi*ro*man*ci*a \cmark & pi*ro*man*ci-a \xmark \\
piromaníaco & pi*ro*ma*ní-a*co \xmark & pi*ro*ma*ní*a*co \cmark \\
piromania & pi*ro*ma*ni*a \cmark & pi*ro*ma*ni-a \xmark \\
pirômetro & pi-rô*me*tro \xmark & pi*rô*me*tro \cmark \\
pirotecnia & pi*ro*tec*ni*a \cmark & pi*ro*tec*ni-a \xmark \\
piruá & pi*ru-á \xmark & pi*ru*á \cmark \\
piscatória & pis*ca*tó*ri.a \xmark & pis*ca*tó*ria \cmark \\
piscatório & pis*ca*tó*ri.o \xmark & pis*ca*tó*rio \cmark \\
pisoteio & pi*so*tei*o \cmark & pi*so*tei-o \xmark \\
pítia & pí*ti.a \xmark & pí*tia \cmark \\
pitiríase & pi*ti*rí-a*se \xmark & pi*ti*rí*a*se \cmark \\
pium & pi-um \xmark & pi*um \cmark \\
piúria & pi-ú*ri.a \xmark & pi*ú*ria \cmark \\
pivô & pi-vô \xmark & pi*vô \cmark \\
pixaim & pi*xa-im \xmark & pi*xa-im \xmark \\
pizaria & pi*za*ri*a \cmark & pi*za*ri-a \xmark \\
pizzaria & piz*za*ri*a \cmark & piz*za*ri-a \xmark \\
placentário & pla*cen*tá*ri.o \xmark & pla*cen*tá*rio \cmark \\
plagiário & pla*gi-á*ri.o \xmark & pla*gi*á*rio \cmark \\
plagioclásio & pla*gi*o*clá*si.o \xmark & pla*gi*o*clá*sio \cmark \\
plágio & plá*gi.o \xmark & plá*gio \cmark \\
planctônico & planc-tô*ni*co \xmark & planc*tô*ni*co \cmark \\
planetário & pla*ne*tá*ri.o \xmark & pla*ne*tá*rio \cmark \\
planície & pla*ní*ci.e \xmark & pla*ní*cie \cmark \\
planimetria & pla*ni*me*tri*a \cmark & pla*ni*me*tri-a \xmark \\
planisfério & pla*nis*fé*ri.o \xmark & pla*nis*fé*rio \cmark \\
plantio & plan*ti*o \cmark & plan*ti-o \xmark \\
plasmódio & plas*mó*di.o \xmark & plas*mó*dio \cmark \\
plateia & pla*tei*a \cmark & pla*tei-a \xmark \\
platônico & pla-tô*ni*co \xmark & pla*tô*ni*co \cmark \\
platô & pla-tô \xmark & pla*tô \cmark \\
plebeísmo & ple*be-ís*mo \xmark & ple*be*ís*mo \cmark \\
pleiotropia & plei*o*tro*pi*a \cmark & plei*o*tro*pi-a \xmark \\
plenário & ple*ná*ri.o \xmark & ple*ná*rio \cmark \\
plenilúnio & ple*ni*lú*ni.o \xmark & ple*ni*lú*nio \cmark \\
plenipotenciário & ple*ni*po*ten*ci-á*ri.o \xmark & ple*ni*po*ten*ci*á*rio \cmark \\
pleurisia & pleu*ri*si*a \cmark & pleu*ri*si-a \xmark \\
plúmbeo & plúm*be.o \xmark & plúm*beo \cmark \\
pluripartidário & plu*ri*par*ti*dá*ri.o \xmark & plu*ri*par*ti*dá*rio \cmark \\
plutocracia & plu*to*cra*ci*a \cmark & plu*to*cra*ci-a \xmark \\
plutônio & plu-tô*ni.o \xmark & plu*tô*nio \cmark \\
pluviometria & plu*vi*o*me*tri*a \cmark & plu*vi*o*me*tri-a \xmark \\
plúvio & plú*vi.o \xmark & plú*vio \cmark \\
pneuma & p.neu*ma \xmark & pneu*ma \cmark \\
pneumática & p.neu*má*ti*ca \xmark & pneu*má*ti*ca \cmark \\
pneumático & p.neu*má*ti*co \xmark & pneu*má*ti*co \cmark \\
pneumatologia & p.neu*ma*to*lo*gi*a \xmark & pneu*ma*to*lo*gi-a \xmark \\
pneumococo & p.neu*mo*co*co \xmark & pneu*mo*co*co \cmark \\
pneumoconiose & p.neu*mo*co*ni*o*se \xmark & pneu*mo*co*ni*o*se \cmark \\
pneumogástrico & p.neu*mo*gás*tri*co \xmark & pneu*mo*gás*tri*co \cmark \\
pneumologia & p.neu*mo*lo*gi*a \xmark & pneu*mo*lo*gi-a \xmark \\
pneumologista & p.neu*mo*lo*gis*ta \xmark & pneu*mo*lo*gis*ta \cmark \\
pneumonia & p.neu*mo*ni*a \xmark & pneu*mo*ni-a \xmark \\
pneumônico & p.neu-mô*ni*co \xmark & pneu*mô*ni*co \cmark \\
pneumonite & p.neu*mo*ni*te \xmark & pneu*mo*ni*te \cmark \\
pneumotórax & p.neu*mo*tó*rax \xmark & pneu*mo*tó*rax \cmark \\
pneu & p.neu \xmark & pneu \cmark \\
poaia & po*ai*a \cmark & po*ai-a \xmark \\
poa & po*a \cmark & po-a \xmark \\
poá & po-á \xmark & po*á \cmark \\
poderio & po*de*ri*o \cmark & po*de*ri-o \xmark \\
pódio & pó*di.o \xmark & pó*dio \cmark \\
podologia & po*do*lo*gi*a \cmark & po*do*lo*gi-a \xmark \\
poesia & po*e*si*a \cmark & po*e*si-a \xmark \\
poética & po-é*ti*ca \xmark & po*é*ti*ca \cmark \\
poético & po-é*ti*co \xmark & po*é*ti*co \cmark \\
poia & poi*a \cmark & poi-a \xmark \\
poio & poi*o \cmark & poi-o \xmark \\
polarimetria & po*la*ri*me*tri*a \cmark & po*la*ri*me*tri-a \xmark \\
poliandria & po*li*an*dri*a \cmark & po*li*an*dri-a \xmark \\
polia & po*li*a \cmark & po*li-a \xmark \\
polícia & po*lí*ci.a \xmark & po*lí*cia \cmark \\
policitemia & po*li*ci*te*mi*a \cmark & po*li*ci*te*mi-a \xmark \\
policromia & po*li*cro*mi*a \cmark & po*li*cro*mi-a \xmark \\
polidactilia & po*li*dac*ti*li*a \cmark & po*li*dac*ti*li-a \xmark \\
polidipsia & po*li*dip*si*a \cmark & po*li*dip*si-a \xmark \\
poliéster & po*li-és*ter \xmark & po*li*és*ter \cmark \\
polifagia & po*li*fa*gi*a \cmark & po*li*fa*gi-a \xmark \\
polifônico & po*li-fô*ni*co \xmark & po*li*fô*ni*co \cmark \\
poliginia & po*li*gi*ni*a \cmark & po*li*gi*ni-a \xmark \\
polimatia & po*li*ma*ti*a \cmark & po*li*ma*ti-a \xmark \\
polimorfia & po*li*mor*fi*a \cmark & po*li*mor*fi-a \xmark \\
polinésio & po*li*né*si.o \xmark & po*li*né*sio \cmark \\
polinômio & po*li-nô*mi.o \xmark & po*li*nô*mio \cmark \\
pólio & pó*li.o \xmark & pó*lio \cmark \\
poliploidia & po*li*ploi*di*a \cmark & po*li*ploi*di-a \xmark \\
polirritmia & po*lir*rit*mi*a \cmark & po*lir*rit*mi-a \xmark \\
polissacarídeo & po*lis*sa*ca*rí*de.o \xmark & po*lis*sa*ca*rí*deo \cmark \\
polissemia & po*lis*se*mi*a \cmark & po*lis*se*mi-a \xmark \\
polissonografia & po*lis*so*no*gra*fi*a \cmark & po*lis*so*no*gra*fi-a \xmark \\
politeísmo & po*li*te-ís*mo \xmark & po*li*te*ís*mo \cmark \\
politeísta & po*li*te-ís*ta \xmark & po*li*te*ís*ta \cmark \\
politiquês & po*li*ti-quês \xmark & po*li*ti*quês \cmark \\
poliuretano & po*li-u*re*ta*no \xmark & po*li*u*re*ta*no \cmark \\
poliúria & po*li-ú*ri.a \xmark & po*li*ú*ria \cmark \\
polivalência & po*li*va*lên*ci.a \xmark & po*li*va*lên*cia \cmark \\
polônio & po-lô*ni.o \xmark & po*lô*nio \cmark \\
poluição & po*lu-i*ção \xmark & po*lu*i*ção \cmark \\
poluidor & po*lu-i*dor \xmark & po*lu-i*dor \xmark \\
poluir & po*lu-ir \xmark & po*lu*ir \cmark \\
pomologia & po*mo*lo*gi*a \cmark & po*mo*lo*gi-a \xmark \\
pontaria & pon*ta*ri*a \cmark & pon*ta*ri-a \xmark \\
pontifício & pon*ti*fí*ci.o \xmark & pon*ti*fí*cio \cmark \\
poplíteo & po*plí*te.o \xmark & po*plí*teo \cmark \\
poraquê & po*ra-quê \xmark & po*ra*quê \cmark \\
porcaria & por*ca*ri*a \cmark & por*ca*ri-a \xmark \\
porciúncula & por*ci-ún*cu*la \xmark & por*ci*ún*cu*la \cmark \\
porfia & por*fi*a \cmark & por*fi-a \xmark \\
porfiria & por*fi*ri*a \cmark & por*fi*ri-a \xmark \\
pornografia & por*no*gra*fi*a \cmark & por*no*gra*fi-a \xmark \\
pornô & por-nô \xmark & por*nô \cmark \\
porquê & por-quê \xmark & por*quê \cmark \\
portaria & por*ta*ri*a \cmark & por*ta*ri-a \xmark \\
portfólio & port*fó*li.o \xmark & port*fó*lio \cmark \\
portuário & por*tu-á*ri.o \xmark & por*tu*á*rio \cmark \\
posologia & po*so*lo*gi*a \cmark & po*so*lo*gi-a \xmark \\
possuído & pos*su-í*do \xmark & pos*su*í*do \cmark \\
possuidor & pos*su-i*dor \xmark & pos*su*i*dor \cmark \\
possuir & pos*su-ir \xmark & pos*su*ir \cmark \\
postectomia & pos*tec*to*mi*a \cmark & pos*tec*to*mi-a \xmark \\
potássio & po*tás*si.o \xmark & po*tás*sio \cmark \\
potência & po*tên*ci.a \xmark & po*tên*cia \cmark \\
pousio & pou*si*o \cmark & pou*si-o \xmark \\
póvoa & pó*vo.a \xmark & pó*voa \cmark \\
pradaria & pra*da*ri*a \cmark & pra*da*ri-a \xmark \\
praia & prai*a \cmark & prai-a \xmark \\
prataria & pra*ta*ri*a \cmark & pra*ta*ri-a \xmark \\
preâmbulo & pre-âm*bu*lo \xmark & pre*âm*bu*lo \cmark \\
preá & pre-á \xmark & pre*á \cmark \\
precário & pre*cá*ri.o \xmark & pre*cá*rio \cmark \\
precatória & pre*ca*tó*ri.a \xmark & pre*ca*tó*ria \cmark \\
precatório & pre*ca*tó*ri.o \xmark & pre*ca*tó*rio \cmark \\
precedência & pre*ce*dên*ci.a \xmark & pre*ce*dên*cia \cmark \\
precipício & pre*ci*pí*ci.o \xmark & pre*ci*pí*cio \cmark \\
precípuo & pre*cí*pu.o \xmark & pre*cí*puo \cmark \\
predatório & pre*da*tó*ri.o \xmark & pre*da*tó*rio \cmark \\
prédio & pré*di.o \xmark & pré*dio \cmark \\
predominância & pre*do*mi*nân*ci.a \xmark & pre*do*mi*nân*cia \cmark \\
predomínio & pre*do*mí*ni.o \xmark & pre*do*mí*nio \cmark \\
preeminência & pre*e*mi*nên*ci.a \xmark & pre*e*mi*nên*cia \cmark \\
preênsil & pre-ên*sil \xmark & pre*ên*sil \cmark \\
preexistência & pre*e*xis*tên*ci.a \xmark & pre*e*xis*tên*cia \cmark \\
prefácio & pre*fá*ci.o \xmark & pre*fá*cio \cmark \\
preferência & pre*fe*rên*ci.a \xmark & pre*fe*rên*cia \cmark \\
pregaria & pre*ga*ri*a \cmark & pre*ga*ri-a \xmark \\
pregnância & preg*nân*ci.a \xmark & preg*nân*cia \cmark \\
preia & prei*a \cmark & prei-a \xmark \\
prejuízo & pre*ju-í*zo \xmark & pre*ju*í*zo \cmark \\
prelatício & pre*la*tí*ci.o \xmark & pre*la*tí*cio \cmark \\
prelazia & pre*la*zi*a \cmark & pre*la*zi-a \xmark \\
prélio & pré*li.o \xmark & pré*lio \cmark \\
prelúdio & pre*lú*di.o \xmark & pre*lú*dio \cmark \\
premência & pre*mên*ci.a \xmark & pre*mên*cia \cmark \\
premonitório & pre*mo*ni*tó*ri.o \xmark & pre*mo*ni*tó*rio \cmark \\
prenúncio & pre*nún*ci.o \xmark & pre*nún*cio \cmark \\
preparatório & pre*pa*ra*tó*ri.o \xmark & pre*pa*ra*tó*rio \cmark \\
preponderância & pre*pon*de*rân*ci.a \xmark & pre*pon*de*rân*cia \cmark \\
prepotência & pre*po*tên*ci.a \xmark & pre*po*tên*cia \cmark \\
prepúcio & pre*pú*ci.o \xmark & pre*pú*cio \cmark \\
presbiopia & pres*bi*o*pi*a \cmark & pres*bi*o*pi-a \xmark \\
presbitério & pres*bi*té*ri.o \xmark & pres*bi*té*rio \cmark \\
presciência & pres*ci*ên*ci.a \xmark & pres*ci*ên*cia \cmark \\
presépio & pre*sé*pi.o \xmark & pre*sé*pio \cmark \\
presidência & pre*si*dên*ci.a \xmark & pre*si*dên*cia \cmark \\
presidenciável & pre*si*den*ci-á*vel \xmark & pre*si*den*ci*á*vel \cmark \\
presidiário & pre*si*di-á*ri.o \xmark & pre*si*di*á*rio \cmark \\
presídio & pre*sí*di.o \xmark & pre*sí*dio \cmark \\
presságio & pres*sá*gi.o \xmark & pres*sá*gio \cmark \\
prestígio & pres*tí*gi.o \xmark & pres*tí*gio \cmark \\
presúria & pre*sú*ri.a \xmark & pre*sú*ria \cmark \\
pretoria & pre*to*ri*a \cmark & pre*to*ri-a \xmark \\
pretório & pre*tó*ri.o \xmark & pre*tó*rio \cmark \\
prevalência & pre*va*lên*ci.a \xmark & pre*va*lên*cia \cmark \\
preventório & pre*ven*tó*ri.o \xmark & pre*ven*tó*rio \cmark \\
previdenciário & pre*vi*den*ci-á*ri.o \xmark & pre*vi*den*ci*á*rio \cmark \\
prévio & pré*vi.o \xmark & pré*vio \cmark \\
priápico & pri-á*pi*co \xmark & pri*á*pi*co \cmark \\
primária & pri*má*ri.a \xmark & pri*má*ria \cmark \\
primário & pri*má*ri.o \xmark & pri*má*rio \cmark \\
primatologia & pri*ma*to*lo*gi*a \cmark & pri*ma*to*lo*gi-a \xmark \\
primazia & pri*ma*zi*a \cmark & pri*ma*zi-a \xmark \\
primicério & pri*mi*cé*ri.o \xmark & pri*mi*cé*rio \cmark \\
primórdio & pri*mór*di.o \xmark & pri*mór*dio \cmark \\
princípio & prin*cí*pi.o \xmark & prin*cí*pio \cmark \\
príon & prí-on \xmark & prí*on \cmark \\
prioritário & pri*o*ri*tá*ri.o \xmark & pri*o*ri*tá*rio \cmark \\
privilégio & pri*vi*lé*gi.o \xmark & pri*vi*lé*gio \cmark \\
proa & pro*a \cmark & pro-a \xmark \\
probatório & pro*ba*tó*ri.o \xmark & pro*ba*tó*rio \cmark \\
procaína & pro*ca-í*na \xmark & pro*ca*í*na \cmark \\
procariótico & pro*ca*ri-ó*ti*co \xmark & pro*ca*ri*ó*ti*co \cmark \\
procedência & pro*ce*dên*ci.a \xmark & pro*ce*dên*cia \cmark \\
procônsul & pro-côn*sul \xmark & pro*côn*sul \cmark \\
proctologia & proc*to*lo*gi*a \cmark & proc*to*lo*gi-a \xmark \\
procuradoria & pro*cu*ra*do*ri*a \cmark & pro*cu*ra*do*ri-a \xmark \\
prodígio & pro*dí*gi.o \xmark & pro*dí*gio \cmark \\
proeminência & pro*e*mi*nên*ci.a \xmark & pro*e*mi*nên*cia \cmark \\
proêmio & pro-ê*mi.o \xmark & pro*ê*mio \cmark \\
profecia & pro*fe*ci*a \cmark & pro*fe*ci-a \xmark \\
proficiência & pro*fi*ci*ên*ci.a \xmark & pro*fi*ci*ên*cia \cmark \\
profícuo & pro*fí*cu.o \xmark & pro*fí*cuo \cmark \\
profilaxia & pro*fi*la*xi*a \cmark & pro*fi*la*xi-a \xmark \\
progênie & pro*gê*ni.e \xmark & pro*gê*nie \cmark \\
proibição & pro-i*bi*ção \xmark & pro*i*bi*ção \cmark \\
proibido & pro-i*bi*do \xmark & pro*i*bi*do \cmark \\
proibir & pro-i*bir \xmark & pro*i*bir \cmark \\
proibitivo & pro-i*bi*ti*vo \xmark & pro*i*bi*ti*vo \cmark \\
prolegômenos & pro*le-gô*me*nos \xmark & pro*le*gô*me*nos \cmark \\
proletário & pro*le*tá*ri.o \xmark & pro*le*tá*rio \cmark \\
promécio & pro*mé*ci.o \xmark & pro*mé*cio \cmark \\
promissória & pro*mis*só*ri.a \xmark & pro*mis*só*ria \cmark \\
promontório & pro*mon*tó*ri.o \xmark & pro*mon*tó*rio \cmark \\
promotoria & pro*mo*to*ri*a \cmark & pro*mo*to*ri-a \xmark \\
prontuário & pron*tu-á*ri.o \xmark & pron*tu*á*rio \cmark \\
pronúncia & pro*nún*ci.a \xmark & pro*nún*cia \cmark \\
pronunciável & pro*nun*ci-á*vel \xmark & pro*nun*ci*á*vel \cmark \\
propiciatório & pro*pi*ci*a*tó*ri.o \xmark & pro*pi*ci*a*tó*rio \cmark \\
propício & pro*pí*ci.o \xmark & pro*pí*cio \cmark \\
proprietário & pro*pri*e*tá*ri.o \xmark & pro*pri*e*tá*rio \cmark \\
próprio & pró*pri.o \xmark & pró*prio \cmark \\
prosaísmo & pro*sa-ís*mo \xmark & pro*sa*ís*mo \cmark \\
proscênio & pros*cê*ni.o \xmark & pros*cê*nio \cmark \\
prosódia & pro*só*di.a \xmark & pro*só*dia \cmark \\
prosopagnosia & pro*so*pag*no*si*a \cmark & pro*so*pag*no*si-a \xmark \\
prosopografia & pro*so*po*gra*fi*a \cmark & pro*so*po*gra*fi-a \xmark \\
prosopopeia & pro*so*po*pei*a \cmark & pro*so*po*pei-a \xmark \\
prostatectomia & pros*ta*tec*to*mi*a \cmark & pros*ta*tec*to*mi-a \xmark \\
prostituição & pros*ti*tu-i*ção \xmark & pros*ti*tu*i*ção \cmark \\
prostituído & pros*ti*tu-í*do \xmark & pros*ti*tu*í*do \cmark \\
prostituir & pros*ti*tu-ir \xmark & pros*ti*tu*ir \cmark \\
protactínio & pro*tac*tí*ni.o \xmark & pro*tac*tí*nio \cmark \\
proteína & pro*te-í*na \xmark & pro*te*í*na \cmark \\
proteólise & pro*te-ó*li*se \xmark & pro*te*ó*li*se \cmark \\
prótio & pró*ti.o \xmark & pró*tio \cmark \\
protofonia & pro*to*fo*ni*a \cmark & pro*to*fo*ni-a \xmark \\
protoginia & pro*to*gi*ni*a \cmark & pro*to*gi*ni-a \xmark \\
protonotário & pro*to*no*tá*ri.o \xmark & pro*to*no*tá*rio \cmark \\
protozoário & pro*to*zo-á*ri.o \xmark & pro*to*zo*á*rio \cmark \\
protozoologia & pro*to*zo*o*lo*gi*a \cmark & pro*to*zo*o*lo*gi-a \xmark \\
protuberância & pro*tu*be*rân*ci.a \xmark & pro*tu*be*rân*cia \cmark \\
provedoria & pro*ve*do*ri*a \cmark & pro*ve*do*ri-a \xmark \\
proveniência & pro*ve*ni*ên*ci.a \xmark & pro*ve*ni*ên*cia \cmark \\
provérbio & pro*vér*bi.o \xmark & pro*vér*bio \cmark \\
providência & pro*vi*dên*ci.a \xmark & pro*vi*dên*cia \cmark \\
província & pro*vín*ci.a \xmark & pro*vín*cia \cmark \\
provisório & pro*vi*só*ri.o \xmark & pro*vi*só*rio \cmark \\
provocatório & pro*vo*ca*tó*ri.o \xmark & pro*vo*ca*tó*rio \cmark \\
prudência & pru*dên*ci.a \xmark & pru*dên*cia \cmark \\
pseudociência & p.seu*do*ci*ên*ci.a \xmark & pseu*do*ci*ên*cia \cmark \\
pseudocientífico & p.seu*do*ci*en*tí*fi*co \xmark & pseu*do*ci*en*tí*fi*co \cmark \\
pseudofruto & p.seu*do*fru*to \xmark & pseu*do*fru*to \cmark \\
pseudônimo & p.seu-dô*ni*mo \xmark & pseu*dô*ni*mo \cmark \\
psicadélico & p.si*ca*dé*li*co \xmark & psi*ca*dé*li*co \cmark \\
psicanálise & p.si*ca*ná*li*se \xmark & psi*ca*ná*li*se \cmark \\
psicanalista & p.si*ca*na*lis*ta \xmark & psi*ca*na*lis*ta \cmark \\
psicanalítico & p.si*ca*na*lí*ti*co \xmark & psi*ca*na*lí*ti*co \cmark \\
psicoativo & p.si*co*a*ti*vo \xmark & psi*co*a*ti*vo \cmark \\
psicocirurgia & p.si*co*ci*rur*gi*a \xmark & psi*co*ci*rur*gi-a \xmark \\
psicodélico & p.si*co*dé*li*co \xmark & psi*co*dé*li*co \cmark \\
psicodelismo & p.si*co*de*lis*mo \xmark & psi*co*de*lis*mo \cmark \\
psicodrama & p.si*co*dra*ma \xmark & psi*co*dra*ma \cmark \\
psicofarmacologia & p.si*co*far*ma*co*lo*gi*a \xmark & psi*co*far*ma*co*lo*gi-a \xmark \\
psicofísico & p.si*co*fí*si*co \xmark & psi*co*fí*si*co \cmark \\
psicofisiologia & p.si*co*fi*si*o*lo*gi*a \xmark & psi*co*fi*si*o*lo*gi-a \xmark \\
psicofonia & p.si*co*fo*ni*a \xmark & psi*co*fo*ni-a \xmark \\
psicografia & p.si*co*gra*fi*a \xmark & psi*co*gra*fi-a \xmark \\
psicógrafo & p.si*có*gra*fo \xmark & psi*có*gra*fo \cmark \\
psicolinguista & p.si*co*lin*guis*ta \xmark & psi*co*lin*guis*ta \cmark \\
psicolinguística & p.si*co*lin*guís*ti*ca \xmark & psi*co*lin*guís*ti*ca \cmark \\
psicologia & p.si*co*lo*gi*a \xmark & psi*co*lo*gi-a \xmark \\
psicologicamente & p.si*co*lo*gi*ca*men*te \xmark & psi*co*lo*gi*ca*men*te \cmark \\
psicológico & p.si*co*ló*gi*co \xmark & psi*co*ló*gi*co \cmark \\
psicologismo & p.si*co*lo*gis*mo \xmark & psi*co*lo*gis*mo \cmark \\
psicologista & p.si*co*lo*gis*ta \xmark & psi*co*lo*gis*ta \cmark \\
psicologizar & p.si*co*lo*gi*zar \xmark & psi*co*lo*gi*zar \cmark \\
psicólogo & p.si*có*lo*go \xmark & psi*có*lo*go \cmark \\
psicometria & p.si*co*me*tri*a \xmark & psi*co*me*tri-a \xmark \\
psicométrico & p.si*co*mé*tri*co \xmark & psi*co*mé*tri*co \cmark \\
psicomotor & p.si*co*mo*tor \xmark & psi*co*mo*tor \cmark \\
psicomotricidade & p.si*co*mo*tri*ci*da*de \xmark & psi*co*mo*tri*ci*da*de \cmark \\
psicomotricista & p.si*co*mo*tri*cis*ta \xmark & psi*co*mo*tri*cis*ta \cmark \\
psicopata & p.si*co*pa*ta \xmark & psi*co*pa*ta \cmark \\
psicopatia & p.si*co*pa*ti*a \xmark & psi*co*pa*ti-a \xmark \\
psicopático & p.si*co*pá*ti*co \xmark & psi*co*pá*ti*co \cmark \\
psicopatologia & p.si*co*pa*to*lo*gi*a \xmark & psi*co*pa*to*lo*gi-a \xmark \\
psicopatológico & p.si*co*pa*to*ló*gi*co \xmark & psi*co*pa*to*ló*gi*co \cmark \\
psicopedagogia & p.si*co*pe*da*go*gi*a \xmark & psi*co*pe*da*go*gi-a \xmark \\
psicopedagógico & p.si*co*pe*da*gó*gi*co \xmark & psi*co*pe*da*gó*gi*co \cmark \\
psicopedagogo & p.si*co*pe*da*go*go \xmark & psi*co*pe*da*go*go \cmark \\
psicopompo & p.si*co*pom*po \xmark & psi*co*pom*po \cmark \\
psicoprofilaxia & p.si*co*pro*fi*la*xi*a \xmark & psi*co*pro*fi*la*xi-a \xmark \\
psicose & p.si*co*se \xmark & psi*co*se \cmark \\
psicossexual & p.si*cos*se*xu*al \xmark & psi*cos*se*xu*al \cmark \\
psicossomática & p.si*cos*so*má*ti*ca \xmark & psi*cos*so*má*ti*ca \cmark \\
psicossomático & p.si*cos*so*má*ti*co \xmark & psi*cos*so*má*ti*co \cmark \\
psicotécnico & p.si*co*téc*ni*co \xmark & psi*co*téc*ni*co \cmark \\
psicoterapeuta & p.si*co*te*ra*peu*ta \xmark & psi*co*te*ra*peu*ta \cmark \\
psicoterapêutico & p.si*co*te*ra*pêu*ti*co \xmark & psi*co*te*ra*pêu*ti*co \cmark \\
psicoterapia & p.si*co*te*ra*pi*a \xmark & psi*co*te*ra*pi-a \xmark \\
psicoterápico & p.si*co*te*rá*pi*co \xmark & psi*co*te*rá*pi*co \cmark \\
psicótico & p.si*có*ti*co \xmark & psi*có*ti*co \cmark \\
psicotrópico & p.si*co*tró*pi*co \xmark & psi*co*tró*pi*co \cmark \\
psique & p.si*que \xmark & psi*que \cmark \\
psiquê & p.si-quê \xmark & psi*quê \cmark \\
psiquiatra & p.si*qui*a*tra \xmark & psi*qui*a*tra \cmark \\
psiquiatria & p.si*qui*a*tri*a \xmark & psi*qui*a*tri-a \xmark \\
psiquiátrico & p.si*qui-á*tri*co \xmark & psi*qui*á*tri*co \cmark \\
psíquico & p.sí*qui*co \xmark & psí*qui*co \cmark \\
psiquismo & p.si*quis*mo \xmark & psi*quis*mo \cmark \\
psitacídeo & p.si*ta*cí*de.o \xmark & psi*ta*cí*deo \cmark \\
psitaciforme & p.si*ta*ci*for*me \xmark & psi*ta*ci*for*me \cmark \\
psitacismo & p.si*ta*cis*mo \xmark & psi*ta*cis*mo \cmark \\
psoas & p.so*as \xmark & p.so*as \xmark \\
psoríase & p.so*rí-a*se \xmark & p.so*rí*a*se \xmark \\
pterígio & p.te*rí*gi.o \xmark & pte*rí*gio \cmark \\
pterodáctilo & p.te*ro*dác*ti*lo \xmark & pte*ro*dác*ti*lo \cmark \\
pterossauro & p.te*ros*sau*ro \xmark & pte*ros*sau*ro \cmark \\
ptialina & p.ti*a*li*na \xmark & pti*a*li*na \cmark \\
ptolemaico & p.to*le*mai*co \xmark & pto*le*mai*co \cmark \\
ptose & p.to*se \xmark & pto*se \cmark \\
pua & pu*a \cmark & pu-a \xmark \\
pubescência & pu*bes*cên*ci.a \xmark & pu*bes*cên*cia \cmark \\
publicitário & pu*bli*ci*tá*ri*o \cmark & pu*bli*ci*tá*ri-o \xmark \\
pudicícia & pu*di*cí*ci.a \xmark & pu*di*cí*cia \cmark \\
puerpério & pu*er*pé*ri.o \xmark & pu*er*pé*rio \cmark \\
puído & pu-í*do \xmark & pu*í*do \cmark \\
pulôver & pu-lô*ver \xmark & pu*lô*ver \cmark \\
pungência & pun*gên*ci.a \xmark & pun*gên*cia \cmark \\
purgatório & pur*ga*tó*ri.o \xmark & pur*ga*tó*rio \cmark \\
purificatório & pu*ri*fi*ca*tó*ri.o \xmark & pu*ri*fi*ca*tó*rio \cmark \\
purpúreo & pur*pú*re.o \xmark & pur*pú*reo \cmark \\
putaria & pu*ta*ri*a \cmark & pu*ta*ri-a \xmark \\
putrião & pu*tri-ão \xmark & pu*tri*ão \cmark \\
quadricromia & qua*dri*cro*mi*a \cmark & qua*dri*cro*mi-a \xmark \\
quadriênio & qua*dri*ê*ni.o \xmark & qua*dri*ê*nio \cmark \\
quadrifólio & qua*dri*fó*li.o \xmark & qua*dri*fó*lio \cmark \\
quadringentenário & qua*drin*gen*te*ná*ri.o \xmark & qua*drin*gen*te*ná*rio \cmark \\
quadrívio & qua*drí*vi.o \xmark & qua*drí*vio \cmark \\
quantia & quan*ti*a \cmark & quan*ti-a \xmark \\
quartzito & quart*zi*to \cmark & quart*zi*to \cmark \\
quartzo & quart*zo \cmark & quart*zo \cmark \\
quaternário & qua*ter*ná*ri.o \xmark & qua*ter*ná*rio \cmark \\
quaternião & qua*ter*ni-ão \xmark & qua*ter*ni*ão \cmark \\
quatriênio & qua*tri*ê*ni.o \xmark & qua*tri*ê*nio \cmark \\
quéchua & qué*chu*a \cmark & qué*chu-a \xmark \\
queijaria & quei*ja*ri*a \cmark & quei*ja*ri-a \xmark \\
querência & que*rên*ci.a \xmark & que*rên*cia \cmark \\
questionário & ques*ti*o*ná*ri.o \xmark & ques*ti*o*ná*rio \cmark \\
questiúncula & ques*ti-ún*cu*la \xmark & ques*ti*ún*cu*la \cmark \\
quiáltera & qui-ál*te*ra \xmark & qui*ál*te*ra \cmark \\
quíchua & quí*chu.a \xmark & quí*chua \cmark \\
quilohertz & qui*lo-hertz \xmark & qui*lo-hertz \xmark \\
quilômetro & qui-lô*me*tro \xmark & qui*lô*me*tro \cmark \\
quilowatt & qui*lo-watt \xmark & qui*lo-watt \xmark \\
quimiotaxia & qui*mi*o*ta*xi*a \cmark & qui*mi*o*ta*xi-a \xmark \\
quimioterapia & qui*mi*o*te*ra*pi*a \cmark & qui*mi*o*te*ra*pi-a \xmark \\
quinário & qui*ná*ri.o \xmark & qui*ná*rio \cmark \\
quinoa & qui*no*a \cmark & qui*no-a \xmark \\
quinquênio & quin-quê*ni.o \xmark & quin*quê*nio \cmark \\
quinquilharia & quin*qui*lha*ri*a \cmark & quin*qui*lha*ri-a \xmark \\
quintessência & quin*tes*sên*ci.a \xmark & quin*tes*sên*cia \cmark \\
quintílio & quin*tí*li.o \xmark & quin*tí*lio \cmark \\
quinzenário & quin*ze*ná*ri.o \xmark & quin*ze*ná*rio \cmark \\
quiproquó & qui*pro-quó \xmark & qui*pro*quó \cmark \\
quiromancia & qui*ro*man*ci*a \cmark & qui*ro*man*ci-a \xmark \\
quiropraxia & qui*ro*pra*xi*a \cmark & qui*ro*pra*xi-a \xmark \\
raciocínio & ra*ci*o*cí*ni.o \xmark & ra*ci*o*cí*nio \cmark \\
radiância & ra*di*ân*ci.a \xmark & ra*di*ân*cia \cmark \\
radiestesia & ra*di*es*te*si*a \cmark & ra*di*es*te*si-a \xmark \\
radioastronomia & ra*di*o*as*tro*no*mi*a \cmark & ra*di*o*as*tro*no*mi-a \xmark \\
radiobiologia & ra*di*o*bi*o*lo*gi*a \cmark & ra*di*o*bi*o*lo*gi-a \xmark \\
radiofonia & ra*di*o*fo*ni*a \cmark & ra*di*o*fo*ni-a \xmark \\
radiofrequência & ra*di*o*fre-quên*ci.a \xmark & ra*di*o*fre*quên*cia \cmark \\
radiogoniometria & ra*di*o*go*ni*o*me*tri*a \cmark & ra*di*o*go*ni*o*me*tri-a \xmark \\
radiografia & ra*di*o*gra*fi*a \cmark & ra*di*o*gra*fi-a \xmark \\
radiólise & ra*di-ó*li*se \xmark & ra*di*ó*li*se \cmark \\
radiologia & ra*di*o*lo*gi*a \cmark & ra*di*o*lo*gi-a \xmark \\
radiometria & ra*di*o*me*tri*a \cmark & ra*di*o*me*tri-a \xmark \\
radionuclídeo & ra*di*o*nu*clí*de.o \xmark & ra*di*o*nu*clí*deo \cmark \\
radioquímica & ra*di*o-quí*mi*ca \xmark & ra*di*o*quí*mi*ca \cmark \\
rádio & rá*di.o \xmark & rá*dio \cmark \\
radioscopia & ra*di*os*co*pi*a \cmark & ra*di*os*co*pi-a \xmark \\
radiotelefonia & ra*di*o*te*le*fo*ni*a \cmark & ra*di*o*te*le*fo*ni-a \xmark \\
radiotelegrafia & ra*di*o*te*le*gra*fi*a \cmark & ra*di*o*te*le*gra*fi-a \xmark \\
radiotelescópio & ra*di*o*te*les*có*pi.o \xmark & ra*di*o*te*les*có*pio \cmark \\
radioterapia & ra*di*o*te*ra*pi*a \cmark & ra*di*o*te*ra*pi-a \xmark \\
ráfia & rá*fi*a \cmark & rá*fi-a \xmark \\
raia & rai*a \cmark & rai-a \xmark \\
rainha & ra-i*nha \xmark & ra*i*nha \cmark \\
raio & rai*o \cmark & rai-o \xmark \\
raiz & ra-iz \xmark & ra*iz \cmark \\
ralídeo & ra*lí*de.o \xmark & ra*lí*deo \cmark \\
ramaria & ra*ma*ri*a \cmark & ra*ma*ri-a \xmark \\
ranário & ra*ná*ri.o \xmark & ra*ná*rio \cmark \\
rancharia & ran*cha*ri*a \cmark & ran*cha*ri-a \xmark \\
râncio & rân*ci.o \xmark & rân*cio \cmark \\
randômico & ran-dô*mi*co \xmark & ran*dô*mi*co \cmark \\
rapsódia & rap*só*di.a \xmark & rap*só*dia \cmark \\
raquianestesia & ra*qui*a*nes*te*si*a \cmark & ra*qui*a*nes*te*si-a \xmark \\
raquítico & ra-quí*ti*co \xmark & ra*quí*ti*co \cmark \\
rastreio & ras*trei*o \cmark & ras*trei-o \xmark \\
rataria & ra*ta*ri*a \cmark & ra*ta*ri-a \xmark \\
rateio & ra*tei*o \cmark & ra*tei-o \xmark \\
ravióli & ra*vi-ó*li \xmark & ra*vi*ó*li \cmark \\
razia & ra*zi*a \cmark & ra*zi-a \xmark \\
razoável & ra*zo-á*vel \xmark & ra*zo*á*vel \cmark \\
reacionário & re*a*ci*o*ná*ri.o \xmark & re*a*ci*o*ná*rio \cmark \\
rebeldia & re*bel*di*a \cmark & re*bel*di-a \xmark \\
rebelião & re*be*li-ão \xmark & re*be*li*ão \cmark \\
recaída & re*ca-í*da \xmark & re*ca*í*da \cmark \\
recair & re*ca-ir \xmark & re*ca*ir \cmark \\
recebedoria & re*ce*be*do*ri*a \cmark & re*ce*be*do*ri-a \xmark \\
receio & re*cei*o \cmark & re*cei-o \xmark \\
receituário & re*cei*tu-á*ri.o \xmark & re*cei*tu*á*rio \cmark \\
recheio & re*chei*o \cmark & re*chei-o \xmark \\
reciário & re*ci-á*ri.o \xmark & re*ci*á*rio \cmark \\
recôncavo & re-côn*ca*vo \xmark & re*côn*ca*vo \cmark \\
recôndito & re-côn*di*to \xmark & re*côn*di*to \cmark \\
reconstituição & re*cons*ti*tu-i*ção \xmark & re*cons*ti*tu*i*ção \cmark \\
reconstituinte & re*cons*ti*tu-in*te \xmark & re*cons*ti*tu*in*te \cmark \\
reconstituir & re*cons*ti*tu-ir \xmark & re*cons*ti*tu*ir \cmark \\
reconstruído & re*cons*tru-í*do \xmark & re*cons*tru*í*do \cmark \\
reconstruir & re*cons*tru-ir \xmark & re*cons*tru*ir \cmark \\
recorrência & re*cor*rên*ci.a \xmark & re*cor*rên*cia \cmark \\
recreio & re*crei*o \cmark & re*crei-o \xmark \\
recria & re*cri*a \cmark & re*cri-a \xmark \\
recua & re*cu*a \cmark & re*cu-a \xmark \\
recuo & re*cu*o \cmark & re*cu-o \xmark \\
rédea & ré*de*a \cmark & ré*de-a \xmark \\
redemoinho & re*de*mo-i*nho \xmark & re*de*mo*i*nho \cmark \\
redistribuir & re*dis*tri*bu-ir \xmark & re*dis*tri*bu*ir \cmark \\
redundância & re*dun*dân*ci.a \xmark & re*dun*dân*cia \cmark \\
reengenharia & re*en*ge*nha*ri*a \cmark & re*en*ge*nha*ri-a \xmark \\
reentrância & re*en*trân*ci.a \xmark & re*en*trân*cia \cmark \\
reenvio & re*en*vi*o \cmark & re*en*vi-o \xmark \\
reequilíbrio & re*e*qui*lí*bri.o \xmark & re*e*qui*lí*brio \cmark \\
refeitório & re*fei*tó*ri.o \xmark & re*fei*tó*rio \cmark \\
referência & re*fe*rên*ci.a \xmark & re*fe*rên*cia \cmark \\
referendário & re*fe*ren*dá*ri.o \xmark & re*fe*ren*dá*rio \cmark \\
refinaria & re*fi*na*ri*a \cmark & re*fi*na*ri-a \xmark \\
reflexologia & re*fle*xo*lo*gi*a \cmark & re*fle*xo*lo*gi-a \xmark \\
reflexoterapia & re*fle*xo*te*ra*pi*a \cmark & re*fle*xo*te*ra*pi-a \xmark \\
refluir & re*flu-ir \xmark & re*flu*ir \cmark \\
reformatório & re*for*ma*tó*ri.o \xmark & re*for*ma*tó*rio \cmark \\
refratário & re*fra*tá*ri.o \xmark & re*fra*tá*rio \cmark \\
refrigério & re*fri*gé*ri.o \xmark & re*fri*gé*rio \cmark \\
refúgio & re*fú*gi.o \xmark & re*fú*gio \cmark \\
regadio & re*ga*di*o \cmark & re*ga*di-o \xmark \\
regalia & re*ga*li*a \cmark & re*ga*li-a \xmark \\
regateio & re*ga*tei*o \cmark & re*ga*tei-o \xmark \\
regência & re*gên*ci.a \xmark & re*gên*cia \cmark \\
região & re*gi-ão \xmark & re*gi*ão \cmark \\
régia & ré*gi.a \xmark & ré*gia \cmark \\
regicídio & re*gi*cí*di.o \xmark & re*gi*cí*dio \cmark \\
régio & ré*gi.o \xmark & ré*gio \cmark \\
reimplantação & re-im*plan*ta*ção \xmark & re*im*plan*ta*ção \cmark \\
reimplantar & re-im*plan*tar \xmark & re*im*plan*tar \cmark \\
reimportar & re-im*por*tar \xmark & re*im*por*tar \cmark \\
reimpressão & re-im*pres*são \xmark & re*im*pres*são \cmark \\
reimpresso & re-im*pres*so \xmark & re*im*pres*so \cmark \\
reimprimir & re-im*pri*mir \xmark & re*im*pri*mir \cmark \\
reincidente & re-in*ci*den*te \xmark & re*in*ci*den*te \cmark \\
reincidir & re-in*ci*dir \xmark & re*in*ci*dir \cmark \\
reincorporação & re-in*cor*po*ra*ção \xmark & re*in*cor*po*ra*ção \cmark \\
reincorporar & re-in*cor*po*rar \xmark & re*in*cor*po*rar \cmark \\
reindexação & re-in*de*xa*ção \xmark & re-in*de*xa*ção \xmark \\
reindexar & re-in*de*xar \xmark & re-in*de*xar \xmark \\
reingressar & re-in*gres*sar \xmark & re*in*gres*sar \cmark \\
reingresso & re-in*gres*so \xmark & re*in*gres*so \cmark \\
reiniciar & re-i*ni*ci*ar \xmark & re-i*ni*ci*ar \xmark \\
reinquirir & re-in*qui*rir \xmark & re-in*qui*rir \xmark \\
reinscrever & re-ins*cre*ver \xmark & re*ins*cre*ver \cmark \\
reinserção & re-in*ser*ção \xmark & re*in*ser*ção \cmark \\
reinserir & re-in*se*rir \xmark & re*in*se*rir \cmark \\
reinstalação & re-ins*ta*la*ção \xmark & re*ins*ta*la*ção \cmark \\
reinstalar & re-ins*ta*lar \xmark & re*ins*ta*lar \cmark \\
reinstituir & re-ins*ti*tu-ir \xmark & re*ins*ti*tu*ir \cmark \\
reintegração & re-in*te*gra*ção \xmark & re*in*te*gra*ção \cmark \\
reintegrar & re-in*te*grar \xmark & re*in*te*grar \cmark \\
reinterpretação & re-in*ter*pre*ta*ção \xmark & re*in*ter*pre*ta*ção \cmark \\
reinterpretar & re-in*ter*pre*tar \xmark & re*in*ter*pre*tar \cmark \\
reintrodução & re-in*tro*du*ção \xmark & re*in*tro*du*ção \cmark \\
reintroduzir & re-in*tro*du*zir \xmark & re*in*tro*du*zir \cmark \\
reinvenção & re-in*ven*ção \xmark & re*in*ven*ção \cmark \\
reinventar & re-in*ven*tar \xmark & re*in*ven*tar \cmark \\
reinvestir & re-in*ves*tir \xmark & re*in*ves*tir \cmark \\
reio & rei*o \cmark & rei-o \xmark \\
reitoria & rei*to*ri*a \cmark & rei*to*ri-a \xmark \\
reivindicatório & rei*vin*di*ca*tó*ri*o \cmark & rei*vin*di*ca*tó*ri-o \xmark \\
relatoria & re*la*to*ri*a \cmark & re*la*to*ri-a \xmark \\
relatório & re*la*tó*ri.o \xmark & re*la*tó*rio \cmark \\
relevância & re*le*vân*ci.a \xmark & re*le*vân*cia \cmark \\
relicário & re*li*cá*ri.o \xmark & re*li*cá*rio \cmark \\
religião & re*li*gi-ão \xmark & re*li*gi*ão \cmark \\
relíquia & re*lí*qui.a \xmark & re*lí*quia \cmark \\
relógio & re*ló*gi.o \xmark & re*ló*gio \cmark \\
relojoaria & re*lo*jo*a*ri*a \cmark & re*lo*jo*a*ri-a \xmark \\
relutância & re*lu*tân*ci.a \xmark & re*lu*tân*cia \cmark \\
remanência & re*ma*nên*ci.a \xmark & re*ma*nên*cia \cmark \\
remediável & re*me*di-á*vel \xmark & re*me*di*á*vel \cmark \\
remédio & re*mé*di.o \xmark & re*mé*dio \cmark \\
remígio & re*mí*gi.o \xmark & re*mí*gio \cmark \\
reminiscência & re*mi*nis*cên*ci.a \xmark & re*mi*nis*cên*cia \cmark \\
remuneratório & re*mu*ne*ra*tó*ri.o \xmark & re*mu*ne*ra*tó*rio \cmark \\
rênio & rê*ni.o \xmark & rê*nio \cmark \\
renitência & re*ni*tên*ci.a \xmark & re*ni*tên*cia \cmark \\
renúncia & re*nún*ci.a \xmark & re*nún*cia \cmark \\
reologia & re*o*lo*gi*a \cmark & re*o*lo*gi-a \xmark \\
repertório & re*per*tó*ri.o \xmark & re*per*tó*rio \cmark \\
repetência & re*pe*tên*ci.a \xmark & re*pe*tên*cia \cmark \\
replantio & re*plan*ti*o \cmark & re*plan*ti-o \xmark \\
repositório & re*po*si*tó*ri.o \xmark & re*po*si*tó*rio \cmark \\
represália & re*pre*sá*li.a \xmark & re*pre*sá*lia \cmark \\
reprografia & re*pro*gra*fi*a \cmark & re*pro*gra*fi-a \xmark \\
repúdio & re*pú*di.o \xmark & re*pú*dio \cmark \\
repugnância & re*pug*nân*ci.a \xmark & re*pug*nân*cia \cmark \\
reservatório & re*ser*va*tó*ri.o \xmark & re*ser*va*tó*rio \cmark \\
residência & re*si*dên*ci.a \xmark & re*si*dên*cia \cmark \\
resiliência & re*si*li*ên*ci.a \xmark & re*si*li*ên*cia \cmark \\
resistência & re*sis*tên*ci.a \xmark & re*sis*tên*cia \cmark \\
respiratório & res*pi*ra*tó*ri.o \xmark & res*pi*ra*tó*rio \cmark \\
responsório & res*pon*só*ri.o \xmark & res*pon*só*rio \cmark \\
resquício & res-quí*ci.o \xmark & res*quí*cio \cmark \\
ressonância & res*so*nân*ci.a \xmark & res*so*nân*cia \cmark \\
ressurgência & res*sur*gên*ci.a \xmark & res*sur*gên*cia \cmark \\
restituição & res*ti*tu-i*ção \xmark & res*ti*tu*i*ção \cmark \\
restituir & res*ti*tu-ir \xmark & res*ti*tu*ir \cmark \\
restituível & res*ti*tu-í*vel \xmark & res*ti*tu*í*vel \cmark \\
retaliatório & re*ta*li*a*tó*ri.o \xmark & re*ta*li*a*tó*rio \cmark \\
retardatário & re*tar*da*tá*ri.o \xmark & re*tar*da*tá*rio \cmark \\
reticência & re*ti*cên*ci.a \xmark & re*ti*cên*cia \cmark \\
retilíneo & re*ti*lí*ne.o \xmark & re*ti*lí*neo \cmark \\
retinopatia & re*ti*no*pa*ti*a \cmark & re*ti*no*pa*ti-a \xmark \\
retraído & re*tra-í*do \xmark & re*tra*í*do \cmark \\
retrair & re*tra-ir \xmark & re*tra*ir \cmark \\
retribuição & re*tri*bu-i*ção \xmark & re*tri*bu*i*ção \cmark \\
retribuir & re*tri*bu-ir \xmark & re*tri*bu*ir \cmark \\
retrosaria & re*tro*sa*ri*a \cmark & re*tro*sa*ri-a \xmark \\
retumbância & re*tum*bân*ci.a \xmark & re*tum*bân*cia \cmark \\
reumatologia & reu*ma*to*lo*gi*a \cmark & reu*ma*to*lo*gi-a \xmark \\
reunir & re-u*nir \xmark & re-u*nir \xmark \\
reurbanização & re-ur*ba*ni*za*ção \xmark & re-ur*ba*ni*za*ção \xmark \\
reurbanizar & re-ur*ba*ni*zar \xmark & re-ur*ba*ni*zar \xmark \\
reutilizar & re-u*ti*li*zar \xmark & re-u*ti*li*zar \xmark \\
revelia & re*ve*li*a \cmark & re*ve*li-a \xmark \\
reverência & re*ve*rên*ci.a \xmark & re*ve*rên*cia \cmark \\
revisório & re*vi*só*ri.o \xmark & re*vi*só*rio \cmark \\
revivência & re*vi*vên*ci.a \xmark & re*vi*vên*cia \cmark \\
revivescência & re*vi*ves*cên*ci.a \xmark & re*vi*ves*cên*cia \cmark \\
revogatório & re*vo*ga*tó*ri.o \xmark & re*vo*ga*tó*rio \cmark \\
revolucionário & re*vo*lu*ci*o*ná*ri.o \xmark & re*vo*lu*ci*o*ná*rio \cmark \\
ria & ri*a \cmark & ri-a \xmark \\
ridicularia & ri*di*cu*la*ri*a \cmark & ri*di*cu*la*ri-a \xmark \\
rinoplastia & ri*no*plas*ti*a \cmark & ri*no*plas*ti-a \xmark \\
rinorreia & ri*nor*rei*a \cmark & ri*nor*rei-a \xmark \\
rinoscopia & ri*nos*co*pi*a \cmark & ri*nos*co*pi-a \xmark \\
rio & ri*o \cmark & ri-o \xmark \\
ritidoplastia & ri*ti*do*plas*ti*a \cmark & ri*ti*do*plas*ti-a \xmark \\
robô & ro-bô \xmark & ro*bô \cmark \\
rocio & ro*ci*o \cmark & ro*ci-o \xmark \\
rócio & ró*ci.o \xmark & ró*cio \cmark \\
rodamoinho & ro*da*mo-i*nho \xmark & ro*da*mo*i*nho \cmark \\
rodeio & ro*dei*o \cmark & ro*dei-o \xmark \\
ródio & ró*di.o \xmark & ró*dio \cmark \\
rodízio & ro*dí*zi.o \xmark & ro*dí*zio \cmark \\
rodopio & ro*do*pi*o \cmark & ro*do*pi-o \xmark \\
rodoviária & ro*do*vi-á*ri.a \xmark & ro*do*vi*á*ria \cmark \\
rodoviário & ro*do*vi-á*ri.o \xmark & ro*do*vi*á*rio \cmark \\
rodovia & ro*do*vi*a \cmark & ro*do*vi-a \xmark \\
rogatória & ro*ga*tó*ri.a \xmark & ro*ga*tó*ria \cmark \\
roído & ro-í*do \xmark & ro*í*do \cmark \\
romaria & ro*ma*ri*a \cmark & ro*ma*ri-a \xmark \\
rosácea & ro*sá*ce.a \xmark & ro*sá*cea \cmark \\
rosário & ro*sá*ri.o \xmark & ro*sá*rio \cmark \\
roséola & ro*sé-o*la \xmark & ro*sé*o*la \cmark \\
róseo & ró*se.o \xmark & ró*seo \cmark \\
rossio & ros*si*o \cmark & ros*si-o \xmark \\
rotário & ro*tá*ri.o \xmark & ro*tá*rio \cmark \\
rotatório & ro*ta*tó*ri.o \xmark & ro*ta*tó*rio \cmark \\
rotisseria & ro*tis*se*ri*a \cmark & ro*tis*se*ri-a \xmark \\
rouparia & rou*pa*ri*a \cmark & rou*pa*ri-a \xmark \\
ruão & ru-ão \xmark & ru*ão \cmark \\
rua & ru*a \cmark & ru-a \xmark \\
rubéola & ru*bé-o*la \xmark & ru*bé*o*la \cmark \\
rúbeo & rú*be.o \xmark & rú*beo \cmark \\
rubiácea & ru*bi-á*ce.a \xmark & ru*bi*á*cea \cmark \\
rubídio & ru*bí*di.o \xmark & ru*bí*dio \cmark \\
rufião & ru*fi-ão \xmark & ru*fi*ão \cmark \\
rúfio & rú*fi.o \xmark & rú*fio \cmark \\
ruído & ru-í*do \xmark & ru*í*do \cmark \\
ruim & ru-im \xmark & ru-im \xmark \\
ruína & ru-í*na \xmark & ru*í*na \cmark \\
ruindade & ru-in*da*de \xmark & ru*in*da*de \cmark \\
ruinoso & ru-i*no*so \xmark & ru*i*no*so \cmark \\
ruir & ru-ir \xmark & ru*ir \cmark \\
rupia & ru*pi*a \cmark & ru*pi-a \xmark \\
rúpia & rú*pi.a \xmark & rú*pia \cmark \\
russofilia & rus*so*fi*li*a \cmark & rus*so*fi*li-a \xmark \\
russofobia & rus*so*fo*bi*a \cmark & rus*so*fo*bi-a \xmark \\
rutênio & ru*tê*ni.o \xmark & ru*tê*nio \cmark \\
rutherford & ru-ther*ford \xmark & ru-ther*ford \xmark \\
sabedoria & sa*be*do*ri*a \cmark & sa*be*do*ri-a \xmark \\
sabiá & sa*bi-á \xmark & sa*bi*á \cmark \\
sábia & sá*bi.a \xmark & sá*bia \cmark \\
sábio & sá*bi.o \xmark & sá*bio \cmark \\
saboaria & sa*bo*a*ri*a \cmark & sa*bo*a*ri-a \xmark \\
sacaria & sa*ca*ri*a \cmark & sa*ca*ri-a \xmark \\
sacerdócio & sa*cer*dó*ci.o \xmark & sa*cer*dó*cio \cmark \\
sacramentário & sa*cra*men*tá*ri*o \cmark & sa*cra*men*tá*ri-o \xmark \\
sacrário & sa*crá*ri.o \xmark & sa*crá*rio \cmark \\
sacrifício & sa*cri*fí*ci.o \xmark & sa*cri*fí*cio \cmark \\
sacrilégio & sa*cri*lé*gi.o \xmark & sa*cri*lé*gio \cmark \\
sacristia & sa*cris*ti*a \cmark & sa*cris*ti-a \xmark \\
sadio & sa*di*o \cmark & sa*di-o \xmark \\
safaria & sa*fa*ri*a \cmark & sa*fa*ri-a \xmark \\
sagitário & sa*gi*tá*ri.o \xmark & sa*gi*tá*rio \cmark \\
saião & sai-ão \xmark & sai*ão \cmark \\
saia & sai*a \cmark & sai-a \xmark \\
saída & sa-í*da \xmark & sa*í*da \cmark \\
saído & sa-í*do \xmark & sa*í*do \cmark \\
sainha & sa-i*nha \xmark & sa*i*nha \cmark \\
sainte & sa-in*te \xmark & sa-in*te \xmark \\
saio & sai*o \cmark & sai-o \xmark \\
saíra & sa-í*ra \xmark & sa*í*ra \cmark \\
sair & sa-ir \xmark & sa*ir \cmark \\
saí & sa-í \xmark & sa*í \cmark \\
salácia & sa*lá*ci.a \xmark & sa*lá*cia \cmark \\
salafrário & sa*la*frá*ri.o \xmark & sa*la*frá*rio \cmark \\
salário & sa*lá*ri.o \xmark & sa*lá*rio \cmark \\
saliência & sa*li*ên*ci.a \xmark & sa*li*ên*cia \cmark \\
sálio & sá*li.o \xmark & sá*lio \cmark \\
salmodia & sal*mo*di*a \cmark & sal*mo*di-a \xmark \\
saloio & sa*loi*o \cmark & sa*loi-o \xmark \\
salomônico & sa*lo-mô*ni*co \xmark & sa*lo*mô*ni*co \cmark \\
salsicharia & sal*si*cha*ri*a \cmark & sal*si*cha*ri-a \xmark \\
saltério & sal*té*ri.o \xmark & sal*té*rio \cmark \\
samambaia & sa*mam*bai*a \cmark & sa*mam*bai-a \xmark \\
samário & sa*má*ri.o \xmark & sa*má*rio \cmark \\
samaúma & sa*ma-ú*ma \xmark & sa*ma*ú*ma \cmark \\
sambaíba & sam*ba-í*ba \xmark & sam*ba*í*ba \cmark \\
sanatório & sa*na*tó*ri.o \xmark & sa*na*tó*rio \cmark \\
sandália & san*dá*li.a \xmark & san*dá*lia \cmark \\
sandia & san*di*a \cmark & san*di-a \xmark \\
sanduicheria & san*du-i*che*ri*a \xmark & san*du*i*che*ri-a \xmark \\
sanduíche & san*du-í*che \xmark & san*du*í*che \cmark \\
sangria & san*gri*a \cmark & san*gri-a \xmark \\
sanguinária & san*gui*ná*ri.a \xmark & san*gui*ná*ria \cmark \\
sanguinário & san*gui*ná*ri.o \xmark & san*gui*ná*rio \cmark \\
sanguínea & san*guí*ne.a \xmark & san*guí*nea \cmark \\
sanguíneo & san*guí*ne.o \xmark & san*guí*neo \cmark \\
sanitário & sa*ni*tá*ri.o \xmark & sa*ni*tá*rio \cmark \\
santuário & san*tu-á*ri.o \xmark & san*tu*á*rio \cmark \\
sapataria & sa*pa*ta*ri*a \cmark & sa*pa*ta*ri-a \xmark \\
sapiência & sa*pi*ên*ci.a \xmark & sa*pi*ên*cia \cmark \\
saponáceo & sa*po*ná*ce.o \xmark & sa*po*ná*ceo \cmark \\
sapucaia & sa*pu*cai*a \cmark & sa*pu*cai-a \xmark \\
saquê & sa-quê \xmark & sa*quê \cmark \\
sardônica & sar-dô*ni*ca \xmark & sar*dô*ni*ca \cmark \\
sardônico & sar-dô*ni*co \xmark & sar*dô*ni*co \cmark \\
sartório & sar*tó*ri.o \xmark & sar*tó*rio \cmark \\
saruê & sa*ru-ê \xmark & sa*ru*ê \cmark \\
satiríase & sa*ti*rí-a*se \xmark & sa*ti*rí*a*se \cmark \\
satisfatório & sa*tis*fa*tó*ri.o \xmark & sa*tis*fa*tó*rio \cmark \\
satrapia & sa*tra*pi*a \cmark & sa*tra*pi-a \xmark \\
saúde & sa-ú*de \xmark & sa*ú*de \cmark \\
saúva & sa-ú*va \xmark & sa*ú*va \cmark \\
saxônico & sa-xô*ni*co \xmark & sa*xô*ni*co \cmark \\
seabórgio & se*a*bór*gi.o \xmark & se*a*bór*gio \cmark \\
sebáceo & se*bá*ce.o \xmark & se*bá*ceo \cmark \\
sebastião & se*bas*ti-ão \xmark & se*bas*ti*ão \cmark \\
seborreia & se*bor*rei*a \cmark & se*bor*rei-a \xmark \\
sécia & sé*ci.a \xmark & sé*cia \cmark \\
secretaria & se*cre*ta*ri*a \cmark & se*cre*ta*ri-a \xmark \\
secretária & se*cre*tá*ri.a \xmark & se*cre*tá*ria \cmark \\
secretário & se*cre*tá*ri.o \xmark & se*cre*tá*rio \cmark \\
sectário & sec*tá*ri.o \xmark & sec*tá*rio \cmark \\
secundário & se*cun*dá*ri.o \xmark & se*cun*dá*rio \cmark \\
securitário & se*cu*ri*tá*ri.o \xmark & se*cu*ri*tá*rio \cmark \\
sedentário & se*den*tá*ri.o \xmark & se*den*tá*rio \cmark \\
seio & sei*o \cmark & sei-o \xmark \\
selaria & se*la*ri*a \cmark & se*la*ri-a \xmark \\
selênio & se*lê*ni.o \xmark & se*lê*nio \cmark \\
selenografia & se*le*no*gra*fi*a \cmark & se*le*no*gra*fi-a \xmark \\
selvageria & sel*va*ge*ri*a \cmark & sel*va*ge*ri-a \xmark \\
selvajaria & sel*va*ja*ri*a \cmark & sel*va*ja*ri-a \xmark \\
semanário & se*ma*ná*ri.o \xmark & se*ma*ná*rio \cmark \\
sematologia & se*ma*to*lo*gi*a \cmark & se*ma*to*lo*gi-a \xmark \\
semiárido & se*mi-á*ri*do \xmark & se*mi*á*ri*do \cmark \\
semicircunferência & se*mi*cir*cun*fe*rên*ci.a \xmark & se*mi*cir*cun*fe*rên*cia \cmark \\
semicolcheia & se*mi*col*chei*a \cmark & se*mi*col*chei-a \xmark \\
seminário & se*mi*ná*ri.o \xmark & se*mi*ná*rio \cmark \\
semiologia & se*mi*o*lo*gi*a \cmark & se*mi*o*lo*gi-a \xmark \\
semiólogo & se*mi-ó*lo*go \xmark & se*mi*ó*lo*go \cmark \\
semiótica & se*mi-ó*ti*ca \xmark & se*mi*ó*ti*ca \cmark \\
semiótico & se*mi-ó*ti*co \xmark & se*mi*ó*ti*co \cmark \\
semipermeável & se*mi*per*me-á*vel \xmark & se*mi*per*me*á*vel \cmark \\
senescência & se*nes*cên*ci.a \xmark & se*nes*cên*cia \cmark \\
senhoria & se*nho*ri*a \cmark & se*nho*ri-a \xmark \\
senhorio & se*nho*ri*o \cmark & se*nho*ri-o \xmark \\
sensaboria & sen*sa*bo*ri*a \cmark & sen*sa*bo*ri-a \xmark \\
sensório & sen*só*ri.o \xmark & sen*só*rio \cmark \\
sépia & sé*pi.a \xmark & sé*pia \cmark \\
sepsia & sep*si*a \cmark & sep*si-a \xmark \\
septicemia & sep*ti*ce*mi*a \cmark & sep*ti*ce*mi-a \xmark \\
septuagenário & sep*tu*a*ge*ná*ri.o \xmark & sep*tu*a*ge*ná*rio \cmark \\
sequência & se-quên*ci.a \xmark & se*quên*cia \cmark \\
sequoia & se*quoi*a \cmark & se*quoi-a \xmark \\
serápia & se*rá*pi.a \xmark & se*rá*pia \cmark \\
sereia & se*rei*a \cmark & se*rei-a \xmark \\
seríceo & se*rí*ce.o \xmark & se*rí*ceo \cmark \\
série & sé*ri.e \xmark & sé*rie \cmark \\
serigrafia & se*ri*gra*fi*a \cmark & se*ri*gra*fi-a \xmark \\
sério & sé*ri.o \xmark & sé*rio \cmark \\
serôdio & se-rô*di.o \xmark & se*rô*dio \cmark \\
serologia & se*ro*lo*gi*a \cmark & se*ro*lo*gi-a \xmark \\
serpentário & ser*pen*tá*ri.o \xmark & ser*pen*tá*rio \cmark \\
serralharia & ser*ra*lha*ri*a \cmark & ser*ra*lha*ri-a \xmark \\
serralheria & ser*ra*lhe*ri*a \cmark & ser*ra*lhe*ri-a \xmark \\
serrania & ser*ra*ni*a \cmark & ser*ra*ni-a \xmark \\
serraria & ser*ra*ri*a \cmark & ser*ra*ri-a \xmark \\
serventia & ser*ven*ti*a \cmark & ser*ven*ti-a \xmark \\
serventuário & ser*ven*tu-á*ri.o \xmark & ser*ven*tu*á*rio \cmark \\
sérvio & sér*vi.o \xmark & sér*vio \cmark \\
servofreio & ser*vo*frei*o \cmark & ser*vo*frei-o \xmark \\
sesmaria & ses*ma*ri*a \cmark & ses*ma*ri-a \xmark \\
sesquicentenário & ses*qui*cen*te*ná*ri.o \xmark & ses*qui*cen*te*ná*rio \cmark \\
sestércio & ses*tér*ci.o \xmark & ses*tér*cio \cmark \\
setenário & se*te*ná*ri.o \xmark & se*te*ná*rio \cmark \\
setentrião & se*ten*tri-ão \xmark & se*ten*tri*ão \cmark \\
setia & se*ti*a \cmark & se*ti-a \xmark \\
sexagenário & se*xa*ge*ná*ri.o \xmark & se*xa*ge*ná*rio \cmark \\
sexologia & se*xo*lo*gi*a \cmark & se*xo*lo*gi-a \xmark \\
siálico & si-á*li*co \xmark & si*á*li*co \cmark \\
sialorreia & si*a*lor*rei*a \cmark & si*a*lor*rei-a \xmark \\
sibilância & si*bi*lân*ci.a \xmark & si*bi*lân*cia \cmark \\
sicário & si*cá*ri.o \xmark & si*cá*rio \cmark \\
sicômoro & si-cô*mo*ro \xmark & si*cô*mo*ro \cmark \\
sideromancia & si*de*ro*man*ci*a \cmark & si*de*ro*man*ci-a \xmark \\
siderurgia & si*de*rur*gi*a \cmark & si*de*rur*gi-a \xmark \\
sievert & si.e*vert \xmark & si.e*vert \xmark \\
sigilografia & si*gi*lo*gra*fi*a \cmark & si*gi*lo*gra*fi-a \xmark \\
signatário & sig*na*tá*ri.o \xmark & sig*na*tá*rio \cmark \\
significância & sig*ni*fi*cân*ci.a \xmark & sig*ni*fi*cân*cia \cmark \\
silabário & si*la*bá*ri.o \xmark & si*la*bá*rio \cmark \\
silêncio & si*lên*ci.o \xmark & si*lên*cio \cmark \\
silharia & si*lha*ri*a \cmark & si*lha*ri-a \xmark \\
silício & si*lí*ci.o \xmark & si*lí*cio \cmark \\
sílvia & síl*vi.a \xmark & síl*via \cmark \\
simbologia & sim*bo*lo*gi*a \cmark & sim*bo*lo*gi-a \xmark \\
simetria & si*me*tri*a \cmark & si*me*tri-a \xmark \\
símio & sí*mi.o \xmark & sí*mio \cmark \\
simonia & si*mo*ni*a \cmark & si*mo*ni-a \xmark \\
simpatia & sim*pa*ti*a \cmark & sim*pa*ti-a \xmark \\
simpatria & sim*pa*tri*a \cmark & sim*pa*tri-a \xmark \\
simplório & sim*pló*ri.o \xmark & sim*pló*rio \cmark \\
simpósio & sim*pó*si.o \xmark & sim*pó*sio \cmark \\
simultâneo & si*mul*tâ*ne.o \xmark & si*mul*tâ*neo \cmark \\
sincronia & sin*cro*ni*a \cmark & sin*cro*ni-a \xmark \\
sindicância & sin*di*cân*ci.a \xmark & sin*di*cân*cia \cmark \\
sinédrio & si*né*dri.o \xmark & si*né*drio \cmark \\
sinergia & si*ner*gi*a \cmark & si*ner*gi-a \xmark \\
sinestesia & si*nes*te*si*a \cmark & si*nes*te*si-a \xmark \\
sinfonia & sin*fo*ni*a \cmark & sin*fo*ni-a \xmark \\
sinfônica & sin-fô*ni*ca \xmark & sin*fô*ni*ca \cmark \\
sinfônico & sin-fô*ni*co \xmark & sin*fô*ni*co \cmark \\
sinologia & si*no*lo*gi*a \cmark & si*no*lo*gi-a \xmark \\
sinonímia & si*no*ní*mi.a \xmark & si*no*ní*mia \cmark \\
sinônimo & si-nô*ni*mo \xmark & si*nô*ni*mo \cmark \\
sintomatologia & sin*to*ma*to*lo*gi*a \cmark & sin*to*ma*to*lo*gi-a \xmark \\
sintonia & sin*to*ni*a \cmark & sin*to*ni-a \xmark \\
siríaco & si*rí-a*co \xmark & si*rí*a*co \cmark \\
sírio & sí*ri.o \xmark & sí*rio \cmark \\
sismologia & sis*mo*lo*gi*a \cmark & sis*mo*lo*gi-a \xmark \\
sismômetro & sis-mô*me*tro \xmark & sis*mô*me*tro \cmark \\
sítio & sí*ti.o \xmark & sí*tio \cmark \\
sizígia & si*zí*gi.a \xmark & si*zí*gia \cmark \\
soberania & so*be*ra*ni*a \cmark & so*be*ra*ni-a \xmark \\
sobranceria & so*bran*ce*ri*a \cmark & so*bran*ce*ri-a \xmark \\
sobressaia & so*bres*sai*a \cmark & so*bres*sai-a \xmark \\
sobressair & so*bres*sa-ir \xmark & so*bres*sa*ir \cmark \\
sobrevivência & so*bre*vi*vên*ci.a \xmark & so*bre*vi*vên*cia \cmark \\
sobrevoo & so*bre*vo*o \cmark & so*bre*vo-o \xmark \\
sóbrio & só*bri.o \xmark & só*brio \cmark \\
sociável & so*ci-á*vel \xmark & so*ci*á*vel \cmark \\
societário & so*ci*e*tá*ri.o \xmark & so*ci*e*tá*rio \cmark \\
sociologia & so*ci*o*lo*gi*a \cmark & so*ci*o*lo*gi-a \xmark \\
sociólogo & so*ci-ó*lo*go \xmark & so*ci*ó*lo*go \cmark \\
sociopatia & so*ci*o*pa*ti*a \cmark & so*ci*o*pa*ti-a \xmark \\
sócio & só*ci.o \xmark & só*cio \cmark \\
sodalício & so*da*lí*ci.o \xmark & so*da*lí*cio \cmark \\
sódio & só*di.o \xmark & só*dio \cmark \\
sodomia & so*do*mi*a \cmark & so*do*mi-a \xmark \\
solanácea & so*la*ná*ce.a \xmark & so*la*ná*cea \cmark \\
solaria & so*la*ri*a \cmark & so*la*ri-a \xmark \\
solário & so*lá*ri.o \xmark & so*lá*rio \cmark \\
solicitadoria & so*li*ci*ta*do*ri*a \cmark & so*li*ci*ta*do*ri-a \xmark \\
solidário & so*li*dá*ri.o \xmark & so*li*dá*rio \cmark \\
solilóquio & so*li*ló*qui.o \xmark & so*li*ló*quio \cmark \\
sólio & só*li.o \xmark & só*lio \cmark \\
solitária & so*li*tá*ri.a \xmark & so*li*tá*ria \cmark \\
solitário & so*li*tá*ri.o \xmark & so*li*tá*rio \cmark \\
solstício & sols*tí*ci.o \xmark & sols*tí*cio \cmark \\
solvência & sol*vên*ci.a \xmark & sol*vên*cia \cmark \\
somatório & so*ma*tó*ri.o \xmark & so*ma*tó*rio \cmark \\
sombrio & som*bri*o \cmark & som*bri-o \xmark \\
sonolência & so*no*lên*ci.a \xmark & so*no*lên*cia \cmark \\
sonômetro & so-nô*me*tro \xmark & so*nô*me*tro \cmark \\
sonoplastia & so*no*plas*ti*a \cmark & so*no*plas*ti-a \xmark \\
sorologia & so*ro*lo*gi*a \cmark & so*ro*lo*gi-a \xmark \\
soroterapia & so*ro*te*ra*pi*a \cmark & so*ro*te*ra*pi-a \xmark \\
sorteio & sor*tei*o \cmark & sor*tei-o \xmark \\
sortilégio & sor*ti*lé*gi.o \xmark & sor*ti*lé*gio \cmark \\
sorveteria & sor*ve*te*ri*a \cmark & sor*ve*te*ri-a \xmark \\
sósia & só*si.a \xmark & só*sia \cmark \\
soslaio & sos*lai*o \cmark & sos*lai-o \xmark \\
soteriologia & so*te*ri*o*lo*gi*a \cmark & so*te*ri*o*lo*gi-a \xmark \\
soviético & so*vi-é*ti*co \xmark & so*vi*é*ti*co \cmark \\
stalinismo & s.ta*li*nis*mo \xmark & sta*li*nis*mo \cmark \\
stalinista & s.ta*li*nis*ta \xmark & sta*li*nis*ta \cmark \\
suaíle & su*a-í*le \xmark & su*a*í*le \cmark \\
suaíli & su*a-í*li \xmark & su*a*í*li \cmark \\
suão & su-ão \xmark & su*ão \cmark \\
suástica & su-ás*ti*ca \xmark & su*ás*ti*ca \cmark \\
sua & su*a \cmark & su-a \xmark \\
suã & su-ã \xmark & su*ã \cmark \\
subaquático & su*ba-quá*ti*co \xmark & su*ba*quá*ti*co \cmark \\
subatômico & su*ba-tô*mi*co \xmark & su*ba*tô*mi*co \cmark \\
subcategoria & sub*ca*te*go*ri*a \cmark & sub*ca*te*go*ri-a \xmark \\
subcomissário & sub*co*mis*sá*ri.o \xmark & sub*co*mis*sá*rio \cmark \\
subconsciência & sub*cons*ci*ên*ci.a \xmark & sub*cons*ci*ên*cia \cmark \\
subcutâneo & sub*cu*tâ*ne.o \xmark & sub*cu*tâ*neo \cmark \\
subdiácono & sub*di-á*co*no \xmark & sub*di*á*co*no \cmark \\
subespécie & su*bes*pé*ci.e \xmark & su*bes*pé*cie \cmark \\
subfamília & sub*fa*mí*li.a \xmark & sub*fa*mí*lia \cmark \\
sublevação & su.b-le*va*ção \xmark & sub*le*va*ção \cmark \\
sublevado & su.b-le*va*do \xmark & sub*le*va*do \cmark \\
sublevar & su.b-le*var \xmark & sub*le*var \cmark \\
subliminal & su.b-li*mi*nal \xmark & su.b-li*mi*nal \xmark \\
subliminar & su.b-li*mi*nar \xmark & su.b-li*mi*nar \xmark \\
sublinear & su.b-li*ne*ar \xmark & su.b-li*ne*ar \xmark \\
sublingual & su.b-lin*gual \xmark & su.b-lin*gual \xmark \\
sublinha & su.b-li*nha \xmark & su.b-li*nha \xmark \\
subliteratura & su.b-li*te*ra*tu*ra \xmark & su.b-li*te*ra*tu*ra \xmark \\
sublocação & su.b-lo*ca*ção \xmark & sub*lo*ca*ção \cmark \\
sublocar & su.b-lo*car \xmark & sub*lo*car \cmark \\
sublocatário & su.b-lo*ca*tá*ri.o \xmark & sub*lo*ca*tá*rio \cmark \\
sublunar & su.b-lu*nar \xmark & sub*lu*nar \cmark \\
subluxação & su.b-lu*xa*ção \xmark & sub*lu*xa*ção \cmark \\
subsecretário & sub*se*cre*tá*ri.o \xmark & sub*se*cre*tá*rio \cmark \\
subsequência & sub*se-quên*ci.a \xmark & sub*se*quên*cia \cmark \\
subserviência & sub*ser*vi*ên*ci.a \xmark & sub*ser*vi*ên*cia \cmark \\
subsidência & sub*si*dên*ci.a \xmark & sub*si*dên*cia \cmark \\
subsidiário & sub*si*di-á*ri.o \xmark & sub*si*di*á*rio \cmark \\
subsídio & sub*sí*di.o \xmark & sub*sí*dio \cmark \\
subsistência & sub*sis*tên*ci.a \xmark & sub*sis*tên*cia \cmark \\
subsônico & sub-sô*ni*co \xmark & sub*sô*ni*co \cmark \\
substância & subs*tân*ci.a \xmark & subs*tân*cia \cmark \\
substituição & subs*ti*tu-i*ção \xmark & subs*ti*tu*i*ção \cmark \\
substituído & subs*ti*tu-í*do \xmark & subs*ti*tu*í*do \cmark \\
substituinte & subs*ti*tu-in*te \xmark & subs*ti*tu*in*te \cmark \\
substituir & subs*ti*tu-ir \xmark & subs*ti*tu*ir \cmark \\
substituível & subs*ti*tu-í*vel \xmark & subs*ti*tu*í*vel \cmark \\
subterrâneo & sub*ter*râ*ne.o \xmark & sub*ter*râ*neo \cmark \\
subtraído & sub*tra-í*do \xmark & sub*tra*í*do \cmark \\
subtrair & sub*tra-ir \xmark & sub*tra*ir \cmark \\
subúrbio & su*búr*bi.o \xmark & su*búr*bio \cmark \\
sucedâneo & su*ce*dâ*ne.o \xmark & su*ce*dâ*neo \cmark \\
sucessório & su*ces*só*ri.o \xmark & su*ces*só*rio \cmark \\
súcia & sú*ci.a \xmark & sú*cia \cmark \\
suculência & su*cu*lên*ci.a \xmark & su*cu*lên*cia \cmark \\
sucumbência & su*cum*bên*ci.a \xmark & su*cum*bên*cia \cmark \\
sucuriú & su*cu*ri-ú \xmark & su*cu*ri*ú \cmark \\
sudário & su*dá*ri.o \xmark & su*dá*rio \cmark \\
suéter & su-é*ter \xmark & su*é*ter \cmark \\
suficiência & su*fi*ci*ên*ci.a \xmark & su*fi*ci*ên*cia \cmark \\
sufragâneo & su*fra*gâ*ne.o \xmark & su*fra*gâ*neo \cmark \\
sufrágio & su*frá*gi.o \xmark & su*frá*gio \cmark \\
suíça & su-í*ça \xmark & su*í*ça \cmark \\
suicida & su-i*ci*da \xmark & su*i*ci*da \cmark \\
suicídio & su-i*cí*di*o \xmark & su*i*cí*di-o \xmark \\
suíço & su-í*ço \xmark & su*í*ço \cmark \\
suídeo & su-í*de.o \xmark & su*í*deo \cmark \\
suingar & su-in*gar \xmark & su*in*gar \cmark \\
suingue & su-in*gue \xmark & su*in*gue \cmark \\
suinicultura & su-i*ni*cul*tu*ra \xmark & su*i*ni*cul*tu*ra \cmark \\
suinocultura & su-i*no*cul*tu*ra \xmark & su*i*no*cul*tu*ra \cmark \\
suíno & su-í*no \xmark & su*í*no \cmark \\
suíte & su-í*te \xmark & su*í*te \cmark \\
sulfúreo & sul*fú*re.o \xmark & sul*fú*reo \cmark \\
sumário & su*má*ri.o \xmark & su*má*rio \cmark \\
sumaúma & su*ma-ú*ma \xmark & su*ma*ú*ma \cmark \\
sumério & su*mé*ri.o \xmark & su*mé*rio \cmark \\
sumô & su-mô \xmark & su*mô \cmark \\
suntuário & sun*tu-á*ri.o \xmark & sun*tu*á*rio \cmark \\
supedâneo & su*pe*dâ*ne.o \xmark & su*pe*dâ*neo \cmark \\
superabundância & su*pe*ra*bun*dân*ci.a \xmark & su*pe*ra*bun*dân*cia \cmark \\
superavitário & su*pe*ra*vi*tá*ri.o \xmark & su*pe*ra*vi*tá*rio \cmark \\
supercampeão & su*per*cam*pe-ão \xmark & su*per*cam*pe*ão \cmark \\
supercílio & su*per*cí*li.o \xmark & su*per*cí*lio \cmark \\
superfície & su*per*fí*ci.e \xmark & su*per*fí*cie \cmark \\
supérfluo & su*pér*flu.o \xmark & su*pér*fluo \cmark \\
superintendência & su*pe*rin*ten*dên*ci.a \xmark & su*pe*rin*ten*dên*cia \cmark \\
supernumerário & su*per*nu*me*rá*ri.o \xmark & su*per*nu*me*rá*rio \cmark \\
superpotência & su*per*po*tên*ci.a \xmark & su*per*po*tên*cia \cmark \\
supersônico & su*per-sô*ni*co \xmark & su*per*sô*ni*co \cmark \\
superveniência & su*per*ve*ni*ên*ci.a \xmark & su*per*ve*ni*ên*cia \cmark \\
supervivência & su*per*vi*vên*ci.a \xmark & su*per*vi*vên*cia \cmark \\
suplência & su*plên*ci.a \xmark & su*plên*cia \cmark \\
suplício & su*plí*ci.o \xmark & su*plí*cio \cmark \\
supositório & su*po*si*tó*ri.o \xmark & su*po*si*tó*rio \cmark \\
supranumerário & su*pra*nu*me*rá*ri.o \xmark & su*pra*nu*me*rá*rio \cmark \\
suprapartidário & su*pra*par*ti*dá*ri.o \xmark & su*pra*par*ti*dá*rio \cmark \\
supremacia & su*pre*ma*ci*a \cmark & su*pre*ma*ci-a \xmark \\
surucuá & su*ru*cu-á \xmark & su*ru*cu*á \cmark \\
suserania & su*se*ra*ni*a \cmark & su*se*ra*ni-a \xmark \\
suspensório & sus*pen*só*ri.o \xmark & sus*pen*só*rio \cmark \\
sustância & sus*tân*ci.a \xmark & sus*tân*cia \cmark \\
sutiã & su*ti-ã \xmark & su*ti*ã \cmark \\
tabacaria & ta*ba*ca*ri*a \cmark & ta*ba*ca*ri-a \xmark \\
tabelião & ta*be*li-ão \xmark & ta*be*li*ão \cmark \\
taboa & ta*bo*a \cmark & ta*bo-a \xmark \\
tabuão & ta*bu-ão \xmark & ta*bu*ão \cmark \\
tabua & ta*bu*a \cmark & ta*bu-a \xmark \\
tabuinha & ta*bu-i*nha \xmark & ta*bu*i*nha \cmark \\
tabulário & ta*bu*lá*ri.o \xmark & ta*bu*lá*rio \cmark \\
tacômetro & ta-cô*me*tro \xmark & ta*cô*me*tro \cmark \\
tainha & ta-i*nha \xmark & ta*i*nha \cmark \\
taiuiá & tai-ui-á \xmark & tai*ui*á \cmark \\
taiwanês & tai-wa*nês \xmark & tai-wa*nês \xmark \\
talassemia & ta*las*se*mi*a \cmark & ta*las*se*mi-a \xmark \\
talassocracia & ta*las*so*cra*ci*a \cmark & ta*las*so*cra*ci-a \xmark \\
talassoterapia & ta*las*so*te*ra*pi*a \cmark & ta*las*so*te*ra*pi-a \xmark \\
talharia & ta*lha*ri*a \cmark & ta*lha*ri-a \xmark \\
talião & ta*li-ão \xmark & ta*li*ão \cmark \\
tália & tá*li.a \xmark & tá*lia \cmark \\
tálio & tá*li.o \xmark & tá*lio \cmark \\
talonário & ta*lo*ná*ri.o \xmark & ta*lo*ná*rio \cmark \\
tâmia & tâ*mi.a \xmark & tâ*mia \cmark \\
tanatofobia & ta*na*to*fo*bi*a \cmark & ta*na*to*fo*bi-a \xmark \\
tanatologia & ta*na*to*lo*gi*a \cmark & ta*na*to*lo*gi-a \xmark \\
tangência & tan*gên*ci.a \xmark & tan*gên*cia \cmark \\
tanoaria & ta*no*a*ri*a \cmark & ta*no*a*ri-a \xmark \\
taoismo & ta.o-is*mo \xmark & ta.o-is*mo \xmark \\
tapeçaria & ta*pe*ça*ri*a \cmark & ta*pe*ça*ri-a \xmark \\
tapuia & ta*pui*a \cmark & ta*pui-a \xmark \\
tapuio & ta*pui*o \cmark & ta*pui-o \xmark \\
taquicardia & ta*qui*car*di*a \cmark & ta*qui*car*di-a \xmark \\
taquigrafia & ta*qui*gra*fi*a \cmark & ta*qui*gra*fi-a \xmark \\
taquígrafo & ta-quí*gra*fo \xmark & ta*quí*gra*fo \cmark \\
taquímetro & ta-quí*me*tro \xmark & ta*quí*me*tro \cmark \\
taquipneia & ta*quip*nei*a \cmark & ta*quip*nei-a \xmark \\
tardio & tar*di*o \cmark & tar*di-o \xmark \\
tarifário & ta*ri*fá*ri.o \xmark & ta*ri*fá*rio \cmark \\
tarô & ta-rô \xmark & ta*rô \cmark \\
tasmânia & tas*mâ*ni.a \xmark & tas*mâ*nia \cmark \\
tataravô & ta*ta*ra-vô \xmark & ta*ta*ra*vô \cmark \\
tatuí & ta*tu-í \xmark & ta*tu*í \cmark \\
taumaturgia & tau*ma*tur*gi*a \cmark & tau*ma*tur*gi-a \xmark \\
tauromaquia & tau*ro*ma*qui*a \cmark & tau*ro*ma*qui-a \xmark \\
tautologia & tau*to*lo*gi*a \cmark & tau*to*lo*gi-a \xmark \\
tautomeria & tau*to*me*ri*a \cmark & tau*to*me*ri-a \xmark \\
tautômero & tau-tô*me*ro \xmark & tau*tô*me*ro \cmark \\
taxionomia & ta*xi*o*no*mi*a \cmark & ta*xi*o*no*mi-a \xmark \\
taxonomia & ta*xo*no*mi*a \cmark & ta*xo*no*mi-a \xmark \\
taxonômico & ta*xo-nô*mi*co \xmark & ta*xo*nô*mi*co \cmark \\
tchau & t.chau \xmark & tchau \cmark \\
tchecoslovaco & t.che*cos*lo*va*co \xmark & tche*cos*lo*va*co \cmark \\
tcheco & t.che*co \xmark & tche*co \cmark \\
tecnécio & tec*né*ci.o \xmark & tec*né*cio \cmark \\
tecnocracia & tec*no*cra*ci*a \cmark & tec*no*cra*ci-a \xmark \\
tecnofobia & tec*no*fo*bi*a \cmark & tec*no*fo*bi-a \xmark \\
tecnologia & tec*no*lo*gi*a \cmark & tec*no*lo*gi-a \xmark \\
tectônico & tec-tô*ni*co \xmark & tec*tô*ni*co \cmark \\
tédio & té*di.o \xmark & té*dio \cmark \\
teia & tei*a \cmark & tei-a \xmark \\
teimosia & tei*mo*si*a \cmark & tei*mo*si-a \xmark \\
teísmo & te-ís*mo \xmark & te*ís*mo \cmark \\
teísta & te-ís*ta \xmark & te*ís*ta \cmark \\
teiú & tei-ú \xmark & tei*ú \cmark \\
telangiectasia & te*lan*gi*ec*ta*si*a \cmark & te*lan*gi*ec*ta*si-a \xmark \\
telecinesia & te*le*ci*ne*si*a \cmark & te*le*ci*ne*si-a \xmark \\
teleconferência & te*le*con*fe*rên*ci.a \xmark & te*le*con*fe*rên*cia \cmark \\
teledramaturgia & te*le*dra*ma*tur*gi*a \cmark & te*le*dra*ma*tur*gi-a \xmark \\
telefonia & te*le*fo*ni*a \cmark & te*le*fo*ni-a \xmark \\
telefônico & te*le-fô*ni*co \xmark & te*le*fô*ni*co \cmark \\
telegrafia & te*le*gra*fi*a \cmark & te*le*gra*fi-a \xmark \\
teleinformática & te*le-in*for*má*ti*ca \xmark & te*le*in*for*má*ti*ca \cmark \\
telemetria & te*le*me*tri*a \cmark & te*le*me*tri-a \xmark \\
teleologia & te*le*o*lo*gi*a \cmark & te*le*o*lo*gi-a \xmark \\
teleósteo & te*le-ós*te.o \xmark & te*le*ós*teo \cmark \\
telepatia & te*le*pa*ti*a \cmark & te*le*pa*ti-a \xmark \\
telescopia & te*les*co*pi*a \cmark & te*les*co*pi-a \xmark \\
telescópio & te*les*có*pi.o \xmark & te*les*có*pio \cmark \\
telômero & te-lô*me*ro \xmark & te*lô*me*ro \cmark \\
telúrio & te*lú*ri.o \xmark & te*lú*rio \cmark \\
temário & te*má*ri.o \xmark & te*má*rio \cmark \\
temerário & te*me*rá*ri.o \xmark & te*me*rá*rio \cmark \\
templário & tem*plá*ri.o \xmark & tem*plá*rio \cmark \\
temporário & tem*po*rá*ri.o \xmark & tem*po*rá*rio \cmark \\
tendência & ten*dên*ci.a \xmark & ten*dên*cia \cmark \\
tenência & te*nên*ci.a \xmark & te*nên*cia \cmark \\
teníase & te*ní-a*se \xmark & te*ní*a*se \cmark \\
tentório & ten*tó*ri.o \xmark & ten*tó*rio \cmark \\
tênue & tê*nu.e \xmark & tê*nue \cmark \\
teocracia & te*o*cra*ci*a \cmark & te*o*cra*ci-a \xmark \\
teodiceia & te*o*di*cei*a \cmark & te*o*di*cei-a \xmark \\
teofania & te*o*fa*ni*a \cmark & te*o*fa*ni-a \xmark \\
teófobo & te-ó*fo*bo \xmark & te*ó*fo*bo \cmark \\
teogonia & te*o*go*ni*a \cmark & te*o*go*ni-a \xmark \\
teologia & te*o*lo*gi*a \cmark & te*o*lo*gi-a \xmark \\
teólogo & te-ó*lo*go \xmark & te*ó*lo*go \cmark \\
teônimo & te-ô*ni*mo \xmark & te*ô*ni*mo \cmark \\
teoria & te*o*ri*a \cmark & te*o*ri-a \xmark \\
teórica & te-ó*ri*ca \xmark & te*ó*ri*ca \cmark \\
teórico & te-ó*ri*co \xmark & te*ó*ri*co \cmark \\
teosofia & te*o*so*fi*a \cmark & te*o*so*fi-a \xmark \\
teósofo & te-ó*so*fo \xmark & te*ó*so*fo \cmark \\
terapia & te*ra*pi*a \cmark & te*ra*pi-a \xmark \\
teratologia & te*ra*to*lo*gi*a \cmark & te*ra*to*lo*gi-a \xmark \\
térbio & tér*bi.o \xmark & tér*bio \cmark \\
terciário & ter*ci-á*ri.o \xmark & ter*ci*á*rio \cmark \\
tércia & tér*ci.a \xmark & tér*cia \cmark \\
tércio & tér*ci.o \xmark & tér*cio \cmark \\
terminologia & ter*mi*no*lo*gi*a \cmark & ter*mi*no*lo*gi-a \xmark \\
termografia & ter*mo*gra*fi*a \cmark & ter*mo*gra*fi-a \xmark \\
termologia & ter*mo*lo*gi*a \cmark & ter*mo*lo*gi-a \xmark \\
termômetro & ter-mô*me*tro \xmark & ter*mô*me*tro \cmark \\
termoquímica & ter*mo-quí*mi*ca \xmark & ter*mo*quí*mi*ca \cmark \\
termoquímico & ter*mo-quí*mi*co \xmark & ter*mo*quí*mi*co \cmark \\
ternário & ter*ná*ri.o \xmark & ter*ná*rio \cmark \\
terráqueo & ter*rá*que.o \xmark & ter*rá*queo \cmark \\
terrário & ter*rá*ri.o \xmark & ter*rá*rio \cmark \\
térreo & tér*re.o \xmark & tér*reo \cmark \\
território & ter*ri*tó*ri.o \xmark & ter*ri*tó*rio \cmark \\
tertúlia & ter*tú*li.a \xmark & ter*tú*lia \cmark \\
tesouraria & te*sou*ra*ri*a \cmark & te*sou*ra*ri-a \xmark \\
tessálio & tes*sá*li.o \xmark & tes*sá*lio \cmark \\
testamentário & tes*ta*men*tá*ri.o \xmark & tes*ta*men*tá*rio \cmark \\
tetania & te*ta*ni*a \cmark & te*ta*ni-a \xmark \\
tetracampeão & te*tra*cam*pe-ão \xmark & te*tra*cam*pe*ão \cmark \\
tetraédrico & te*tra-é*dri*co \xmark & te*tra*é*dri*co \cmark \\
tetraplegia & te*tra*ple*gi*a \cmark & te*tra*ple*gi-a \xmark \\
tetrarquia & te*trar*qui*a \cmark & te*trar*qui-a \xmark \\
tetravô & te*tra-vô \xmark & te*tra*vô \cmark \\
teurgia & te-ur*gi*a \xmark & te-ur*gi-a \xmark \\
teúrgico & te-úr*gi*co \xmark & te*úr*gi*co \cmark \\
teutônico & teu-tô*ni*co \xmark & teu*tô*ni*co \cmark \\
tia & ti*a \cmark & ti-a \xmark \\
tíbia & tí*bi.a \xmark & tí*bia \cmark \\
tíbio & tí*bi.o \xmark & tí*bio \cmark \\
tié & ti-é \xmark & ti*é \cmark \\
tília & tí*li.a \xmark & tí*lia \cmark \\
timbaúba & tim*ba-ú*ba \xmark & tim*ba*ú*ba \cmark \\
timocracia & ti*mo*cra*ci*a \cmark & ti*mo*cra*ci-a \xmark \\
tinamídeo & ti*na*mí*de.o \xmark & ti*na*mí*deo \cmark \\
tinturaria & tin*tu*ra*ri*a \cmark & tin*tu*ra*ri-a \xmark \\
tio & ti*o \cmark & ti-o \xmark \\
tipografia & ti*po*gra*fi*a \cmark & ti*po*gra*fi-a \xmark \\
tipoia & ti*poi*a \cmark & ti*poi-a \xmark \\
tipologia & ti*po*lo*gi*a \cmark & ti*po*lo*gi-a \xmark \\
tiptologia & tip*to*lo*gi*a \cmark & tip*to*lo*gi-a \xmark \\
tirania & ti*ra*ni*a \cmark & ti*ra*ni-a \xmark \\
tiranicídio & ti*ra*ni*cí*di.o \xmark & ti*ra*ni*cí*dio \cmark \\
tireoidectomia & ti*re*oi*dec*to*mi*a \cmark & ti*re*oi*dec*to*mi-a \xmark \\
tirocínio & ti*ro*cí*ni.o \xmark & ti*ro*cí*nio \cmark \\
tiromancia & ti*ro*man*ci*a \cmark & ti*ro*man*ci-a \xmark \\
tiroteio & ti*ro*tei*o \cmark & ti*ro*tei-o \xmark \\
tisiologia & ti*si*o*lo*gi*a \cmark & ti*si*o*lo*gi-a \xmark \\
titânio & ti*tâ*ni.o \xmark & ti*tâ*nio \cmark \\
titia & ti*ti*a \cmark & ti*ti-a \xmark \\
titio & ti*ti*o \cmark & ti*ti-o \xmark \\
tiú & ti-ú \xmark & ti*ú \cmark \\
tmese & t.me*se \xmark & tme*se \cmark \\
toa & to*a \cmark & to-a \xmark \\
toboágua & to*bo-á*gua \xmark & to*bo*á*gua \cmark \\
tocaia & to*cai*a \cmark & to*cai-a \xmark \\
todavia & to*da*vi*a \cmark & to*da*vi-a \xmark \\
tolerância & to*le*rân*ci.a \xmark & to*le*rân*cia \cmark \\
tomografia & to*mo*gra*fi*a \cmark & to*mo*gra*fi-a \xmark \\
tonelaria & to*ne*la*ri*a \cmark & to*ne*la*ri-a \xmark \\
tonsilectomia & ton*si*lec*to*mi*a \cmark & ton*si*lec*to*mi-a \xmark \\
topázio & to*pá*zi.o \xmark & to*pá*zio \cmark \\
topiaria & to*pi*a*ri*a \cmark & to*pi*a*ri-a \xmark \\
topografia & to*po*gra*fi*a \cmark & to*po*gra*fi-a \xmark \\
topologia & to*po*lo*gi*a \cmark & to*po*lo*gi-a \xmark \\
toponímia & to*po*ní*mi.a \xmark & to*po*ní*mia \cmark \\
topônimo & to-pô*ni*mo \xmark & to*pô*ni*mo \cmark \\
toracoscopia & to*ra*cos*co*pi*a \cmark & to*ra*cos*co*pi-a \xmark \\
toracotomia & to*ra*co*to*mi*a \cmark & to*ra*co*to*mi-a \xmark \\
tório & tó*ri*o \cmark & tó*ri-o \xmark \\
tormentório & tor*men*tó*ri.o \xmark & tor*men*tó*rio \cmark \\
tornearia & tor*ne*a*ri*a \cmark & tor*ne*a*ri-a \xmark \\
torquês & tor-quês \xmark & tor*quês \cmark \\
torreão & tor*re-ão \xmark & tor*re*ão \cmark \\
tosquia & tos*qui*a \cmark & tos*qui-a \xmark \\
totalitário & to*ta*li*tá*ri.o \xmark & to*ta*li*tá*rio \cmark \\
toureio & tou*rei*o \cmark & tou*rei-o \xmark \\
toxicodependência & to*xi*co*de*pen*dên*ci.a \xmark & to*xi*co*de*pen*dên*cia \cmark \\
toxicologia & to*xi*co*lo*gi*a \cmark & to*xi*co*lo*gi-a \xmark \\
toxicomania & to*xi*co*ma*ni*a \cmark & to*xi*co*ma*ni-a \xmark \\
toxicômano & to*xi-cô*ma*no \xmark & to*xi*cô*ma*no \cmark \\
trábea & trá*be.a \xmark & trá*bea \cmark \\
trácio & trá*ci*o \cmark & trá*ci-o \xmark \\
tragédia & tra*gé*di.a \xmark & tra*gé*dia \cmark \\
tragediógrafo & tra*ge*di-ó*gra*fo \xmark & tra*ge*di*ó*gra*fo \cmark \\
tragicomédia & tra*gi*co*mé*di.a \xmark & tra*gi*co*mé*dia \cmark \\
tragicômico & tra*gi-cô*mi*co \xmark & tra*gi*cô*mi*co \cmark \\
traído & tra-í*do \xmark & tra*í*do \cmark \\
traíra & tra-í*ra \xmark & tra*í*ra \cmark \\
trair & tra-ir \xmark & tra*ir \cmark \\
trajetória & tra*je*tó*ri.a \xmark & tra*je*tó*ria \cmark \\
tramoia & tra*moi*a \cmark & tra*moi-a \xmark \\
transcendência & trans*cen*dên*ci.a \xmark & trans*cen*dên*cia \cmark \\
transeunte & tran*se-un*te \xmark & tran*se-un*te \xmark \\
transferência & trans*fe*rên*ci.a \xmark & trans*fe*rên*cia \cmark \\
transigência & tran*si*gên*ci.a \xmark & tran*si*gên*cia \cmark \\
transitório & tran*si*tó*ri.o \xmark & tran*si*tó*rio \cmark \\
transmitância & trans*mi*tân*ci.a \xmark & trans*mi*tân*cia \cmark \\
transoceânico & tran*so*ce-â*ni*co \xmark & tran*so*ce*â*ni*co \cmark \\
transparência & trans*pa*rên*ci.a \xmark & trans*pa*rên*cia \cmark \\
transumância & tran*su*mân*ci.a \xmark & tran*su*mân*cia \cmark \\
tranvia & tran*vi*a \cmark & tran*vi-a \xmark \\
trapézio & tra*pé*zi.o \xmark & tra*pé*zio \cmark \\
traqueia & tra*quei*a \cmark & tra*quei-a \xmark \\
traqueotomia & tra*que*o*to*mi*a \cmark & tra*que*o*to*mi-a \xmark \\
traquítico & tra-quí*ti*co \xmark & tra*quí*ti*co \cmark \\
traumatologia & trau*ma*to*lo*gi*a \cmark & trau*ma*to*lo*gi-a \xmark \\
travessia & tra*ves*si*a \cmark & tra*ves*si-a \xmark \\
trematódeo & tre*ma*tó*de.o \xmark & tre*ma*tó*deo \cmark \\
tríade & trí-a*de \xmark & trí*a*de \cmark \\
triádico & tri-á*di*co \xmark & tri*á*di*co \cmark \\
triarquia & tri*ar*qui*a \cmark & tri*ar*qui-a \xmark \\
triásico & tri-á*si*co \xmark & tri*á*si*co \cmark \\
triássico & tri-ás*si*co \xmark & tri*ás*si*co \cmark \\
tribunício & tri*bu*ní*ci.o \xmark & tri*bu*ní*cio \cmark \\
tributário & tri*bu*tá*ri.o \xmark & tri*bu*tá*rio \cmark \\
tricampeão & tri*cam*pe-ão \xmark & tri*cam*pe*ão \cmark \\
tricentenário & tri*cen*te*ná*ri.o \xmark & tri*cen*te*ná*rio \cmark \\
triclínio & tri*clí*ni.o \xmark & tri*clí*nio \cmark \\
tricórnio & tri*cór*ni.o \xmark & tri*cór*nio \cmark \\
tricotilomania & tri*co*ti*lo*ma*ni*a \cmark & tri*co*ti*lo*ma*ni-a \xmark \\
tricotomia & tri*co*to*mi*a \cmark & tri*co*to*mi-a \xmark \\
tricô & tri-cô \xmark & tri*cô \cmark \\
tricromia & tri*cro*mi*a \cmark & tri*cro*mi-a \xmark \\
tríduo & trí*du.o \xmark & trí*duo \cmark \\
triênio & tri*ê*ni.o \xmark & tri*ê*nio \cmark \\
trifólio & tri*fó*li.o \xmark & tri*fó*lio \cmark \\
trifório & tri*fó*ri.o \xmark & tri*fó*rio \cmark \\
trigonometria & tri*go*no*me*tri*a \cmark & tri*go*no*me*tri-a \xmark \\
trilogia & tri*lo*gi*a \cmark & tri*lo*gi-a \xmark \\
trinitário & tri*ni*tá*ri.o \xmark & tri*ni*tá*rio \cmark \\
trinômio & tri-nô*mi.o \xmark & tri*nô*mio \cmark \\
tríodo & trí-o*do \xmark & trí*o*do \cmark \\
trio & tri*o \cmark & tri-o \xmark \\
trióxido & tri-ó*xi*do \xmark & tri*ó*xi*do \cmark \\
tripanossomíase & tri*pa*nos*so*mí-a*se \xmark & tri*pa*nos*so*mí*a*se \cmark \\
trisavô & tri*sa-vô \xmark & tri*sa*vô \cmark \\
trissomia & tris*so*mi*a \cmark & tris*so*mi-a \xmark \\
trítio & trí*ti.o \xmark & trí*tio \cmark \\
triunfador & tri-un*fa*dor \xmark & tri*un*fa*dor \cmark \\
triunfalismo & tri-un*fa*lis*mo \xmark & tri*un*fa*lis*mo \cmark \\
triunfalista & tri-un*fa*lis*ta \xmark & tri*un*fa*lis*ta \cmark \\
triunfal & tri-un*fal \xmark & tri*un*fal \cmark \\
triunfante & tri-un*fan*te \xmark & tri*un*fan*te \cmark \\
triunfar & tri-un*far \xmark & tri*un*far \cmark \\
triunfo & tri-un*fo \xmark & tri*un*fo \cmark \\
triunvirato & tri-un*vi*ra*to \xmark & tri*un*vi*ra*to \cmark \\
triúnviro & tri-ún*vi*ro \xmark & tri*ún*vi*ro \cmark \\
trívio & trí*vi.o \xmark & trí*vio \cmark \\
tróclea & tró*cle.a \xmark & tró*clea \cmark \\
troia & troi*a \cmark & troi-a \xmark \\
trombocitopenia & trom*bo*ci*to*pe*ni*a \cmark & trom*bo*ci*to*pe*ni-a \xmark \\
tropologia & tro*po*lo*gi*a \cmark & tro*po*lo*gi-a \xmark \\
truculência & tru*cu*lên*ci.a \xmark & tru*cu*lên*cia \cmark \\
truísmo & tru-ís*mo \xmark & tru*ís*mo \cmark \\
tsonga & t.son*ga \xmark & tson*ga \cmark \\
tsunami & t.su*na*mi \xmark & tsu*na*mi \cmark \\
tua & tu*a \cmark & tu-a \xmark \\
tugúrio & tu*gú*ri.o \xmark & tu*gú*rio \cmark \\
tuia & tui*a \cmark & tu.i-a \xmark \\
tuim & tu-im \xmark & tu*im \cmark \\
tuiuiú & tui-ui-ú \xmark & tu.i*ui*ú \xmark \\
túlio & tú*li.o \xmark & tú*lio \cmark \\
tumescência & tu*mes*cên*ci.a \xmark & tu*mes*cên*cia \cmark \\
tumultuário & tu*mul*tu-á*ri.o \xmark & tu*mul*tu*á*rio \cmark \\
tungstênio & tungs*tê*ni.o \xmark & tungs*tê*nio \cmark \\
tupaia & tu*pai*a \cmark & tu*pai-a \xmark \\
turboélice & tur*bo-é*li*ce \xmark & tur*bo*é*li*ce \cmark \\
turbulência & tur*bu*lên*ci.a \xmark & tur*bu*lên*cia \cmark \\
turgência & tur*gên*ci.a \xmark & tur*gên*cia \cmark \\
turgescência & tur*ges*cên*ci.a \xmark & tur*ges*cên*cia \cmark \\
turíngia & tu*rín*gi.a \xmark & tu*rín*gia \cmark \\
tutoria & tu*to*ri*a \cmark & tu*to*ri-a \xmark \\
tuxaua & tu*xau*a \cmark & tu*xau-a \xmark \\
tzarista & t.za*ris*ta \xmark & tza*ris*ta \cmark \\
ufania & u*fa*ni*a \cmark & u*fa*ni-a \xmark \\
ufologia & u*fo*lo*gi*a \cmark & u*fo*lo*gi-a \xmark \\
uísque & u-ís*que \xmark & u*ís*que \cmark \\
uíste & u-ís*te \xmark & u*ís*te \cmark \\
ultrassonografia & ul*tras*so*no*gra*fi*a \cmark & ul*tras*so*no*gra*fi-a \xmark \\
umbaúba & um*ba-ú*ba \xmark & um*ba*ú*ba \cmark \\
umbria & um*bri*a \cmark & um*bri-a \xmark \\
úmbrio & úm*bri.o \xmark & úm*brio \cmark \\
união & u*ni-ão \xmark & u*ni*ão \cmark \\
unicórnio & u*ni*cór*ni.o \xmark & u*ni*cór*nio \cmark \\
unitário & u*ni*tá*ri.o \xmark & u*ni*tá*rio \cmark \\
universitário & u*ni*ver*si*tá*ri.o \xmark & u*ni*ver*si*tá*rio \cmark \\
urânia & u*râ*ni.a \xmark & u*râ*nia \cmark \\
urânio & u*râ*ni.o \xmark & u*râ*nio \cmark \\
uranometria & u*ra*no*me*tri*a \cmark & u*ra*no*me*tri-a \xmark \\
ureia & u*rei*a \cmark & u*rei-a \xmark \\
uremia & u*re*mi*a \cmark & u*re*mi-a \xmark \\
urgência & ur*gên*ci.a \xmark & ur*gên*cia \cmark \\
urinário & u*ri*ná*ri.o \xmark & u*ri*ná*rio \cmark \\
urnário & ur*ná*ri*o \cmark & ur*ná*ri-o \xmark \\
urologia & u*ro*lo*gi*a \cmark & u*ro*lo*gi-a \xmark \\
uropatágio & u*ro*pa*tá*gi.o \xmark & u*ro*pa*tá*gio \cmark \\
uropígio & u*ro*pí*gi.o \xmark & u*ro*pí*gio \cmark \\
urticária & ur*ti*cá*ri.a \xmark & ur*ti*cá*ria \cmark \\
uruguaio & u*ru*guai*o \cmark & u*ru*guai-o \xmark \\
usuário & u*su-á*ri.o \xmark & u*su*á*rio \cmark \\
usucapião & u*su*ca*pi-ão \xmark & u*su*ca*pi*ão \cmark \\
usufruir & u*su*fru-ir \xmark & u*su*fru*ir \cmark \\
usufrutuário & u*su*fru*tu-á*ri.o \xmark & u*su*fru*tu*á*rio \cmark \\
usurário & u*su*rá*ri.o \xmark & u*su*rá*rio \cmark \\
utensílio & u*ten*sí*li.o \xmark & u*ten*sí*lio \cmark \\
utilitário & u*ti*li*tá*ri.o \xmark & u*ti*li*tá*rio \cmark \\
utopia & u*to*pi*a \cmark & u*to*pi-a \xmark \\
uvaia & u*vai*a \cmark & u*vai-a \xmark \\
úvea & ú*ve*a \cmark & ú*ve-a \xmark \\
uveíte & u*ve-í*te \xmark & u*ve*í*te \cmark \\
vacância & va*cân*ci.a \xmark & va*cân*cia \cmark \\
vacaria & va*ca*ri*a \cmark & va*ca*ri-a \xmark \\
vacínio & va*cí*ni.o \xmark & va*cí*nio \cmark \\
vacuidade & va*cu-i*da*de \xmark & va*cu-i*da*de \xmark \\
vacúolo & va*cú-o*lo \xmark & va*cú*o*lo \cmark \\
vácuo & vá*cu.o \xmark & vá*cuo \cmark \\
vadiaria & va*di*a*ri*a \cmark & va*di*a*ri-a \xmark \\
vadia & va*di*a \cmark & va*di-a \xmark \\
vaia & vai*a \cmark & vai-a \xmark \\
valáquio & va*lá*qui.o \xmark & va*lá*quio \cmark \\
valência & va*lên*ci.a \xmark & va*lên*cia \cmark \\
valentia & va*len*ti*a \cmark & va*len*ti-a \xmark \\
valetudinário & va*le*tu*di*ná*ri.o \xmark & va*le*tu*di*ná*rio \cmark \\
valia & va*li*a \cmark & va*li-a \xmark \\
vanádio & va*ná*di.o \xmark & va*ná*dio \cmark \\
vanglória & van*gló*ri.a \xmark & van*gló*ria \cmark \\
vanilóquio & va*ni*ló*qui.o \xmark & va*ni*ló*quio \cmark \\
vareio & va*rei*o \cmark & va*rei-o \xmark \\
variância & va*ri*ân*ci.a \xmark & va*ri*ân*cia \cmark \\
vária & vá*ri.a \xmark & vá*ria \cmark \\
variável & va*ri-á*vel \xmark & va*ri*á*vel \cmark \\
varíola & va*rí-o*la \xmark & va*rí*o*la \cmark \\
vário & vá*ri.o \xmark & vá*rio \cmark \\
varonia & va*ro*ni*a \cmark & va*ro*ni-a \xmark \\
várzea & vár*ze.a \xmark & vár*zea \cmark \\
vascaíno & vas*ca-í*no \xmark & vas*ca*í*no \cmark \\
vasectomia & va*sec*to*mi*a \cmark & va*sec*to*mi-a \xmark \\
vaticínio & va*ti*cí*ni.o \xmark & va*ti*cí*nio \cmark \\
vazia & va*zi*a \cmark & va*zi-a \xmark \\
vazio & va*zi*o \cmark & va*zi-o \xmark \\
vedoria & ve*do*ri*a \cmark & ve*do*ri-a \xmark \\
veemência & ve*e*mên*ci.a \xmark & ve*e*mên*cia \cmark \\
veia & vei*a \cmark & vei-a \xmark \\
veículo & ve-í*cu*lo \xmark & ve*í*cu*lo \cmark \\
velário & ve*lá*ri.o \xmark & ve*lá*rio \cmark \\
velhacaria & ve*lha*ca*ri*a \cmark & ve*lha*ca*ri-a \xmark \\
velharia & ve*lha*ri*a \cmark & ve*lha*ri-a \xmark \\
velocimetria & ve*lo*ci*me*tri*a \cmark & ve*lo*ci*me*tri-a \xmark \\
velório & ve*ló*ri.o \xmark & ve*ló*rio \cmark \\
venefício & ve*ne*fí*ci.o \xmark & ve*ne*fí*cio \cmark \\
venereologia & ve*ne*re*o*lo*gi*a \cmark & ve*ne*re*o*lo*gi-a \xmark \\
venéreo & ve*né*re.o \xmark & ve*né*reo \cmark \\
vênia & vê*ni.a \xmark & vê*nia \cmark \\
ventania & ven*ta*ni*a \cmark & ven*ta*ni-a \xmark \\
ventoinha & ven*to-i*nha \xmark & ven*to*i*nha \cmark \\
ventriloquia & ven*tri*lo*qui*a \cmark & ven*tri*lo*qui-a \xmark \\
veraneio & ve*ra*nei*o \cmark & ve*ra*nei-o \xmark \\
verborragia & ver*bor*ra*gi*a \cmark & ver*bor*ra*gi-a \xmark \\
vergência & ver*gên*ci.a \xmark & ver*gên*cia \cmark \\
verônica & ve-rô*ni*ca \xmark & ve*rô*ni*ca \cmark \\
vestiaria & ves*ti*a*ri*a \cmark & ves*ti*a*ri-a \xmark \\
vestiário & ves*ti-á*ri.o \xmark & ves*ti*á*rio \cmark \\
véstia & vés*ti*a \cmark & vés*ti-a \xmark \\
vestígio & ves*tí*gi.o \xmark & ves*tí*gio \cmark \\
vestuário & ves*tu-á*ri.o \xmark & ves*tu*á*rio \cmark \\
veterinária & ve*te*ri*ná*ri.a \xmark & ve*te*ri*ná*ria \cmark \\
veterinário & ve*te*ri*ná*ri.o \xmark & ve*te*ri*ná*rio \cmark \\
vexatório & ve*xa*tó*ri.o \xmark & ve*xa*tó*rio \cmark \\
vexilologia & ve*xi*lo*lo*gi*a \cmark & ve*xi*lo*lo*gi-a \xmark \\
viário & vi-á*ri.o \xmark & vi*á*rio \cmark \\
viático & vi-á*ti*co \xmark & vi*á*ti*co \cmark \\
viável & vi-á*vel \xmark & vi*á*vel \cmark \\
via & vi*a \cmark & vi-a \xmark \\
vibratório & vi*bra*tó*ri.o \xmark & vi*bra*tó*rio \cmark \\
vibrião & vi*bri-ão \xmark & vi*bri*ão \cmark \\
vicário & vi*cá*ri.o \xmark & vi*cá*rio \cmark \\
vício & ví*ci.o \xmark & ví*cio \cmark \\
vidência & vi*dên*ci.a \xmark & vi*dên*cia \cmark \\
videoconferência & vi*de*o*con*fe*rên*ci.a \xmark & vi*de*o*con*fe*rên*cia \cmark \\
vídeo & ví*de.o \xmark & ví*deo \cmark \\
vidraçaria & vi*dra*ça*ri*a \cmark & vi*dra*ça*ri-a \xmark \\
vidraria & vi*dra*ri*a \cmark & vi*dra*ri-a \xmark \\
viés & vi-és \xmark & vi*és \cmark \\
vietcongue & vi*et*con*gue \cmark & vi*et-con*gue \xmark \\
vigairaria & vi*gai*ra*ri*a \cmark & vi*gai*ra*ri-a \xmark \\
vigária & vi*gá*ri.a \xmark & vi*gá*ria \cmark \\
vigário & vi*gá*ri.o \xmark & vi*gá*rio \cmark \\
vigência & vi*gên*ci.a \xmark & vi*gên*cia \cmark \\
vigia & vi*gi*a \cmark & vi*gi-a \xmark \\
vigilância & vi*gi*lân*ci.a \xmark & vi*gi*lân*cia \cmark \\
vigília & vi*gí*li.a \xmark & vi*gí*lia \cmark \\
vigorexia & vi*go*re*xi*a \cmark & vi*go*re*xi-a \xmark \\
vilania & vi*la*ni*a \cmark & vi*la*ni-a \xmark \\
vilipêndio & vi*li*pên*di.o \xmark & vi*li*pên*dio \cmark \\
vináceo & vi*ná*ce.o \xmark & vi*ná*ceo \cmark \\
violácea & vi*o*lá*ce*a \cmark & vi*o*lá*ce-a \xmark \\
violáceo & vi*o*lá*ce.o \xmark & vi*o*lá*ceo \cmark \\
violência & vi*o*lên*ci.a \xmark & vi*o*lên*cia \cmark \\
virologia & vi*ro*lo*gi*a \cmark & vi*ro*lo*gi-a \xmark \\
virulência & vi*ru*lên*ci.a \xmark & vi*ru*lên*cia \cmark \\
visionário & vi*si*o*ná*ri.o \xmark & vi*si*o*ná*rio \cmark \\
vistoria & vis*to*ri*a \cmark & vis*to*ri-a \xmark \\
vitalício & vi*ta*lí*ci.o \xmark & vi*ta*lí*cio \cmark \\
vitimologia & vi*ti*mo*lo*gi*a \cmark & vi*ti*mo*lo*gi-a \xmark \\
vitória & vi*tó*ri.a \xmark & vi*tó*ria \cmark \\
vítreo & ví*tre.o \xmark & ví*treo \cmark \\
vitriólico & vi*tri-ó*li*co \xmark & vi*tri*ó*li*co \cmark \\
vitríolo & vi*trí-o*lo \xmark & vi*trí*o*lo \cmark \\
vitupério & vi*tu*pé*ri.o \xmark & vi*tu*pé*rio \cmark \\
viúva & vi-ú*va \xmark & vi*ú*va \cmark \\
viúvo & vi-ú*vo \xmark & vi*ú*vo \cmark \\
vivência & vi*vên*ci.a \xmark & vi*vên*cia \cmark \\
vixnuísmo & vix*nu-ís*mo \xmark & vix*nu*ís*mo \cmark \\
vocabulário & vo*ca*bu*lá*ri.o \xmark & vo*ca*bu*lá*rio \cmark \\
voivodia & voi*vo*di*a \cmark & voi*vo*di-a \xmark \\
voleio & vo*lei*o \cmark & vo*lei-o \xmark \\
volemia & vo*le*mi*a \cmark & vo*le*mi-a \xmark \\
volfrâmio & vol*frâ*mi.o \xmark & vol*frâ*mio \cmark \\
volteio & vol*tei*o \cmark & vol*tei-o \xmark \\
volumetria & vo*lu*me*tri*a \cmark & vo*lu*me*tri-a \xmark \\
voluntário & vo*lun*tá*ri.o \xmark & vo*lun*tá*rio \cmark \\
volúpia & vo*lú*pi.a \xmark & vo*lú*pia \cmark \\
voo & vo*o \cmark & vo-o \xmark \\
vovô & vo-vô \xmark & vo*vô \cmark \\
voyeurismo & voy-eu*ris*mo \xmark & voy-eu*ris*mo \xmark \\
voyeurista & voy-eu*ris*ta \xmark & voy-eu*ris*ta \xmark \\
voyeurístico & voy-eu*rís*ti*co \xmark & voy-eu*rís*ti*co \xmark \\
vozerio & vo*ze*ri*o \cmark & vo*ze*ri-o \xmark \\
vreia & vrei*a \cmark & vrei-a \xmark \\
vulcanologia & vul*ca*no*lo*gi*a \cmark & vul*ca*no*lo*gi-a \xmark \\
xaria & xa*ri*a \cmark & xa*ri-a \xmark \\
xariá & xa*ri-á \xmark & xa*ri*á \cmark \\
xátria & xá*tri.a \xmark & xá*tria \cmark \\
xenofilia & xe*no*fi*li*a \cmark & xe*no*fi*li-a \xmark \\
xenofobia & xe*no*fo*bi*a \cmark & xe*no*fo*bi-a \xmark \\
xenomania & xe*no*ma*ni*a \cmark & xe*no*ma*ni-a \xmark \\
xenônio & xe-nô*ni.o \xmark & xe*nô*nio \cmark \\
xerocópia & xe*ro*có*pi.a \xmark & xe*ro*có*pia \cmark \\
xeroftalmia & xe*rof*tal*mi*a \cmark & xe*rof*tal*mi-a \xmark \\
xerografia & xe*ro*gra*fi*a \cmark & xe*ro*gra*fi-a \xmark \\
xerostomia & xe*ros*to*mi*a \cmark & xe*ros*to*mi-a \xmark \\
xibiu & xi*biu \cmark & xi*bi.u \xmark \\
xilocaína & xi*lo*ca-í*na \xmark & xi*lo*ca*í*na \cmark \\
xilografia & xi*lo*gra*fi*a \cmark & xi*lo*gra*fi-a \xmark \\
xilomancia & xi*lo*man*ci*a \cmark & xi*lo*man*ci-a \xmark \\
xintoísmo & xin*to-ís*mo \xmark & xin*to*ís*mo \cmark \\
xintoísta & xin*to-ís*ta \xmark & xin*to*ís*ta \cmark \\
xivaísmo & xi*va-ís*mo \xmark & xi*va*ís*mo \cmark \\
xivaíta & xi*va-í*ta \xmark & xi*va*í*ta \cmark \\
zagaia & za*gai*a \cmark & za*gai-a \xmark \\
zaragatoa & za*ra*ga*to*a \cmark & za*ra*ga*to-a \xmark \\
zebuíno & ze*bu-í*no \xmark & ze*bu*í*no \cmark \\
zeladoria & ze*la*do*ri*a \cmark & ze*la*do*ri-a \xmark \\
zimbório & zim*bó*ri.o \xmark & zim*bó*rio \cmark \\
zircônio & zir-cô*ni.o \xmark & zir*cô*nio \cmark \\
zizânia & zi*zâ*ni.a \xmark & zi*zâ*nia \cmark \\
zodíaco & zo*dí-a*co \xmark & zo*dí*a*co \cmark \\
zombaria & zom*ba*ri*a \cmark & zom*ba*ri-a \xmark \\
zoofilia & zo*o*fi*li*a \cmark & zo*o*fi*li-a \xmark \\
zoogeografia & zo*o*ge*o*gra*fi*a \cmark & zo*o*ge*o*gra*fi-a \xmark \\
zoolagnia & zo*o*lag*ni*a \cmark & zo*o*lag*ni-a \xmark \\
zoologia & zo*o*lo*gi*a \cmark & zo*o*lo*gi-a \xmark \\
zoósporo & zo-ós*po*ro \xmark & zo*ós*po*ro \cmark \\
zootecnia & zo*o*tec*ni*a \cmark & zo*o*tec*ni-a \xmark \\
zoo & zo*o \cmark & zo-o \xmark \\
zoroástrico & zo*ro-ás*tri*co \xmark & zo*ro*ás*tri*co \cmark \\
zurraria & zur*ra*ri*a \cmark & zur*ra*ri-a \xmark \\
total correct & 1848 & 3137 \\

\end{longtable}


\printbibliography

\end{document}
