% xelatex --shell-escape article.tex 
% pandoc -s article.tex --pdf-engine=xelatex --pdf-engine-opt=-shell-escape -o article.html
\documentclass{article}
\usepackage[english]{babel}
%\usepackage{testhyphens}
\usepackage{multicol}
\usepackage{multirow}
\usepackage{verbatim}
\usepackage{hyperref}
\usepackage{amssymb}% http://ctan.org/pkg/amssymb
\usepackage{pifont}% http://ctan.org/pkg/pifont
\newcommand{\cmark}{\ding{51}}%
\newcommand{\xmark}{\ding{55}}%
\usepackage{longtable}
\usepackage[style=authoryear, backend=biber]{biblatex}
\addbibresource{references.bib}
\usepackage[inline]{enumitem}
\usepackage{listings}
\usepackage{hyperref}
\usepackage{cleveref}
\usepackage{fancyvrb}
\usepackage{soul}
\usepackage{siunitx}
\sisetup{
  group-separator = {\,},
  group-minimum-digits = 3,
  output-decimal-marker = {.},
}
\usepackage{booktabs}

\usepackage[textsize=tiny,textwidth=3.5cm,color=blue!10]{todonotes}

%%%%%%% https://tex.stackexchange.com/questions/16790/write18-capturing-shell-script-output-as-command-variable
\usepackage{xparse}
\ExplSyntaxOn
\NewDocumentCommand{\captureshell}{som}
 {
  \sdaau_captureshell:Ne \l__sdaau_captureshell_out_tl { #3 }
  \IfBooleanT { #1 }
   {% we may need to stringify the result
    \tl_set:Nx \l__sdaau_captureshell_out_tl
     { \tl_to_str:N \l__sdaau_captureshell_out_tl }
   }
  \IfNoValueTF { #2 }
   {
    \tl_use:N \l__sdaau_captureshell_out_tl
   }
   {
    \tl_set_eq:NN #2 \l__sdaau_captureshell_out_tl
   }
 }

\tl_new:N \l__sdaau_captureshell_out_tl

\cs_new_protected:Nn \sdaau_captureshell:Nn
 {
  \sys_get_shell:nnN { #2 } { } #1
  \tl_trim_spaces:N #1 % remove leading and trailing spaces
 }
\cs_generate_variant:Nn \sdaau_captureshell:Nn { Ne }
\ExplSyntaxOff
%%%%%%%



\title{Improving the portuguese hyphenation rules}
\author{Leonardo Araújo \and Aline de Lima Benevides}
\begin{document}
\VerbatimFootnotes
\maketitle

\captureshell*[\NumberOfDefaultRules]{wc -l ../data/default.TeX.pt-br.patterns | cut -d' ' -f1}

\captureshell*[\NumberOfNewRules]{wc -l ../data/patch.TeX.pt-br.patterns | cut -d' ' -f1}

\captureshell*[\DictionarySize]{../scripts/dicstats.sh} % number of words that have at least one hyphenation
\captureshell*[\CorpusSize]{../scripts/dicstats.sh -1} % total number of words in the corpus used (same as $ wc -l ../data/hyphenatios.csv)

% https://www.texdev.net/2009/10/06/what-does-write18-mean/
% number of words that hyphenations were found in N online dictionaries
\captureshell*[\NumberOfSixHyphens]{../scripts/dicstats.sh 6}  
\captureshell*[\NumberOfFiveHyphens]{../scripts/dicstats.sh 5}
\captureshell*[\NumberOfFourHyphen]{../scripts/dicstats.sh 4}
\captureshell*[\NumberOfThreeHyphens]{../scripts/dicstats.sh 3}
\captureshell*[\NumberOfTwoHyphens]{../scripts/dicstats.sh 2}
\captureshell*[\NumberOfOneHyphens]{../scripts/dicstats.sh 1}
\captureshell*[\NumberOfNoHyphens]{../scripts/dicstats.sh 0}
% run to update values
%\immediate\write18{../scripts/hyphenationagreements.sh > ../data/hyphagreements}
\captureshell*[\NumberOfSixAgrees]{cat ../data/hyphenations6.dic | wc -l}
% run to update hyphen5 dic
%\immediate\write18{../scripts/get51.sh > ../data/hyphenations5.dic}
%\immediate\write18{../manualcorrections.sh ../data/hyphenations5.dic}
\captureshell*[\NumberOfFiveAgrees]{cat ../data/hyphenations5.dic | wc -l} %#{../scripts/gethyphenationagreements.sh 5}
\captureshell*[\NumberOfSixFiveAgrees]{cat ../data/hyphenations6.dic ../data/hyphenations5.dic | wc -l}
\captureshell*[\NumberOfFourAgrees]{../scripts/gethyphenationagreements.sh 4}
\captureshell*[\NumberOfThreeAgrees]{../scripts/gethyphenationagreements.sh 3}
\captureshell*[\NumberOfTwoAgrees]{../scripts/gethyphenationagreements.sh 2}
\captureshell*[\NumberOfOneAgrees]{../scripts/gethyphenationagreements.sh 1}

\captureshell*[\ExceptionRulesCV]{../scripts/listrules.sh 1}
\captureshell*[\ExceptionRulesLR]{../scripts/listrules.sh 2[lr]}
\captureshell*[\ExceptionRulesCLNH]{../scripts/listrules.sh [cln]2h}
\captureshell*[\ExceptionRulesThree]{../scripts/listrules.sh 3}
\captureshell*[\ExceptionRulesFour]{../scripts/listrules.sh 4}


\captureshell*[\NumberOfAppProp]{grep ':' ../data/hyphenations.csv | wc -l} % number of apparent proparoxytones

% number of words correctly hyphenated using default TeX rules when analysing the words in hyphenations6.dic
\captureshell*[\NumberOfCorrectSix]{../scripts/resultfilemetrics.sh -f ../data/hyphenations6result-default-patterns.dic -c} 

% number of words with wrong hyphenations using default TeX rules when analysing the words in hyphenations6.dic
\captureshell*[\NumberOfWrongSix]{../scripts/resultfilemetrics.sh -f ../data/hyphenations6result-default-patterns.dic -w}

% number of words with missing hyphenations using default TeX rules when analysing the words in hyphenations6.dic
\captureshell*[\NumberOfMissingSix]{../scripts/resultfilemetrics.sh -f ../data/hyphenations6result-default-patterns.dic -m}

\newcounter{numberRulesGroups}


\begin{abstract}
    Portuguese hyphenation rules are available for more than 35
    years and have done a good job. Nonetheless they still make mistakes and
    leave some hyphenation points unmarked. Although most undetected
    hyphenation points are located near word boundaries, what will be
    irrelevant for \TeX{} typographic purposes, they are still useful to
    hyphenate proper nouns, new words or pseudoword, and for usage in other
    applications, such as text-to-speech conversion. A list of
    \num{\DictionarySize} hyphenated words acquired from online dictionaries was
    used along \emph{patgen} to create improved rules, leading to better hyphenation
    of Portuguese words.
\end{abstract}

\section{Introduction}\label{sec-intro}

Hyphenation in text wrapping was not used for a long time. Words should fit
entirely in a line, or they would be broken in arbitrary places.  Initially, no
markers were used to indicate word wrapping, leading to potential confusion and
unintended interpretations. As a result, orthographers advocated for the
introduction of a sign to indicate such breaks. Portuguese faced the same
gradual introduction of a hyphenation sign to mark words wrapping across lines.
Even though the usage of a hyphenation sign ($=$) was advocated by
orthographers \parencite{gandavo1574}, few documents used such sign until the end of
the 18th century \parencite{araujo2015}.

In some cases, hyphenation hinders smooth reading and should be avoided in
child literature. In opposition, both excessive spacing and insufficient
spacing between words also impose difficulty in the reading process, making
hyphenation fundamental when texts use short line lengths. As a line gets
shorter, the number of breaking candidates between words decreases, leading to
awkward spaces between words and among letters. For that matter, automatic
hyphenation plays an important role in good typesetting.

%Automatic hyphenation plays an important role in good typesetting and
%it is fundamental when a text uses short line lengths. As a line gets
%shorter, the number of breaking candidates between words decreases,
%creating awkward spaces between words and among letters.

\TeX{} is a typesetting system which carefully deals with these issues,
automatically arranging text on a page to create a good reading experience.
Automatic hyphenation is an important part on this process,
promoting an even-tempered distribution of elements on the page.
%Hyphenation is therefore an important part on this process.
%A good readability depends on an even-tempered distribution of elements
%on the page.
Line height and line length, paragraph length, font size and typeface,
letter-spacing, and word-spacing are some factors which influence the text legibility
and readability. Space between words should not be too long creating lakes and
rivers in the text, nor too tight, impairing the legibility and readability.

Another important matter to consider is ambiguities that might be created when
a word is partitioned during the hyphenation process. In English we should
avoid hyphenations such as \emph{re-cover}, \emph{re-form}, \emph{re-sign}, \emph{the-rapist},
\emph{depart-mental}, and \emph{mans-laughter} (in Portuguese, some examples are:
\emph{de-putada}, \emph{fede-ração}, \emph{acu-mula}, \emph{após-tolo},
\emph{cú-bico}). Hyphenations that might lead the reader to pronounce a word
incorrectly should also be prevented. That is the case of
\emph{considera-tion}, in Enslish (and \emph{pe-rigo} in Portuguese).

In some situations, hyphenation is also a matter of style. Some partitioning
choices sound better than others. These conflicting alternatives typically
arises when a words has many possible hyphenation points. Consider
\emph{ar-chae-ol-o-gist} (\emph{or ar-che-ol-o-gist}), which is preferably partitioned as
\emph{archae-ologist} (or \emph{arche-ologist}) in opposition to \emph{archaeol-ogist} (or
\emph{archeol-ogist}) or \emph{archaeolo-gist} (or \emph{archeolo-gist}). It is preferable to keep
whole morphemes together. In the previous example: \emph{archae} (or \emph{arche}, meaning
``ancient'', ``primitive'')
%\footnote{archae- comes from the Ancient Greek ἀρχαῖος (arkhaîos, `ancient',
%`primitive'), from ἀρχή (arkhḗ, `beginning').}
%https://en.wiktionary.org/wiki/archae-
and \emph{-ologist} (``one who studies the topic''). In Portuguese, it is also more
elegant to avoid splits between double consonants or vowels, even if an
hyphenation point do exists between those letters. For example,
\emph{pressu-rizar} is preferable over \emph{pres-surizar} and
\emph{empreen-dedor} is preferable over \emph{empre-endedor}. Even so,
exceptions exists, it is preferable to partition \emph{micro-organismo}
rather than \emph{microor-ganismo} (keeping morphemes together is favored
over splitting a double vowel). Numerous factors come into play when choosing
the optimal hyphenation point for a word.

The general rule for Portuguese hyphenation is to split a word into its
syllables. A syllable is made of a mandatory nucleus, filled by a vowel, and
optional peripheral consonants (before or after the nucleus). In some
situations, the syllabic division does not respect the ethnologic constituents.
The usage of the prefixes \emph{bis-} and \emph{in-} are examples of this
circumstance. The correct syllabifications are \emph{bi-sa-vô},
\emph{i-nobs-tan-te}\footnote{The rule of syllable
division could lead to two possible partitions: \emph{i-nobs-tan-te} and
\emph{in-obs-tan-te}, but the first is preferable.}, and \emph{i-na-ti-vo}, 
where the prefixes are split into two syllables. 
But, as pointed out previously, it is preferable to keep
morphemes together rather than splitting them apart. Therefore we should
favor the hyphenation \emph{bisa-vô} over \emph{bi-savô}, \emph{inobs-tante} 
over \emph{i-nobstante}, and \emph{ina-tivo} over \emph{i-nativo}. 
This last example could also lead to a misunderstanding
since the word \emph{nativo} (\emph{native} in English) emerges from this word
break. 

Each language has its hyphenation rules, which can be categorized into two
groups: those driven by morphology (etymology) and those driven by
pronunciation.  An algorithmic approach employs a logical system to analyze
words and apply the hyphenation rules of a specific language. Since hyphenation
rules vary significantly between languages, a unique algorithm must be
developed for each one. While a logic-based system can be efficient and
compact, it still needs to address exceptions through hard-coded rules.

%The automation of this process might use a dictionary based approach, which
%will restrict the hyphenation possibilities to those entries in the dictionary,
%an algorithmic base approach, which might be applied to whatever sequence found
%in a text, or a mix of both approaches. 
%%An algorithmic approach might use a logic system to analyse words and apply
%%the hyphenation rules of a given language. 
%A rule base model might include the recognition of prefixes,
%suffixes, morphemes, or some sequences suitable for the inclusion of a break.
%Another approach is the usage of pattern matching. Given a corpus of
%hyphenation examples in a language, it is necessary to identify those sequence
%of letters that define a point where hyphenation is suitable or not. Patterns
%might include prefixes, suffixes, exceptions and special hyphenation rules of a
%given language. Such approach might be used for different languages, by
%providing hyphenation examples in a target language to extract patterns and
%rules.

The automation of this process can employ various approaches, including:
\begin{description}
\item[Dictionary-Based Approach:] This method restricts hyphenation
    possibilities to entries found in the dictionary.
\item[Algorithmic-Based Approach:] This approach can be applied to any sequence
    encountered in a text.
\item[Rule-Based Model:] This model recognizes prefixes, suffixes, morphemes,
    or specific sequences suitable for hyphenation.
\item[Pattern Matching:] By using a corpus of hyphenation examples in a
    language, this approach identifies letter sequences that determine suitable
    hyphenation points. Patterns encompass prefixes, suffixes, exceptions, and
    special hyphenation rules of the language.
\item[Mixed Approach:] This approach combines two or more of the previously
    described methods to enhance hyphenation accuracy and flexibility.
\end{description}

Analyzing only the immediate surroundings may not always suffice to determine a
potential hyphenation point. For instance, consider \emph{de-moc-ra-cy} and \emph{dem-o-crat},
where the immediate surrounding of the letter \emph{e} does not provide a clear
indication of the hyphenation point. 
Even when facing chosen patterns that typically make hyphenation
straightforward, exceptions can arise. Take the sequence \emph{tion}, for example.
It may seem logical to place a break before this pattern, but the word \emph{cation}
is hyphenated as \emph{cat-ion}, highlighting the influence of etymology on the way
a word is split.

A word with multiple meanings may have distinct hyphenations based on its intended meaning.
`For instance, the Swedish word form \emph{glassko} has three different
meanings, and can be hyphenated as \emph{glas-sko} (glass shoe),
\emph{glass-ko} (ice cream cow) and in the non-standard way,
\emph{glass-sko} (ice cream shoe)' \parencite{nemeth2006}. In Portuguese, the
word \emph{sublinha} might be hiphenated in two different ways:
\emph{su-bli-nha} when representing the inflected form of the verb
\emph{sublinhar} (to underline) or \emph{sub-li-nha} when refering
to the line under (underline as a noun).

% hyphenation might change with syntactic function
% part of speech (PoS)

Machine learning techniques were employed to hyphenate Norwegian text
\parencite{kristensen2001}. The study revealed that, overall, the \TeX{} approach
outperformed a neural network.  Both methods demonstrated similar performance
in identifying correct hyphenation points and minimizing incorrect hyphenations
when tested on a small word list (with the neural network slightly
outperforming in correctly recognizing hyphenation points).  However, when
tested on a larger word list, the \TeX{} approach proved superior in avoiding
incorrect hyphenations compared to the neural network.


The original \TeX{} hyphenation algorithm, introduced by %Knuth in 1977
\textcite{knuth1977}, primarily focused on the English language and employed three
main rules:
\begin{enumerate*}[label=(\arabic*)]
 \item suffix removal; 
 \item prefix removal; and 
 \item the Vowel-Consonant-Consonant-Vowel (VCCV) breaking rule\footnote{The 
     VCCV rule in hyphenation patterns, places the syllable boundary between 
     two consecutive consonants when they appear between two vowels. For example, 
     in the word \emph{sudden}, the syllable break occurs between the \emph{d} and 
     the second \emph{d}, making it \emph{sud-den}. 
     The VCCV pattern is a relatively common syllable division pattern in English.
     This rule ensures accurate 
     hyphenation, maintaining word readability and pronunciation when words are 
     split at the end of a line in printed text.}. 
\end{enumerate*}
Tests demonstrated that it could identify 40\% of allowable hyphen locations
\parencite{liang1983}.  The \TeX{} hyphenation algorithm later adopted the approach
proposed by Frank M. Liang, which involves competing patterns.  The \TeX{}82
algorithm employs five alternating levels of hyphenating and inhibiting
patterns. The program for pattern generation, known as PATGEN, was created by
%Liang 
\parencite{liangbreitenlohner1999} and has been utilized to generate
hyphenating patterns for numerous languages
\parencite{sojka1995,sojka1995a,sojka2005thesis,sojka2003,scannell2003}. It involves
sweeping a database of hyphenated words in a language to identify both
hyphenating and inhibiting patterns, ultimately creating a list of competing
patterns for that specific language.

The effective hyphenation of words by \TeX{} will actually depends on the following factors:
\begin{enumerate*}[label=(\arabic*)]
    \item document language, which will determine which set of patterns to apply;
    \item characters used, since some might block hyphenation at their edges;
    \item the value of the internal variables \verb|lefthyphenmin| and \verb|righthyphenmin|\footnote{
	The variables \verb|lefthyphenmin| and \verb|righthyphenmin| are language dependant and
	are defined in \emph{tlpobj} files (\verb|/usr/local/texlive/20XX/tlpkg/tlpobj/hyphen-xxxxxx.tlpobj|). 
	Default values varies in the range from 1 to 3. 
    	English and Portuguese, for example, use \verb|lefthyphenmin=2| and \verb|righthyphenmin=3|.},
	% ls /usr/local/texlive/2023/tlpkg/tlpobj/*.tlpobj | while read -r filename; do grep -Po "lefthyphenmin=[0-9]\s+righthyphenmin=[0-9]" $filename; done | tr '\t' ' ' | tr -s ' ' | sort | uniq -c
	%      17 lefthyphenmin=1 righthyphenmin=1
        %       6 lefthyphenmin=1 righthyphenmin=2
        %       1 lefthyphenmin=1 righthyphenmin=3
        %      49 lefthyphenmin=2 righthyphenmin=2
        %       9 lefthyphenmin=2 righthyphenmin=3
        which defines the minimum sequence length of characters at the left and right borders
        before any hyphenation is allowed.
\end{enumerate*}

Despite the fact that \TeX{} hyphenation algorithm and rules are old, they are,
to these days, the most frequently used approach, even outside the \TeX{}'s
world. The grounds for this is Hunspell, a spell checker and morphological
analyzer that is adopted in many softwares (e.g. LibreOffice, OpenOffice.org,
Mozilla Firefox, Mozilla Thunderbird, Google Chrome, macOS, InDesign, memoQ,
Opera, Affinity Publisher, among others \parencite{hunspell}). Hunspell uses \TeX{}
hyphenation rules \parencite{hunspellhyphen,levien1998}, making \TeX{} hyphenation
widespread in the computer world. That is a result of \TeX{} approach
simplicity and versatility.  The algorithm works effectively, as it already
supports rules for 66 languages \parencite{texhyphenrules}, and offers the
flexibility to create rules for any currently unsupported languages.

Unfortunately, certain hyphenation rules cannot be implemented using the \TeX{}
hyphenation algorithm. In languages like German, hyphenation can lead to letter
change or insertion. Additionally, compound words lack hyphens, resulting in
extended letter sequences without visible separation and even repetitions of
the same letter, as seen in examples like \emph{Wasserrinne} and
\emph{Schifffahrt}.  Furthermore, the German spelling reform made some changes,
making it necessary to create a different set of rules for German hyphenation.
For example, the word \emph{Schiffahrt} should be hyphenated as
\emph{Schiff-fahrt}, preserving the \emph{f}s from each word that makes this
compound. The hyphenation should insert an \emph{f} that is not part of the
written form. That was not a problem for the old written form of the word:
\emph{Schifffahrt}. Also, the old hyphenating rules of German grammar stated
the hyphenation \emph{Bäk-ker} for the word \emph{Bäcker}, \emph{Zuk-ker} for
the word \emph{Zucker} and \emph{pak-ken} for the word \emph{packen}. Now those
words are hyphenated as \emph{Bä-cker}, \emph{Zu-cker} and \emph{pa-cken},
respectively. Some words have also changed their hyphenation point after the
spelling reform. For example, \emph{Fen-ster} became \emph{Fens-ter} and
\emph{mei-stens} became \emph{meis-tens}. Some problems in compound word
hyphenation in \TeX{} are discussed in \textcite{sojka1995a}.


% Some hyphenation rules are not possible with TeX algorithm.
% For example, the german word Schiffahrt should be hyphenated as Schiff-fahrt.
% Although it is no longer the recommend written form, it was the most usual
% until 2003 (see graph in schifffart.png from google ngrams).
% Also the old hyphenating rules of German grammatic stated the following
% hyphenations Bäk-ker (for the word Bäcker, what is now hyphenated as Bä-cker),
% Zuk-ker (for the word Zucker, what is now hyphenated as Zu-cker),
% pak-ken (for packen, and now pa-cken) and trok-ken (for trocken, now tro-cken).


%which contain archeo can be alternatively spelt with archaeo or archæo.


%GÂNDAVO, Pero de Magalhães de. Regras que ensinam a maneira de escrever e
%orthographia da lingoa portuguesa, etc. Lisboa: Antonio Gonsalvez, 1574

%  HIFENIZAÇÃO EM PORTUGUÊS
% Antonio Martins de ARAÚJO and Toru MARUYAMA



Moreover, as mentioned earlier, certain hyphenations may be preferred for
stylistic reasons, or to avoid ambiguity, or for better reading experience.
Some words might have multiple hyphenations, depending on the intended meaning.
In such cases, \TeX{} may not efficiently address the hyphenation challenge, as
it would necessitate case-by-case handling.


\section{Patterns for \TeX{} hyphenation}\label{sec-patt4TeX} 

To simplify, if we consider only the Latin alphabet, with no diacritcs, the patterns used 
in \TeX{} hyphenation are of the form: \texttt{\^{}\textbackslash{}.?[0-9]?([a-z]+[0-9]?)+\textbackslash{}.?\$},
where we have described it using a regular expression\footnote{
    Regular expressions (regex) are powerful search patterns used in text processing to find, match, and manipulate strings of text. They are a fundamental tool in programming and are supported in many programming languages.
    The regex given here uses the Perl syntax and might be broken in the following
    parts: \texttt{\^{}} and \texttt{\$} mark the start and end of the string;
    \texttt{\textbackslash{}.?} specifies an optional period; \texttt{[0-9]?} allows for an optional single digit;
    \texttt{([a-z]+[0-9]?)+} matches sequences of lowercase letters interleaved with optional digits.
}.
One example of such pattern is \texttt{4z1z2}, which is composed of a sequence
of letters and numbers. Odd numbers indicate a
good hyphenation point, whereas even numbers indicate a bad place to break.
The given example states that the sequence has a good breaking point between
the first and the second \emph{z} and an hyphenation should be inhibited before
the first \emph{z} and after the second \emph{z}. For example, the hyphenation
of the word \emph{piz-za}, \emph{fiz-zle} and \emph{mez-zanine} use this rule,
where we see the hyphen placed between the two \emph{z}'s and no hyphen before,
nor after the \emph{z}'s. Patterns may also use period symbol (\emph{.}) to
indicate word boundaries. The pattern \texttt{.sh2} applies to beginning of
words, implying that the \emph{s} and the \emph{h} should stick together in
beginning of a word and an hyphenation should also be inhibited after the
\emph{h}. For example, this pattern is used in \emph{Sher-lock}.

Hyphenation rules are organized in levels, from 1 to 9, where odd numbers
represent hyphenating levels and even numbers represent inhibiting levels. Each
level works as an exception level of it predecessor. For example, the rule
\texttt{sh1er} indicate a good hyphenation point between the \emph{h} and the
\emph{e} in the sequence \emph{sher}. A rule at a higher level, as
\texttt{.sh2}, implies an exception to the lower level rule. When we see
\emph{sher}, in the beginning of a word, the rule \texttt{.sh2} applies and
hyphenation proposed by the lower level rule \texttt{sh1er} should be hindered.
That is the case in the hyphenation of the word \emph{Sher-lock}. The full
example is provided in \Cref{sherlockhyphenation}, where we might see all pertinent
English rules taking place in the hyphenation of \emph{Sher-lock} and \emph{sher-iff}.

\noindent\begin{minipage}{\linewidth}
\begin{lstlisting}[language={}, caption={Example of rules applyied in the
hyphenation of the word \emph{Sherlock}. Example done using a port of \TeX{}'s
hyphenation algorithm to Go provided at
\url{https://github.com/speedata/hyphenation}.}, label=sherlockhyphenation]
   .   s   h   e   r   l   o   c   k   .
     0   0   2   |   |   |   |   |   |    .sh2
     0   2   0   |   |   |   |   |   |    s2h
     0   0   1   0   0   |   |   |   |    sh1er
     |   |   |   0   1   0   |   |   |    r1l
     |   |   |   0   3   0   4   |   |    r3lo4
     |   |   |   |   |   |   0   0   1    ck1
max: 0   2   2   0   3   0   4   0   1
final: s   h   e   r - l   o   c   k -
\end{lstlisting}
\end{minipage}

A pattern will consist of a string made of characters (from the language
alphabet) possibly with a number in between, expressing the
hyphenation/inhibition level and possibly word boundaries marker (the period)
at the pattern edges. When there is no number between characters in a pattern,
a zero is assumed, which means \emph{undefined} and no hyphenation point will
be suggest at that location.



\section{PATGEN}\label{sec-patgen} 

PATGEN utilizes a list of hyphenated words to define rules at various levels
and lengths, based on the patterns found in the data. It starts with short
patterns and incrementally increases their length until reaching the maximum
pattern length allowed by the user.  The objective is to keep the patterns as
concise as possible, as this enhances their generalizability.  As it advances
and incorporates longer patterns, PATGEN establishes exceptions.  In certain
cases, analyzing long patterns may be necessary, as some hyphenation points
could depend on characters far away from the breaking point\footnote{Some
    examples of hyphenation dependency on characters far from the break point:
    \emph{dem-o-crat} and \emph{de-moc-ra-cy}; \emph{as-pi-rin} and
    \emph{aspir-ing}; \emph{de-mon-stra-tive} and \emph{dem-on-stra-tion}.
}.


PATEGEN works on glyph\footnote{\emph{Glyph} is commonly used in linguistics,
typography, and computer graphics to refer to a specific graphical
representation of a character or symbol, which can be the entire symbol or a
distinct visual element within it.} indices rather than character codes. Each
glyph is represented by a single byte. That amount to 256 indices, where 13 of
them are reserved for the digits 0-9 and the characters `.', `-', and `*'. The
remaining 243 are used to represent symbols of the given language.  To run
PATGEN, a translation file is necessary. This file defines the values of
certain language-specific parameters (in the first line) and enumerates the
various forms in which language symbols may appear (all subsequent lines).  In
the first line, positions 1 and 2 are used to set the value of
\verb|lefthyphenmin|, and positions 3 and 4 are used to set the value of
\verb|righthyphenmin|.  These values determine the minimum length of a string
that may be generated by a hyphenation procedure. To set a single-digit value,
leave the first position blank, i.e., place a space in position 1 and 3 for
\verb|lefthyphenmin| and \verb|righthyphenmin|, respectively.  Positions 5, 6
and 7 are used to define alternative values for the special characters `.', `-'
and `*'.  The following lines use a delimiter to enclose each `letter' of the
desired language alphabet, including its alternative representations.  The
first position of the line defines the delimiter, and each symbol of the
language can occupy as many positions as necessary, as long as the reserved
value for the delimiter is not used in the symbol's definition.  Consider the
following example for defining the letter `e' in Portuguese: 
\begin{verbatim}
XeXEXéXêXÉXÊX\'{e}X\^{e}X\'{E}X\^{E}XX 
\end{verbatim}
It represents the many ways in which the letter `e' might be found: lowercase,
uppercase, with or without acute or circumflex accents. Note that we have used
the direct input (using UTF-8 or other encoding that support the accented `e'
character) and also the compositional counterpart using the escape codes that
define each accent.  In this definition, we have assumed that the many forms in
which we may find the character \emph{e} will be equivalent for
hyphenation (pattern matching) purpose. As another example, see the next line
which defines the character $\pi$ (taken from \textcite{haralambous2021}):
\begin{verbatim} 
#p#P#\varpi ## 
\end{verbatim}

PATGEN also needs a dictionary file, which is a list of pre-hyphenated words
from which PATGEN extracts patterns to create hyphenation rules. To ensure
PATGEN's proper functionality, the translation file and the dictionary file
must utilize the same encoding, even if it is a multi-byte encoding. The
translation file describes how to handle byte sequences representing a glyph,
and PATGEN will work seamlessly when there are at most 243 symbols in the given
language.

%Those rules are organized
%in different levels, from 1 to 9, where odd numbers represent hyphenating
%levels and even numbers represent inhibiting levels. Each level works as an
%inhibiting level of it predecessor. 
%A pattern will consist of a string made of
%characters (from the language alphabet) possibly with a number in between,
%expressing the hyphenation/inhibition level and possibly word boundaries marker
%(the period) at the pattern edges. When there is no number between characters
%in a pattern, a zero is assumed, which means \emph{undefined} and no
%hyphenation point will be suggest at that location.




\section{\TeX{} hyphenation rules for portuguese}\label{sec-tex-hyphen-pt}
\textcite{rezende1987} was the first to create patterns for Portuguese hyphenation
in \TeX{}. The pattern set was last updated in 2015, incorporating
contributions from %José João Dias Almeida 
\textcite{hyphpt}, resulting in a total
of 307 rules\footnote{The set of rules for Portuguese hyphenation is short when
compared to rules in other languages. For example, English currently has 4938
rules, Russian has 7023 rules and German has 34011 rules.}. 
These rules effectively hyphenate the majority of Portuguese words.

However, in light of certain cases, we propose an analysis of the default rules
to identify areas for improvement. By addressing specific issues and
considering non-typical patterns, we aim to enhance hyphenation accuracy. The
methodology used for this analysis is described in Section
\ref{sec-methodology}.

%
% FILE="../data/default.TeX.pt-br.patterns"; grep "^1..$" "$FILE" | awk '{print substr($0, 0, 2)}' | sort | uniq -c | while read line; do p=$(echo $line | cut -d' ' -f2); c=$(grep "^$p[^0-9]$" "$FILE" | awk '{print substr($0, 3)}' | tr '\n' ' '); printf "$line $c \n"; done | cut -d' ' -f1 | paste -sd+ | bc
%
Out of the default \TeX{} rules, 252 (82\%) follow the pattern \verb|1CV|, which
represents the recurring CV syllables in Portuguese.  As for the consonants,
there are typically 18 considered (\verb|b|, \verb|c|, \verb|ç|, \verb|d|,
\verb|f|, \verb|g|, \verb|j|, \verb|k|, \verb|l|, \verb|m|, \verb|n|, \verb|p|,
\verb|r|, \verb|s|, \verb|t|, \verb|v|, \verb|x|, and \verb|z|) that can
combine with 14 vowels (\verb|a|, \verb|e|, \verb|i|, \verb|o|, \verb|u|,
\verb|á|, \verb|â|, \verb|ã|, \verb|é|, \verb|í|, \verb|ó|, \verb|ú|, \verb|ê|,
and \verb|õ|), resulting in these CV patterns that indicate favorable
hyphenation points before the consonants\footnote{The complete list of these
252 patterns is given here: \ExceptionRulesCV{}.}. 
It is worth noting that there might
be other rules or exceptions in the hyphenation patterns not covered by these
default rules.  To accommodate exceptions, \textcite{rezende1987} proposes the 
following rules: 
\begin{itemize}
    \item 20 rules that created for cases involving consonants \verb|b|, \verb|c|, \verb|d|, \verb|f|, \verb|g|, \verb|k|, \verb|p|, \verb|t|, \verb|v|, or \verb|w| followed by \verb|l| or \verb|r|\footnote{The 20 additional rules are: \ExceptionRulesLR{}.};
    \item 3 rules for \verb|c|, \verb|l| or \verb|n| followed by \verb|h|\footnote{Those 3 rules are: \ExceptionRulesCLNH{}.}.
    % FILE="../data/default.TeX.pt-br.patterns"; grep -v "^1..$" "$FILE" | grep "2" | awk '{print substr($0, 2, 1)}' | sort | uniq | while read line; doc=$(grep "^1${line}2.$" "$FILE" | tr -d '12' | awk '{print substr($0, 2, 1)}' | tr '\n' ' '); printf "$line $c\n"; done
    \item 23 distinct patterns were introduced to indicate hyphenation points between vowels or between \verb|c|'s, \verb|r|'s, and \verb|s|'s\footnote{Those patterns are: \ExceptionRulesThree{}.}. 
    % FILE="../data/default.TeX.pt-br.patterns"; grep -v "^1..$" "$FILE" | grep -v "2" | grep "3" | wc -l
    \item 8 patterns adhere to the \verb|1[gq]u4V|, signaling beneficial hyphenation points before \verb|g| or \verb|q|, followed by a sequence of \verb|u| and a vowel, with an inhibiting point between the \verb|u| and the subsequent vowel\footnote{They are the following: \ExceptionRulesFour{}.}.
    \item 1 pattern represented as \verb|1-|, denoting that a hyphen indeed serves as a beneficial hyphenation point.
\end{itemize}

% - contabilizar a quantidade de palavras que não são contempladas pela proposta do autor
% - analisar a necessidade de excluir alguma regra (ao invés de apenas acrescentar novas)


\section{Methodology}\label{sec-methodology}

To assess the performance of each set of rules, we require a collection of
correctly hyphenated Portuguese words. We employed word frequency data to
ensure we selected representative, commonly used words, avoiding excessively
rare ones. Our comprehensive word list was curated using online dictionaries,
Portuguese corpora, and verified against Portuguese grammars to validate
hyphenation rules. We also performed manual corrections where necessary.
We should not expect an entire corpus to follow a set of conventions for a written 
language nor uniformity among a variety of corpora, especially those harvested from the Internet.
``It is notoriously difficult to prescribe rules governing the use of a written language; it is even more difficult
to get people to `follow the rules.' This is in large part due to the nature of written language, in which
the conventions are not always in line with actual usage and are subject to frequent change'' \parencite[p.~14]{palmer2010}.


\subsection{Data collection}\label{sec-data-coll}

Initially, we selected
\href{https://www.linguateca.pt/cetenfolha/index_info.html}{CETENFolha}
as the source corpus, but this approach presented two issues: many
words were still missing, and the corpus was based on text from 1994, prior to
the implementation of the Orthographic Agreement. The Orthographic Agreement
of the Portuguese Language was conducted in 1990, and its transition period
started in 2009, becoming mandatory in 2016. However, even after the agreement,
some words continue to have different spellings in the participating countries.
To address these idiosyncrasies, we aim to develop hyphenation rules that
accommodate variations in spelling. For instance, the word \emph{reception}
might be written as \emph{receção} (Portugal) or \emph{recepção} (Brazil); the
word \emph{action} might be written as \emph{acção} (Portugal) or \emph{ação}
(Brazil); and the word \emph{project} might be written as \emph{projecto}
(Portugal) or \emph{projeto} (Brazil).

% We have used the Portuguese Wikipedia dump as our Portuguese corpus. Texts in Wikipedia might be written by Brazilian or Portuguese contributors, the both spellings may appear.

% cat /ms/downloads/samples/wikipedia/ptwiki-latest-pages-articles-multistream_wordlist.txt | tr -dc '0-9\n' | paste -sd+ | bc
% 302226482
% cat /ms/downloads/samples/wikipedia/ptwiki-latest-pages-articles-multistream_wordlist.txt | awk '{CUMCOUNT+=$1; if(CUMCOUNT < 287115158) print;}' >/tmp/ptwikitop95.txt
% cat /ms/downloads/samples/wikipedia/ptwiki-latest-pages-articles-multistream_wordlist.txt | awk '{CUMCOUNT+=$1; if(CUMCOUNT < 299204217) print;}' >/tmp/ptwikitop99.txt
% cat /tmp/ptwikitop99.txt | tr -d "0-9 " |  while read -r word; do if ! grep -q "^${word}," ../data/hyphenations.csv; then echo $word; fi; done > /tmp/ptwikitop99_notinlist.txt
% cat /home/leoca/ee/research/hyphenation/listapalavrasnet.txt | awk '{print tolower($0)}' | while read -r word; do if ! grep -qP "^\s*[0-9]+\s*${word}$" /ms/downloads/samples/wikipedia/ptwiki-latest-pages-articles-multistream_wordlist.txt; then echo $word; fi; done > /tmp/list_of_palavrasnet_not_in_wikipedia.txt

% wc -l /ms/downloads/samples/wikipedia/ptwiki-latest-pages-articles-multistream_wordlist.txt
% 1746947
% wc -l /tmp/ptwikitop99_notinlist.txt
% 42017
% from 1746947 of wikipedia, only 42017 are not in list

Due to the peculiarities mentioned, we made the decision to integrate the word
list from \href{https://www.palavras.net/}{Palavras NET}, and subsequently, we
augmented it with words sourced from the
\href{https://pt.wikipedia.org}{Portuguese Wikipedia} dump. From the Wikipedia
data, we narrowed down the selection to a subset of \num{50721} words,
accounting for 95\% of occurrences in the corpus. This threshold was set to
filter out typos and infrequent words. The initial word list became quite
extensive, comprising \num{\CorpusSize} words. However, we refined the list by
retaining only those words for which we could find hyphenation data in at least
one of the following online dictionaries:
\href{https://michaelis.uol.com.br/}{Michaelis},
\href{https://dicionario.priberam.org/}{Priberam},
\href{https://pt.wiktionary.org}{Wikcionário},
\href{https://aulete.com.br/}{Aulete},
\href{http://www.portaldalinguaportuguesa.org/}{Portal da Língua Portuguesa}, and
\href{https://www.dicio.com.br/}{Dicio}.
Consequently, this curation process resulted in a final dictionary containing
\num{\DictionarySize} words.  

Table \ref{tblstats} presents the number of distinct words with a given number
of hyphenations found in the dictionaries.  Additionally, it displays the
number of words that received the highest agreed-upon hyphenation.  For
instance, the word \emph{como} was consistently hyphenated as \emph{co-mo} in
all six dictionaries, therefore, it contributes to both the counts in the
column representing six hyphenations (first column).  On the other hand, the
word \emph{sua} received different hyphenations: \emph{su-a} in four
dictionaries, \emph{sua} in one dictionary, and one dictionary did not provide
any hyphenation result.  This leads to an increment in the first line, second
column (since hyphenations were found in five dictionaries) and the second
line, third column (since the most frequent hyphenation was found in four
dictionaries). It is evident now that the number of identical hyphenations can
exceed the total number of hyphenations found, especially for smaller values.

\begin{table}
\centering
\caption{Considering the six dictionaries used, the first line of this table
presents the number words that a given number of hyphenations were found in the
dictionaries. The second line presents the number of words that have a given
number of hyphenation as its most frequent form found in the
dictionaries.}\label{tblstats}
\small
\begin{tabular}{@{ }l@{ } *{7}{@{ }S[table-format=6]@{ }}}
    hyphenations & 6 & 5 & 4 & 3 & 2 & 1 & 0 \\
    \hline
    hyphenations found & \NumberOfSixHyphens{} & \NumberOfFiveHyphens{} &
    \NumberOfFourHyphen{} & \NumberOfThreeHyphens{} & \NumberOfTwoHyphens{} &
    \NumberOfOneHyphens{} & \NumberOfNoHyphens{} \\

    same hyphenations & \NumberOfSixAgrees{} & \NumberOfFiveAgrees{} &
    \NumberOfFourAgrees{} & \NumberOfThreeAgrees{} & \NumberOfTwoAgrees{} &
    \NumberOfOneAgrees{} & n/a
\end{tabular}
\end{table}

In summary, our data compilation involved \emph{CETENFolha}, \emph{Palavras
NET}, and \emph{Wikipedia}, and hyphenated the words using the original \TeX{}
hyphenation rules.  Subsequently, we compared the hyphenated results with
those obtained from the dictionary. By conducting a thorough performance
appraisal, we systematized the errors and delved into the relevant literature
to identify potential rules that could complement and enhance the existing
hyphenation system.


\section{Writing systems}\label{sec-writing-systems}

There is a diverse array of writing systems, categorized into logographic,
syllabic, and alphabetic systems. Although distinct, these systems can be built
on an interplay of these categories \parencite{coulmas2003,palmer2010}. 
The principles guiding an alphabetic or phonemic writing system vary significantly 
based on the language and its history. Here are some key points worth highlighting:
\begin{enumerate}[label=\arabic*)]
    \item The most common principle involves representing the sounds of a
	language with written symbols, using one or a combination of symbols to
	represent each sound (e.g., the orthographic system of Portuguese uses
	an alphabet of 27 letters to represent different sounds of the
	language).
    \item Etymology also plays a crucial role, where the spelling of a word
	reflects its origins and historical development (e.g., the French
	language often reflects the Latin roots of words in its spelling).
    \item Morphology serves as a guide in structuring many languages, using
	specific letters or symbols to indicate word endings, prefixes, or
	suffixes (e.g., Russian uses different forms of the Cyrillic alphabet
	to indicate gender and case in its nouns).
    \item The evolution of a language over time also contributes to its written
	form (e.g., the spelling of many English words has changed over time to
	reflect changes in pronunciation).
\end{enumerate}
Orthographic systems for languages with logographic writing systems, such as
Chinese ideograms, cuneiform writing, and Egyptian hieroglyphs, differ
significantly from alphabetic writing systems. In logographic systems,
individual symbols or characters represent entire words or ideas, rather than
phonemes or sounds. As a result, the principles guiding their orthographic
systems are based more on semantic and visual principles than on phonemic
principles.
Hyphenation is a feature of alphabetic scripts where words are composed of letters, 
and spaces or hyphens are used to separate words, syllables, or parts of words. 
Logographic writing systems, such as the Chinese, 
function differently and do not employ hyphenation \parencite{honorof2006}.



\section{Grammar rules for hyphenation in Portuguese}\label{sec-grammar-port}

Portuguese employs an alphabetical writing system, which means its spelling is
guided by phonological principles \parencite{cagliari2015}. The correlation between
spelling and pronunciation influences word hyphenation, as words are divided
into syllables based on the phonemic system. It is important to note that
different languages follow diverse principles for word division. For example,
English is primarily guided by morphological principles, evident in words like
\emph{walk-ing}, \emph{un-happy}, \emph{work-s}, and \emph{ear-ly}.
Additionally, other factors also impact word hyphenation in English, such as 
the distinction between long and short vowels, which function within the context
of open or closed syllable, repectively; and the presence of doubled consonants and
digraphs \parencite{lin2011,yavas2020}. 
Furthermore, there exist numerous exceptions that may be categorized into rules.
% https://english.stackexchange.com/questions/385/what-are-the-rules-for-splitting-words-at-the-end-of-a-line

Merely stating that an orthographic system is guided by phonological issues
does not necessarily mean that its hyphenation rules directly mirror the
phonetic counterpart. This is notably apparent in Portuguese, where a strict
one-to-one correspondence between letters and sounds is not always observed.
The orthographic system operates according to its specific rules. For example,
consider consonant clusters that create a single sound (digraphs) in words like
\emph{\underline{ch}ave}, \emph{i\underline{lh}a}, \emph{\underline{sh}ampoo}, \emph{ca\underline{rr}o}, and \emph{ma\underline{ss}a}.
While these digraphs are pronounced as one sound within a single syllable,
their representation in writing determines how they are divided. Specifically,
different consonants within a digraph must remain together, whereas identical
consonants are separated. As a result, we observe hyphenations like
\emph{cha-ve}, \emph{i-lha}, and \emph{sham-poo}, but \emph{car-ro} and \emph{mas-sa}. 

%When there is no space left in line for a whole word, it might be split in two
%using a hyphen as an indication of such procedure. 

In Portuguese, hyphenations are allowed on syllables boundaries and, in
general, follow phonological principles.  According to the Grammar
\parencite{cunha2016,bergstrom2011,cegalla2008}, some rules might still apply:
\begin{enumerate}
    \item\label{rule-di-triphthong} diphthong or triphthong should not be split
	(e.g. \emph{m\underline{ui}-to}, \emph{Pa-ra-g\underline{uai}});
    \item\label{rule-unstressed} the sequences \emph{ia}, \emph{ie}, \emph{io}, \emph{oa}, \emph{ua},
	\emph{ue} and \emph{uo}, when in final unstressed position, should not be split 
	(e.g. \emph{gló-r\underline{ia}}, \emph{vi-tó-r\underline{ia}}, 
	\emph{cá-r\underline{ie}}, \emph{es-pé-c\underline{ie}}, 
	\emph{Má-r\underline{io}}, \emph{má-g\underline{oa}}, 
	\emph{ré-g\underline{ua}}, \emph{tê-n\underline{ue}}, 
	\emph{con-tí-g\underline{uo}}, \emph{am-bí-g\underline{uo}});
    \item\label{rule-c-clusters} consonant clusters starting a syllable should not be
	split (e.g. \emph{\underline{pn}eu-má-ti-co}, \emph{\underline{ps}i-có-lo-go}, 
	\emph{\underline{mn}e-mô-ni-co});
    \item\label{rules-digraphs-ns} the digraphs	\emph{ch}, \emph{lh}, \emph{nh}
	should not be split (e.g. \emph{ra-\underline{ch}ar},
	\emph{a-bro-\underline{lh}os}, \emph{ma-\underline{nh}ã};
    \item\label{rule-guqu} bigrams like \emph{gu} and \emph{qu} whose vowel \emph{u} is not 
	pronounced are never separated from the vowel or diphthong that follows it 
	(e.g. \emph{ne-\underline{gue}}, \emph{pe-\underline{que}});
    \item\label{rule-nasalization} since they are digraphs, a vowel and its following 
	nasalization marker (a graphic nasal consonant) should not be split (e.g. \emph{\underline{am}-bição},
	\emph{m\underline{an}-cha});
    \item\label{rule-decreasing} decreasing diphthongs should not be split (e.g. 
	\emph{\underline{ai}-ro-so}, \emph{ca-d\underline{ei}-ra}, 
	\emph{o-ra-ç\underline{ão}});
    \item\label{rule-rising} rising diphthong should not be split (e.g.
	\emph{a-b\underline{ai}-xo}; \emph{c\underline{au}-te-la}, 
	\emph{pa-p\underline{éi}s}, \emph{cha-p\underline{éu}}, 
	\emph{pre-f\underline{ei}-to}, \emph{r\underline{eu}-nir},
	\emph{n\underline{oi}-te}; \emph{ca-la-b\underline{ou}-ço}, 
	\emph{as-te-r\underline{ói}-de}; 
	\emph{re-tri-b\underline{ui}});
    \item\label{rule-singlev} disyllables whose syllable has a single vowel should 
	not be split (e.g. \emph{\underline{a}to}, \emph{ru\underline{a}}, 
	\emph{\underline{ó}dio}, \emph{\underline{u}nha});
    \item\label{rule-orphan} words with more than two syllables, when divided, cannot 
	isolate a syllable composed of a single vowel (e.g. \emph{\underline{a}gos-to}, 
	\emph{la-go\underline{a}}, \emph{\underline{i}da-de});


    \hspace{10em} \hbox to 5cm{\leaders\hbox to 10pt{\hss . \hss}\hfil} 

    \item\label{rule-hiatus} hiatus vowels and those vowel sequences where each vowel
	belongs to a different syllable should be split (e.g.
	\emph{sa\underline{-ú-}de}, \emph{ra\underline{-i-}nha}, \emph{d\underline{o-e}r},
	\emph{v\underline{o-o}s}), the same procedure is used splitting diphthongs
	in different syllables (e.g. \emph{c\underline{ai-ai}s}) or diphthong and
	vowel in different syllables (e.g. \emph{en-s\underline{ai-o}s});
    \item\label{rule-consonants} consonant sequences, when in different syllables, should
	be split (e.g. \emph{a\underline{f-t}a}, \emph{a\underline{b-d}i-car},
	\emph{re\underline{s-c}i-são}, \emph{a\underline{b-s}o-lu-to});
    \item\label{rule-digraphs} the following consonant digraphs should be split:
	\emph{rr}, \emph{ss}, \emph{mm}, \emph{nn}, \emph{sc}, \emph{sç} and
	\emph{xc} (e.g. \emph{te\underline{r-r}a}, \emph{pro-fe\underline{s-s}or},
	\emph{co-mu\underline{\emph{m-m}}en-te}, \emph{co\underline{n-n}os-co},
	\emph{de\underline{s-c}er}, \emph{cre\underline{s-ç}a}, \emph{e\underline{x-c}e-der}).
\end{enumerate}

Rules \ref{rule-singlev} and \ref{rule-orphan} are primarily aimed at ensuring
proper readability of the text, aligning with \TeX{} approach to deal with
widows and orphans. As mentioned in \Cref{sec-intro}, the variables
\verb|lefthyphenmin| and \verb|righthyphenmin| are used to control the minimum
length for fragments of hyphenated words. For that reason, rules
\ref{rule-singlev} and \ref{rule-orphan} are fiddling rules in \TeX{}
hyphenation, and that is not reason to take than in account in our analysis to
improve Portuguese hyphenation rules. Furthermore, get the full hyphenation of
words might be useful for text-to-speech applications \cite{libossek2000,trogkanis2010}.

Additionally, it is advisable to refrain from
splitting disyllables consisting of four letters (e.g., \emph{para},
\emph{como}, \emph{cede}). This considerations also lead to more aesthetically
pleasing and intelligible text, and \TeX{}'s control of isolated fragments already
address this issue.

%Note that rules \ref{rule-singlev} and \ref{rule-orphan} are fiddling rules in
%\TeX{} hyphenation, since \TeX{} already has directives to avoid hyphenated
%widows and orphans.  As mentioned in \Cref{sec-intro}, the variables
%\verb|lefthyphenmin| and \verb|righthyphenmin| are used to control the minimum
%length for fragments of hyphenated words.  For this reason, we will not take
%into account rules \ref{rule-singlev} and \ref{rule-orphan} during the analysis
%and development of the hyphenation rules.

In some situations a hyphen should be repeated at the start of the following
line. They are: 
\begin{enumerate}
\item cases where compound words using a hyphen are split across lines (e.g.,
    \emph{couve-/-flor}, \emph{ex-/-presidente}); and
\item cases where splitting a pronoun could result in a different meaning (e.g.,
    \emph{prazer de ver-/-me}\footnote{Possible conveyed meanings: 
    \emph{pleasure in seeing me} or \emph{worm's pleasure}.}). 
\end{enumerate}

%To ensure proper hyphenation and readability, certain guidelines should be
%followed. Avoiding a single vowel at the end or start of a line is crucial
%(e.g., \emph{san-guí-nea} should be split only on the provided hyphens,
%avoiding a split that leaves the final vowel \emph{a} alone at the start of a
%new line: \emph{sanguíne-a}). Additionally, it is advisable to refrain from
%splitting disyllables consisting of four letters (e.g., \emph{para},
%\emph{como}, \emph{cede}).  These considerations lead to more aesthetically
%pleasing and intelligible text.

Systematizing the rules that guide syllable boundaries and, consequently,
hyphenation in Portuguese proved to be fundamental for the improvement of the
\TeX{} hyphenator rules, since they indicate the possible or prohibited contexts
of hyphenation. Expomos, a seguir, uma análise comparativa...

... to continue
\vspace{10ex}



\section{Creating the dictionary}

A bash script, named \texttt{gethyphenations.sh}, was developed to crawl six
online dictionaries and extract word hyphenations whenever available. The
process involves launching parallel threads for each word in the list, querying
a specific online dictionary, and retrieving the corresponding hyphenation. For
this purpose, individual scripts were created for each dictionary:
\texttt{getmichaelishyphenation.sh} for \href{https://michaelis.uol.com.br/}{Michaelis},
\texttt{getpriberamhyphenation.sh} for \href{https://dicionario.priberam.org/}{Priberam},
\texttt{getwiktionaryhyphenation.sh} for \href{https://pt.wiktionary.org}{Wikcionário},
\texttt{getauletehyphenation.sh} for \href{https://aulete.com.br/}{Aulete},
\texttt{getportalhyphenations.sh} for \href{http://www.portaldalinguaportuguesa.org/}{Portal da Língua Portuguesa}, and
\texttt{getdiciohyphenation.sh} for \href{https://www.dicio.com.br/}{Dicio}.
This systematic approach ensures efficient and accurate extraction of
hyphenations from various online sources. The resulting data is stored in a CSV\footnote{
    Comma-separated values (CSV) is a text file format defined in \href{https://datatracker.ietf.org/doc/html/rfc4180}{RFC 4180}.
    In CSV, commas are used to separate values, and each line in the file represents a new record. Records consist of fields separated by delimiters, typically a single comma.
} file named \texttt{hyphenations.csv}.






\subsection{Establishing a gold standard}

To construct our gold standard for hyphenation, we will compare the
hyphenations offered by all six dictionaries. We'll determine the total count
of unique hyphenations across these dictionaries using the script
\texttt{hyphenationagreements.sh}.

As our initial hyphenation reference, we will consider those cases where all
six dictionaries agree. This list comprises \num{\NumberOfSixAgrees} words,
and their hyphenations are stored in \texttt{hyphenations6.dic}.


For any disagreement, we must be cautious and examine each case.
We might be tempted to chose the majority vote approach, selection
those hyphenations that have the most supporters, but it is unclear 
how the hyphenations on those online dictionaries were curated, 
many might use algorithmic approach, leading to potential flaws shared
among them. 
For example, 5 dictionaries use the hyphenation 
\emph{quart-zo} while just one uses the rightful one, \emph{quar-tzo}.
The same happens with \emph{de-sem-ba-i-nhar} and \emph{de-sem-bai-nhar},
the former one receiving the majority of the votes, despite being wrong.
Some strings might accept more than one hyphenation. For example, 
\emph{sub-li-nha} and \emph{su-bli-nha} represent two different words.
The former is the substantive \emph{underline} and the latter is a flexed form of the verb \emph{to underline}.
\todo{Aline, segue a lista dos casos duvidosos que encontrei com votos 5 a 1. Marquei em negrito aqueles
    que considero ser a forma correta.
    quart-zo(5), \textbf{quar-tzo}(1) (assim também trocar quart-zi-to por \textbf{quar-tzi-to)}; \textbf{sub-li-nha}(5), \textbf{su-bli-nha}(1); 
    de-sem-ba-i-nhar(5), \textbf{de-sem-bai-nhar}(1); ap-nei-a(5), \textbf{a-pnei-a}(1); 
    disp-nei-a(5), dispneia(1) (\textbf{dis-pnei-a}, forma não encontrada nos dicionários); 
    hi-pe-ral-ge-si-a(5), \textbf{hi-per-al-ge-si-a}(1);
    in-de-is-cen-te(5), \textbf{in-deis-cen-te}(1); jo-in-vi-len-se(5), \textbf{join-vi-len-se}(1); 
    ne-er-lan-dês(5), \textbf{neer-lan-dês}(1). Os demais casos de 5 a 1 são proparoxítonas aparentes.
}
The procedure to create the second list of hyphenated words was:
\begin{enumerate*}[label=\arabic*)]
  \item select those hyphenations agreed by five dictionaries that have no other hyphenation proposal (5 to 0 votes);
  \item from those that have another proposal, check if it is an apparent proparoxytone,
      \begin{enumerate*}[label=\alph*)]
	  \item if it has a diacritic, select the paroxytone option,
	  \item if not, select the proparoxytone option.
      \end{enumerate*}
  \item if it is not an apparent proparoxytone case, select the hyphenation with five votes.
\end{enumerate*}
This procedure is implemented in the script \texttt{get51.sh}.
Just a few hyphenations were manually corrected. They are: \emph{quar-tzo}, 
\emph{quar-tzi-to}, \emph{su-bli-nha} (was added), \emph{de-sem-bai-nhar},
\emph{a-pnei-a}, \emph{dis-pnei-a}, \emph{hi-per-al-ge-si-a}, \emph{join-vi-len-se}, and \emph{neer-lan-dês}.
The final list is stored in \texttt{hyphenations5.dic} and holds \num{\NumberOfFiveAgrees} additional hyphenations.
Using the two lists created so far, we have in total \num{\NumberOfSixFiveAgrees} hyphenations.

% results from script ./get51.sh
% cases where majority vote might go wrong:
% sub-li-nha(5),su-bli-nha(1) % both are valid
% de-sem-ba-i-nhar(5),de-sem-bai-nhar(1) % de-sem-bai-nhar seems to be the correct choice
% ap-nei-a(5),a-pnei-a(1) % ??
% disp-nei-a(5),dispneia(1) or dis-pnei-a?
% hi-pe-ral-ge-si-a(5),hi-per-al-ge-si-a(1)
% in-de-is-cen-te(5),in-deis-cen-te(1)
% jo-in-vi-len-se(5),join-vi-len-se(1)
% ne-er-lan-dês(5),neer-lan-dês(1)
% the remaining are apparent proparoxytones


For those hyphenations that have 4 votes, we used basically the same approach
as before. We noticed many words had only 4 votes (no alternative hyphenation was provided
by any dictionary), and they were hyphenated as proparoxytones. They were, in fact, 
apparent proparoxytones. Those cases were easy to target by looking
for diacritics in the third to last syllable and a sequence \emph{V-V} in the end.
We transformed those cases into paroxytones by removing the last hyphen.
The remaining cases, where an alternative hyphenation was found, we checked for
diacritics, selecting the paroxytone option when found and selection the
proparoxytone option when not. For the reaming cases with alternative option,
we simply select the most voted one. This procedure is implemented in script 
\texttt{get4.sh}.

% analisar os resultados das hifenizações através da análise de frequência das sílabas
% as silabas de baixa frequência podem ser indicativos de erros... verificar a origem delas
% cat ../data/hyphenations6.dic ../data/hyphenations5.dic ../data/hyphenations4.dic | tr '-' '\n' | sort | uniq -c | sort -nr > /tmp/syls
% grep -P "\s+1\s[^0-9]+$" /tmp/syls | tr -d '1' | sed 's/^\s\+//g' | while read syl; do grep "^$syl-\|-$syl-\|-$syl$" ../data/hyphenations6.dic; done > /tmp/rare
% grep -P "\s+1\s[^0-9]+$" /tmp/syls | tr -d '1' | sed 's/^\s\+//g' | while read syl; do grep "^$syl-\|-$syl-\|-$syl$" ../data/hyphenations5.dic; done >> /tmp/rare
% grep -P "\s+1\s[^0-9]+$" /tmp/syls | tr -d '1' | sed 's/^\s\+//g' | while read syl; do grep "^$syl-\|-$syl-\|-$syl$" ../data/hyphenations4.dic; done >> /tmp/rare
% grep -P "\s+1\s[^0-9]+$" /tmp/syls | tr -d '1' | sed 's/^\s\+//g' | while read syl; do grep --color=always "^$syl-\|-$syl-\|-$syl$" <(cat ../data/hyphenations6.dic ../data/hyphenations5.dic ../data/hyphenations4.dic); done | ansi2html > /tmp/rare.html


% fun-ci-o-ná-ri-o > fun-ci-o-ná-rio
% co-mí-ci-o > co-mí-cio
% con-vê-ni-o > con-vê-nio
% cri-té-ri-o > cri-té-rio
% ple-ná-ri-o > ple-ná-rio
% cre-di-á-ri-o > cre-di-á-rio
% es-tá-gi-o > es-tá-gio
% in-ter-ban-cá-ri-o > in-ter-ban-cá-rio
% ur-gên-ci-a > ur-gên-cia
% i-mo-bi-li-á-ri-a > i-mo-bi-li-á-ria
% re-vo-lu-ci-o-ná-ri-o > re-vo-lu-ci-o-ná-rio
% ve-ló-ri-o > ve-ló-rio
% es-có-ci-a > es-có-cia
% for-mu-lá-ri-o > for-mu-lá-rio
% or-ça-men-tá-ri-o > or-ça-men-tá-rio



% ag-nós-ti-co(5),a-gnós-ti-co(1)   << ?
% co-i-bir(5),coi-bir(1)            << both are correct?
% drui-da(5),dru-i-da(1)            << drui-da is correct, since an hiatus would require an accent dru-í-da
% flui-do(5),flu-i-do(1)            << as there is no diacritic it should be flui-do
%					fluido (sem acento) - substantivo: flui-do (ditongo) -- exemplo: O fluido de freio acabou.
%					fluido (com acento) - forma verbal: flu-í-do (hiato) -- exemplo: A discussão para aprovar reformas no país não tem fluído como esperavam os governantes.
% sub-li-nha(5),su-bli-nha(1)       << sub-linha or su-blinha (sublinhar) 
% man-ti-do(5) mantido(1)           << mantido clearly was not hyphenated 
% quart-zo(5),quar-tzo(1)           << quar-tzo sounds like the correct choice
% den-ti-frí-ci-o(5),den-ti-frí-cio(1) << it should be marked as apparent proparoxytones but no one did


%compare_strings() {
%    # Trim leading and trailing whitespace from the first string
%    string1_trimmed="${1#"${1%%[![:space:]]*}"}"
%    string1_trimmed="${string1_trimmed%"${string1_trimmed##*[![:space:]]}"}"
%
%    # Trim leading and trailing whitespace from the second string
%    string2_trimmed="${2#"${2%%[![:space:]]*}"}"
%    string2_trimmed="${string2_trimmed%"${string2_trimmed##*[![:space:]]}"}"
%
%    if [ "$string1_trimmed" == "$string2_trimmed" ]; then
%        return 0  # Strings are equal
%    else
%        return 1  # Strings are different
%    fi
%}

% grep "(5)" ../data/hyphagreements | grep "(1)" | head | tr -d '()0-9' | while read line; do word1=$(echo $line | cut -d, -f1); word2=$(echo $line | cut -d, -f2); count1=$(grep -o '-' <<< "$word1" | wc -l); count2=$(grep -o '-' <<< "$word2" | wc -l); if [ $count1 -gt $count2 ]; then w1=$(echo $word1 | sed 's/\(.*\)-/\1/'); compare_strings "$w1" "word2"; result=$?; if [ $result -ne 0 ]; then echo "c1: $line $w1 $word2"; od -c <<< "$w1"; od -c <<< "$word2"; fi; elif [ $count1 -lt $count2 ]; then w2=$(echo $word2 | sed 's/\(.*\)-/\1/'); compare_strings "$w2" "word1"; result=$?; if [ $result -ne 0 ]; then echo "c2: $line $word1 $w2"; fi; else compare_strings "$word1" "word2"; result=$?; if [ $result -ne 0 ]; then echo $line; fi; fi; done


After performing all procedures above described to create our golden hyphenation
dictionary, we might still need manual corrections.
All manual corrections are written in the file \texttt{replacements.txt} and
the script \texttt{manualcorrections.sh} is used to perform these corrections.





In some dictionaries, words are usually marked with a colon (`:') to indicate a
dubious ending, classifying them as `apparent proparoxytones'. In Portuguese,
words ending in \emph{ea}, \emph{eo}, \emph{ia}, \emph{ie}, \emph{io},
\emph{ua}, \emph{ue}, or \emph{uo}, with the stressed syllable before these
endings, can be considered either paroxytones ending in a rising diphthong or
proparoxytones ending in a hiatus. The words \emph{his-tó-ri:a} and \emph{on-du-la-tó-ri:o}
are examples of words marked as apparent proparoxytones, meaning they might
be regarded as paroxytones \emph{his-tó-ria} and \emph{on-du-la-tó-rio} or
as proparoxytones \emph{his-tó-ri-a} and \emph{on-du-la-tó-ri-o}.
The script \texttt{hyphenationagreements.sh} might deal with these apparent proparoxytones
in four different ways: we might keep the hyphenations with the colon mark to
indicate an apparent proparoxytones; or we might only the paroxytones or only the proparoxytones
counterpart; or we may count both paroxytones and proparoxytones.
As there is no consensus in the literature, we have decided to count both
hyphenations. In our hyphenation dictionary, we have found \NumberOfAppProp{} words
marked as apparent proparoxytones.
 



\section{Updating the rules}
Let us remember that the standard \TeX{} hyphenation rules, proposed by
\textcite{rezende1987,hyphpt}, when tested against a set of \num{\DictionarySize}
words, present 1,368 (????) errors, of which 1,273 (????) correspond to places
where the hyphenation was not carried out and 125 (????) to erroneous markings.
Faced with the errors made by the standard hyphenator, a set of \num{\NumberOfNewRules} rules were
established, which reduced the errors to 132 (???), with 118 (????) places
where the hyphenation was not marked and 35 (????) where there was erroneous
marking. Such rules are presented and exemplified below, as well as the
necessary exception rules and an example in which the application of the rule
is demonstrated.




% get the error rate from default TeX rules in this dic file
% ./checkhyphenation.sh correct test
% ./gocreateresultfile.sh ../data/hyphenations6.dic ../data/default.TeX.pt-br.patterns > ../data/hyphenations6result-default-patterns.dic
% grep -v "[\.\-]" ../data/hyphenations6result-default-patterns.dic | wc -l
% 15536
% NumberOfCorrectSix NumberOfWrongSix NumberOfMissingSix
% correct: 15536; wrong: 30 (30); missing: 277 (276,1)
Using the default \TeX{} hyphenation rules we have found \NumberOfCorrectSix{}
words correctly hyphenated, \NumberOfWrongSix{} words with wrong hyphenation
points introduced by the default rules, and \NumberOfMissingSix{} words with at
least a missing hyphenation point (among those words only \texttt{ba-í-a} had
two missing hyphenation points).

% cat ../data/default.TeX.pt-br.patterns ../data/patch.TeX.pt-br.patterns > ../data/patched-default.TeX.pt-br.patterns
% ./gocreateresultfile.sh ../data/hyphenations6.dic ../data/patched-default.TeX.pt-br.patterns > ../data/hyphenations6result-patched-patterns.dic
% ./resultfilemetrics.sh -f ../data/hyphenations6result-patched-patterns.dic -a
% correct: 15826; wrong: 2 (2); missing: 13 (13)



\subsection{Results from the new set of rules}
Although the XXX erroneous words have some common characteristics, which would
allow, to some extent, to reduce the amount of hyphenation errors, there is a
limitation arising from the way \TeX{}'s rules are conceived. We present,
below, the systematists found in these data that could be predicted by another
structure of rules:

\begin{description} 
    \item[Morphological determination:] Prefixes such as \emph{re-},
	\emph{sub-}, \emph{cyber-}, and \emph{auto-} require separating the
	prefix from its stem, what runs into phonological issues. For example,
	\emph{reiniciar} and \emph{sublinhar}, as they contain prefixes, should
	be hyphenated as \emph{re-i-ni-ci-ar} and \emph{sub-li-nhar},
	respecting the morphological formation of the word. However,
	considering that the Portuguese language hyphenates its words from
	syllabic phonological correlates, words that require morphological
	information, such as those, do not have their hyphenation performed
	properly. They add up, in our corpus, to twelve words (eight words with
	\emph{re-} and one case with \emph{auto-}, \emph{meta-}, \emph{sub-},
	\emph{cyber-}, \emph{geo-} or \emph{neuro-}, those results are presented
	in Table \ref{tab-mor-syl}).
	% $ { head -n 1 ../data/hyphenations.csv; grep "^sublinhar," ../data/hyphenations.csv; } | column -t -s,
	% su-bli-nhar (priberam, wiktionary, aulete)
	% sub-li-nhar (michaelis, portal, dicio)

	% $ { head -n 1 ../data/hyphenations.csv; grep "^auto[a-z]\+," ../data/hyphenations.csv | grep -v ",,,";} | column -t -s,

	\begin{table}
	\scriptsize
	\caption{Hyphenation in the dictionaries of words having conflict between morphological and syllabic information.}
	\label{tab-mor-syl}
        \begin{tabular}{ll*{6}{c}}
	\toprule
	    word                   & hyphenation  & \multicolumn{6}{c}{dictionary} \\
             &                                    & michaelis & priberam & wikitionary & aulete & portal & dicio \\
	\midrule
        \multirow{2}{*}{sublinhar} & su-bli-nhar  &           & x        & x           & x      &        &       \\
				   & sub-li-nhar  & x         &          &             &        & x      &       \\
	\multirow{2}{*}{reiniciar} & rei-ni-ci-ar &           &          & x           &        & x      &       \\
				   & re-i-ni-ci-ar & x        & x        &             & x      &        & x     \\
	\bottomrule
	\end{tabular}
	\end{table}

    \item[Foreignness:] There is a group of 20 words in the corpus that are
	terminologies or words that were incorporated into the Portuguese
	language without full phonological adaptation, such as
	\emph{darwinismo}, \emph{quilowatt}, and \emph{esfiha}. The lack of
	adaptation makes the phonological pattern very specific to the word,
	making it impossible to incorporate in \TeX{}'s rules alongside the
	other rules. The solution is to add them to the exception word list.

    \item[Word-initial consonant clusters:] Portuguese has few cases of
	consonant clusters at the beginning of a word. They, in general, are
	etymological remnants and are currently unproductive in the language,
	since there are no neologisms with this pattern. Encounters like
	\emph{ps-} and \emph{pn-} are more frequent, as they are present in
	words like \emph{psicologia} and \emph{pneu}, which have an average
	frequency in the language. They might be predicted by rules, which
	comprise XXX words. However, there are consonant clusters that are
	found in very specific and low frequency words. Although possible, it
	is not worth adding very specific rules using the consonant clusters
	contained in the words \emph{dzeta}, \emph{gnu}, \emph{cnidário},
	\emph{ftálico}, and \emph{gnaisse} -- that is, just five words.

    \item[Abbreviations, Acronyms, or Initialisms:] 
       Whether for efficiency, convenience, clarity, or specialized jargon, 
       it is common to use shortened versions of words or phrases. Abbreviation is a method employed to achieve this shortening. In our Portuguese corpus, we find examples such as 
       \emph{etc.}\footnote{Latin expression \emph{et cetera}, meaning `and other similar things'.}, 
       \emph{Dr.}\footnote{\emph{doutor} (doctor, person with PhD title, but popularly used to designate an erudite individual)}, 
       \emph{Exmo.}\footnote{\emph{Excelentíssimo} (honourable)},
       \emph{cap.}\footnote{\emph{capítulo} (chapter)}, \emph{Univ.}\footnote{Universidade (university)}, 
       \emph{ed.}\footnote{\emph{edição} (edition)}, 
       \emph{s.n.}\footnote{sine nomine, Latin expression meaning `without a name', mostly used in the context of publishing.}.

       Another form of abbreviation is an initialism, which consists of using
       the initial letters of words to create a shortened version. However,
       initialisms may not always conform to the hyphenation rules described in
       this work, as they do not necessarily follow the orthographic or
       phonotactic standards of the language. Some abbreviations found in the
       corpus include 
       \emph{SESC}\footnote{Serviço Social do Comércio}, 
       \emph{INSS}\footnote{\emph{Instituto Nacional do Seguro Social} (National Institute of Social Security)}, 
       \emph{PCdoB}\footnote{\emph{Partido Comunista do Brasil} (Communist Party of Brazil)},
       \emph{PM}\footnote{\emph{Polícia Militar} (military police)}, and 
       \emph{UFRJ}\footnote{Universidade Federal do Rio de Janeiro}.

       Acronyms are a specific type of abbreviation where the first letters (or
       groups of letters) of each word are combined to form a new pronounceable
       word. In the corpus, we encounter examples like 
       \emph{Anatel}\footnote{\emph{Agência Nacional de Telecomunicações} (National Telecommunications Agency)},
       \emph{Ovni}\footnote{\emph{Objeto voador não identificado} (unidentified flying object - UFO)},
       \emph{Sida}\footnote{\emph{Síndrome da Imunodeficiência Adquirida} (acquired immunodeficiency syndrome - AIDS)} (in Portugal) and
       \emph{Mercosul}\footnote{\emph{Mercado Comum do Sul} (Southern Common Market).}.

       These various forms of abbreviation play an important role in written
       language, providing concise ways to represent longer words or phrases.
       It is important to note that abbreviations, acronyms, and initialisms
       are generally treated as single units and are not hyphenated.

\end{description}


\section{Creating a new set of rules using Patgen}



\section{Comparing the set of rules} 
Let us remember that the standard \TeX{} hyphenation rules, proposed by
\textcite{hyphpt}, amount to XXX errors, in a set of XXXX words.  Among those, XXX
correspond to places where the hyphenation was not carried out and XXX to
mismarkings. Faced with the errors made by the standard hyphenator, a set of
\num{\NumberOfNewRules} rules were established, which reduced the errors to XXXX,
with XXXX unmarked hyphenation places and XXX mismarkings.  Such rules are
presented and exemplified below, as well as the necessary exception rules and
an example in which the application of the rule is demonstrated.


\section{Universal syllabic pattern generation}
Universal syllabic pattern generation by Jakub Máca, Petr Sojka, Ondřej Sojka (TUG 2023).


\section{Old Text}


Foi criado um conjunto de \num{\NumberOfNewRules} regras para aprimorar as hifenizações no
português. Analisando um conjunto de \num{37798} palavras, as regras padrão apresentam
1368 erros (1273 pontos de hifenização não marcados e 125 pontos marcados
erroneamente)\footnote{A soma não é 1368 pois temos 30 casos em que os dois
tipos de erros ocorrem.}, enquanto as novas regras apresentaram 132 erros (118
pontos de hifenização não marcados e 35 pontos marcados
erroneamente)\footnote{35 palavras apresentam ambos tipos de erro.}.
%%% <<< fazer correção após correção manual das hifenizações tidas como corretas.

Foi criado um conjunto de 37798 palavras obtidas do corpus
\href{https://www.linguateca.pt/cetenfolha/index_info.html}{CETENFolha},
juntamente com palavras do \href{https://www.palavras.net/}{Palavras NET} e
algumas palavras do corpus baseado na
\href{https://pt.wikipedia.org}{Wikipedia}. 
Foram mantidas apenas as palavras
para as quais encontrou-se a hifenização nos seguintes dicionários online:
\href{https://michaelis.uol.com.br/}{Michaelis},
\href{https://dicionario.priberam.org/}{Priberam},
\href{https://pt.wiktionary.org}{Wikcionário},
\href{https://aulete.com.br/}{Aulete},
\href{http://www.portaldalinguaportuguesa.org/}{Portal da Língua Portuguesa} e
\href{https://www.dicio.com.br/}{Dicio}.
Adotamos como hifenização `correta' aquela que ao menos 4 dicionários
concordavam. Algumas hifenizações foram realizadas manualmente pelo autor.
Considerando a frequência de ocorrência das palavras nos corpus, o conjuto de
palavras escolhido representa 55\% do total das ocorrências relativo ao
Wikipedia e 58\% em relação ao CETENFolha.
% Corpus Wikipedia
%
% 166341424 / 302226482 = 0.55
% 
% soma da freq. de occ. das palavras utilizadas
% cat portuguese_4_utf8.dic portuguese_extra_utf8.dic | tr -d '-' | while read -r word; do grep "[0-9]\+ ${word}$" /ms/downloads/samples/wikipedia/ptwiki-latest-pages-articles-multistream_wordlist.txt | head -n 1 | tr -dc '0-9\n'; done | paste -s -d+ - | bc 
% = 166341424
%
% soma da freq. de occ. de todas palavras no corpus
% cat /ms/downloads/samples/wikipedia/ptwiki-latest-pages-articles-multistream_wordlist.txt | tr -dc '0-9\n' | paste -s -d+ - | bc
% = 302226482
%
% CETENFolha
% 13314084 / 22728804 = 0.5857
Este percentual é relativamente baixo, mas deve-se ao fato de que não obtivemos
sucesso em encontrar a hifenização de muitas palavras nos dicionários
analisados e também por não observarmos um consenso de ao menos 4 dicionários
com uma mesma hifenização.

Analisando os erros resultantes da hifenização utilizando as regras padrão do
TeX, foram estabelecidas algumas regras como forma de abarcar um conjunto maior
de palavras. Tais regras são elencadas a seguir, expondo também as regras de
exceção necessárias e um exemplo onde a regra será utilizada.

\begin{enumerate}
\item\label{rulegrp_psi} regra: \texttt{p2si}, \texttt{p2sí} -- psicologia, psíquico\\
exceção: \texttt{p3sia}, \texttt{p3sin} -- epilepsia, rodopsina

\item\label{rulegrp_tc} regra: \texttt{t2c} -- tchau%, tcheco

\item\label{rulegrp_pneu} regra: \texttt{p2neu} -- pneumonia
    
\item\label{rulegrp_gno} regra: \texttt{.g2no}, \texttt{.g2nó} -- gnomo, gnóstico

\item\label{rulegrp_pt} regra: \texttt{.p2t} -- ptose%, pterossauro

\item\label{rulegrp_cza} regra: \texttt{c2za} -- czar

\item\label{rulegrp_s2} regra: \texttt{.s2} -- stalinismo

\item\label{rulegrp_t2} regra: \texttt{.t2} -- tsunami

\item\label{rulegrp_mn} regra: \texttt{.m2n} -- mnemônico

\item\label{rulegrp_zz} regra: \texttt{z1z}, \texttt{p1p}, \texttt{z2z.} -- pizza, shopping, jazz

\item\label{rulegrp_sub} regra: \texttt{su2b3r}, \texttt{su2b3l} -- sublunar, subrotina
exceção: \texttt{su3b4li1nh}, \texttt{su3b4li1ma}, \texttt{su3b4li1me}, \texttt{su3b4li1mid} -- sublinhar, sublimar, sublime, sublimidade

\item\label{rulegrp_so} regra: \texttt{1sô} -- subsônico

\item\label{rulegrp_pseu} regra: \texttt{1p2seu1d} -- pseudônimo

\item\label{rulegrp_ein} regra: \texttt{e1in1c}, \texttt{e1in1f}, \texttt{e1in1g}, \texttt{e1in1s}, \texttt{e1in1t}, \texttt{e1in1v} -- reincidência, reinfecção, reingressa, reinserção, reintegração, reinventar
exceção: \texttt{e1in2st}, \texttt{e1in2sc} -- reinstalado, reinscrever 

\item\label{rulegrp_oin} regra: \texttt{o1in} -- coincidência, agroindustrial, bioinformática, amendoins, gastrointestinal
exceção: \texttt{o1ina} -- boina

\item\label{rulegrp_eim} regra: \texttt{e1imp} -- reimpresso%, coimbra, coimperador
%exceção: \texttt{que2i}, \texttt{te2im} -- queimar, queimadura, teimosia, tirateima

\item\label{rulegrp_aa} regra: \texttt{a1â}, \texttt{a1ã}, \texttt{a1é}, \texttt{a1í}, \texttt{a1ó}, \texttt{a1ô}, \texttt{a1ú}, \texttt{e1á}, \texttt{e1â}, \texttt{e1ã}, \texttt{e1ê}, \texttt{e1í}, \texttt{e1ó}, \texttt{e1ô}, \texttt{e1ú}, \texttt{i1á}, \texttt{i1ã}, \texttt{i1ã}, \texttt{i1é}, \texttt{i1ó}, \texttt{i1ú}, \texttt{o1á}, \texttt{o1ã}, \texttt{o1é}, \texttt{o1ê}, \texttt{o1í}, \texttt{o1ó}, \texttt{u1á}, \texttt{u1ã}, \texttt{u1é}, \texttt{u1ê}, \texttt{u1í} --  abraâmico, abraão, aéreo, país, caótico, faraônico, saúde, balneário, oceânico, campeã, veêm, veículo, teórico, napoleônico, conteúdo, diário, região, soviético, periódico, viúva, razoável, joão, poético, boêmia, heroísmo, alcoólico, usuário, itapuã, suécia, cauê, suíça \\
exceção: \texttt{1gu2é}, \texttt{1gu2ê}, \texttt{1gu2í}, \texttt{1qu2á}, \texttt{1qu2é}, \texttt{1qu2ê}, \texttt{1qu2í} -- alguém, português, linguística, aquático, inquérito, sequência, química

\item\label{rulegrp_air} regra: \texttt{a1ir.}, \texttt{u1ir.} -- sair, diminuir

\item\label{rulegrp_ia} regra: \texttt{í1a} -- baía

\item\label{rulegrp_ain} regra: \texttt{a1ind}, \texttt{a1i1nh} -- ainda, rainha

\item\label{rulegrp_oin} regra: \texttt{o1i1nh} -- moinho

\item\label{rulegrp_aui} regra: \texttt{au1i1c}, \texttt{du1i1c}, \texttt{u1i1ç}, \texttt{u1i1d}, \texttt{cu2i}, \texttt{dru2i}, \texttt{flu2id}, \texttt{bu1i1n}, \texttt{cu1i1n}, \texttt{fu1i1n}, \texttt{nu1i1n}, \texttt{ru1i1na}, \texttt{ru1i1no}, \texttt{su1i1ti}, \texttt{tu1i1ti}, \texttt{u1iz} -- cacauicultor, sanduicheira, constituição, continuidade, cuidador, druida, fluido, tabuinha, picuinha, fuinha, genuinamente, arruinado, arruinou, jesuitismo, intuitivo, juizado 
exceção: \texttt{cu3i1da1de}, \texttt{bu1i2n1d}, \texttt{bu1i2n1t}  -- acuidade, contribuindo, contribuinte

\item\label{rulegrp_gua} regra: \texttt{1gu4á}, \texttt{1gu4ã}, \texttt{1qu4ã} -- jaraguá, saguão, quão

\item\label{rulegrp_iur} regra: \texttt{i1ur} -- diurno

\item\label{rulegrp_io} regra: \texttt{í1o} -- íon

\item\label{rulegrp_uo} regra: \texttt{ú1o} -- flúor

\item\label{rulegrp_proi} regra: \texttt{pro1i1b} -- proibição

\item\label{rulegrp_iun} regra: \texttt{i1un} -- triunfar

\item\label{rulegrp_quo} regra: \texttt{1qu4ó}, \texttt{1qu4â} -- quórum, equânime

\item\label{rulegrp_eo} regra: \texttt{é3o} -- alvéolo

\item\label{rulegrp_ia} \st{regra: \texttt{i4a.}, \texttt{i4e.}, \texttt{i4o.}, \texttt{o4a.}, \texttt{u4a.} -- economia, espécie, vazio, destoa, institua}

\item\label{rulegrp_lo} regra: \texttt{1lô} -- camelô

\item\label{rulegrp_co} regra: \texttt{1cô} -- recôncavo

\item\label{rulegrp_bo} regra: \texttt{1bô}, \texttt{1dô}, \texttt{1fô}, \texttt{1gô}, \texttt{1pô}, \texttt{1mô}, \texttt{1nô}, \texttt{1rô}, \texttt{1tô}, \texttt{1vô}, \texttt{1xô}, \texttt{1zô} -- robô, judô,  telefônica, xangô, capô, sumô, econômico, tarô, chatô, vovô, saxônia, amazônia

\item\label{rulegrp_cco} regra: \texttt{1çô} -- maçônico

% As sequências gráficas ia, ie, io, oa, ua quando em posição
% final átona, que são normalmente pronunciadas como ditongos
% crescentes, mas que podem corresponder foneticamente a duas
% vogais (hiato), sobretudo quando pronunciadas pausadamente:
% vitó-ria, espé-cie, exercí-cio, nó-doa, lé-gua, tê-nue, ambí-guo.
    \setcounter{numberRulesGroups}{\value{enumi}}
\end{enumerate}

The \NumberOfNewRules{} rules were grouped above in a list of
\arabic{numberRulesGroups} types of rules. They may be further organized into four
large groups. The first, which comprises rules \ref{rulegrp_psi} to
\ref{rulegrp_zz}, includes consonant clusters such as \emph{czar}, \emph{ptose}
and \emph{gnomo}. They, unlike the examples that will be exposed in section
\ref{secxxx} (exceptions - item 9 currently), present a set of derived words,
which makes their marking advantageous in view of the number of cases that are
included in this marking rule. The second group, comprising rules
\ref{rulegrp_sub} to \ref{rulegrp_eim}, delimits the morphological boundary
between prefixes and radicals. As noted, although phonological issues guide the
separation of numerous words in Portuguese, there are also those that are
guided by morphology. This is the case of words that have the prefixes
\emph{sub-} and \emph{re-}, such as \emph{sublunar} and \emph{reinserção}. The
third group, comprising rules \ref{rulegrp_aa} to \ref{rulegrp_ia}, seeks to
understand a set of words that have vowel combinations that do not follow the
general rules. This is because the Portuguese language has vowel encounters
with the second vowel graphically marked that can be separated, forming
hiatuses, such as \emph{caótico}, \emph{balneário} and \emph{razoável}, while
there are also words with a similar structure that constitute a diphthong, such
as \emph{português}, \emph{alguém} and \emph{linguística}. It is remarkable, of
course, that the latter are formed by the digraphs \emph{qu-} and \emph{gu-},
while the former by vowels other than \emph{i} and \emph{u}. The fourth and
last group, in turn, which comprises rules \ref{rulegrp_lo} to
\ref{rulegrp_cco}, which are counterparts of rules that were already in the
default rules, but did not contemplate the cases with certain accents. They
were then added to encompass words such as \emph{camelô}, \emph{recôncavo},
\emph{amazônia}, and \emph{maçônico}.

It is important to highlight that the words included in these rules are, in general, of low frequency and were incorporated into the Portuguese language without a phonotactic adaptation, which causes these idiosyncrasies and exceptions to the language. They are characterized as marked cases, since it is not possible to defend that they reproduce a phonological pattern of the language because they are not productive, that is, they are not taken as an example and/or derive new words.


O conjunto de \NumberOfNewRules{} regras é apresentado a seguir:
\begin{multicols}{6}
    \footnotesize
    \verbatiminput{../data/patch.TeX.pt-br.patterns}
\end{multicols}

A Tabela \ref{tab-resultados} apresenta os erros cometidos no conjunto das
37798 palavras analisadas. As palavras omitidas são aquelas que foram
hifenizadas corretamente pelos dois conjuntos de regra. Utilizamos aqui a
marcação adotada pelo \emph{Patgen}: \texttt{*} para indicar pontos de
hifenização corretamente marcados, \texttt{.} para indicar pontos de hifenização
marcados erroneamente e \texttt{-} para indicar pontos de hifenização não
encontrados. Os pontos marcados erroneamente e aqueles não encontrados são
considerados erros. Quando ocorre algum erro de hifenização, marcaremos a
palavra com \xmark. Quando não houver erro algum, marcaremos a palavra com
\cmark. Ao final, podemos verificar que as regras padrão apresentaram 1368 erros
(3,62\%) e as novas regras 132 erros (0,35\%).

\begin{longtable}{l l l}
\caption{Comparativo dos erros cometidos pelos dois conjuntos de regras de
hifenização.}\label{tab-resultados}\\
    \hline
    palavra & regras padrão & regras adicionais \\
    \hline
    \endhead
    aarônico & a*a-rô*ni*co \xmark & a*a*rô*ni*co \cmark \\
abadia & a*ba*di*a \cmark & a*ba*di-a \xmark \\
abdômen & ab-dô*men \xmark & ab*dô*men \cmark \\
abdominoplastia & ab*do*mi*no*plas*ti*a \cmark & ab*do*mi*no*plas*ti-a \xmark \\
abecedário & a*be*ce*dá*ri.o \xmark & a*be*ce*dá*rio \cmark \\
abiótico & a*bi-ó*ti*co \xmark & a*bi*ó*ti*co \cmark \\
abissínio & a*bis*sí*ni.o \xmark & a*bis*sí*nio \cmark \\
abluir & a*blu-ir \xmark & a*blu*ir \cmark \\
aboio & a*boi*o \cmark & a*boi-o \xmark \\
abraâmico & a*bra-â*mi*co \xmark & a*bra*â*mi*co \cmark \\
abrangência & a*bran*gên*ci.a \xmark & a*bran*gên*cia \cmark \\
abreugrafia & a*breu*gra*fi*a \cmark & a*breu*gra*fi-a \xmark \\
abrótea & a*bró*te.a \xmark & a*bró*tea \cmark \\
abrupto & a.b-rup*to \xmark & a.b-rup*to \xmark \\
absenteísmo & ab*sen*te-ís*mo \xmark & ab*sen*te*ís*mo \cmark \\
abstinência & abs*ti*nên*ci.a \xmark & abs*ti*nên*cia \cmark \\
abstraído & abs*tra-í*do \xmark & abs*tra*í*do \cmark \\
abstrair & abs*tra-ir \xmark & abs*tra*ir \cmark \\
abulia & a*bu*li*a \cmark & a*bu*li-a \xmark \\
abundância & a*bun*dân*ci*a \cmark & a*bun*dân*ci-a \xmark \\
acácia & a*cá*ci.a \xmark & a*cá*cia \cmark \\
academia & a*ca*de*mi*a \cmark & a*ca*de*mi-a \xmark \\
açaí & a*ça-í \xmark & a*ça*í \cmark \\
acatisia & a*ca*ti*si*a \cmark & a*ca*ti*si-a \xmark \\
acefalia & a*ce*fa*li*a \cmark & a*ce*fa*li-a \xmark \\
acéquia & a*cé*qui.a \xmark & a*cé*quia \cmark \\
acessório & a*ces*só*ri.o \xmark & a*ces*só*rio \cmark \\
acetaldeído & a*ce*tal*de-í*do \xmark & a*ce*tal*de*í*do \cmark \\
acidemia & a*ci*de*mi*a \cmark & a*ci*de*mi-a \xmark \\
acônito & a-cô*ni*to \xmark & a*cô*ni*to \cmark \\
acordeão & a*cor*de-ão \xmark & a*cor*de*ão \cmark \\
açoteia & a*ço*tei*a \cmark & a*ço*tei-a \xmark \\
acrobacia & a*cro*ba*ci*a \cmark & a*cro*ba*ci-a \xmark \\
acrofobia & a*cro*fo*bi*a \cmark & a*cro*fo*bi-a \xmark \\
acromegalia & a*cro*me*ga*li*a \cmark & a*cro*me*ga*li-a \xmark \\
actínio & ac*tí*ni.o \xmark & ac*tí*nio \cmark \\
acuidade & a*cu-i*da*de \xmark & a*cu-i*da*de \xmark \\
acurácia & a*cu*rá*ci.a \xmark & a*cu*rá*cia \cmark \\
adágio & a*dá*gi.o \xmark & a*dá*gio \cmark \\
adail & a*da-il \xmark & a*da-il \xmark \\
adimplência & a*dim*plên*ci.a \xmark & a*dim*plên*cia \cmark \\
adjutório & ad*ju*tó*ri.o \xmark & ad*ju*tó*rio \cmark \\
adolescência & a*do*les*cên*ci.a \xmark & a*do*les*cên*cia \cmark \\
adriático & a*dri-á*ti*co \xmark & a*dri*á*ti*co \cmark \\
adultério & a*dul*té*ri.o \xmark & a*dul*té*rio \cmark \\
adventício & ad*ven*tí*ci.o \xmark & ad*ven*tí*cio \cmark \\
advérbio & ad*vér*bi.o \xmark & ad*vér*bio \cmark \\
adversário & ad*ver*sá*ri.o \xmark & ad*ver*sá*rio \cmark \\
advertência & ad*ver*tên*ci.a \xmark & ad*ver*tên*cia \cmark \\
advocacia & ad*vo*ca*ci*a \cmark & ad*vo*ca*ci-a \xmark \\
aéreo & a-é*re.o \xmark & a*é*reo \cmark \\
aeróbio & a*e*ró*bi.o \xmark & a*e*ró*bio \cmark \\
aerofobia & a*e*ro*fo*bi*a \cmark & a*e*ro*fo*bi-a \xmark \\
aerofotografia & a*e*ro*fo*to*gra*fi*a \cmark & a*e*ro*fo*to*gra*fi-a \xmark \\
aerografia & a*e*ro*gra*fi*a \cmark & a*e*ro*gra*fi-a \xmark \\
aeronomia & a*e*ro*no*mi*a \cmark & a*e*ro*no*mi-a \xmark \\
aeroportuário & a*e*ro*por*tu-á*ri.o \xmark & a*e*ro*por*tu*á*rio \cmark \\
aerovia & a*e*ro*vi*a \cmark & a*e*ro*vi-a \xmark \\
aeroviário & a*e*ro*vi-á*ri.o \xmark & a*e*ro*vi*á*rio \cmark \\
aético & a-é*ti*co \xmark & a*é*ti*co \cmark \\
afasia & a*fa*si*a \cmark & a*fa*si-a \xmark \\
afélio & a*fé*li.o \xmark & a*fé*lio \cmark \\
afluência & a*flu-ên*ci.a \xmark & a*flu*ên*cia \cmark \\
afluir & a*flu-ir \xmark & a*flu*ir \cmark \\
aforia & a*fo*ri*a \cmark & a*fo*ri-a \xmark \\
afrodisíaco & a*fro*di*sí-a*co \xmark & a*fro*di*sí*a*co \cmark \\
agamia & a*ga*mi*a \cmark & a*ga*mi-a \xmark \\
agência & a*gên*ci.a \xmark & a*gên*cia \cmark \\
agenesia & a*ge*ne*si*a \cmark & a*ge*ne*si-a \xmark \\
aglossia & a*glos*si*a \cmark & a*glos*si-a \xmark \\
agnosia & ag*no*si*a \cmark & ag*no*si-a \xmark \\
agogô & a*go-gô \xmark & a*go*gô \cmark \\
agonia & a*go*ni*a \cmark & a*go*ni-a \xmark \\
agorafobia & a*go*ra*fo*bi*a \cmark & a*go*ra*fo*bi-a \xmark \\
agrário & a*grá*ri.o \xmark & a*grá*rio \cmark \\
agrião & a*gri-ão \xmark & a*gri*ão \cmark \\
agronomia & a*gro*no*mi*a \cmark & a*gro*no*mi-a \xmark \\
agronômico & a*gro-nô*mi*co \xmark & a*gro*nô*mi*co \cmark \\
águia & á*gui.a \xmark & á*guia \cmark \\
aia & ai*a \cmark & ai-a \xmark \\
aí & a-í \xmark & a*í \cmark \\
ainda & a-in*da \xmark & a*in*da \cmark \\
aio & ai*o \cmark & ai-o \xmark \\
ajuizar & a*ju-i*zar \xmark & a*ju*i*zar \cmark \\
alaúde & a*la-ú*de \xmark & a*la*ú*de \cmark \\
albergaria & al*ber*ga*ri*a \cmark & al*ber*ga*ri-a \xmark \\
albugínea & al*bu*gí*ne*a \cmark & al*bu*gí*ne-a \xmark \\
alcaidaria & al*cai*da*ri*a \cmark & al*cai*da*ri-a \xmark \\
alcaravia & al*ca*ra*vi*a \cmark & al*ca*ra*vi-a \xmark \\
alcaria & al*ca*ri*a \cmark & al*ca*ri-a \xmark \\
alcateia & al*ca*tei*a \cmark & al*ca*tei-a \xmark \\
alcíone & al*cí-o*ne \xmark & al*cí*o*ne \cmark \\
alcoólatra & al*co-ó*la*tra \xmark & al*co*ó*la*tra \cmark \\
alcoolemia & al*co*o*le*mi*a \cmark & al*co*o*le*mi-a \xmark \\
alcoólico & al*co-ó*li*co \xmark & al*co*ó*li*co \cmark \\
aldeão & al*de-ão \xmark & al*de*ão \cmark \\
aldeia & al*dei*a \cmark & al*dei-a \xmark \\
aldeído & al*de-í*do \xmark & al*de*í*do \cmark \\
aleatório & a*le*a*tó*ri.o \xmark & a*le*a*tó*rio \cmark \\
alegoria & a*le*go*ri*a \cmark & a*le*go*ri-a \xmark \\
alegria & a*le*gri*a \cmark & a*le*gri-a \xmark \\
aleia & a*lei*a \cmark & a*lei-a \xmark \\
aleivosia & a*lei*vo*si*a \cmark & a*lei*vo*si-a \xmark \\
alelopatia & a*le*lo*pa*ti*a \cmark & a*le*lo*pa*ti-a \xmark \\
aleluia & a*le*lui*a \cmark & a*le*lui-a \xmark \\
alergia & a*ler*gi*a \cmark & a*ler*gi-a \xmark \\
alergologia & a*ler*go*lo*gi*a \cmark & a*ler*go*lo*gi-a \xmark \\
aletria & a*le*tri*a \cmark & a*le*tri-a \xmark \\
aleúte & a*le-ú*te \xmark & a*le*ú*te \cmark \\
alexia & a*le*xi*a \cmark & a*le*xi-a \xmark \\
alfaia & al*fai*a \cmark & al*fai-a \xmark \\
alfaiataria & al*fai*a*ta*ri*a \cmark & al*fai*a*ta*ri-a \xmark \\
alfarrábio & al*far*rá*bi.o \xmark & al*far*rá*bio \cmark \\
alforria & al*for*ri*a \cmark & al*for*ri-a \xmark \\
algaravia & al*ga*ra*vi*a \cmark & al*ga*ra*vi-a \xmark \\
algarvio & al*gar*vi*o \cmark & al*gar*vi-o \xmark \\
algofobia & al*go*fo*bi*a \cmark & al*go*fo*bi-a \xmark \\
alheio & a*lhei*o \cmark & a*lhei-o \xmark \\
aliá & a*li-á \xmark & a*li*á \cmark \\
aliáceo & a*li-á*ce.o \xmark & a*li*á*ceo \cmark \\
aliás & a*li-ás \xmark & a*li*ás \cmark \\
alimária & a*li*má*ri.a \xmark & a*li*má*ria \cmark \\
alimentício & a*li*men*tí*ci.o \xmark & a*li*men*tí*cio \cmark \\
alínea & a*lí*ne.a \xmark & a*lí*nea \cmark \\
alívio & a*lí*vi.o \xmark & a*lí*vio \cmark \\
almôndega & al-môn*de*ga \xmark & al*môn*de*ga \cmark \\
alô & a-lô \xmark & a*lô \cmark \\
alódio & a*ló*di.o \xmark & a*ló*dio \cmark \\
aloé & a*lo-é \xmark & a*lo*é \cmark \\
aloés & a*lo-és \xmark & a*lo*és \cmark \\
alometria & a*lo*me*tri*a \cmark & a*lo*me*tri-a \xmark \\
alopatia & a*lo*pa*ti*a \cmark & a*lo*pa*ti-a \xmark \\
alopecia & a*lo*pe*ci*a \cmark & a*lo*pe*ci-a \xmark \\
alquimia & al*qui*mi*a \cmark & al*qui*mi-a \xmark \\
alquímico & al-quí*mi*co \xmark & al*quí*mi*co \cmark \\
alteia & al*tei*a \cmark & al*tei-a \xmark \\
alternância & al*ter*nân*ci.a \xmark & al*ter*nân*cia \cmark \\
altimetria & al*ti*me*tri*a \cmark & al*ti*me*tri-a \xmark \\
altruísmo & al*tru-ís*mo \xmark & al*tru*ís*mo \cmark \\
altruísta & al*tru-ís*ta \xmark & al*tru*ís*ta \cmark \\
alucinatório & a*lu*ci*na*tó*ri.o \xmark & a*lu*ci*na*tó*rio \cmark \\
aluimento & a*lu-i*men*to \xmark & a*lu-i*men*to \xmark \\
aluir & a*lu-ir \xmark & a*lu*ir \cmark \\
alumínio & a*lu*mí*ni.o \xmark & a*lu*mí*nio \cmark \\
aluvião & a*lu*vi-ão \xmark & a*lu*vi*ão \cmark \\
alvedrio & al*ve*dri*o \cmark & al*ve*dri-o \xmark \\
alvenaria & al*ve*na*ri*a \cmark & al*ve*na*ri-a \xmark \\
alvéolo & al*vé-o*lo \xmark & al*vé*o*lo \cmark \\
amásia & a*má*si.a \xmark & a*má*sia \cmark \\
amásio & a*má*si.o \xmark & a*má*sio \cmark \\
amazônico & a*ma-zô*ni*co \xmark & a*ma*zô*ni*co \cmark \\
ambidestria & am*bi*des*tri*a \cmark & am*bi*des*tri-a \xmark \\
ambrosia & am*bro*si*a \cmark & am*bro*si-a \xmark \\
ambulância & am*bu*lân*ci.a \xmark & am*bu*lân*cia \cmark \\
ambulatório & am*bu*la*tó*ri.o \xmark & am*bu*la*tó*rio \cmark \\
amebíase & a*me*bí-a*se \xmark & a*me*bí*a*se \cmark \\
ameia & a*mei*a \cmark & a*mei-a \xmark \\
amélia & a*mé*li.a \xmark & a*mé*lia \cmark \\
amêndoa & a*mên*do.a \xmark & a*mên*doa \cmark \\
amendoim & a*men*do-im \xmark & a*men*do*im \cmark \\
amenorreia & a*me*nor*rei*a \cmark & a*me*nor*rei-a \xmark \\
amerício & a*me*rí*ci.o \xmark & a*me*rí*cio \cmark \\
ameríndio & a*me*rín*di.o \xmark & a*me*rín*dio \cmark \\
aminoácido & a*mi*no-á*ci*do \xmark & a*mi*no*á*ci*do \cmark \\
amiúde & a*mi-ú*de \xmark & a*mi*ú*de \cmark \\
amoníaco & a*mo*ní-a*co \xmark & a*mo*ní*a*co \cmark \\
amontoa & a*mon*to*a \cmark & a*mon*to-a \xmark \\
ampelografia & am*pe*lo*gra*fi*a \cmark & am*pe*lo*gra*fi-a \xmark \\
anaeróbio & a*na*e*ró*bi.o \xmark & a*na*e*ró*bio \cmark \\
anafilaxia & a*na*fi*la*xi*a \cmark & a*na*fi*la*xi-a \xmark \\
anafrodisíaco & a*na*fro*di*sí-a*co \xmark & a*na*fro*di*sí*a*co \cmark \\
analgesia & a*nal*ge*si*a \cmark & a*nal*ge*si-a \xmark \\
analogia & a*na*lo*gi*a \cmark & a*na*lo*gi-a \xmark \\
anarquia & a*nar*qui*a \cmark & a*nar*qui-a \xmark \\
anatomia & a*na*to*mi*a \cmark & a*na*to*mi-a \xmark \\
anatômico & a*na-tô*mi*co \xmark & a*na*tô*mi*co \cmark \\
ancião & an*ci-ão \xmark & an*ci*ão \cmark \\
ancilostomíase & an*ci*los*to*mí-a*se \xmark & an*ci*los*to*mí*a*se \cmark \\
androfobia & an*dro*fo*bi*a \cmark & an*dro*fo*bi-a \xmark \\
androginia & an*dro*gi*ni*a \cmark & an*dro*gi*ni-a \xmark \\
andrologia & an*dro*lo*gi*a \cmark & an*dro*lo*gi-a \xmark \\
anedonia & a*ne*do*ni*a \cmark & a*ne*do*ni-a \xmark \\
anelídeo & a*ne*lí*de.o \xmark & a*ne*lí*deo \cmark \\
anemia & a*ne*mi*a \cmark & a*ne*mi-a \xmark \\
anemofilia & a*ne*mo*fi*li*a \cmark & a*ne*mo*fi*li-a \xmark \\
anemômetro & a*ne-mô*me*tro \xmark & a*ne*mô*me*tro \cmark \\
anestesia & a*nes*te*si*a \cmark & a*nes*te*si-a \xmark \\
anestesiologia & a*nes*te*si*o*lo*gi*a \cmark & a*nes*te*si*o*lo*gi-a \xmark \\
aneuploidia & a*neu*ploi*di*a \cmark & a*neu*ploi*di-a \xmark \\
anfíbio & an*fí*bi.o \xmark & an*fí*bio \cmark \\
anfibologia & an*fi*bo*lo*gi*a \cmark & an*fi*bo*lo*gi-a \xmark \\
anfitrião & an*fi*tri-ão \xmark & an*fi*tri*ão \cmark \\
angiologia & an*gi*o*lo*gi*a \cmark & an*gi*o*lo*gi-a \xmark \\
angioplastia & an*gi*o*plas*ti*a \cmark & an*gi*o*plas*ti-a \xmark \\
angústia & an*gús*ti.a \xmark & an*gús*tia \cmark \\
anisogamia & a*ni*so*ga*mi*a \cmark & a*ni*so*ga*mi-a \xmark \\
anisotropia & a*ni*so*tro*pi*a \cmark & a*ni*so*tro*pi-a \xmark \\
anistia & a*nis*ti*a \cmark & a*nis*ti-a \xmark \\
aniversário & a*ni*ver*sá*ri.o \xmark & a*ni*ver*sá*rio \cmark \\
anodinia & a*no*di*ni*a \cmark & a*no*di*ni-a \xmark \\
anomalia & a*no*ma*li*a \cmark & a*no*ma*li-a \xmark \\
anomia & a*no*mi*a \cmark & a*no*mi-a \xmark \\
anônimo & a-nô*ni*mo \xmark & a*nô*ni*mo \cmark \\
anorexia & a*no*re*xi*a \cmark & a*no*re*xi-a \xmark \\
anosmia & a*nos*mi*a \cmark & a*nos*mi-a \xmark \\
anoxia & a*no*xi*a \cmark & a*no*xi-a \xmark \\
anseio & an*sei*o \cmark & an*sei-o \xmark \\
antagônico & an*ta-gô*ni*co \xmark & an*ta*gô*ni*co \cmark \\
antecedência & an*te*ce*dên*ci.a \xmark & an*te*ce*dên*cia \cmark \\
antiácido & an*ti-á*ci*do \xmark & an*ti*á*ci*do \cmark \\
antialcoólico & an*ti*al*co-ó*li*co \xmark & an*ti*al*co*ó*li*co \cmark \\
antibiótico & an*ti*bi-ó*ti*co \xmark & an*ti*bi*ó*ti*co \cmark \\
antinomia & an*ti*no*mi*a \cmark & an*ti*no*mi-a \xmark \\
antipatia & an*ti*pa*ti*a \cmark & an*ti*pa*ti-a \xmark \\
antipatriótico & an*ti*pa*tri-ó*ti*co \xmark & an*ti*pa*tri*ó*ti*co \cmark \\
antiquário & an*ti-quá*ri.o \xmark & an*ti*quá*rio \cmark \\
antiquíssimo & an*ti-quís*si*mo \xmark & an*ti*quís*si*mo \cmark \\
antissepsia & an*tis*sep*si*a \cmark & an*tis*sep*si-a \xmark \\
antologia & an*to*lo*gi*a \cmark & an*to*lo*gi-a \xmark \\
antônimo & an-tô*ni*mo \xmark & an*tô*ni*mo \cmark \\
antonomásia & an*to*no*má*si.a \xmark & an*to*no*má*sia \cmark \\
antropofagia & an*tro*po*fa*gi*a \cmark & an*tro*po*fa*gi-a \xmark \\
antropologia & an*tro*po*lo*gi*a \cmark & an*tro*po*lo*gi-a \xmark \\
antropometria & an*tro*po*me*tri*a \cmark & an*tro*po*me*tri-a \xmark \\
antroponímia & an*tro*po*ní*mi.a \xmark & an*tro*po*ní*mia \cmark \\
antropônimo & an*tro-pô*ni*mo \xmark & an*tro*pô*ni*mo \cmark \\
antroposofia & an*tro*po*so*fi*a \cmark & an*tro*po*so*fi-a \xmark \\
anuário & a*nu-á*ri.o \xmark & a*nu*á*rio \cmark \\
anuência & a*nu-ên*ci.a \xmark & a*nu*ên*cia \cmark \\
anuidade & a*nu-i*da*de \xmark & a*nu*i*da*de \cmark \\
anuir & a*nu-ir \xmark & a*nu*ir \cmark \\
anúncio & a*nún*ci.o \xmark & a*nún*cio \cmark \\
aórtico & a-ór*ti*co \xmark & a*ór*ti*co \cmark \\
apanágio & a*pa*ná*gi.o \xmark & a*pa*ná*gio \cmark \\
aparência & a*pa*rên*ci.a \xmark & a*pa*rên*cia \cmark \\
apatia & a*pa*ti*a \cmark & a*pa*ti-a \xmark \\
apendicectomia & a*pen*di*cec*to*mi*a \cmark & a*pen*di*cec*to*mi-a \xmark \\
apetência & a*pe*tên*ci.a \xmark & a*pe*tên*cia \cmark \\
aplasia & a*pla*si*a \cmark & a*pla*si-a \xmark \\
apneia & ap*nei*a \cmark & ap*nei-a \xmark \\
apofonia & a*po*fo*ni*a \cmark & a*po*fo*ni-a \xmark \\
apoio & a*poi*o \cmark & a*poi-o \xmark \\
apologia & a*po*lo*gi*a \cmark & a*po*lo*gi-a \xmark \\
apoplexia & a*po*ple*xi*a \cmark & a*po*ple*xi-a \xmark \\
aporia & a*po*ri*a \cmark & a*po*ri-a \xmark \\
aposentadoria & a*po*sen*ta*do*ri*a \cmark & a*po*sen*ta*do*ri-a \xmark \\
apostasia & a*pos*ta*si*a \cmark & a*pos*ta*si-a \xmark \\
apraxia & a*pra*xi*a \cmark & a*pra*xi-a \xmark \\
apreciável & a*pre*ci-á*vel \xmark & a*pre*ci*á*vel \cmark \\
apuí & a*pu-í \xmark & a*pu*í \cmark \\
aquário & a-quá*ri.o \xmark & a*quá*rio \cmark \\
aquático & a-quá*ti*co \xmark & a*quá*ti*co \cmark \\
aquém & a-quém \xmark & a*quém \cmark \\
aquícola & a-quí*co*la \xmark & a*quí*co*la \cmark \\
aquiescência & a*qui*es*cên*ci.a \xmark & a*qui*es*cên*cia \cmark \\
aquífero & a-quí*fe*ro \xmark & a*quí*fe*ro \cmark \\
aquileia & a*qui*lei*a \cmark & a*qui*lei-a \xmark \\
arábias & a*rá*bi.as \xmark & a*rá*bi.as \xmark \\
aracnofobia & a*rac*no*fo*bi*a \cmark & a*rac*no*fo*bi-a \xmark \\
aracnologia & a*rac*no*lo*gi*a \cmark & a*rac*no*lo*gi-a \xmark \\
araucária & a*rau*cá*ri.a \xmark & a*rau*cá*ria \cmark \\
arbitrário & ar*bi*trá*ri.o \xmark & ar*bi*trá*rio \cmark \\
arbítrio & ar*bí*tri.o \xmark & ar*bí*trio \cmark \\
arbóreo & ar*bó*re.o \xmark & ar*bó*reo \cmark \\
arcaísmo & ar*ca-ís*mo \xmark & ar*ca*ís*mo \cmark \\
ardência & ar*dên*ci.a \xmark & ar*dên*cia \cmark \\
ardósia & ar*dó*si.a \xmark & ar*dó*sia \cmark \\
árduo & ár*du.o \xmark & ár*duo \cmark \\
área & á*re.a \xmark & á*rea \cmark \\
areia & a*rei*a \cmark & a*rei-a \xmark \\
argênteo & ar*gên*te*o \cmark & ar*gên*te-o \xmark \\
argúcia & ar*gú*ci.a \xmark & ar*gú*cia \cmark \\
aríete & a*rí-e*te \xmark & a*rí*e*te \cmark \\
aristocracia & a*ris*to*cra*ci*a \cmark & a*ris*to*cra*ci-a \xmark \\
armário & ar*má*ri.o \xmark & ar*má*rio \cmark \\
armistício & ar*mis*tí*ci.o \xmark & ar*mis*tí*cio \cmark \\
aromaterapia & a*ro*ma*te*ra*pi*a \cmark & a*ro*ma*te*ra*pi-a \xmark \\
arqueologia & ar*que*o*lo*gi*a \cmark & ar*que*o*lo*gi-a \xmark \\
arqueólogo & ar*que-ó*lo*go \xmark & ar*que*ó*lo*go \cmark \\
arquétipo & ar-qué*ti*po \xmark & ar*qué*ti*po \cmark \\
arquidiácono & ar*qui*di-á*co*no \xmark & ar*qui*di*á*co*no \cmark \\
arquitetônica & ar*qui*te-tô*ni*ca \xmark & ar*qui*te*tô*ni*ca \cmark \\
arquitetônico & ar*qui*te-tô*ni*co \xmark & ar*qui*te*tô*ni*co \cmark \\
arrais & ar*ra-is \xmark & ar*ra-is \xmark \\
arreio & ar*rei*o \cmark & ar*rei-o \xmark \\
arrelia & ar*re*li*a \cmark & ar*re*li-a \xmark \\
arrepio & ar*re*pi*o \cmark & ar*re*pi-o \xmark \\
arritmia & ar*rit*mi*a \cmark & ar*rit*mi-a \xmark \\
arroio & ar*roi*o \cmark & ar*roi-o \xmark \\
arruinar & ar*ru-i*nar \xmark & ar*ru*i*nar \cmark \\
artemísia & ar*te*mí*si.a \xmark & ar*te*mí*sia \cmark \\
articulatório & ar*ti*cu*la*tó*ri*o \cmark & ar*ti*cu*la*tó*ri-o \xmark \\
artifício & ar*ti*fí*ci.o \xmark & ar*ti*fí*cio \cmark \\
artilharia & ar*ti*lha*ri*a \cmark & ar*ti*lha*ri-a \xmark \\
artralgia & ar*tral*gi*a \cmark & ar*tral*gi-a \xmark \\
artroscopia & ar*tros*co*pi*a \cmark & ar*tros*co*pi-a \xmark \\
ascaridíase & as*ca*ri*dí-a*se \xmark & as*ca*ri*dí*a*se \cmark \\
ascendência & as*cen*dên*ci.a \xmark & as*cen*dên*cia \cmark \\
asfixia & as*fi*xi*a \cmark & as*fi*xi-a \xmark \\
asiática & a*si-á*ti*ca \xmark & a*si*á*ti*ca \cmark \\
asiático & a*si-á*ti*co \xmark & a*si*á*ti*co \cmark \\
assassínio & as*sas*sí*ni.o \xmark & as*sas*sí*nio \cmark \\
assédio & as*sé*di.o \xmark & as*sé*dio \cmark \\
assepsia & as*sep*si*a \cmark & as*sep*si-a \xmark \\
assessoria & as*ses*so*ri*a \cmark & as*ses*so*ri-a \xmark \\
assiduidade & as*si*du-i*da*de \xmark & as*si*du*i*da*de \cmark \\
assíduo & as*sí*du.o \xmark & as*sí*duo \cmark \\
assimetria & as*si*me*tri*a \cmark & as*si*me*tri-a \xmark \\
assistência & as*sis*tên*ci.a \xmark & as*sis*tên*cia \cmark \\
associável & as*so*ci-á*vel \xmark & as*so*ci*á*vel \cmark \\
assonância & as*so*nân*ci.a \xmark & as*so*nân*cia \cmark \\
astenia & as*te*ni*a \cmark & as*te*ni-a \xmark \\
astrolábio & as*tro*lá*bi.o \xmark & as*tro*lá*bio \cmark \\
astrologia & as*tro*lo*gi*a \cmark & as*tro*lo*gi-a \xmark \\
astronomia & as*tro*no*mi*a \cmark & as*tro*no*mi-a \xmark \\
astronômico & as*tro-nô*mi*co \xmark & as*tro*nô*mi*co \cmark \\
astúcia & as*tú*ci.a \xmark & as*tú*cia \cmark \\
atalaia & a*ta*lai*a \cmark & a*ta*lai-a \xmark \\
ataraxia & a*ta*ra*xi*a \cmark & a*ta*ra*xi-a \xmark \\
ataúde & a*ta-ú*de \xmark & a*ta*ú*de \cmark \\
ateísmo & a*te-ís*mo \xmark & a*te*ís*mo \cmark \\
ateísta & a*te-ís*ta \xmark & a*te*ís*ta \cmark \\
atelectasia & a*te*lec*ta*si*a \cmark & a*te*lec*ta*si-a \xmark \\
atentatório & a*ten*ta*tó*ri.o \xmark & a*ten*ta*tó*rio \cmark \\
atômico & a-tô*mi*co \xmark & a*tô*mi*co \cmark \\
atonia & a*to*ni*a \cmark & a*to*ni-a \xmark \\
atrabiliário & a*tra*bi*li-á*ri.o \xmark & a*tra*bi*li*á*rio \cmark \\
atraído & a*tra-í*do \xmark & a*tra*í*do \cmark \\
atrair & a*tra-ir \xmark & a*tra*ir \cmark \\
atresia & a*tre*si*a \cmark & a*tre*si-a \xmark \\
atribuição & a*tri*bu-i*ção \xmark & a*tri*bu*i*ção \cmark \\
atribuir & a*tri*bu-ir \xmark & a*tri*bu*ir \cmark \\
atribuível & a*tri*bu-í*vel \xmark & a*tri*bu*í*vel \cmark \\
átrio & á*tri.o \xmark & á*trio \cmark \\
atrofia & a*tro*fi*a \cmark & a*tro*fi-a \xmark \\
audiência & au*di*ên*ci*a \cmark & au*di*ên*ci-a \xmark \\
áudio & áu*di.o \xmark & áu*dio \cmark \\
audiologia & au*di*o*lo*gi*a \cmark & au*di*o*lo*gi-a \xmark \\
auditoria & au*di*to*ri*a \cmark & au*di*to*ri-a \xmark \\
auditório & au*di*tó*ri.o \xmark & au*di*tó*rio \cmark \\
augúrio & au*gú*ri.o \xmark & au*gú*rio \cmark \\
áureo & áu*re.o \xmark & áu*reo \cmark \\
auréola & au*ré-o*la \xmark & au*ré*o*la \cmark \\
ausência & au*sên*ci.a \xmark & au*sên*cia \cmark \\
auspício & aus*pí*ci.o \xmark & aus*pí*cio \cmark \\
austríaco & aus*trí-a*co \xmark & aus*trí*a*co \cmark \\
autarquia & au*tar*qui*a \cmark & au*tar*qui-a \xmark \\
autobiografia & au*to*bi*o*gra*fi*a \cmark & au*to*bi*o*gra*fi-a \xmark \\
autocracia & au*to*cra*ci*a \cmark & au*to*cra*ci-a \xmark \\
autofagia & au*to*fa*gi*a \cmark & au*to*fa*gi-a \xmark \\
autografia & au*to*gra*fi*a \cmark & au*to*gra*fi-a \xmark \\
autônimo & au-tô*ni*mo \xmark & au*tô*ni*mo \cmark \\
autonomia & au*to*no*mi*a \cmark & au*to*no*mi-a \xmark \\
autônomo & au-tô*no*mo \xmark & au*tô*no*mo \cmark \\
autópsia & au*tóp*si.a \xmark & au*tóp*sia \cmark \\
autoria & au*to*ri*a \cmark & au*to*ri-a \xmark \\
autoritário & au*to*ri*tá*ri.o \xmark & au*to*ri*tá*rio \cmark \\
autotomia & au*to*to*mi*a \cmark & au*to*to*mi-a \xmark \\
auxílio & au*xí*li.o \xmark & au*xí*lio \cmark \\
avaria & a*va*ri*a \cmark & a*va*ri-a \xmark \\
aveia & a*vei*a \cmark & a*vei-a \xmark \\
avião & a*vi-ão \xmark & a*vi*ão \cmark \\
aviário & a*vi-á*ri.o \xmark & a*vi*á*rio \cmark \\
avô & a-vô \xmark & a*vô \cmark \\
axiologia & a*xi*o*lo*gi*a \cmark & a*xi*o*lo*gi-a \xmark \\
azálea & a*zá*le*a \cmark & a*zá*le-a \xmark \\
azia & a*zi*a \cmark & a*zi-a \xmark \\
azulejaria & a*zu*le*ja*ri*a \cmark & a*zu*le*ja*ri-a \xmark \\
babalaô & ba*ba*la-ô \xmark & ba*ba*la*ô \cmark \\
babuíno & ba*bu-í*no \xmark & ba*bu*í*no \cmark \\
bacia & ba*ci*a \cmark & ba*ci-a \xmark \\
bacio & ba*ci*o \cmark & ba*ci-o \xmark \\
bactéria & bac*té*ri.a \xmark & bac*té*ria \cmark \\
bacteriófago & bac*te*ri-ó*fa*go \xmark & bac*te*ri*ó*fa*go \cmark \\
bacteriologia & bac*te*ri*o*lo*gi*a \cmark & bac*te*ri*o*lo*gi-a \xmark \\
bafio & ba*fi*o \cmark & ba*fi-o \xmark \\
bafômetro & ba-fô*me*tro \xmark & ba*fô*me*tro \cmark \\
bahamense & ba-ha*men*se \xmark & ba-ha*men*se \xmark \\
baia & bai*a \cmark & bai-a \xmark \\
baía & ba-í-a \xmark & ba*í-a \xmark \\
baião & bai-ão \xmark & bai*ão \cmark \\
bainha & ba-i*nha \xmark & ba*i*nha \cmark \\
baio & bai*o \cmark & bai-o \xmark \\
baiuca & bai-u*ca \xmark & bai*u*ca \cmark \\
baixaria & bai*xa*ri*a \cmark & bai*xa*ri-a \xmark \\
baixio & bai*xi*o \cmark & bai*xi-o \xmark \\
balaio & ba*lai*o \cmark & ba*lai-o \xmark \\
balaustrada & ba*la-us*tra*da \xmark & ba*la-us*tra*da \xmark \\
balaústre & ba*la-ús*tre \xmark & ba*la*ús*tre \cmark \\
balboa & bal*bo*a \cmark & bal*bo-a \xmark \\
balbucio & bal*bu*ci*o \cmark & bal*bu*ci-o \xmark \\
balbúrdia & bal*búr*di.a \xmark & bal*búr*dia \cmark \\
baldio & bal*di*o \cmark & bal*di-o \xmark \\
baleia & ba*lei*a \cmark & ba*lei-a \xmark \\
balneário & bal*ne-á*ri*o \xmark & bal*ne*á*ri-o \xmark \\
bálteo & bál*te*o \cmark & bál*te-o \xmark \\
bamboleio & bam*bo*lei*o \cmark & bam*bo*lei-o \xmark \\
bancário & ban*cá*ri.o \xmark & ban*cá*rio \cmark \\
bangalô & ban*ga-lô \xmark & ban*ga*lô \cmark \\
barataria & ba*ra*ta*ri*a \cmark & ba*ra*ta*ri-a \xmark \\
barateio & ba*ra*tei*o \cmark & ba*ra*tei-o \xmark \\
barbárie & bar*bá*ri.e \xmark & bar*bá*rie \cmark \\
barbearia & bar*be*a*ri*a \cmark & bar*be*a*ri-a \xmark \\
bário & bá*ri.o \xmark & bá*rio \cmark \\
bastião & bas*ti-ão \xmark & bas*ti*ão \cmark \\
bateia & ba*tei*a \cmark & ba*tei-a \xmark \\
bateria & ba*te*ri*a \cmark & ba*te*ri-a \xmark \\
batimetria & ba*ti*me*tri*a \cmark & ba*ti*me*tri-a \xmark \\
batráquio & ba*trá*qui.o \xmark & ba*trá*quio \cmark \\
batuíra & ba*tu-í*ra \xmark & ba*tu*í*ra \cmark \\
baú & ba-ú \xmark & ba*ú \cmark \\
bazófia & ba*zó*fi.a \xmark & ba*zó*fia \cmark \\
belisário & be*li*sá*ri.o \xmark & be*li*sá*rio \cmark \\
beneficência & be*ne*fi*cên*ci.a \xmark & be*ne*fi*cên*cia \cmark \\
beneficiário & be*ne*fi*ci-á*ri*o \xmark & be*ne*fi*ci*á*ri-o \xmark \\
benefício & be*ne*fí*ci.o \xmark & be*ne*fí*cio \cmark \\
benevolência & be*ne*vo*lên*ci.a \xmark & be*ne*vo*lên*cia \cmark \\
benjoim & ben*jo-im \xmark & ben*jo*im \cmark \\
beócio & be-ó*ci.o \xmark & be*ó*cio \cmark \\
berçário & ber*çá*ri.o \xmark & ber*çá*rio \cmark \\
berílio & be*rí*li.o \xmark & be*rí*lio \cmark \\
béstia & bés*ti*a \cmark & bés*ti-a \xmark \\
bia & bi*a \cmark & bi-a \xmark \\
bibelô & bi*be-lô \xmark & bi*be*lô \cmark \\
bibliofilia & bi*bli*o*fi*li*a \cmark & bi*bli*o*fi*li-a \xmark \\
bibliófilo & bi*bli-ó*fi*lo \xmark & bi*bli*ó*fi*lo \cmark \\
bibliografia & bi*bli*o*gra*fi*a \cmark & bi*bli*o*gra*fi-a \xmark \\
bibliologia & bi*bli*o*lo*gi*a \cmark & bi*bli*o*lo*gi-a \xmark \\
biblioteconomia & bi*bli*o*te*co*no*mi*a \cmark & bi*bli*o*te*co*no*mi-a \xmark \\
bicampeão & bi*cam*pe-ão \xmark & bi*cam*pe*ão \cmark \\
bicentenário & bi*cen*te*ná*ri.o \xmark & bi*cen*te*ná*rio \cmark \\
bicicletário & bi*ci*cle*tá*ri.o \xmark & bi*ci*cle*tá*rio \cmark \\
bicromia & bi*cro*mi*a \cmark & bi*cro*mi-a \xmark \\
bigamia & bi*ga*mi*a \cmark & bi*ga*mi-a \xmark \\
bijuteria & bi*ju*te*ri*a \cmark & bi*ju*te*ri-a \xmark \\
bilheteria & bi*lhe*te*ri*a \cmark & bi*lhe*te*ri-a \xmark \\
binário & bi*ná*ri.o \xmark & bi*ná*rio \cmark \\
bioenergia & bi*o*e*ner*gi*a \cmark & bi*o*e*ner*gi-a \xmark \\
bioética & bi*o-é*ti*ca \xmark & bi*o*é*ti*ca \cmark \\
biófilo & bi-ó*fi*lo \xmark & bi*ó*fi*lo \cmark \\
biogeografia & bi*o*ge*o*gra*fi*a \cmark & bi*o*ge*o*gra*fi-a \xmark \\
biografia & bi*o*gra*fi*a \cmark & bi*o*gra*fi-a \xmark \\
biógrafo & bi-ó*gra*fo \xmark & bi*ó*gra*fo \cmark \\
biologia & bi*o*lo*gi*a \cmark & bi*o*lo*gi-a \xmark \\
biólogo & bi-ó*lo*go \xmark & bi*ó*lo*go \cmark \\
bioquímica & bi*o-quí*mi*ca \xmark & bi*o*quí*mi*ca \cmark \\
biotecnologia & bi*o*tec*no*lo*gi*a \cmark & bi*o*tec*no*lo*gi-a \xmark \\
biotério & bi*o*té*ri.o \xmark & bi*o*té*rio \cmark \\
biótico & bi-ó*ti*co \xmark & bi*ó*ti*co \cmark \\
biótipo & bi-ó*ti*po \xmark & bi*ó*ti*po \cmark \\
biquíni & bi-quí*ni \xmark & bi*quí*ni \cmark \\
birô & bi-rô \xmark & bi*rô \cmark \\
bisavô & bi*sa-vô \xmark & bi*sa*vô \cmark \\
biscaia & bis*cai*a \cmark & bis*cai-a \xmark \\
biscainho & bis*ca-i*nho \xmark & bis*ca*i*nho \cmark \\
bizarria & bi*zar*ri*a \cmark & bi*zar*ri-a \xmark \\
bloqueio & blo*quei*o \cmark & blo*quei-o \xmark \\
boa & bo*a \cmark & bo-a \xmark \\
boá & bo-á \xmark & bo*á \cmark \\
bocaiuva & bo*cai-u*va \xmark & bo*cai*u*va \cmark \\
boião & boi-ão \xmark & boi*ão \cmark \\
boiuna & boi-u*na \xmark & boi*u*na \cmark \\
boleia & bo*lei*a \cmark & bo*lei-a \xmark \\
bombardeio & bom*bar*dei*o \cmark & bom*bar*dei-o \xmark \\
bongô & bon-gô \xmark & bon*gô \cmark \\
bonomia & bo*no*mi*a \cmark & bo*no*mi-a \xmark \\
bordô & bor-dô \xmark & bor*dô \cmark \\
borracharia & bor*ra*cha*ri*a \cmark & bor*ra*cha*ri-a \xmark \\
bósnio & bós*ni.o \xmark & bós*nio \cmark \\
boticário & bo*ti*cá*ri.o \xmark & bo*ti*cá*rio \cmark \\
bovídeo & bo*ví*de.o \xmark & bo*ví*deo \cmark \\
bradicardia & bra*di*car*di*a \cmark & bra*di*car*di-a \xmark \\
brasílio & bra*sí*li.o \xmark & bra*sí*lio \cmark \\
bravio & bra*vi*o \cmark & bra*vi-o \xmark \\
brechtiano & bre.ch*ti*a*no \xmark & bre.ch*ti*a*no \xmark \\
breviário & bre*vi-á*ri.o \xmark & bre*vi*á*rio \cmark \\
brio & bri*o \cmark & bri-o \xmark \\
broa & bro*a \cmark & bro-a \xmark \\
bromatologia & bro*ma*to*lo*gi*a \cmark & bro*ma*to*lo*gi-a \xmark \\
bromélia & bro*mé*li.a \xmark & bro*mé*lia \cmark \\
brônquio & brôn*qui.o \xmark & brôn*quio \cmark \\
bronquíolo & bron-quí-o*lo \xmark & bron*quí*o*lo \cmark \\
bruxaria & bru*xa*ri*a \cmark & bru*xa*ri-a \xmark \\
bué & bu-é \xmark & bu*é \cmark \\
bugio & bu*gi*o \cmark & bu*gi-o \xmark \\
buir & bu-ir \xmark & bu*ir \cmark \\
bulício & bu*lí*ci.o \xmark & bu*lí*cio \cmark \\
bulimia & bu*li*mi*a \cmark & bu*li*mi-a \xmark \\
buquê & bu-quê \xmark & bu*quê \cmark \\
burguesia & bur*gue*si*a \cmark & bur*gue*si-a \xmark \\
burocracia & bu*ro*cra*ci*a \cmark & bu*ro*cra*ci-a \xmark \\
búteo & bú*te*o \cmark & bú*te-o \xmark \\
butiá & bu*ti-á \xmark & bu*ti*á \cmark \\
búzio & bú*zi.o \xmark & bú*zio \cmark \\
cacaio & ca*cai*o \cmark & ca*cai-o \xmark \\
cacatua & ca*ca*tu*a \cmark & ca*ca*tu-a \xmark \\
cacofonia & ca*co*fo*ni*a \cmark & ca*co*fo*ni-a \xmark \\
cadeia & ca*dei*a \cmark & ca*dei-a \xmark \\
cadência & ca*dên*ci.a \xmark & ca*dên*cia \cmark \\
cafarnaum & ca*far*na-um \xmark & ca*far*na-um \xmark \\
cafeína & ca*fe-í*na \xmark & ca*fe*í*na \cmark \\
cafeteria & ca*fe*te*ri*a \cmark & ca*fe*te*ri-a \xmark \\
cafraria & ca*fra*ri*a \cmark & ca*fra*ri-a \xmark \\
caíco & ca-í*co \xmark & ca*í*co \cmark \\
caída & ca-í*da \xmark & ca*í*da \cmark \\
caído & ca-í*do \xmark & ca*í*do \cmark \\
caim & ca-im \xmark & ca-im \xmark \\
caíque & ca-í*que \xmark & ca*í*que \cmark \\
cair & ca-ir \xmark & ca*ir \cmark \\
cajuína & ca*ju-í*na \xmark & ca*ju*í*na \cmark \\
calafrio & ca*la*fri*o \cmark & ca*la*fri-o \xmark \\
calcário & cal*cá*ri*o \cmark & cal*cá*ri-o \xmark \\
cálcio & cál*ci.o \xmark & cál*cio \cmark \\
calcografia & cal*co*gra*fi*a \cmark & cal*co*gra*fi-a \xmark \\
caleidoscópio & ca*lei*dos*có*pi.o \xmark & ca*lei*dos*có*pio \cmark \\
calendário & ca*len*dá*ri.o \xmark & ca*len*dá*rio \cmark \\
caligrafia & ca*li*gra*fi*a \cmark & ca*li*gra*fi-a \xmark \\
calmaria & cal*ma*ri*a \cmark & cal*ma*ri-a \xmark \\
caloria & ca*lo*ri*a \cmark & ca*lo*ri-a \xmark \\
calorimetria & ca*lo*ri*me*tri*a \cmark & ca*lo*ri*me*tri-a \xmark \\
calvário & cal*vá*ri.o \xmark & cal*vá*rio \cmark \\
camaleão & ca*ma*le-ão \xmark & ca*ma*le*ão \cmark \\
cambaio & cam*bai*o \cmark & cam*bai-o \xmark \\
câmbio & câm*bi.o \xmark & câm*bio \cmark \\
camboa & cam*bo*a \cmark & cam*bo-a \xmark \\
cambraia & cam*brai*a \cmark & cam*brai-a \xmark \\
cambuí & cam*bu-í \xmark & cam*bu*í \cmark \\
camélia & ca*mé*li*a \cmark & ca*mé*li-a \xmark \\
camelídeo & ca*me*lí*de.o \xmark & ca*me*lí*deo \cmark \\
camelô & ca*me-lô \xmark & ca*me*lô \cmark \\
camião & ca*mi-ão \xmark & ca*mi*ão \cmark \\
camisaria & ca*mi*sa*ri*a \cmark & ca*mi*sa*ri-a \xmark \\
campainha & cam*pa-i*nha \xmark & cam*pa*i*nha \cmark \\
campanário & cam*pa*ná*ri.o \xmark & cam*pa*ná*rio \cmark \\
campeão & cam*pe-ão \xmark & cam*pe*ão \cmark \\
canaria & ca*na*ri*a \cmark & ca*na*ri-a \xmark \\
canária & ca*ná*ri.a \xmark & ca*ná*ria \cmark \\
cancerologia & can*ce*ro*lo*gi*a \cmark & can*ce*ro*lo*gi-a \xmark \\
candeia & can*dei*a \cmark & can*dei-a \xmark \\
candelária & can*de*lá*ri*a \cmark & can*de*lá*ri-a \xmark \\
candidíase & can*di*dí-a*se \xmark & can*di*dí*a*se \cmark \\
canhoneio & ca*nho*nei*o \cmark & ca*nho*nei-o \xmark \\
canídeo & ca*ní*de.o \xmark & ca*ní*deo \cmark \\
canoa & ca*no*a \cmark & ca*no-a \xmark \\
canoísta & ca*no-ís*ta \xmark & ca*no*ís*ta \cmark \\
cantaria & can*ta*ri*a \cmark & can*ta*ri-a \xmark \\
cantoria & can*to*ri*a \cmark & can*to*ri-a \xmark \\
caótico & ca-ó*ti*co \xmark & ca*ó*ti*co \cmark \\
capadócio & ca*pa*dó*ci.o \xmark & ca*pa*dó*cio \cmark \\
capatazia & ca*pa*ta*zi*a \cmark & ca*pa*ta*zi-a \xmark \\
capelania & ca*pe*la*ni*a \cmark & ca*pe*la*ni-a \xmark \\
capicua & ca*pi*cu*a \cmark & ca*pi*cu-a \xmark \\
capitania & ca*pi*ta*ni*a \cmark & ca*pi*ta*ni-a \xmark \\
capitânia & ca*pi*tâ*ni.a \xmark & ca*pi*tâ*nia \cmark \\
capô & ca-pô \xmark & ca*pô \cmark \\
caquexia & ca*que*xi*a \cmark & ca*que*xi-a \xmark \\
caraíba & ca*ra-í*ba \xmark & ca*ra*í*ba \cmark \\
carbonária & car*bo*ná*ri.a \xmark & car*bo*ná*ria \cmark \\
carbonário & car*bo*ná*ri.o \xmark & car*bo*ná*rio \cmark \\
carbônico & car-bô*ni*co \xmark & car*bô*ni*co \cmark \\
carcerário & car*ce*rá*ri.o \xmark & car*ce*rá*rio \cmark \\
cardápio & car*dá*pi.o \xmark & car*dá*pio \cmark \\
cardíaco & car*dí-a*co \xmark & car*dí*a*co \cmark \\
cardiologia & car*di*o*lo*gi*a \cmark & car*di*o*lo*gi-a \xmark \\
carência & ca*rên*ci.a \xmark & ca*rên*cia \cmark \\
carestia & ca*res*ti*a \cmark & ca*res*ti-a \xmark \\
carícia & ca*rí*ci.a \xmark & ca*rí*cia \cmark \\
cárie & cá*ri.e \xmark & cá*rie \cmark \\
cariótipo & ca*ri-ó*ti*po \xmark & ca*ri*ó*ti*po \cmark \\
carnaúba & car*na-ú*ba \xmark & car*na*ú*ba \cmark \\
carolíngio & ca*ro*lín*gi.o \xmark & ca*ro*lín*gio \cmark \\
carotídeo & ca*ro*tí*de.o \xmark & ca*ro*tí*deo \cmark \\
carpintaria & car*pin*ta*ri*a \cmark & car*pin*ta*ri-a \xmark \\
carroceria & car*ro*ce*ri*a \cmark & car*ro*ce*ri-a \xmark \\
cartapácio & car*ta*pá*ci.o \xmark & car*ta*pá*cio \cmark \\
cartografia & car*to*gra*fi*a \cmark & car*to*gra*fi-a \xmark \\
cartomancia & car*to*man*ci*a \cmark & car*to*man*ci-a \xmark \\
cartório & car*tó*ri.o \xmark & car*tó*rio \cmark \\
carvoaria & car*vo*a*ri*a \cmark & car*vo*a*ri-a \xmark \\
casaria & ca*sa*ri*a \cmark & ca*sa*ri-a \xmark \\
casario & ca*sa*ri*o \cmark & ca*sa*ri-o \xmark \\
caseína & ca*se-í*na \xmark & ca*se*í*na \cmark \\
casório & ca*só*ri.o \xmark & ca*só*rio \cmark \\
cássia & cás*si.a \xmark & cás*sia \cmark \\
casuísmo & ca*su-ís*mo \xmark & ca*su*ís*mo \cmark \\
casuísta & ca*su-ís*ta \xmark & ca*su*ís*ta \cmark \\
casuística & ca*su-ís*ti*ca \xmark & ca*su*ís*ti*ca \cmark \\
casuístico & ca*su-ís*ti*co \xmark & ca*su*ís*ti*co \cmark \\
catalepsia & ca*ta*lep*si*a \cmark & ca*ta*lep*si-a \xmark \\
catatonia & ca*ta*to*ni*a \cmark & ca*ta*to*ni-a \xmark \\
catequético & ca*te-qué*ti*co \xmark & ca*te*qué*ti*co \cmark \\
catião & ca*ti-ão \xmark & ca*ti*ão \cmark \\
catilinária & ca*ti*li*ná*ri.a \xmark & ca*ti*li*ná*ria \cmark \\
catoptromancia & ca*top*tro*man*ci*a \cmark & ca*top*tro*man*ci-a \xmark \\
catraia & ca*trai*a \cmark & ca*trai-a \xmark \\
cauã & cau-ã \xmark & cau*ã \cmark \\
cauim & cau-im \xmark & cau*im \cmark \\
caúna & ca-ú*na \xmark & ca*ú*na \cmark \\
cavalaria & ca*va*la*ri*a \cmark & ca*va*la*ri-a \xmark \\
cávea & cá*ve*a \cmark & cá*ve-a \xmark \\
ceceio & ce*cei*o \cmark & ce*cei-o \xmark \\
cefaleia & ce*fa*lei*a \cmark & ce*fa*lei-a \xmark \\
ceia & cei*a \cmark & cei-a \xmark \\
cemitério & ce*mi*té*ri.o \xmark & ce*mi*té*rio \cmark \\
cenário & ce*ná*ri.o \xmark & ce*ná*rio \cmark \\
cenóbio & ce*nó*bi.o \xmark & ce*nó*bio \cmark \\
cenografia & ce*no*gra*fi*a \cmark & ce*no*gra*fi-a \xmark \\
cenotáfio & ce*no*tá*fi.o \xmark & ce*no*tá*fio \cmark \\
censitário & cen*si*tá*ri.o \xmark & cen*si*tá*rio \cmark \\
centáurea & cen*táu*re.a \xmark & cen*táu*rea \cmark \\
centeio & cen*tei*o \cmark & cen*tei-o \xmark \\
centenário & cen*te*ná*ri.o \xmark & cen*te*ná*rio \cmark \\
centopeia & cen*to*pei*a \cmark & cen*to*pei-a \xmark \\
centúria & cen*tú*ri.a \xmark & cen*tú*ria \cmark \\
centurião & cen*tu*ri-ão \xmark & cen*tu*ri*ão \cmark \\
cércea & cér*ce.a \xmark & cér*cea \cmark \\
cerefólio & ce*re*fó*li.o \xmark & ce*re*fó*lio \cmark \\
cério & cé*ri.o \xmark & cé*rio \cmark \\
cerúleo & ce*rú*le.o \xmark & ce*rú*leo \cmark \\
cervejaria & cer*ve*ja*ri*a \cmark & cer*ve*ja*ri-a \xmark \\
cesárea & ce*sá*re.a \xmark & ce*sá*rea \cmark \\
césio & cé*si.o \xmark & cé*sio \cmark \\
cessionário & ces*si*o*ná*ri.o \xmark & ces*si*o*ná*rio \cmark \\
cetáceo & ce*tá*ce.o \xmark & ce*tá*ceo \cmark \\
chancelaria & chan*ce*la*ri*a \cmark & chan*ce*la*ri-a \xmark \\
charrua & char*ru*a \cmark & char*ru-a \xmark \\
charutaria & cha*ru*ta*ri*a \cmark & cha*ru*ta*ri-a \xmark \\
chatô & cha-tô \xmark & cha*tô \cmark \\
chefia & che*fi*a \cmark & che*fi-a \xmark \\
cheia & chei*a \cmark & chei-a \xmark \\
cheio & chei*o \cmark & chei-o \xmark \\
chicória & chi*có*ri.a \xmark & chi*có*ria \cmark \\
chio & chi*o \cmark & chi-o \xmark \\
choperia & cho*pe*ri*a \cmark & cho*pe*ri-a \xmark \\
churrascaria & chur*ras*ca*ri*a \cmark & chur*ras*ca*ri-a \xmark \\
ciático & ci-á*ti*co \xmark & ci*á*ti*co \cmark \\
ciclotimia & ci*clo*ti*mi*a \cmark & ci*clo*ti*mi-a \xmark \\
ciclovia & ci*clo*vi*a \cmark & ci*clo*vi-a \xmark \\
cidadania & ci*da*da*ni*a \cmark & ci*da*da*ni-a \xmark \\
ciência & ci*ên*ci.a \xmark & ci*ên*cia \cmark \\
cílio & cí*li.o \xmark & cí*lio \cmark \\
cimério & ci*mé*ri.o \xmark & ci*mé*rio \cmark \\
cinábrio & ci*ná*bri.o \xmark & ci*ná*brio \cmark \\
cinefilia & ci*ne*fi*li*a \cmark & ci*ne*fi*li-a \xmark \\
cinematografia & ci*ne*ma*to*gra*fi*a \cmark & ci*ne*ma*to*gra*fi-a \xmark \\
cinescópio & ci*nes*có*pi.o \xmark & ci*nes*có*pio \cmark \\
cinestesia & ci*nes*te*si*a \cmark & ci*nes*te*si-a \xmark \\
cinofilia & ci*no*fi*li*a \cmark & ci*no*fi*li-a \xmark \\
cinologia & ci*no*lo*gi*a \cmark & ci*no*lo*gi-a \xmark \\
cinquentenário & cin*quen*te*ná*ri.o \xmark & cin*quen*te*ná*rio \cmark \\
cintilografia & cin*ti*lo*gra*fi*a \cmark & cin*ti*lo*gra*fi-a \xmark \\
cio & ci*o \cmark & ci-o \xmark \\
circunferência & cir*cun*fe*rên*ci.a \xmark & cir*cun*fe*rên*cia \cmark \\
circunstância & cir*cuns*tân*ci.a \xmark & cir*cuns*tân*cia \cmark \\
círio & cí*ri.o \xmark & cí*rio \cmark \\
cirurgia & ci*rur*gi*a \cmark & ci*rur*gi-a \xmark \\
cirurgião & ci*rur*gi-ão \xmark & ci*rur*gi*ão \cmark \\
cistoscopia & cis*tos*co*pi*a \cmark & cis*tos*co*pi-a \xmark \\
citânia & ci*tâ*ni.a \xmark & ci*tâ*nia \cmark \\
citologia & ci*to*lo*gi*a \cmark & ci*to*lo*gi-a \xmark \\
ciúme & ci-ú*me \xmark & ci*ú*me \cmark \\
cizânia & ci*zâ*ni.a \xmark & ci*zâ*nia \cmark \\
claraboia & cla*ra*boi*a \cmark & cla*ra*boi-a \xmark \\
clarividência & cla*ri*vi*dên*ci.a \xmark & cla*ri*vi*dên*cia \cmark \\
claustrofobia & claus*tro*fo*bi*a \cmark & claus*tro*fo*bi-a \xmark \\
clavicórdio & cla*vi*cór*di.o \xmark & cla*vi*cór*dio \cmark \\
clemência & cle*mên*ci.a \xmark & cle*mên*cia \cmark \\
cleptocracia & clep*to*cra*ci*a \cmark & clep*to*cra*ci-a \xmark \\
cleptomania & clep*to*ma*ni*a \cmark & clep*to*ma*ni-a \xmark \\
climatologia & cli*ma*to*lo*gi*a \cmark & cli*ma*to*lo*gi-a \xmark \\
clorofórmio & clo*ro*fór*mi.o \xmark & clo*ro*fór*mio \cmark \\
cnidário & c.ni*dá*ri.o \xmark & c.ni*dá*rio \xmark \\
coa & co*a \cmark & co-a \xmark \\
coágulo & co-á*gu*lo \xmark & co*á*gu*lo \cmark \\
cobaia & co*bai*a \cmark & co*bai-a \xmark \\
cobardia & co*bar*di*a \cmark & co*bar*di-a \xmark \\
cocaína & co*ca-í*na \xmark & co*ca*í*na \cmark \\
cocô & co-cô \xmark & co*cô \cmark \\
côdea & cô*de.a \xmark & cô*dea \cmark \\
codeína & co*de-í*na \xmark & co*de*í*na \cmark \\
coerência & co*e*rên*ci.a \xmark & co*e*rên*cia \cmark \\
coetâneo & co*e*tâ*ne*o \cmark & co*e*tâ*ne-o \xmark \\
coexistência & co*e*xis*tên*ci.a \xmark & co*e*xis*tên*cia \cmark \\
coibir & co-i*bir \xmark & co-i*bir \xmark \\
coincidência & co-in*ci*dên*ci.a \xmark & co*in*ci*dên*cia \cmark \\
coincidente & co-in*ci*den*te \xmark & co*in*ci*den*te \cmark \\
coincidir & co-in*ci*dir \xmark & co*in*ci*dir \cmark \\
colcheia & col*chei*a \cmark & col*chei-a \xmark \\
colecistectomia & co*le*cis*tec*to*mi*a \cmark & co*le*cis*tec*to*mi-a \xmark \\
colégio & co*lé*gi.o \xmark & co*lé*gio \cmark \\
coleóptero & co*le-óp*te*ro \xmark & co*le*óp*te*ro \cmark \\
colmeia & col*mei*a \cmark & col*mei-a \xmark \\
colonia & co*lo*ni*a \cmark & co*lo*ni-a \xmark \\
colonoscopia & co*lo*nos*co*pi*a \cmark & co*lo*nos*co*pi-a \xmark \\
colóquio & co*ló*qui.o \xmark & co*ló*quio \cmark \\
columbofilia & co*lum*bo*fi*li*a \cmark & co*lum*bo*fi*li-a \xmark \\
combinatório & com*bi*na*tó*ri.o \xmark & com*bi*na*tó*rio \cmark \\
comboio & com*boi*o \cmark & com*boi-o \xmark \\
comédia & co*mé*di.a \xmark & co*mé*dia \cmark \\
comediógrafo & co*me*di-ó*gra*fo \xmark & co*me*di*ó*gra*fo \cmark \\
comentário & co*men*tá*ri.o \xmark & co*men*tá*rio \cmark \\
comerciário & co*mer*ci-á*ri.o \xmark & co*mer*ci*á*rio \cmark \\
comércio & co*mér*ci.o \xmark & co*mér*cio \cmark \\
comício & co*mí*ci.o \xmark & co*mí*cio \cmark \\
comissário & co*mis*sá*ri.o \xmark & co*mis*sá*rio \cmark \\
companhia & com*pa*nhi*a \cmark & com*pa*nhi-a \xmark \\
comparsaria & com*par*sa*ri*a \cmark & com*par*sa*ri-a \xmark \\
compêndio & com*pên*di.o \xmark & com*pên*dio \cmark \\
compensatório & com*pen*sa*tó*ri.o \xmark & com*pen*sa*tó*rio \cmark \\
competência & com*pe*tên*ci.a \xmark & com*pe*tên*cia \cmark \\
complacência & com*pla*cên*ci.a \xmark & com*pla*cên*cia \cmark \\
compulsória & com*pul*só*ri.a \xmark & com*pul*só*ria \cmark \\
compulsório & com*pul*só*ri.o \xmark & com*pul*só*rio \cmark \\
conciliábulo & con*ci*li-á*bu*lo \xmark & con*ci*li*á*bu*lo \cmark \\
conciliável & con*ci*li-á*vel \xmark & con*ci*li*á*vel \cmark \\
concílio & con*cí*li.o \xmark & con*cí*lio \cmark \\
concluído & con*clu-í*do \xmark & con*clu*í*do \cmark \\
concluinte & con*clu-in*te \xmark & con*clu*in*te \cmark \\
concluir & con*clu-ir \xmark & con*clu*ir \cmark \\
concordância & con*cor*dân*ci.a \xmark & con*cor*dân*cia \cmark \\
concórdia & con*cór*di.a \xmark & con*cór*dia \cmark \\
concorrência & con*cor*rên*ci.a \xmark & con*cor*rên*cia \cmark \\
concupiscência & con*cu*pis*cên*ci.a \xmark & con*cu*pis*cên*cia \cmark \\
condenatório & con*de*na*tó*ri.o \xmark & con*de*na*tó*rio \cmark \\
condescendência & con*des*cen*dên*ci.a \xmark & con*des*cen*dên*cia \cmark \\
condoído & con*do-í*do \xmark & con*do*í*do \cmark \\
condolência & con*do*lên*ci.a \xmark & con*do*lên*cia \cmark \\
condomínio & con*do*mí*ni.o \xmark & con*do*mí*nio \cmark \\
confeitaria & con*fei*ta*ri*a \cmark & con*fei*ta*ri-a \xmark \\
conferência & con*fe*rên*ci.a \xmark & con*fe*rên*cia \cmark \\
confessionário & con*fes*si*o*ná*ri.o \xmark & con*fes*si*o*ná*rio \cmark \\
confiável & con*fi-á*vel \xmark & con*fi*á*vel \cmark \\
confidência & con*fi*dên*ci.a \xmark & con*fi*dên*cia \cmark \\
confluência & con*flu-ên*ci.a \xmark & con*flu*ên*cia \cmark \\
confluir & con*flu-ir \xmark & con*flu*ir \cmark \\
confraria & con*fra*ri*a \cmark & con*fra*ri-a \xmark \\
congruência & con*gru-ên*ci.a \xmark & con*gru*ên*cia \cmark \\
conluio & con*lui*o \cmark & con*lui-o \xmark \\
consanguíneo & con*san*guí*ne.o \xmark & con*san*guí*neo \cmark \\
consciência & cons*ci*ên*ci.a \xmark & cons*ci*ên*cia \cmark \\
cônscio & côns*ci*o \cmark & côns*ci-o \xmark \\
consentâneo & con*sen*tâ*ne.o \xmark & con*sen*tâ*neo \cmark \\
consequência & con*se-quên*ci.a \xmark & con*se*quên*cia \cmark \\
conservatório & con*ser*va*tó*ri.o \xmark & con*ser*va*tó*rio \cmark \\
consílio & con*sí*li.o \xmark & con*sí*lio \cmark \\
consistência & con*sis*tên*ci.a \xmark & con*sis*tên*cia \cmark \\
consistório & con*sis*tó*ri*o \cmark & con*sis*tó*ri-o \xmark \\
consonância & con*so*nân*ci*a \cmark & con*so*nân*ci-a \xmark \\
consórcio & con*sór*ci.o \xmark & con*sór*cio \cmark \\
conspícuo & cons*pí*cu.o \xmark & cons*pí*cuo \cmark \\
constância & cons*tân*ci.a \xmark & cons*tân*cia \cmark \\
constituição & cons*ti*tu-i*ção \xmark & cons*ti*tu*i*ção \cmark \\
constituído & cons*ti*tu-í*do \xmark & cons*ti*tu*í*do \cmark \\
constituinte & cons*ti*tu-in*te \xmark & cons*ti*tu*in*te \cmark \\
constituir & cons*ti*tu-ir \xmark & cons*ti*tu*ir \cmark \\
construído & cons*tru-í*do \xmark & cons*tru*í*do \cmark \\
construir & cons*tru-ir \xmark & cons*tru*ir \cmark \\
consuetudinário & con*su*e*tu*di*ná*ri.o \xmark & con*su*e*tu*di*ná*rio \cmark \\
consultoria & con*sul*to*ri*a \cmark & con*sul*to*ri-a \xmark \\
consultório & con*sul*tó*ri.o \xmark & con*sul*tó*rio \cmark \\
contágio & con*tá*gi.o \xmark & con*tá*gio \cmark \\
contaria & con*ta*ri*a \cmark & con*ta*ri-a \xmark \\
contemporâneo & con*tem*po*râ*ne.o \xmark & con*tem*po*râ*neo \cmark \\
conteúdo & con*te-ú*do \xmark & con*te*ú*do \cmark \\
continência & con*ti*nên*ci.a \xmark & con*ti*nên*cia \cmark \\
contingência & con*tin*gên*ci.a \xmark & con*tin*gên*cia \cmark \\
contínua & con*tí*nu.a \xmark & con*tí*nua \cmark \\
contínuo & con*tí*nu.o \xmark & con*tí*nuo \cmark \\
contraditório & con*tra*di*tó*ri.o \xmark & con*tra*di*tó*rio \cmark \\
contraído & con*tra-í*do \xmark & con*tra*í*do \cmark \\
contrair & con*tra-ir \xmark & con*tra*ir \cmark \\
contrário & con*trá*ri.o \xmark & con*trá*rio \cmark \\
contribuição & con*tri*bu-i*ção \xmark & con*tri*bu*i*ção \cmark \\
contribuidor & con*tri*bu-i*dor \xmark & con*tri*bu*i*dor \cmark \\
contribuinte & con*tri*bu-in*te \xmark & con*tri*bu*in*te \cmark \\
contribuir & con*tri*bu-ir \xmark & con*tri*bu*ir \cmark \\
controladoria & con*tro*la*do*ri*a \cmark & con*tro*la*do*ri-a \xmark \\
controvérsia & con*tro*vér*si.a \xmark & con*tro*vér*sia \cmark \\
contumácia & con*tu*má*ci.a \xmark & con*tu*má*cia \cmark \\
contundência & con*tun*dên*ci*a \cmark & con*tun*dên*ci-a \xmark \\
conúbio & co*nú*bi.o \xmark & co*nú*bio \cmark \\
conveniência & con*ve*ni*ên*ci.a \xmark & con*ve*ni*ên*cia \cmark \\
convênio & con*vê*ni.o \xmark & con*vê*nio \cmark \\
convergência & con*ver*gên*ci.a \xmark & con*ver*gên*cia \cmark \\
convivência & con*vi*vên*ci.a \xmark & con*vi*vên*cia \cmark \\
convívio & con*ví*vi*o \cmark & con*ví*vi-o \xmark \\
coordenadoria & co*or*de*na*do*ri*a \cmark & co*or*de*na*do*ri-a \xmark \\
copaíba & co*pa-í*ba \xmark & co*pa*í*ba \cmark \\
cópia & có*pi.a \xmark & có*pia \cmark \\
coprofilia & co*pro*fi*li*a \cmark & co*pro*fi*li-a \xmark \\
coprolalia & co*pro*la*li*a \cmark & co*pro*la*li-a \xmark \\
coreia & co*rei*a \cmark & co*rei-a \xmark \\
coreografia & co*re*o*gra*fi*a \cmark & co*re*o*gra*fi-a \xmark \\
coreógrafo & co*re-ó*gra*fo \xmark & co*re*ó*gra*fo \cmark \\
coríntio & co*rín*ti.o \xmark & co*rín*tio \cmark \\
córnea & cór*ne.a \xmark & cór*nea \cmark \\
córneo & cór*ne.o \xmark & cór*neo \cmark \\
cornucópia & cor*nu*có*pi.a \xmark & cor*nu*có*pia \cmark \\
coroa & co*ro*a \cmark & co*ro-a \xmark \\
coroinha & co*ro-i*nha \xmark & co*ro*i*nha \cmark \\
corolário & co*ro*lá*ri.o \xmark & co*ro*lá*rio \cmark \\
coronária & co*ro*ná*ri.a \xmark & co*ro*ná*ria \cmark \\
coronário & co*ro*ná*ri.o \xmark & co*ro*ná*rio \cmark \\
corpóreo & cor*pó*re.o \xmark & cor*pó*reo \cmark \\
corregedoria & cor*re*ge*do*ri*a \cmark & cor*re*ge*do*ri-a \xmark \\
correia & cor*rei*a \cmark & cor*rei-a \xmark \\
correio & cor*rei*o \cmark & cor*rei-o \xmark \\
correligionário & cor*re*li*gi*o*ná*ri.o \xmark & cor*re*li*gi*o*ná*rio \cmark \\
correria & cor*re*ri*a \cmark & cor*re*ri-a \xmark \\
correspondência & cor*res*pon*dên*ci.a \xmark & cor*res*pon*dên*cia \cmark \\
corroído & cor*ro-í*do \xmark & cor*ro*í*do \cmark \\
corsário & cor*sá*ri.o \xmark & cor*sá*rio \cmark \\
cortesia & cor*te*si*a \cmark & cor*te*si-a \xmark \\
corveia & cor*vei*a \cmark & cor*vei-a \xmark \\
cosmogonia & cos*mo*go*ni*a \cmark & cos*mo*go*ni-a \xmark \\
cosmografia & cos*mo*gra*fi*a \cmark & cos*mo*gra*fi-a \xmark \\
cosmologia & cos*mo*lo*gi*a \cmark & cos*mo*lo*gi-a \xmark \\
cotia & co*ti*a \cmark & co*ti-a \xmark \\
cotovia & co*to*vi*a \cmark & co*to*vi-a \xmark \\
coudelaria & cou*de*la*ri*a \cmark & cou*de*la*ri-a \xmark \\
covardia & co*var*di*a \cmark & co*var*di-a \xmark \\
coxia & co*xi*a \cmark & co*xi-a \xmark \\
crânio & crâ*ni.o \xmark & crâ*nio \cmark \\
crediário & cre*di-á*ri.o \xmark & cre*di*á*rio \cmark \\
cretáceo & cre*tá*ce.o \xmark & cre*tá*ceo \cmark \\
cria & cri*a \cmark & cri-a \xmark \\
criciúma & cri*ci-ú*ma \xmark & cri*ci*ú*ma \cmark \\
criminologia & cri*mi*no*lo*gi*a \cmark & cri*mi*no*lo*gi-a \xmark \\
criptologia & crip*to*lo*gi*a \cmark & crip*to*lo*gi-a \xmark \\
criptônimo & crip-tô*ni*mo \xmark & crip*tô*ni*mo \cmark \\
cristalografia & cris*ta*lo*gra*fi*a \cmark & cris*ta*lo*gra*fi-a \xmark \\
cristologia & cris*to*lo*gi*a \cmark & cris*to*lo*gi-a \xmark \\
critério & cri*té*ri.o \xmark & cri*té*rio \cmark \\
cromatografia & cro*ma*to*gra*fi*a \cmark & cro*ma*to*gra*fi-a \xmark \\
crômio & crô*mi.o \xmark & crô*mio \cmark \\
cromolitografia & cro*mo*li*to*gra*fi*a \cmark & cro*mo*li*to*gra*fi-a \xmark \\
cromoterapia & cro*mo*te*ra*pi*a \cmark & cro*mo*te*ra*pi-a \xmark \\
cronofotografia & cro*no*fo*to*gra*fi*a \cmark & cro*no*fo*to*gra*fi-a \xmark \\
cronografia & cro*no*gra*fi*a \cmark & cro*no*gra*fi-a \xmark \\
cronologia & cro*no*lo*gi*a \cmark & cro*no*lo*gi-a \xmark \\
cronometria & cro*no*me*tri*a \cmark & cro*no*me*tri-a \xmark \\
cronômetro & cro-nô*me*tro \xmark & cro*nô*me*tro \cmark \\
crustáceo & crus*tá*ce.o \xmark & crus*tá*ceo \cmark \\
cucurbitácea & cu*cur*bi*tá*ce.a \xmark & cu*cur*bi*tá*cea \cmark \\
cuia & cui*a \cmark & cui-a \xmark \\
cuíca & cu-í*ca \xmark & cu*í*ca \cmark \\
culinária & cu*li*ná*ri.a \xmark & cu*li*ná*ria \cmark \\
culinário & cu*li*ná*ri.o \xmark & cu*li*ná*rio \cmark \\
culminância & cul*mi*nân*ci.a \xmark & cul*mi*nân*cia \cmark \\
curadoria & cu*ra*do*ri*a \cmark & cu*ra*do*ri-a \xmark \\
cúria & cú*ri.a \xmark & cú*ria \cmark \\
curião & cu*ri-ão \xmark & cu*ri*ão \cmark \\
curie & cu*ri*e \cmark & cu*ri-e \xmark \\
curió & cu*ri-ó \xmark & cu*ri*ó \cmark \\
cúrio & cú*ri.o \xmark & cú*rio \cmark \\
curvilíneo & cur*vi*lí*ne.o \xmark & cur*vi*lí*neo \cmark \\
custeio & cus*tei*o \cmark & cus*tei-o \xmark \\
custódia & cus*tó*di.a \xmark & cus*tó*dia \cmark \\
custódio & cus*tó*di.o \xmark & cus*tó*dio \cmark \\
cutelaria & cu*te*la*ri*a \cmark & cu*te*la*ri-a \xmark \\
cutia & cu*ti*a \cmark & cu*ti-a \xmark \\
czarina & c.za*ri*na \xmark & cza*ri*na \cmark \\
czarismo & c.za*ris*mo \xmark & cza*ris*mo \cmark \\
czarista & c.za*ris*ta \xmark & cza*ris*ta \cmark \\
dactiloscopia & dac*ti*los*co*pi*a \cmark & dac*ti*los*co*pi-a \xmark \\
dadaísmo & da*da-ís*mo \xmark & da*da*ís*mo \cmark \\
dadaísta & da*da-ís*ta \xmark & da*da*ís*ta \cmark \\
daguerreótipo & da*guer*re-ó*ti*po \xmark & da*guer*re*ó*ti*po \cmark \\
daí & da-í \xmark & da*í \cmark \\
dália & dá*li.a \xmark & dá*lia \cmark \\
daltônico & dal-tô*ni*co \xmark & dal*tô*ni*co \cmark \\
danceteria & dan*ce*te*ri*a \cmark & dan*ce*te*ri-a \xmark \\
darwinismo & dar-wi*nis*mo \xmark & dar-wi*nis*mo \xmark \\
darwinista & dar-wi*nis*ta \xmark & dar-wi*nis*ta \xmark \\
deão & de-ão \xmark & de*ão \cmark \\
decacampeão & de*ca*cam*pe-ão \xmark & de*ca*cam*pe*ão \cmark \\
decadência & de*ca*dên*ci.a \xmark & de*ca*dên*cia \cmark \\
decaída & de*ca-í*da \xmark & de*ca*í*da \cmark \\
decaído & de*ca-í*do \xmark & de*ca*í*do \cmark \\
decair & de*ca-ir \xmark & de*ca*ir \cmark \\
decalcomania & de*cal*co*ma*ni*a \cmark & de*cal*co*ma*ni-a \xmark \\
decência & de*cên*ci.a \xmark & de*cên*cia \cmark \\
decêndio & de*cên*di.o \xmark & de*cên*dio \cmark \\
decisório & de*ci*só*ri.o \xmark & de*ci*só*rio \cmark \\
declínio & de*clí*ni.o \xmark & de*clí*nio \cmark \\
decorrência & de*cor*rên*ci.a \xmark & de*cor*rên*cia \cmark \\
decurião & de*cu*ri-ão \xmark & de*cu*ri*ão \cmark \\
defensoria & de*fen*so*ri*a \cmark & de*fen*so*ri-a \xmark \\
deferência & de*fe*rên*ci.a \xmark & de*fe*rên*cia \cmark \\
deficiência & de*fi*ci*ên*ci.a \xmark & de*fi*ci*ên*cia \cmark \\
deficitário & de*fi*ci*tá*ri.o \xmark & de*fi*ci*tá*rio \cmark \\
deia & dei*a \cmark & dei-a \xmark \\
deísmo & de-ís*mo \xmark & de*ís*mo \cmark \\
deísta & de-ís*ta \xmark & de*ís*ta \cmark \\
delegacia & de*le*ga*ci*a \cmark & de*le*ga*ci-a \xmark \\
deletério & de*le*té*ri.o \xmark & de*le*té*rio \cmark \\
delícia & de*lí*ci.a \xmark & de*lí*cia \cmark \\
delinquência & de*lin-quên*ci.a \xmark & de*lin*quên*cia \cmark \\
delírio & de*lí*ri.o \xmark & de*lí*rio \cmark \\
demagogia & de*ma*go*gi*a \cmark & de*ma*go*gi-a \xmark \\
demasia & de*ma*si*a \cmark & de*ma*si-a \xmark \\
demissionário & de*mis*si*o*ná*ri.o \xmark & de*mis*si*o*ná*rio \cmark \\
demiurgo & de*mi-ur*go \xmark & de*mi*ur*go \cmark \\
democracia & de*mo*cra*ci*a \cmark & de*mo*cra*ci-a \xmark \\
demografia & de*mo*gra*fi*a \cmark & de*mo*gra*fi-a \xmark \\
demoníaco & de*mo*ní-a*co \xmark & de*mo*ní*a*co \cmark \\
demonologia & de*mo*no*lo*gi*a \cmark & de*mo*no*lo*gi-a \xmark \\
dendrocronologia & den*dro*cro*no*lo*gi*a \cmark & den*dro*cro*no*lo*gi-a \xmark \\
dendrologia & den*dro*lo*gi*a \cmark & den*dro*lo*gi-a \xmark \\
densitometria & den*si*to*me*tri*a \cmark & den*si*to*me*tri-a \xmark \\
dentifrício & den*ti*frí*ci.o \xmark & den*ti*frí*cio \cmark \\
denúncia & de*nún*ci.a \xmark & de*nún*cia \cmark \\
deontologia & de*on*to*lo*gi*a \cmark & de*on*to*lo*gi-a \xmark \\
dependência & de*pen*dên*ci.a \xmark & de*pen*dên*cia \cmark \\
depoimento & de*po-i*men*to \xmark & de*po*i*men*to \cmark \\
depositário & de*po*si*tá*ri.o \xmark & de*po*si*tá*rio \cmark \\
dermatologia & der*ma*to*lo*gi*a \cmark & der*ma*to*lo*gi-a \xmark \\
desafio & de*sa*fi*o \cmark & de*sa*fi-o \xmark \\
desarmonia & de*sar*mo*ni*a \cmark & de*sar*mo*ni-a \xmark \\
desbloqueio & des*blo*quei*o \cmark & des*blo*quei-o \xmark \\
descaída & des*ca-í*da \xmark & des*ca*í*da \cmark \\
descair & des*ca-ir \xmark & des*ca*ir \cmark \\
descendência & des*cen*dên*ci.a \xmark & des*cen*dên*cia \cmark \\
desconstruir & des*cons*tru-ir \xmark & des*cons*tru*ir \cmark \\
descontraído & des*con*tra-í*do \xmark & des*con*tra*í*do \cmark \\
descontrair & des*con*tra-ir \xmark & des*con*tra*ir \cmark \\
descortesia & des*cor*te*si*a \cmark & des*cor*te*si-a \xmark \\
deselegância & de*se*le*gân*ci.a \xmark & de*se*le*gân*cia \cmark \\
desembainhar & de*sem*ba-i*nhar \xmark & de*sem*ba*i*nhar \cmark \\
desequilíbrio & de*se*qui*lí*bri.o \xmark & de*se*qui*lí*brio \cmark \\
desfastio & des*fas*ti*o \cmark & des*fas*ti-o \xmark \\
desídia & de*sí*di.a \xmark & de*sí*dia \cmark \\
desinência & de*si*nên*ci.a \xmark & de*si*nên*cia \cmark \\
desistência & de*sis*tên*ci.a \xmark & de*sis*tên*cia \cmark \\
desmaio & des*mai*o \cmark & des*mai-o \xmark \\
desnecessário & des*ne*ces*sá*ri.o \xmark & des*ne*ces*sá*rio \cmark \\
desobediência & de*so*be*di*ên*ci.a \xmark & de*so*be*di*ên*cia \cmark \\
desobstruir & de*sobs*tru-ir \xmark & de*sobs*tru*ir \cmark \\
despoluição & des*po*lu-i*ção \xmark & des*po*lu*i*ção \cmark \\
despoluir & des*po*lu-ir \xmark & des*po*lu*ir \cmark \\
desprestígio & des*pres*tí*gi*o \cmark & des*pres*tí*gi-o \xmark \\
destinatário & des*ti*na*tá*ri.o \xmark & des*ti*na*tá*rio \cmark \\
destituição & des*ti*tu-i*ção \xmark & des*ti*tu*i*ção \cmark \\
destituído & des*ti*tu-í*do \xmark & des*ti*tu*í*do \cmark \\
destituir & des*ti*tu-ir \xmark & des*ti*tu*ir \cmark \\
destruição & des*tru-i*ção \xmark & des*tru*i*ção \cmark \\
destruído & des*tru-í*do \xmark & des*tru*í*do \cmark \\
destruir & des*tru-ir \xmark & des*tru*ir \cmark \\
desvario & des*va*ri*o \cmark & des*va*ri-o \xmark \\
desvio & des*vi*o \cmark & des*vi-o \xmark \\
deutério & deu*té*ri.o \xmark & deu*té*rio \cmark \\
devaneio & de*va*nei*o \cmark & de*va*nei-o \xmark \\
diaconia & di*a*co*ni*a \cmark & di*a*co*ni-a \xmark \\
diácono & di-á*co*no \xmark & di*á*co*no \cmark \\
diacronia & di*a*cro*ni*a \cmark & di*a*cro*ni-a \xmark \\
díade & dí-a*de \xmark & dí*a*de \cmark \\
dia & di*a \cmark & di-a \xmark \\
diáfano & di-á*fa*no \xmark & di*á*fa*no \cmark \\
diálise & di-á*li*se \xmark & di*á*li*se \cmark \\
diálogo & di-á*lo*go \xmark & di*á*lo*go \cmark \\
diária & di-á*ri.a \xmark & di*á*ria \cmark \\
diário & di-á*ri.o \xmark & di*á*rio \cmark \\
diarquia & di*ar*qui*a \cmark & di*ar*qui-a \xmark \\
diarreia & di*ar*rei*a \cmark & di*ar*rei-a \xmark \\
diáspora & di-ás*po*ra \xmark & di*ás*po*ra \cmark \\
diástole & di-ás*to*le \xmark & di*ás*to*le \cmark \\
dicionário & di*ci*o*ná*ri.o \xmark & di*ci*o*ná*rio \cmark \\
dicotomia & di*co*to*mi*a \cmark & di*co*to*mi-a \xmark \\
dicroísmo & di*cro-ís*mo \xmark & di*cro*ís*mo \cmark \\
diérese & di-é*re*se \xmark & di*é*re*se \cmark \\
diferenciável & di*fe*ren*ci-á*vel \xmark & di*fe*ren*ci*á*vel \cmark \\
difluência & di*flu-ên*ci.a \xmark & di*flu*ên*cia \cmark \\
difteria & dif*te*ri*a \cmark & dif*te*ri-a \xmark \\
diglossia & di*glos*si*a \cmark & di*glos*si-a \xmark \\
dignitário & dig*ni*tá*ri.o \xmark & dig*ni*tá*rio \cmark \\
diligência & di*li*gên*ci.a \xmark & di*li*gên*cia \cmark \\
diluído & di*lu-í*do \xmark & di*lu*í*do \cmark \\
diluir & di*lu-ir \xmark & di*lu*ir \cmark \\
dilúvio & di*lú*vi.o \xmark & di*lú*vio \cmark \\
diminuição & di*mi*nu-i*ção \xmark & di*mi*nu*i*ção \cmark \\
diminuído & di*mi*nu-í*do \xmark & di*mi*nu*í*do \cmark \\
diminuir & di*mi*nu-ir \xmark & di*mi*nu*ir \cmark \\
dinamarquês & di*na*mar-quês \xmark & di*na*mar*quês \cmark \\
dinastia & di*nas*ti*a \cmark & di*nas*ti-a \xmark \\
dionisíaco & di*o*ni*sí-a*co \xmark & di*o*ni*sí*a*co \cmark \\
dióxido & di-ó*xi*do \xmark & di*ó*xi*do \cmark \\
diplegia & di*ple*gi*a \cmark & di*ple*gi-a \xmark \\
diplomacia & di*plo*ma*ci*a \cmark & di*plo*ma*ci-a \xmark \\
diretoria & di*re*to*ri*a \cmark & di*re*to*ri-a \xmark \\
diretório & di*re*tó*ri.o \xmark & di*re*tó*rio \cmark \\
disartria & di*sar*tri*a \cmark & di*sar*tri-a \xmark \\
discinesia & dis*ci*ne*si*a \cmark & dis*ci*ne*si-a \xmark \\
discografia & dis*co*gra*fi*a \cmark & dis*co*gra*fi-a \xmark \\
discordância & dis*cor*dân*ci.a \xmark & dis*cor*dân*cia \cmark \\
discórdia & dis*cór*di.a \xmark & dis*cór*dia \cmark \\
discrepância & dis*cre*pân*ci.a \xmark & dis*cre*pân*cia \cmark \\
discricionário & dis*cri*ci*o*ná*ri.o \xmark & dis*cri*ci*o*ná*rio \cmark \\
disenteria & di*sen*te*ri*a \cmark & di*sen*te*ri-a \xmark \\
disfagia & dis*fa*gi*a \cmark & dis*fa*gi-a \xmark \\
disfemia & dis*fe*mi*a \cmark & dis*fe*mi-a \xmark \\
disfonia & dis*fo*ni*a \cmark & dis*fo*ni-a \xmark \\
disforia & dis*fo*ri*a \cmark & dis*fo*ri-a \xmark \\
disgenesia & dis*ge*ne*si*a \cmark & dis*ge*ne*si-a \xmark \\
dislalia & dis*la*li*a \cmark & dis*la*li-a \xmark \\
dislexia & dis*le*xi*a \cmark & dis*le*xi-a \xmark \\
dismenorreia & dis*me*nor*rei*a \cmark & dis*me*nor*rei-a \xmark \\
dispêndio & dis*pên*di.o \xmark & dis*pên*dio \cmark \\
dispepsia & dis*pep*si*a \cmark & dis*pep*si-a \xmark \\
displasia & dis*pla*si*a \cmark & dis*pla*si-a \xmark \\
dispneia & disp*nei*a \cmark & disp*nei-a \xmark \\
disprósio & dis*pró*si.o \xmark & dis*pró*sio \cmark \\
disritmia & dis*rit*mi*a \cmark & dis*rit*mi-a \xmark \\
dissacarídeo & dis*sa*ca*rí*de.o \xmark & dis*sa*ca*rí*deo \cmark \\
dissidência & dis*si*dên*ci.a \xmark & dis*si*dên*cia \cmark \\
dissídio & dis*sí*di.o \xmark & dis*sí*dio \cmark \\
dissimetria & dis*si*me*tri*a \cmark & dis*si*me*tri-a \xmark \\
dissonância & dis*so*nân*ci.a \xmark & dis*so*nân*cia \cmark \\
distanásia & dis*ta*ná*si.a \xmark & dis*ta*ná*sia \cmark \\
distância & dis*tân*ci.a \xmark & dis*tân*cia \cmark \\
distimia & dis*ti*mi*a \cmark & dis*ti*mi-a \xmark \\
distocia & dis*to*ci*a \cmark & dis*to*ci-a \xmark \\
distonia & dis*to*ni*a \cmark & dis*to*ni-a \xmark \\
distraído & dis*tra-í*do \xmark & dis*tra*í*do \cmark \\
distrair & dis*tra-ir \xmark & dis*tra*ir \cmark \\
distribuição & dis*tri*bu-i*ção \xmark & dis*tri*bu*i*ção \cmark \\
distribuído & dis*tri*bu-í*do \xmark & dis*tri*bu*í*do \cmark \\
distribuidora & dis*tri*bu-i*do*ra \xmark & dis*tri*bu*i*do*ra \cmark \\
distribuidor & dis*tri*bu-i*dor \xmark & dis*tri*bu*i*dor \cmark \\
distribuir & dis*tri*bu-ir \xmark & dis*tri*bu*ir \cmark \\
distúrbio & dis*túr*bi.o \xmark & dis*túr*bio \cmark \\
disúria & di*sú*ri.a \xmark & di*sú*ria \cmark \\
diurno & di-ur*no \xmark & di*ur*no \cmark \\
divergência & di*ver*gên*ci.a \xmark & di*ver*gên*cia \cmark \\
divisória & di*vi*só*ri.a \xmark & di*vi*só*ria \cmark \\
divisório & di*vi*só*ri.o \xmark & di*vi*só*rio \cmark \\
divórcio & di*vór*ci.o \xmark & di*vór*cio \cmark \\
doceria & do*ce*ri*a \cmark & do*ce*ri-a \xmark \\
documentário & do*cu*men*tá*ri.o \xmark & do*cu*men*tá*rio \cmark \\
dodecafonia & do*de*ca*fo*ni*a \cmark & do*de*ca*fo*ni-a \xmark \\
dodecafônico & do*de*ca-fô*ni*co \xmark & do*de*ca*fô*ni*co \cmark \\
doentio & do*en*ti*o \cmark & do*en*ti-o \xmark \\
doído & do-í*do \xmark & do*í*do \cmark \\
domiciliário & do*mi*ci*li-á*ri.o \xmark & do*mi*ci*li*á*rio \cmark \\
domicílio & do*mi*cí*li.o \xmark & do*mi*cí*lio \cmark \\
domínio & do*mí*ni.o \xmark & do*mí*nio \cmark \\
donataria & do*na*ta*ri*a \cmark & do*na*ta*ri-a \xmark \\
donatário & do*na*tá*ri.o \xmark & do*na*tá*rio \cmark \\
dormitório & dor*mi*tó*ri*o \cmark & dor*mi*tó*ri-o \xmark \\
doutrinário & dou*tri*ná*ri.o \xmark & dou*tri*ná*rio \cmark \\
doxografia & do*xo*gra*fi*a \cmark & do*xo*gra*fi-a \xmark \\
drágea & drá*ge.a \xmark & drá*gea \cmark \\
dramaturgia & dra*ma*tur*gi*a \cmark & dra*ma*tur*gi-a \xmark \\
dríade & drí-a*de \xmark & drí*a*de \cmark \\
drogaria & dro*ga*ri*a \cmark & dro*ga*ri-a \xmark \\
dromedário & dro*me*dá*ri.o \xmark & dro*me*dá*rio \cmark \\
dúbio & dú*bi.o \xmark & dú*bio \cmark \\
dulcineia & dul*ci*nei*a \cmark & dul*ci*nei-a \xmark \\
duo & du*o \cmark & du-o \xmark \\
duralumínio & du*ra*lu*mí*ni.o \xmark & du*ra*lu*mí*nio \cmark \\
dúzia & dú*zi.a \xmark & dú*zia \cmark \\
dzeta & d.ze*ta \xmark & d.ze*ta \xmark \\
ébrio & é*bri*o \cmark & é*bri-o \xmark \\
ebúrneo & e*búr*ne.o \xmark & e*búr*neo \cmark \\
eclesiástico & e*cle*si-ás*ti*co \xmark & e*cle*si*ás*ti*co \cmark \\
ecografia & e*co*gra*fi*a \cmark & e*co*gra*fi-a \xmark \\
ecolalia & e*co*la*li*a \cmark & e*co*la*li-a \xmark \\
ecologia & e*co*lo*gi*a \cmark & e*co*lo*gi-a \xmark \\
econometria & e*co*no*me*tri*a \cmark & e*co*no*me*tri-a \xmark \\
economia & e*co*no*mi*a \cmark & e*co*no*mi-a \xmark \\
econômico & e*co-nô*mi*co \xmark & e*co*nô*mi*co \cmark \\
ecônomo & e-cô*no*mo \xmark & e*cô*no*mo \cmark \\
ecotoxicologia & e*co*to*xi*co*lo*gi*a \cmark & e*co*to*xi*co*lo*gi-a \xmark \\
ectasia & ec*ta*si*a \cmark & ec*ta*si-a \xmark \\
edifício & e*di*fí*ci.o \xmark & e*di*fí*cio \cmark \\
editoria & e*di*to*ri*a \cmark & e*di*to*ri-a \xmark \\
educandário & e*du*can*dá*ri.o \xmark & e*du*can*dá*rio \cmark \\
eficácia & e*fi*cá*ci.a \xmark & e*fi*cá*cia \cmark \\
eficiência & e*fi*ci*ên*ci.a \xmark & e*fi*ci*ên*cia \cmark \\
efígie & e*fí*gi.e \xmark & e*fí*gie \cmark \\
eflúvio & e*flú*vi.o \xmark & e*flú*vio \cmark \\
egípcio & e*gíp*ci.o \xmark & e*gíp*cio \cmark \\
egiptologia & e*gip*to*lo*gi*a \cmark & e*gip*to*lo*gi-a \xmark \\
egoísmo & e*go-ís*mo \xmark & e*go*ís*mo \cmark \\
egoísta & e*go-ís*ta \xmark & e*go*ís*ta \cmark \\
egoístico & e*go-ís*ti*co \xmark & e*go*ís*ti*co \cmark \\
egrégio & e*gré*gi.o \xmark & e*gré*gio \cmark \\
eia & ei*a \cmark & ei-a \xmark \\
elastômero & e*las-tô*me*ro \xmark & e*las*tô*me*ro \cmark \\
elefantíase & e*le*fan*tí-a*se \xmark & e*le*fan*tí*a*se \cmark \\
elegância & e*le*gân*ci.a \xmark & e*le*gân*cia \cmark \\
elegia & e*le*gi*a \cmark & e*le*gi-a \xmark \\
eletrocardiógrafo & e*le*tro*car*di-ó*gra*fo \xmark & e*le*tro*car*di*ó*gra*fo \cmark \\
eletroímã & e*le*tro-í*mã \xmark & e*le*tro*í*mã \cmark \\
eletroquímica & e*le*tro-quí*mi*ca \xmark & e*le*tro*quí*mi*ca \cmark \\
eliminatória & e*li*mi*na*tó*ri.a \xmark & e*li*mi*na*tó*ria \cmark \\
elogiável & e*lo*gi-á*vel \xmark & e*lo*gi*á*vel \cmark \\
elogio & e*lo*gi*o \cmark & e*lo*gi-o \xmark \\
eloquência & e*lo-quên*ci.a \xmark & e*lo*quên*cia \cmark \\
elucidário & e*lu*ci*dá*ri.o \xmark & e*lu*ci*dá*rio \cmark \\
eluição & e*lu-i*ção \xmark & e*lu*i*ção \cmark \\
embainhar & em*ba-i*nhar \xmark & em*ba*i*nhar \cmark \\
embolia & em*bo*li*a \cmark & em*bo*li-a \xmark \\
embrião & em*bri-ão \xmark & em*bri*ão \cmark \\
embriologia & em*bri*o*lo*gi*a \cmark & em*bri*o*lo*gi-a \xmark \\
embrionário & em*bri*o*ná*ri.o \xmark & em*bri*o*ná*rio \cmark \\
ementário & e*men*tá*ri*o \cmark & e*men*tá*ri-o \xmark \\
emergência & e*mer*gên*ci.a \xmark & e*mer*gên*cia \cmark \\
eminência & e*mi*nên*ci.a \xmark & e*mi*nên*cia \cmark \\
emissário & e*mis*sá*ri.o \xmark & e*mis*sá*rio \cmark \\
empáfia & em*pá*fi.a \xmark & em*pá*fia \cmark \\
empatia & em*pa*ti*a \cmark & em*pa*ti-a \xmark \\
empório & em*pó*ri.o \xmark & em*pó*rio \cmark \\
encefalopatia & en*ce*fa*lo*pa*ti*a \cmark & en*ce*fa*lo*pa*ti-a \xmark \\
enciclopédia & en*ci*clo*pé*di.a \xmark & en*ci*clo*pé*dia \cmark \\
endemia & en*de*mi*a \cmark & en*de*mi-a \xmark \\
endocárdio & en*do*cár*di.o \xmark & en*do*cár*dio \cmark \\
endocrinologia & en*do*cri*no*lo*gi*a \cmark & en*do*cri*no*lo*gi-a \xmark \\
endodontia & en*do*don*ti*a \cmark & en*do*don*ti-a \xmark \\
endogamia & en*do*ga*mi*a \cmark & en*do*ga*mi-a \xmark \\
endométrio & en*do*mé*tri.o \xmark & en*do*mé*trio \cmark \\
endoscopia & en*dos*co*pi*a \cmark & en*dos*co*pi-a \xmark \\
energia & e*ner*gi*a \cmark & e*ner*gi-a \xmark \\
enfermaria & en*fer*ma*ri*a \cmark & en*fer*ma*ri-a \xmark \\
engenharia & en*ge*nha*ri*a \cmark & en*ge*nha*ri-a \xmark \\
enguia & en*gui*a \cmark & en*gui-a \xmark \\
enjoo & en*jo*o \cmark & en*jo-o \xmark \\
enologia & e*no*lo*gi*a \cmark & e*no*lo*gi-a \xmark \\
ensaio & en*sai*o \cmark & en*sai-o \xmark \\
ensaísmo & en*sa-ís*mo \xmark & en*sa*ís*mo \cmark \\
ensaísta & en*sa-ís*ta \xmark & en*sa*ís*ta \cmark \\
entalpia & en*tal*pi*a \cmark & en*tal*pi-a \xmark \\
enteropatia & en*te*ro*pa*ti*a \cmark & en*te*ro*pa*ti-a \xmark \\
entomologia & en*to*mo*lo*gi*a \cmark & en*to*mo*lo*gi-a \xmark \\
entremeio & en*tre*mei*o \cmark & en*tre*mei-o \xmark \\
entropia & en*tro*pi*a \cmark & en*tro*pi-a \xmark \\
entusiástico & en*tu*si-ás*ti*co \xmark & en*tu*si*ás*ti*co \cmark \\
envio & en*vi*o \cmark & en*vi-o \xmark \\
envoltório & en*vol*tó*ri.o \xmark & en*vol*tó*rio \cmark \\
enxertia & en*xer*ti*a \cmark & en*xer*ti-a \xmark \\
enxovia & en*xo*vi*a \cmark & en*xo*vi-a \xmark \\
enzimologia & en*zi*mo*lo*gi*a \cmark & en*zi*mo*lo*gi-a \xmark \\
eólico & e-ó*li*co \xmark & e*ó*li*co \cmark \\
éon & é-on \xmark & é*on \cmark \\
eparquia & e*par*qui*a \cmark & e*par*qui-a \xmark \\
epidemia & e*pi*de*mi*a \cmark & e*pi*de*mi-a \xmark \\
epidemiologia & e*pi*de*mi*o*lo*gi*a \cmark & e*pi*de*mi*o*lo*gi-a \xmark \\
epigrafia & e*pi*gra*fi*a \cmark & e*pi*gra*fi-a \xmark \\
epilepsia & e*pi*lep*si*a \cmark & e*pi*lep*si-a \xmark \\
epistemologia & e*pis*te*mo*lo*gi*a \cmark & e*pis*te*mo*lo*gi-a \xmark \\
epitáfio & e*pi*tá*fi.o \xmark & e*pi*tá*fio \cmark \\
epizootia & e*pi*zo*o*ti*a \cmark & e*pi*zo*o*ti-a \xmark \\
epopeia & e*po*pei*a \cmark & e*po*pei-a \xmark \\
equânime & e-quâ*ni*me \xmark & e*quâ*ni*me \cmark \\
equídeo & e-quí*de.o \xmark & e*quí*deo \cmark \\
equilíbrio & e*qui*lí*bri.o \xmark & e*qui*lí*brio \cmark \\
equinócio & e*qui*nó*ci.o \xmark & e*qui*nó*cio \cmark \\
equivalência & e*qui*va*lên*ci.a \xmark & e*qui*va*lên*cia \cmark \\
equívoco & e-quí*vo*co \xmark & e*quí*vo*co \cmark \\
erário & e*rá*ri.o \xmark & e*rá*rio \cmark \\
érbio & ér*bi.o \xmark & ér*bio \cmark \\
ergometria & er*go*me*tri*a \cmark & er*go*me*tri-a \xmark \\
ergonomia & er*go*no*mi*a \cmark & er*go*no*mi-a \xmark \\
ergonômico & er*go-nô*mi*co \xmark & er*go*nô*mi*co \cmark \\
erotomania & e*ro*to*ma*ni*a \cmark & e*ro*to*ma*ni-a \xmark \\
esbórnia & es*bór*ni.a \xmark & es*bór*nia \cmark \\
escadaria & es*ca*da*ri*a \cmark & es*ca*da*ri-a \xmark \\
escândio & es*cân*di.o \xmark & es*cân*dio \cmark \\
escanteio & es*can*tei*o \cmark & es*can*tei-o \xmark \\
escárnio & es*cár*ni.o \xmark & es*cár*nio \cmark \\
escatologia & es*ca*to*lo*gi*a \cmark & es*ca*to*lo*gi-a \xmark \\
esclerodermia & es*cle*ro*der*mi*a \cmark & es*cle*ro*der*mi-a \xmark \\
escleroterapia & es*cle*ro*te*ra*pi*a \cmark & es*cle*ro*te*ra*pi-a \xmark \\
escócia & es*có*ci.a \xmark & es*có*cia \cmark \\
escólio & es*có*li.o \xmark & es*có*lio \cmark \\
escória & es*có*ri.a \xmark & es*có*ria \cmark \\
escorpião & es*cor*pi-ão \xmark & es*cor*pi*ão \cmark \\
escorregadio & es*cor*re*ga*di*o \cmark & es*cor*re*ga*di-o \xmark \\
escritório & es*cri*tó*ri.o \xmark & es*cri*tó*rio \cmark \\
escriturário & es*cri*tu*rá*ri.o \xmark & es*cri*tu*rá*rio \cmark \\
escrofulária & es*cro*fu*lá*ri.a \xmark & es*cro*fu*lá*ria \cmark \\
escrutínio & es*cru*tí*ni.o \xmark & es*cru*tí*nio \cmark \\
escuderia & es*cu*de*ri*a \cmark & es*cu*de*ri-a \xmark \\
esculápio & es*cu*lá*pi.o \xmark & es*cu*lá*pio \cmark \\
esfigmomanômetro & es*fig*mo*ma-nô*me*tro \xmark & es*fig*mo*ma*nô*me*tro \cmark \\
esguio & es*gui*o \cmark & es*gui-o \xmark \\
esofagectomia & e*so*fa*gec*to*mi*a \cmark & e*so*fa*gec*to*mi-a \xmark \\
especiaria & es*pe*ci*a*ri*a \cmark & es*pe*ci*a*ri-a \xmark \\
espécie & es*pé*ci.e \xmark & es*pé*cie \cmark \\
espia & es*pi*a \cmark & es*pi-a \xmark \\
espião & es*pi-ão \xmark & es*pi*ão \cmark \\
esplenomegalia & es*ple*no*me*ga*li*a \cmark & es*ple*no*me*ga*li-a \xmark \\
esponsais & es*pon*sa-is \xmark & es*pon*sa-is \xmark \\
espontâneo & es*pon*tâ*ne.o \xmark & es*pon*tâ*neo \cmark \\
espúrio & es*pú*ri.o \xmark & es*pú*rio \cmark \\
esquadria & es*qua*dri*a \cmark & es*qua*dri-a \xmark \\
esquálido & es-quá*li*do \xmark & es*quá*li*do \cmark \\
esquizofrenia & es*qui*zo*fre*ni*a \cmark & es*qui*zo*fre*ni-a \xmark \\
essência & es*sên*ci.a \xmark & es*sên*cia \cmark \\
estacaria & es*ta*ca*ri*a \cmark & es*ta*ca*ri-a \xmark \\
estacionário & es*ta*ci*o*ná*ri.o \xmark & es*ta*ci*o*ná*rio \cmark \\
estadia & es*ta*di*a \cmark & es*ta*di-a \xmark \\
estádio & es*tá*di.o \xmark & es*tá*dio \cmark \\
estágio & es*tá*gi.o \xmark & es*tá*gio \cmark \\
estamparia & es*tam*pa*ri*a \cmark & es*tam*pa*ri-a \xmark \\
estância & es*tân*ci.a \xmark & es*tân*cia \cmark \\
estátua & es*tá*tu.a \xmark & es*tá*tua \cmark \\
esteárico & es*te-á*ri*co \xmark & es*te*á*ri*co \cmark \\
esteatorreia & es*te*a*tor*rei*a \cmark & es*te*a*tor*rei-a \xmark \\
esteio & es*tei*o \cmark & es*tei-o \xmark \\
estelionatário & es*te*li*o*na*tá*ri.o \xmark & es*te*li*o*na*tá*rio \cmark \\
estenografia & es*te*no*gra*fi*a \cmark & es*te*no*gra*fi-a \xmark \\
estequiometria & es*te*qui*o*me*tri*a \cmark & es*te*qui*o*me*tri-a \xmark \\
estéreo & es*té*re.o \xmark & es*té*reo \cmark \\
estereótipo & es*te*re-ó*ti*po \xmark & es*te*re*ó*ti*po \cmark \\
estígio & es*tí*gi.o \xmark & es*tí*gio \cmark \\
estio & es*ti*o \cmark & es*ti-o \xmark \\
estipêndio & es*ti*pên*di.o \xmark & es*ti*pên*dio \cmark \\
estômago & es-tô*ma*go \xmark & es*tô*ma*go \cmark \\
estomatologia & es*to*ma*to*lo*gi*a \cmark & es*to*ma*to*lo*gi-a \xmark \\
estratégia & es*tra*té*gi.a \xmark & es*tra*té*gia \cmark \\
estratigrafia & es*tra*ti*gra*fi*a \cmark & es*tra*ti*gra*fi-a \xmark \\
estrebaria & es*tre*ba*ri*a \cmark & es*tre*ba*ri-a \xmark \\
estreia & es*trei*a \cmark & es*trei-a \xmark \\
estria & es*tri*a \cmark & es*tri-a \xmark \\
estroboscópio & es*tro*bos*có*pi.o \xmark & es*tro*bos*có*pio \cmark \\
estrôncio & es*trôn*ci.o \xmark & es*trôn*cio \cmark \\
estuário & es*tu-á*ri.o \xmark & es*tu*á*rio \cmark \\
estúdio & es*tú*di.o \xmark & es*tú*dio \cmark \\
estultícia & es*tul*tí*ci.a \xmark & es*tul*tí*cia \cmark \\
estúrdia & es*túr*di.a \xmark & es*túr*dia \cmark \\
esvair & es*va-ir \xmark & es*va*ir \cmark \\
etéreo & e*té*re.o \xmark & e*té*reo \cmark \\
etimologia & e*ti*mo*lo*gi*a \cmark & e*ti*mo*lo*gi-a \xmark \\
etiologia & e*ti*o*lo*gi*a \cmark & e*ti*o*lo*gi-a \xmark \\
etíope & e*tí-o*pe \xmark & e*tí*o*pe \cmark \\
etnia & et*ni*a \cmark & et*ni-a \xmark \\
etnografia & et*no*gra*fi*a \cmark & et*no*gra*fi-a \xmark \\
etnologia & et*no*lo*gi*a \cmark & et*no*lo*gi-a \xmark \\
etnomusicologia & et*no*mu*si*co*lo*gi*a \cmark & et*no*mu*si*co*lo*gi-a \xmark \\
etnônimo & et-nô*ni*mo \xmark & et*nô*ni*mo \cmark \\
etologia & e*to*lo*gi*a \cmark & e*to*lo*gi-a \xmark \\
eucariótico & eu*ca*ri-ó*ti*co \xmark & eu*ca*ri*ó*ti*co \cmark \\
eucaristia & eu*ca*ris*ti*a \cmark & eu*ca*ris*ti-a \xmark \\
eufonia & eu*fo*ni*a \cmark & eu*fo*ni-a \xmark \\
euforia & eu*fo*ri*a \cmark & eu*fo*ri-a \xmark \\
eugenia & eu*ge*ni*a \cmark & eu*ge*ni-a \xmark \\
eulalia & eu*la*li*a \cmark & eu*la*li-a \xmark \\
eurasiático & eu*ra*si-á*ti*co \xmark & eu*ra*si*á*ti*co \cmark \\
európio & eu*ró*pi.o \xmark & eu*ró*pio \cmark \\
eustasia & eus*ta*si*a \cmark & eus*ta*si-a \xmark \\
eutanásia & eu*ta*ná*si.a \xmark & eu*ta*ná*sia \cmark \\
evidência & e*vi*dên*ci.a \xmark & e*vi*dên*cia \cmark \\
evoé & e*vo-é \xmark & e*vo*é \cmark \\
evoluído & e*vo*lu-í*do \xmark & e*vo*lu*í*do \cmark \\
evoluir & e*vo*lu-ir \xmark & e*vo*lu*ir \cmark \\
excelência & ex*ce*lên*ci.a \xmark & ex*ce*lên*cia \cmark \\
excluído & ex*clu-í*do \xmark & ex*clu*í*do \cmark \\
excluir & ex*clu-ir \xmark & ex*clu*ir \cmark \\
excrescência & ex*cres*cên*ci.a \xmark & ex*cres*cên*cia \cmark \\
exéquias & e*xé*qui.as \xmark & e*xé*qui.as \xmark \\
exequível & e*xe-quí*vel \xmark & e*xe*quí*vel \cmark \\
exigência & e*xi*gên*ci.a \xmark & e*xi*gên*cia \cmark \\
exímio & e*xí*mi.o \xmark & e*xí*mio \cmark \\
existência & e*xis*tên*ci.a \xmark & e*xis*tên*cia \cmark \\
exoftalmia & e*xof*tal*mi*a \cmark & e*xof*tal*mi-a \xmark \\
exogamia & e*xo*ga*mi*a \cmark & e*xo*ga*mi-a \xmark \\
exórdio & e*xór*di.o \xmark & e*xór*dio \cmark \\
expedicionário & ex*pe*di*ci*o*ná*ri.o \xmark & ex*pe*di*ci*o*ná*rio \cmark \\
experiência & ex*pe*ri*ên*ci.a \xmark & ex*pe*ri*ên*cia \cmark \\
extermínio & ex*ter*mí*ni.o \xmark & ex*ter*mí*nio \cmark \\
extrair & ex*tra-ir \xmark & ex*tra*ir \cmark \\
extraordinário & ex*tra*or*di*ná*ri.o \xmark & ex*tra*or*di*ná*rio \cmark \\
extravio & ex*tra*vi*o \cmark & ex*tra*vi-o \xmark \\
exuberância & e*xu*be*rân*ci.a \xmark & e*xu*be*rân*cia \cmark \\
factício & fac*tí*ci.o \xmark & fac*tí*cio \cmark \\
faia & fai*a \cmark & fai-a \xmark \\
faim & fa-im \xmark & fa-im \xmark \\
faísca & fa-ís*ca \xmark & fa*ís*ca \cmark \\
falácia & fa*lá*ci.a \xmark & fa*lá*cia \cmark \\
falcatrua & fal*ca*tru*a \cmark & fal*ca*tru-a \xmark \\
falência & fa*lên*ci.a \xmark & fa*lên*cia \cmark \\
falésia & fa*lé*si.a \xmark & fa*lé*sia \cmark \\
fancaria & fan*ca*ri*a \cmark & fan*ca*ri-a \xmark \\
fantasia & fan*ta*si*a \cmark & fan*ta*si-a \xmark \\
fantasmagoria & fan*tas*ma*go*ri*a \cmark & fan*tas*ma*go*ri-a \xmark \\
faraó & fa*ra-ó \xmark & fa*ra*ó \cmark \\
faríngeo & fa*rín*ge.o \xmark & fa*rín*geo \cmark \\
farisaísmo & fa*ri*sa-ís*mo \xmark & fa*ri*sa*ís*mo \cmark \\
farmácia & far*má*ci.a \xmark & far*má*cia \cmark \\
farmacopeia & far*ma*co*pei*a \cmark & far*ma*co*pei-a \xmark \\
farmacoterapia & far*ma*co*te*ra*pi*a \cmark & far*ma*co*te*ra*pi-a \xmark \\
fascínio & fas*cí*ni.o \xmark & fas*cí*nio \cmark \\
fastígio & fas*tí*gi.o \xmark & fas*tí*gio \cmark \\
fastio & fas*ti*o \cmark & fas*ti-o \xmark \\
fatia & fa*ti*a \cmark & fa*ti-a \xmark \\
faúlha & fa-ú*lha \xmark & fa*ú*lha \cmark \\
fazendário & fa*zen*dá*ri.o \xmark & fa*zen*dá*rio \cmark \\
feérico & fe-é*ri*co \xmark & fe*é*ri*co \cmark \\
feio & fei*o \cmark & fei-o \xmark \\
feitiçaria & fei*ti*ça*ri*a \cmark & fei*ti*ça*ri-a \xmark \\
feitio & fei*ti*o \cmark & fei*ti-o \xmark \\
feitoria & fei*to*ri*a \cmark & fei*to*ri-a \xmark \\
feiura & fei-u*ra \xmark & fei*u*ra \cmark \\
felídeo & fe*lí*de.o \xmark & fe*lí*deo \cmark \\
felonia & fe*lo*ni*a \cmark & fe*lo*ni-a \xmark \\
fêmea & fê*me.a \xmark & fê*mea \cmark \\
fenício & fe*ní*ci.o \xmark & fe*ní*cio \cmark \\
fenologia & fe*no*lo*gi*a \cmark & fe*no*lo*gi-a \xmark \\
fenômeno & fe-nô*me*no \xmark & fe*nô*me*no \cmark \\
fenomenologia & fe*no*me*no*lo*gi*a \cmark & fe*no*me*no*lo*gi-a \xmark \\
féria & fé*ri.a \xmark & fé*ria \cmark \\
férmio & fér*mi.o \xmark & fér*mio \cmark \\
ferraria & fer*ra*ri*a \cmark & fer*ra*ri-a \xmark \\
ferrovia & fer*ro*vi*a \cmark & fer*ro*vi-a \xmark \\
fiável & fi-á*vel \xmark & fi*á*vel \cmark \\
fibromialgia & fi*bro*mi*al*gi*a \cmark & fi*bro*mi*al*gi-a \xmark \\
fichário & fi*chá*ri.o \xmark & fi*chá*rio \cmark \\
ficologia & fi*co*lo*gi*a \cmark & fi*co*lo*gi-a \xmark \\
fictício & fic*tí*ci.o \xmark & fic*tí*cio \cmark \\
fideísmo & fi*de-ís*mo \xmark & fi*de*ís*mo \cmark \\
fidúcia & fi*dú*ci.a \xmark & fi*dú*cia \cmark \\
fiduciário & fi*du*ci-á*ri.o \xmark & fi*du*ci*á*rio \cmark \\
filantropia & fi*lan*tro*pi*a \cmark & fi*lan*tro*pi-a \xmark \\
filarmônica & fi*lar-mô*ni*ca \xmark & fi*lar*mô*ni*ca \cmark \\
filarmônico & fi*lar-mô*ni*co \xmark & fi*lar*mô*ni*co \cmark \\
filatelia & fi*la*te*li*a \cmark & fi*la*te*li-a \xmark \\
filáucia & fi*láu*ci.a \xmark & fi*láu*cia \cmark \\
filmografia & fil*mo*gra*fi*a \cmark & fil*mo*gra*fi-a \xmark \\
filogenia & fi*lo*ge*ni*a \cmark & fi*lo*ge*ni-a \xmark \\
filologia & fi*lo*lo*gi*a \cmark & fi*lo*lo*gi-a \xmark \\
filosofia & fi*lo*so*fi*a \cmark & fi*lo*so*fi-a \xmark \\
fímbria & fím*bri.a \xmark & fím*bria \cmark \\
financiável & fi*nan*ci-á*vel \xmark & fi*nan*ci*á*vel \cmark \\
finório & fi*nó*ri.o \xmark & fi*nó*rio \cmark \\
fio & fi*o \cmark & fi-o \xmark \\
fisiocracia & fi*si*o*cra*ci*a \cmark & fi*si*o*cra*ci-a \xmark \\
fisiologia & fi*si*o*lo*gi*a \cmark & fi*si*o*lo*gi-a \xmark \\
fisiólogo & fi*si-ó*lo*go \xmark & fi*si*ó*lo*go \cmark \\
fisionomia & fi*si*o*no*mi*a \cmark & fi*si*o*no*mi-a \xmark \\
fisioterapia & fi*si*o*te*ra*pi*a \cmark & fi*si*o*te*ra*pi-a \xmark \\
fitopatologia & fi*to*pa*to*lo*gi*a \cmark & fi*to*pa*to*lo*gi-a \xmark \\
fitotecnia & fi*to*tec*ni*a \cmark & fi*to*tec*ni-a \xmark \\
fiúza & fi-ú*za \xmark & fi*ú*za \cmark \\
flatulência & fla*tu*lên*ci.a \xmark & fla*tu*lên*cia \cmark \\
floreio & flo*rei*o \cmark & flo*rei-o \xmark \\
florilégio & flo*ri*lé*gi.o \xmark & flo*ri*lé*gio \cmark \\
fluência & flu-ên*ci.a \xmark & flu*ên*cia \cmark \\
fluidez & flu-i*dez \xmark & flu-i*dez \xmark \\
fluídico & flu-í*di*co \xmark & flu*í*di*co \cmark \\
fluir & flu-ir \xmark & flu*ir \cmark \\
fluorescência & flu*o*res*cên*ci.a \xmark & flu*o*res*cên*cia \cmark \\
flúor & flú-or \xmark & flú*or \cmark \\
fobia & fo*bi*a \cmark & fo*bi-a \xmark \\
folheio & fo*lhei*o \cmark & fo*lhei-o \xmark \\
folia & fo*li*a \cmark & fo*li-a \xmark \\
folião & fo*li-ão \xmark & fo*li*ão \cmark \\
fonografia & fo*no*gra*fi*a \cmark & fo*no*gra*fi-a \xmark \\
fonologia & fo*no*lo*gi*a \cmark & fo*no*lo*gi-a \xmark \\
formaldeído & for*mal*de-í*do \xmark & for*mal*de*í*do \cmark \\
formulário & for*mu*lá*ri.o \xmark & for*mu*lá*rio \cmark \\
fosforescência & fos*fo*res*cên*ci.a \xmark & fos*fo*res*cên*cia \cmark \\
fotocópia & fo*to*có*pi.a \xmark & fo*to*có*pia \cmark \\
fotofobia & fo*to*fo*bi*a \cmark & fo*to*fo*bi-a \xmark \\
fotogenia & fo*to*ge*ni*a \cmark & fo*to*ge*ni-a \xmark \\
fotografia & fo*to*gra*fi*a \cmark & fo*to*gra*fi-a \xmark \\
fotolitografia & fo*to*li*to*gra*fi*a \cmark & fo*to*li*to*gra*fi-a \xmark \\
fotometria & fo*to*me*tri*a \cmark & fo*to*me*tri-a \xmark \\
fotoquímica & fo*to-quí*mi*ca \xmark & fo*to*quí*mi*ca \cmark \\
fotoquímico & fo*to-quí*mi*co \xmark & fo*to*quí*mi*co \cmark \\
fototerapia & fo*to*te*ra*pi*a \cmark & fo*to*te*ra*pi-a \xmark \\
fototipia & fo*to*ti*pi*a \cmark & fo*to*ti*pi-a \xmark \\
fragrância & fra*grân*ci.a \xmark & fra*grân*cia \cmark \\
frâncio & frân*ci.o \xmark & frân*cio \cmark \\
francofonia & fran*co*fo*ni*a \cmark & fran*co*fo*ni-a \xmark \\
franquia & fran*qui*a \cmark & fran*qui-a \xmark \\
fraseologia & fra*se*o*lo*gi*a \cmark & fra*se*o*lo*gi-a \xmark \\
fratricídio & fra*tri*cí*di.o \xmark & fra*tri*cí*dio \cmark \\
freático & fre-á*ti*co \xmark & fre*á*ti*co \cmark \\
freguesia & fre*gue*si*a \cmark & fre*gue*si-a \xmark \\
freio & frei*o \cmark & frei-o \xmark \\
freiria & frei*ri*a \cmark & frei*ri-a \xmark \\
frenologia & fre*no*lo*gi*a \cmark & fre*no*lo*gi-a \xmark \\
frequência & fre-quên*ci.a \xmark & fre*quên*cia \cmark \\
fria & fri*a \cmark & fri-a \xmark \\
friável & fri-á*vel \xmark & fri*á*vel \cmark \\
frio & fri*o \cmark & fri-o \xmark \\
frísio & frí*si.o \xmark & frí*sio \cmark \\
frontaria & fron*ta*ri*a \cmark & fron*ta*ri-a \xmark \\
frontispício & fron*tis*pí*ci.o \xmark & fron*tis*pí*cio \cmark \\
fruição & fru-i*ção \xmark & fru*i*ção \cmark \\
fruir & fru-ir \xmark & fru*ir \cmark \\
frutaria & fru*ta*ri*a \cmark & fru*ta*ri-a \xmark \\
fuá & fu-á \xmark & fu*á \cmark \\
fúfio & fú*fi.o \xmark & fú*fio \cmark \\
fugidio & fu*gi*di*o \cmark & fu*gi*di-o \xmark \\
fuinha & fu-i*nha \xmark & fu*i*nha \cmark \\
funcionário & fun*ci*o*ná*ri.o \xmark & fun*ci*o*ná*rio \cmark \\
funerária & fu*ne*rá*ri.a \xmark & fu*ne*rá*ria \cmark \\
funerário & fu*ne*rá*ri.o \xmark & fu*ne*rá*rio \cmark \\
funéreo & fu*né*re.o \xmark & fu*né*reo \cmark \\
funilaria & fu*ni*la*ri*a \cmark & fu*ni*la*ri-a \xmark \\
fúria & fú*ri.a \xmark & fú*ria \cmark \\
futevôlei & fu*te-vô*lei \xmark & fu*te*vô*lei \cmark \\
futurologia & fu*tu*ro*lo*gi*a \cmark & fu*tu*ro*lo*gi-a \xmark \\
fuzilaria & fu*zi*la*ri*a \cmark & fu*zi*la*ri-a \xmark \\
fuzuê & fu*zu-ê \xmark & fu*zu*ê \cmark \\
gabião & ga*bi-ão \xmark & ga*bi*ão \cmark \\
gadolínio & ga*do*lí*ni.o \xmark & ga*do*lí*nio \cmark \\
gaélico & ga-é*li*co \xmark & ga*é*li*co \cmark \\
gaio & gai*o \cmark & gai-o \xmark \\
galanteio & ga*lan*tei*o \cmark & ga*lan*tei-o \xmark \\
galeão & ga*le-ão \xmark & ga*le*ão \cmark \\
galeria & ga*le*ri*a \cmark & ga*le*ri-a \xmark \\
galhardia & ga*lhar*di*a \cmark & ga*lhar*di-a \xmark \\
galináceo & ga*li*ná*ce.o \xmark & ga*li*ná*ceo \cmark \\
gálio & gá*li.o \xmark & gá*lio \cmark \\
gâmbia & gâm*bi.a \xmark & gâm*bia \cmark \\
gamboa & gam*bo*a \cmark & gam*bo-a \xmark \\
ganância & ga*nân*ci.a \xmark & ga*nân*cia \cmark \\
gandaia & gan*dai*a \cmark & gan*dai-a \xmark \\
gânglio & gân*gli.o \xmark & gân*glio \cmark \\
garantia & ga*ran*ti*a \cmark & ga*ran*ti-a \xmark \\
garoa & ga*ro*a \cmark & ga*ro-a \xmark \\
gastrectomia & gas*trec*to*mi*a \cmark & gas*trec*to*mi-a \xmark \\
gastrointestinal & gas*tro-in*tes*ti*nal \xmark & gas*tro*in*tes*ti*nal \cmark \\
gastronomia & gas*tro*no*mi*a \cmark & gas*tro*no*mi-a \xmark \\
gastronômico & gas*tro-nô*mi*co \xmark & gas*tro*nô*mi*co \cmark \\
gaúcho & ga-ú*cho \xmark & ga*ú*cho \cmark \\
gaudério & gau*dé*ri.o \xmark & gau*dé*rio \cmark \\
gáudio & gáu*di.o \xmark & gáu*dio \cmark \\
gávea & gá*ve.a \xmark & gá*vea \cmark \\
gavião & ga*vi-ão \xmark & ga*vi*ão \cmark \\
gazua & ga*zu*a \cmark & ga*zu-a \xmark \\
geio & gei*o \cmark & gei-o \xmark \\
geleia & ge*lei*a \cmark & ge*lei-a \xmark \\
gelosia & ge*lo*si*a \cmark & ge*lo*si-a \xmark \\
gemologia & ge*mo*lo*gi*a \cmark & ge*mo*lo*gi-a \xmark \\
gendarmaria & gen*dar*ma*ri*a \cmark & gen*dar*ma*ri-a \xmark \\
genealogia & ge*ne*a*lo*gi*a \cmark & ge*ne*a*lo*gi-a \xmark \\
genitália & ge*ni*tá*li.a \xmark & ge*ni*tá*lia \cmark \\
geniturinário & ge*ni*tu*ri*ná*ri.o \xmark & ge*ni*tu*ri*ná*rio \cmark \\
genocídio & ge*no*cí*di.o \xmark & ge*no*cí*dio \cmark \\
gentio & gen*ti*o \cmark & gen*ti-o \xmark \\
genuflexório & ge*nu*fle*xó*ri.o \xmark & ge*nu*fle*xó*rio \cmark \\
genuíno & ge*nu-í*no \xmark & ge*nu*í*no \cmark \\
geodesia & ge*o*de*si*a \cmark & ge*o*de*si-a \xmark \\
geografia & ge*o*gra*fi*a \cmark & ge*o*gra*fi-a \xmark \\
geógrafo & ge-ó*gra*fo \xmark & ge*ó*gra*fo \cmark \\
geologia & ge*o*lo*gi*a \cmark & ge*o*lo*gi-a \xmark \\
geólogo & ge-ó*lo*go \xmark & ge*ó*lo*go \cmark \\
geometria & ge*o*me*tri*a \cmark & ge*o*me*tri-a \xmark \\
gerânio & ge*râ*ni.o \xmark & ge*râ*nio \cmark \\
gerência & ge*rên*ci.a \xmark & ge*rên*cia \cmark \\
geriatria & ge*ri*a*tri*a \cmark & ge*ri*a*tri-a \xmark \\
germania & ger*ma*ni*a \cmark & ger*ma*ni-a \xmark \\
germânio & ger*mâ*ni.o \xmark & ger*mâ*nio \cmark \\
gerontocracia & ge*ron*to*cra*ci*a \cmark & ge*ron*to*cra*ci-a \xmark \\
gerúndio & ge*rún*di.o \xmark & ge*rún*dio \cmark \\
gigolô & gi*go-lô \xmark & gi*go*lô \cmark \\
ginásio & gi*ná*si.o \xmark & gi*ná*sio \cmark \\
ginecologia & gi*ne*co*lo*gi*a \cmark & gi*ne*co*lo*gi-a \xmark \\
ginecomastia & gi*ne*co*mas*ti*a \cmark & gi*ne*co*mas*ti-a \xmark \\
gio & gi*o \cmark & gi-o \xmark \\
giratório & gi*ra*tó*ri.o \xmark & gi*ra*tó*rio \cmark \\
gíria & gí*ri.a \xmark & gí*ria \cmark \\
giroscópio & gi*ros*có*pi.o \xmark & gi*ros*có*pio \cmark \\
glaciário & gla*ci-á*ri.o \xmark & gla*ci*á*rio \cmark \\
gládio & glá*di.o \xmark & glá*dio \cmark \\
gladíolo & gla*dí-o*lo \xmark & gla*dí*o*lo \cmark \\
gláucia & gláu*ci.a \xmark & gláu*cia \cmark \\
glia & gli*a \cmark & gli-a \xmark \\
glicemia & gli*ce*mi*a \cmark & gli*ce*mi-a \xmark \\
glicínia & gli*cí*ni.a \xmark & gli*cí*nia \cmark \\
glicoproteína & gli*co*pro*te-í*na \xmark & gli*co*pro*te*í*na \cmark \\
glória & gló*ri.a \xmark & gló*ria \cmark \\
glossário & glos*sá*ri.o \xmark & glos*sá*rio \cmark \\
glossolalia & glos*so*la*li*a \cmark & glos*so*la*li-a \xmark \\
glúteo & glú*te.o \xmark & glú*teo \cmark \\
glutonaria & glu*to*na*ri*a \cmark & glu*to*na*ri-a \xmark \\
gnaisse & g.nais*se \xmark & g.nais*se \xmark \\
gnomo & g.no*mo \xmark & gno*mo \cmark \\
gnose & g.no*se \xmark & gno*se \cmark \\
gnosiologia & g.no*si*o*lo*gi*a \xmark & gno*si*o*lo*gi-a \xmark \\
gnosticismo & g.nos*ti*cis*mo \xmark & gnos*ti*cis*mo \cmark \\
gnóstico & g.nós*ti*co \xmark & gnós*ti*co \cmark \\
gonorreia & go*nor*rei*a \cmark & go*nor*rei-a \xmark \\
gorjeio & gor*jei*o \cmark & gor*jei-o \xmark \\
grafia & gra*fi*a \cmark & gra*fi-a \xmark \\
grafologia & gra*fo*lo*gi*a \cmark & gra*fo*lo*gi-a \xmark \\
grafomaníaco & gra*fo*ma*ní-a*co \xmark & gra*fo*ma*ní*a*co \cmark \\
gramatologia & gra*ma*to*lo*gi*a \cmark & gra*ma*to*lo*gi-a \xmark \\
granulometria & gra*nu*lo*me*tri*a \cmark & gra*nu*lo*me*tri-a \xmark \\
grapiúna & gra*pi-ú*na \xmark & gra*pi*ú*na \cmark \\
graúdo & gra-ú*do \xmark & gra*ú*do \cmark \\
graúna & gra-ú*na \xmark & gra*ú*na \cmark \\
gregário & gre*gá*ri.o \xmark & gre*gá*rio \cmark \\
gregório & gre*gó*ri.o \xmark & gre*gó*rio \cmark \\
gritaria & gri*ta*ri*a \cmark & gri*ta*ri-a \xmark \\
grosseria & gros*se*ri*a \cmark & gros*se*ri-a \xmark \\
grua & gru*a \cmark & gru-a \xmark \\
guardião & guar*di-ão \xmark & guar*di*ão \cmark \\
guia & gui*a \cmark & gui-a \xmark \\
guião & gui-ão \xmark & gui*ão \cmark \\
guria & gu*ri*a \cmark & gu*ri-a \xmark \\
hagiógrafo & ha*gi-ó*gra*fo \xmark & ha*gi*ó*gra*fo \cmark \\
hamburgueria & ham*bur*gue*ri*a \cmark & ham*bur*gue*ri-a \xmark \\
hanseático & han*se-á*ti*co \xmark & han*se*á*ti*co \cmark \\
hanseníase & han*se*ní-a*se \xmark & han*se*ní*a*se \cmark \\
harmonia & har*mo*ni*a \cmark & har*mo*ni-a \xmark \\
harmônica & har-mô*ni*ca \xmark & har*mô*ni*ca \cmark \\
harmônico & har-mô*ni*co \xmark & har*mô*ni*co \cmark \\
harpia & har*pi*a \cmark & har*pi-a \xmark \\
hebdomadário & heb*do*ma*dá*ri.o \xmark & heb*do*ma*dá*rio \cmark \\
hebraísmo & he*bra-ís*mo \xmark & he*bra*ís*mo \cmark \\
hebraísta & he*bra-ís*ta \xmark & he*bra*ís*ta \cmark \\
hegemonia & he*ge*mo*ni*a \cmark & he*ge*mo*ni-a \xmark \\
hegemônico & he*ge-mô*ni*co \xmark & he*ge*mô*ni*co \cmark \\
helíaco & he*lí-a*co \xmark & he*lí*a*co \cmark \\
heliógrafo & he*li-ó*gra*fo \xmark & he*li*ó*gra*fo \cmark \\
hélio & hé*li.o \xmark & hé*lio \cmark \\
helvécio & hel*vé*ci*o \cmark & hel*vé*ci-o \xmark \\
hematologia & he*ma*to*lo*gi*a \cmark & he*ma*to*lo*gi-a \xmark \\
hemeralopia & he*me*ra*lo*pi*a \cmark & he*me*ra*lo*pi-a \xmark \\
hemiparesia & he*mi*pa*re*si*a \cmark & he*mi*pa*re*si-a \xmark \\
hemiplegia & he*mi*ple*gi*a \cmark & he*mi*ple*gi-a \xmark \\
hemisfério & he*mis*fé*ri.o \xmark & he*mis*fé*rio \cmark \\
hemistíquio & he*mis*tí*qui.o \xmark & he*mis*tí*quio \cmark \\
hemodiálise & he*mo*di-á*li*se \xmark & he*mo*di*á*li*se \cmark \\
hemofilia & he*mo*fi*li*a \cmark & he*mo*fi*li-a \xmark \\
hemorragia & he*mor*ra*gi*a \cmark & he*mor*ra*gi-a \xmark \\
hemostasia & he*mos*ta*si*a \cmark & he*mos*ta*si-a \xmark \\
hemoterapia & he*mo*te*ra*pi*a \cmark & he*mo*te*ra*pi-a \xmark \\
hepatologia & he*pa*to*lo*gi*a \cmark & he*pa*to*lo*gi-a \xmark \\
hepatomegalia & he*pa*to*me*ga*li*a \cmark & he*pa*to*me*ga*li-a \xmark \\
hepatopatia & he*pa*to*pa*ti*a \cmark & he*pa*to*pa*ti-a \xmark \\
heptacampeão & hep*ta*cam*pe-ão \xmark & hep*ta*cam*pe*ão \cmark \\
herbáceo & her*bá*ce.o \xmark & her*bá*ceo \cmark \\
hercúleo & her*cú*le.o \xmark & her*cú*leo \cmark \\
hereditário & he*re*di*tá*ri.o \xmark & he*re*di*tá*rio \cmark \\
heresia & he*re*si*a \cmark & he*re*si-a \xmark \\
hermínio & her*mí*ni.o \xmark & her*mí*nio \cmark \\
hérnia & hér*ni.a \xmark & hér*nia \cmark \\
heroína & he*ro-í*na \xmark & he*ro*í*na \cmark \\
heroísmo & he*ro-ís*mo \xmark & he*ro*ís*mo \cmark \\
herpetologia & her*pe*to*lo*gi*a \cmark & her*pe*to*lo*gi-a \xmark \\
heteria & he*te*ri*a \cmark & he*te*ri-a \xmark \\
heteroátomo & he*te*ro-á*to*mo \xmark & he*te*ro*á*to*mo \cmark \\
heterocromia & he*te*ro*cro*mi*a \cmark & he*te*ro*cro*mi-a \xmark \\
heterodoxia & he*te*ro*do*xi*a \cmark & he*te*ro*do*xi-a \xmark \\
heterofilia & he*te*ro*fi*li*a \cmark & he*te*ro*fi*li-a \xmark \\
heterologia & he*te*ro*lo*gi*a \cmark & he*te*ro*lo*gi-a \xmark \\
heteronímia & he*te*ro*ní*mi.a \xmark & he*te*ro*ní*mia \cmark \\
heterônimo & he*te-rô*ni*mo \xmark & he*te*rô*ni*mo \cmark \\
heteronomia & he*te*ro*no*mi*a \cmark & he*te*ro*no*mi-a \xmark \\
hexacampeão & he*xa*cam*pe-ão \xmark & he*xa*cam*pe*ão \cmark \\
hidroavião & hi*dro*a*vi-ão \xmark & hi*dro*a*vi*ão \cmark \\
hidrocefalia & hi*dro*ce*fa*li*a \cmark & hi*dro*ce*fa*li-a \xmark \\
hidrofobia & hi*dro*fo*bi*a \cmark & hi*dro*fo*bi-a \xmark \\
hidrologia & hi*dro*lo*gi*a \cmark & hi*dro*lo*gi-a \xmark \\
hidromancia & hi*dro*man*ci*a \cmark & hi*dro*man*ci-a \xmark \\
hidropisia & hi*dro*pi*si*a \cmark & hi*dro*pi*si-a \xmark \\
hidropônico & hi*dro-pô*ni*co \xmark & hi*dro*pô*ni*co \cmark \\
hidroterapia & hi*dro*te*ra*pi*a \cmark & hi*dro*te*ra*pi-a \xmark \\
hidrovia & hi*dro*vi*a \cmark & hi*dro*vi-a \xmark \\
hidroviário & hi*dro*vi-á*ri.o \xmark & hi*dro*vi*á*rio \cmark \\
hierarquia & hi*e*rar*qui*a \cmark & hi*e*rar*qui-a \xmark \\
hilária & hi*lá*ri.a \xmark & hi*lá*ria \cmark \\
hilário & hi*lá*ri.o \xmark & hi*lá*rio \cmark \\
hinário & hi*ná*ri.o \xmark & hi*ná*rio \cmark \\
hinduísmo & hin*du-ís*mo \xmark & hin*du*ís*mo \cmark \\
hinologia & hi*no*lo*gi*a \cmark & hi*no*lo*gi-a \xmark \\
hiperalgesia & hi*pe*ral*ge*si*a \cmark & hi*pe*ral*ge*si-a \xmark \\
hiperbóreo & hi*per*bó*re.o \xmark & hi*per*bó*reo \cmark \\
hipercalcemia & hi*per*cal*ce*mi*a \cmark & hi*per*cal*ce*mi-a \xmark \\
hiperdulia & hi*per*du*li*a \cmark & hi*per*du*li-a \xmark \\
hiperemia & hi*pe*re*mi*a \cmark & hi*pe*re*mi-a \xmark \\
hipertermia & hi*per*ter*mi*a \cmark & hi*per*ter*mi-a \xmark \\
hipertonia & hi*per*to*ni*a \cmark & hi*per*to*ni-a \xmark \\
hipertrofia & hi*per*tro*fi*a \cmark & hi*per*tro*fi-a \xmark \\
hipocalcemia & hi*po*cal*ce*mi*a \cmark & hi*po*cal*ce*mi-a \xmark \\
hipocalemia & hi*po*ca*le*mi*a \cmark & hi*po*ca*le*mi-a \xmark \\
hipocondríaco & hi*po*con*drí-a*co \xmark & hi*po*con*drí*a*co \cmark \\
hipocondria & hi*po*con*dri*a \cmark & hi*po*con*dri-a \xmark \\
hipocrisia & hi*po*cri*si*a \cmark & hi*po*cri*si-a \xmark \\
hipoglicemia & hi*po*gli*ce*mi*a \cmark & hi*po*gli*ce*mi-a \xmark \\
hipônimo & hi-pô*ni*mo \xmark & hi*pô*ni*mo \cmark \\
hipoplasia & hi*po*pla*si*a \cmark & hi*po*pla*si-a \xmark \\
hipotermia & hi*po*ter*mi*a \cmark & hi*po*ter*mi-a \xmark \\
hipotonia & hi*po*to*ni*a \cmark & hi*po*to*ni-a \xmark \\
hipotrofia & hi*po*tro*fi*a \cmark & hi*po*tro*fi-a \xmark \\
hipoxemia & hi*po*xe*mi*a \cmark & hi*po*xe*mi-a \xmark \\
histerectomia & his*te*rec*to*mi*a \cmark & his*te*rec*to*mi-a \xmark \\
histeria & his*te*ri*a \cmark & his*te*ri-a \xmark \\
histeroscopia & his*te*ros*co*pi*a \cmark & his*te*ros*co*pi-a \xmark \\
histologia & his*to*lo*gi*a \cmark & his*to*lo*gi-a \xmark \\
histopatologia & his*to*pa*to*lo*gi*a \cmark & his*to*pa*to*lo*gi-a \xmark \\
histoquímica & his*to-quí*mi*ca \xmark & his*to*quí*mi*ca \cmark \\
história & his*tó*ri.a \xmark & his*tó*ria \cmark \\
historiografia & his*to*ri*o*gra*fi*a \cmark & his*to*ri*o*gra*fi-a \xmark \\
hólmio & hól*mi.o \xmark & hól*mio \cmark \\
holografia & ho*lo*gra*fi*a \cmark & ho*lo*gra*fi-a \xmark \\
homeopatia & ho*me*o*pa*ti*a \cmark & ho*me*o*pa*ti-a \xmark \\
homeostasia & ho*me*os*ta*si*a \cmark & ho*me*os*ta*si-a \xmark \\
homicídio & ho*mi*cí*di.o \xmark & ho*mi*cí*dio \cmark \\
homilia & ho*mi*li*a \cmark & ho*mi*li-a \xmark \\
hominídeo & ho*mi*ní*de.o \xmark & ho*mi*ní*deo \cmark \\
homofilia & ho*mo*fi*li*a \cmark & ho*mo*fi*li-a \xmark \\
homofobia & ho*mo*fo*bi*a \cmark & ho*mo*fo*bi-a \xmark \\
homofonia & ho*mo*fo*ni*a \cmark & ho*mo*fo*ni-a \xmark \\
homografia & ho*mo*gra*fi*a \cmark & ho*mo*gra*fi-a \xmark \\
homologia & ho*mo*lo*gi*a \cmark & ho*mo*lo*gi-a \xmark \\
homonímia & ho*mo*ní*mi.a \xmark & ho*mo*ní*mia \cmark \\
homônimo & ho-mô*ni*mo \xmark & ho*mô*ni*mo \cmark \\
homoplasia & ho*mo*pla*si*a \cmark & ho*mo*pla*si-a \xmark \\
homotetia & ho*mo*te*ti*a \cmark & ho*mo*te*ti-a \xmark \\
horário & ho*rá*ri.o \xmark & ho*rá*rio \cmark \\
hortênsia & hor*tên*si.a \xmark & hor*tên*sia \cmark \\
hospedaria & hos*pe*da*ri*a \cmark & hos*pe*da*ri-a \xmark \\
hospício & hos*pí*ci.o \xmark & hos*pí*cio \cmark \\
hóstia & hós*ti.a \xmark & hós*tia \cmark \\
hotelaria & ho*te*la*ri*a \cmark & ho*te*la*ri-a \xmark \\
humanitário & hu*ma*ni*tá*ri.o \xmark & hu*ma*ni*tá*rio \cmark \\
iconoclastia & i*co*no*clas*ti*a \cmark & i*co*no*clas*ti-a \xmark \\
iconografia & i*co*no*gra*fi*a \cmark & i*co*no*gra*fi-a \xmark \\
iconolatria & i*co*no*la*tri*a \cmark & i*co*no*la*tri-a \xmark \\
iconologia & i*co*no*lo*gi*a \cmark & i*co*no*lo*gi-a \xmark \\
icterícia & ic*te*rí*ci.a \xmark & ic*te*rí*cia \cmark \\
ictiologia & ic*ti*o*lo*gi*a \cmark & ic*ti*o*lo*gi-a \xmark \\
ictiólogo & ic*ti-ó*lo*go \xmark & ic*ti*ó*lo*go \cmark \\
ideário & i*de-á*ri.o \xmark & i*de*á*rio \cmark \\
ideia & i*dei*a \cmark & i*dei-a \xmark \\
identitário & i*den*ti*tá*ri.o \xmark & i*den*ti*tá*rio \cmark \\
ideologia & i*de*o*lo*gi*a \cmark & i*de*o*lo*gi-a \xmark \\
ideólogo & i*de-ó*lo*go \xmark & i*de*ó*lo*go \cmark \\
idílio & i*dí*li.o \xmark & i*dí*lio \cmark \\
idiossincrasia & i*di*os*sin*cra*si*a \cmark & i*di*os*sin*cra*si-a \xmark \\
idiotia & i*di*o*ti*a \cmark & i*di*o*ti-a \xmark \\
idolatria & i*do*la*tri*a \cmark & i*do*la*tri-a \xmark \\
ígneo & íg*ne.o \xmark & íg*neo \cmark \\
ignorância & ig*no*rân*ci.a \xmark & ig*no*rân*cia \cmark \\
igualitário & i*gua*li*tá*ri.o \xmark & i*gua*li*tá*rio \cmark \\
iídiche & i-í*di*che \xmark & i*í*di*che \cmark \\
íleo & í*le.o \xmark & í*leo \cmark \\
ilhoa & i*lho*a \cmark & i*lho-a \xmark \\
ilíaco & i*lí-a*co \xmark & i*lí*a*co \cmark \\
ilíada & i*lí-a*da \xmark & i*lí*a*da \cmark \\
iliteracia & i*li*te*ra*ci*a \cmark & i*li*te*ra*ci-a \xmark \\
ilusório & i*lu*só*ri.o \xmark & i*lu*só*rio \cmark \\
imaginária & i*ma*gi*ná*ri.a \xmark & i*ma*gi*ná*ria \cmark \\
imaginário & i*ma*gi*ná*ri.o \xmark & i*ma*gi*ná*rio \cmark \\
imbróglio & im*bró*gli.o \xmark & im*bró*glio \cmark \\
imbuído & im*bu-í*do \xmark & im*bu*í*do \cmark \\
imbuir & im*bu-ir \xmark & im*bu*ir \cmark \\
iminência & i*mi*nên*ci.a \xmark & i*mi*nên*cia \cmark \\
imobiliária & i*mo*bi*li-á*ri.a \xmark & i*mo*bi*li*á*ria \cmark \\
imobiliário & i*mo*bi*li-á*ri.o \xmark & i*mo*bi*li*á*rio \cmark \\
imodéstia & i*mo*dés*ti.a \xmark & i*mo*dés*tia \cmark \\
impaciência & im*pa*ci*ên*ci.a \xmark & im*pa*ci*ên*cia \cmark \\
impenitência & im*pe*ni*tên*ci.a \xmark & im*pe*ni*tên*cia \cmark \\
imperdoável & im*per*do-á*vel \xmark & im*per*do*á*vel \cmark \\
império & im*pé*ri.o \xmark & im*pé*rio \cmark \\
impertinência & im*per*ti*nên*ci.a \xmark & im*per*ti*nên*cia \cmark \\
ímpio & ím*pi.o \xmark & ím*pio \cmark \\
implicância & im*pli*cân*ci.a \xmark & im*pli*cân*cia \cmark \\
importância & im*por*tân*ci.a \xmark & im*por*tân*cia \cmark \\
impotência & im*po*tên*ci.a \xmark & im*po*tên*cia \cmark \\
imprevidência & im*pre*vi*dên*ci.a \xmark & im*pre*vi*dên*cia \cmark \\
impronunciável & im*pro*nun*ci-á*vel \xmark & im*pro*nun*ci*á*vel \cmark \\
impropério & im*pro*pé*ri.o \xmark & im*pro*pé*rio \cmark \\
impróprio & im*pró*pri.o \xmark & im*pró*prio \cmark \\
imprudência & im*pru*dên*ci.a \xmark & im*pru*dên*cia \cmark \\
imunologia & i*mu*no*lo*gi*a \cmark & i*mu*no*lo*gi-a \xmark \\
imunoterapia & i*mu*no*te*ra*pi*a \cmark & i*mu*no*te*ra*pi-a \xmark \\
inadiável & i*na*di-á*vel \xmark & i*na*di*á*vel \cmark \\
inapetência & i*na*pe*tên*ci.a \xmark & i*na*pe*tên*cia \cmark \\
incandescência & in*can*des*cên*ci.a \xmark & in*can*des*cên*cia \cmark \\
incendiário & in*cen*di-á*ri.o \xmark & in*cen*di*á*rio \cmark \\
incêndio & in*cên*di.o \xmark & in*cên*dio \cmark \\
incensário & in*cen*sá*ri.o \xmark & in*cen*sá*rio \cmark \\
incidência & in*ci*dên*ci.a \xmark & in*ci*dên*cia \cmark \\
inclemência & in*cle*mên*ci.a \xmark & in*cle*mên*cia \cmark \\
incluir & in*clu-ir \xmark & in*clu*ir \cmark \\
incômodo & in-cô*mo*do \xmark & in*cô*mo*do \cmark \\
inconciliável & in*con*ci*li-á*vel \xmark & in*con*ci*li*á*vel \cmark \\
inconfidência & in*con*fi*dên*ci.a \xmark & in*con*fi*dên*cia \cmark \\
incongruência & in*con*gru-ên*ci.a \xmark & in*con*gru*ên*cia \cmark \\
inconsciência & in*cons*ci*ên*ci.a \xmark & in*cons*ci*ên*cia \cmark \\
inconsequência & in*con*se-quên*ci.a \xmark & in*con*se*quên*cia \cmark \\
inconspícuo & in*cons*pí*cu.o \xmark & in*cons*pí*cuo \cmark \\
inconstância & in*cons*tân*ci.a \xmark & in*cons*tân*cia \cmark \\
incontinência & in*con*ti*nên*ci.a \xmark & in*con*ti*nên*cia \cmark \\
inconveniência & in*con*ve*ni*ên*ci.a \xmark & in*con*ve*ni*ên*cia \cmark \\
incumbência & in*cum*bên*ci.a \xmark & in*cum*bên*cia \cmark \\
incúria & in*cú*ri.a \xmark & in*cú*ria \cmark \\
indaiá & in*dai-á \xmark & in*dai*á \cmark \\
indeiscente & in*de-is*cen*te \xmark & in*de-is*cen*te \xmark \\
independência & in*de*pen*dên*ci.a \xmark & in*de*pen*dên*cia \cmark \\
indício & in*dí*ci.o \xmark & in*dí*cio \cmark \\
indigência & in*di*gên*ci.a \xmark & in*di*gên*cia \cmark \\
índio & ín*di.o \xmark & ín*dio \cmark \\
indolência & in*do*lên*ci.a \xmark & in*do*lên*cia \cmark \\
indonésio & in*do*né*si.o \xmark & in*do*né*sio \cmark \\
indumentária & in*du*men*tá*ri.a \xmark & in*du*men*tá*ria \cmark \\
indústria & in*dús*tri.a \xmark & in*dús*tria \cmark \\
inédia & i*né*di.a \xmark & i*né*dia \cmark \\
ineficiência & i*ne*fi*ci*ên*ci.a \xmark & i*ne*fi*ci*ên*cia \cmark \\
inegociável & i*ne*go*ci-á*vel \xmark & i*ne*go*ci*á*vel \cmark \\
inequívoco & i*ne-quí*vo*co \xmark & i*ne*quí*vo*co \cmark \\
inércia & i*nér*ci.a \xmark & i*nér*cia \cmark \\
inexequível & i*ne*xe-quí*vel \xmark & i*ne*xe*quí*vel \cmark \\
inexistência & i*ne*xis*tên*ci.a \xmark & i*ne*xis*tên*cia \cmark \\
inexperiência & i*nex*pe*ri*ên*ci.a \xmark & i*nex*pe*ri*ên*cia \cmark \\
infâmia & in*fâ*mi.a \xmark & in*fâ*mia \cmark \\
infância & in*fân*ci.a \xmark & in*fân*cia \cmark \\
infantaria & in*fan*ta*ri*a \cmark & in*fan*ta*ri-a \xmark \\
infanticídio & in*fan*ti*cí*di.o \xmark & in*fan*ti*cí*dio \cmark \\
inferência & in*fe*rên*ci.a \xmark & in*fe*rên*cia \cmark \\
inflacionário & in*fla*ci*o*ná*ri.o \xmark & in*fla*ci*o*ná*rio \cmark \\
inflamatório & in*fla*ma*tó*ri.o \xmark & in*fla*ma*tó*rio \cmark \\
inflorescência & in*flo*res*cên*ci.a \xmark & in*flo*res*cên*cia \cmark \\
influência & in*flu-ên*ci.a \xmark & in*flu*ên*cia \cmark \\
influenciável & in*flu*en*ci-á*vel \xmark & in*flu*en*ci*á*vel \cmark \\
influído & in*flu-í*do \xmark & in*flu*í*do \cmark \\
influir & in*flu-ir \xmark & in*flu*ir \cmark \\
infografia & in*fo*gra*fi*a \cmark & in*fo*gra*fi-a \xmark \\
infortúnio & in*for*tú*ni.o \xmark & in*for*tú*nio \cmark \\
infrutescência & in*fru*tes*cên*ci.a \xmark & in*fru*tes*cên*cia \cmark \\
ingenuidade & in*ge*nu-i*da*de \xmark & in*ge*nu*i*da*de \cmark \\
ingerência & in*ge*rên*ci.a \xmark & in*ge*rên*cia \cmark \\
iniciático & i*ni*ci-á*ti*co \xmark & i*ni*ci*á*ti*co \cmark \\
início & i*ní*ci.o \xmark & i*ní*cio \cmark \\
injúria & in*jú*ri.a \xmark & in*jú*ria \cmark \\
inocência & i*no*cên*ci.a \xmark & i*no*cên*cia \cmark \\
inópia & i*nó*pi.a \xmark & i*nó*pia \cmark \\
insaciável & in*sa*ci-á*vel \xmark & in*sa*ci*á*vel \cmark \\
insídia & in*sí*di*a \cmark & in*sí*di-a \xmark \\
insígnia & in*síg*ni.a \xmark & in*síg*nia \cmark \\
insignificância & in*sig*ni*fi*cân*ci.a \xmark & in*sig*ni*fi*cân*cia \cmark \\
insistência & in*sis*tên*ci.a \xmark & in*sis*tên*cia \cmark \\
insolência & in*so*lên*ci.a \xmark & in*so*lên*cia \cmark \\
insolvência & in*sol*vên*ci.a \xmark & in*sol*vên*cia \cmark \\
inspiratório & ins*pi*ra*tó*ri*o \cmark & ins*pi*ra*tó*ri-o \xmark \\
instância & ins*tân*ci.a \xmark & ins*tân*cia \cmark \\
instantâneo & ins*tan*tâ*ne*o \cmark & ins*tan*tâ*ne-o \xmark \\
instituído & ins*ti*tu-í*do \xmark & ins*ti*tu*í*do \cmark \\
instituir & ins*ti*tu-ir \xmark & ins*ti*tu*ir \cmark \\
instruído & ins*tru-í*do \xmark & ins*tru*í*do \cmark \\
instruir & ins*tru-ir \xmark & ins*tru*ir \cmark \\
insubstituível & in*subs*ti*tu-í*vel \xmark & in*subs*ti*tu*í*vel \cmark \\
insuficiência & in*su*fi*ci*ên*ci.a \xmark & in*su*fi*ci*ên*cia \cmark \\
inteligência & in*te*li*gên*ci.a \xmark & in*te*li*gên*cia \cmark \\
interbancário & in*ter*ban*cá*ri.o \xmark & in*ter*ban*cá*rio \cmark \\
intercambiável & in*ter*cam*bi-á*vel \xmark & in*ter*cam*bi*á*vel \cmark \\
intercâmbio & in*ter*câm*bi.o \xmark & in*ter*câm*bio \cmark \\
interdependência & in*ter*de*pen*dên*ci.a \xmark & in*ter*de*pen*dên*cia \cmark \\
interferência & in*ter*fe*rên*ci.a \xmark & in*ter*fe*rên*cia \cmark \\
interflúvio & in*ter*flú*vi.o \xmark & in*ter*flú*vio \cmark \\
interlúdio & in*ter*lú*di.o \xmark & in*ter*lú*dio \cmark \\
intermediário & in*ter*me*di-á*ri.o \xmark & in*ter*me*di*á*rio \cmark \\
intermédio & in*ter*mé*di.o \xmark & in*ter*mé*dio \cmark \\
intermitência & in*ter*mi*tên*ci.a \xmark & in*ter*mi*tên*cia \cmark \\
interoceânico & in*te*ro*ce-â*ni*co \xmark & in*te*ro*ce*â*ni*co \cmark \\
interplanetário & in*ter*pla*ne*tá*ri.o \xmark & in*ter*pla*ne*tá*rio \cmark \\
interrogatório & in*ter*ro*ga*tó*ri.o \xmark & in*ter*ro*ga*tó*rio \cmark \\
interstício & in*ters*tí*ci.o \xmark & in*ters*tí*cio \cmark \\
interventoria & in*ter*ven*to*ri*a \cmark & in*ter*ven*to*ri-a \xmark \\
intolerância & in*to*le*rân*ci.a \xmark & in*to*le*rân*cia \cmark \\
intransigência & in*tran*si*gên*ci.a \xmark & in*tran*si*gên*cia \cmark \\
intuição & in*tu-i*ção \xmark & in*tu*i*ção \cmark \\
intuir & in*tu-ir \xmark & in*tu*ir \cmark \\
intuitivo & in*tu-i*ti*vo \xmark & in*tu-i*ti*vo \xmark \\
invariância & in*va*ri*ân*ci.a \xmark & in*va*ri*ân*cia \cmark \\
invariável & in*va*ri-á*vel \xmark & in*va*ri*á*vel \cmark \\
inventário & in*ven*tá*ri.o \xmark & in*ven*tá*rio \cmark \\
inviável & in*vi-á*vel \xmark & in*vi*á*vel \cmark \\
involuntário & in*vo*lun*tá*ri.o \xmark & in*vo*lun*tá*rio \cmark \\
iódico & i-ó*di*co \xmark & i*ó*di*co \cmark \\
irídio & i*rí*di.o \xmark & i*rí*dio \cmark \\
ironia & i*ro*ni*a \cmark & i*ro*ni-a \xmark \\
irônico & i-rô*ni*co \xmark & i*rô*ni*co \cmark \\
iroquês & i*ro-quês \xmark & i*ro*quês \cmark \\
irreconciliável & ir*re*con*ci*li-á*vel \xmark & ir*re*con*ci*li*á*vel \cmark \\
irrefreável & ir*re*fre-á*vel \xmark & ir*re*fre*á*vel \cmark \\
irremediável & ir*re*me*di-á*vel \xmark & ir*re*me*di*á*vel \cmark \\
irrenunciável & ir*re*nun*ci-á*vel \xmark & ir*re*nun*ci*á*vel \cmark \\
irretorquível & ir*re*tor-quí*vel \xmark & ir*re*tor*quí*vel \cmark \\
irreverência & ir*re*ve*rên*ci.a \xmark & ir*re*ve*rên*cia \cmark \\
irrisório & ir*ri*só*ri.o \xmark & ir*ri*só*rio \cmark \\
isogamia & i*so*ga*mi*a \cmark & i*so*ga*mi-a \xmark \\
isomeria & i*so*me*ri*a \cmark & i*so*me*ri-a \xmark \\
isometria & i*so*me*tri*a \cmark & i*so*me*tri-a \xmark \\
isomorfia & i*so*mor*fi*a \cmark & i*so*mor*fi-a \xmark \\
isonomia & i*so*no*mi*a \cmark & i*so*no*mi-a \xmark \\
isostasia & i*sos*ta*si*a \cmark & i*sos*ta*si-a \xmark \\
isotônico & i*so-tô*ni*co \xmark & i*so*tô*ni*co \cmark \\
isquemia & is*que*mi*a \cmark & is*que*mi-a \xmark \\
itérbio & i*tér*bi.o \xmark & i*tér*bio \cmark \\
itinerário & i*ti*ne*rá*ri.o \xmark & i*ti*ne*rá*rio \cmark \\
ítrio & í*tri.o \xmark & í*trio \cmark \\
jabô & ja-bô \xmark & ja*bô \cmark \\
jactância & jac*tân*ci.a \xmark & jac*tân*cia \cmark \\
jaculatória & ja*cu*la*tó*ri.a \xmark & ja*cu*la*tó*ria \cmark \\
jandaia & jan*dai*a \cmark & jan*dai-a \xmark \\
jaó & ja-ó \xmark & ja*ó \cmark \\
jaracatiá & ja*ra*ca*ti-á \xmark & ja*ra*ca*ti*á \cmark \\
jaú & ja-ú \xmark & ja*ú \cmark \\
jejunostomia & je*ju*nos*to*mi*a \cmark & je*ju*nos*to*mi-a \xmark \\
jesuíta & je*su-í*ta \xmark & je*su*í*ta \cmark \\
jesuítico & je*su-í*ti*co \xmark & je*su*í*ti*co \cmark \\
jesuitismo & je*su-i*tis*mo \xmark & je*su*i*tis*mo \cmark \\
jia & ji*a \cmark & ji-a \xmark \\
joalheria & jo*a*lhe*ri*a \cmark & jo*a*lhe*ri-a \xmark \\
joia & joi*a \cmark & joi-a \xmark \\
joinvilense & jo-in*vi*len*se \xmark & jo*in*vi*len*se \cmark \\
joio & joi*o \cmark & joi-o \xmark \\
juá & ju-á \xmark & ju*á \cmark \\
judaísmo & ju*da-ís*mo \xmark & ju*da*ís*mo \cmark \\
judia & ju*di*a \cmark & ju*di-a \xmark \\
judiaria & ju*di*a*ri*a \cmark & ju*di*a*ri-a \xmark \\
judiciário & ju*di*ci-á*ri.o \xmark & ju*di*ci*á*rio \cmark \\
judô & ju-dô \xmark & ju*dô \cmark \\
juizado & ju-i*za*do \xmark & ju*i*za*do \cmark \\
juíza & ju-í*za \xmark & ju*í*za \cmark \\
juiz & ju-iz \xmark & ju*iz \cmark \\
juízo & ju-í*zo \xmark & ju*í*zo \cmark \\
juquiá & ju*qui-á \xmark & ju*qui*á \cmark \\
kuwaitiano & ku-wai*ti*a*no \xmark & ku-wai*ti*a*no \xmark \\
lábia & lá*bi.a \xmark & lá*bia \cmark \\
laboratório & la*bo*ra*tó*ri.o \xmark & la*bo*ra*tó*rio \cmark \\
lacaio & la*cai*o \cmark & la*cai-o \xmark \\
laçaria & la*ça*ri*a \cmark & la*ça*ri-a \xmark \\
lacônico & la-cô*ni*co \xmark & la*cô*ni*co \cmark \\
ladainha & la*da-i*nha \xmark & la*da*i*nha \cmark \\
lagoa & la*go*a \cmark & la*go-a \xmark \\
laia & lai*a \cmark & lai-a \xmark \\
lamaísmo & la*ma-ís*mo \xmark & la*ma*ís*mo \cmark \\
lâmia & lâ*mi.a \xmark & lâ*mia \cmark \\
lampião & lam*pi-ão \xmark & lam*pi*ão \cmark \\
lampreia & lam*prei*a \cmark & lam*prei-a \xmark \\
lamúria & la*mú*ri.a \xmark & la*mú*ria \cmark \\
lanifício & la*ni*fí*ci.o \xmark & la*ni*fí*cio \cmark \\
laparoscopia & la*pa*ros*co*pi*a \cmark & la*pa*ros*co*pi-a \xmark \\
laparotomia & la*pa*ro*to*mi*a \cmark & la*pa*ro*to*mi-a \xmark \\
laquê & la-quê \xmark & la*quê \cmark \\
larápio & la*rá*pi.o \xmark & la*rá*pio \cmark \\
laríngeo & la*rín*ge.o \xmark & la*rín*geo \cmark \\
lascívia & las*cí*vi.a \xmark & las*cí*via \cmark \\
latifúndio & la*ti*fún*di.o \xmark & la*ti*fún*dio \cmark \\
latinório & la*ti*nó*ri.o \xmark & la*ti*nó*rio \cmark \\
latrocínio & la*tro*cí*ni.o \xmark & la*tro*cí*nio \cmark \\
laudatório & lau*da*tó*ri.o \xmark & lau*da*tó*rio \cmark \\
laurêncio & lau*rên*ci.o \xmark & lau*rên*cio \cmark \\
lavatório & la*va*tó*ri.o \xmark & la*va*tó*rio \cmark \\
lavradio & la*vra*di*o \cmark & la*vra*di-o \xmark \\
leão & le-ão \xmark & le*ão \cmark \\
legendário & le*gen*dá*ri.o \xmark & le*gen*dá*rio \cmark \\
legião & le*gi-ão \xmark & le*gi*ão \cmark \\
legionário & le*gi*o*ná*ri.o \xmark & le*gi*o*ná*rio \cmark \\
leitaria & lei*ta*ri*a \cmark & lei*ta*ri-a \xmark \\
leitoa & lei*to*a \cmark & lei*to-a \xmark \\
lendário & len*dá*ri.o \xmark & len*dá*rio \cmark \\
leniência & le*ni*ên*ci.a \xmark & le*ni*ên*cia \cmark \\
lenocínio & le*no*cí*ni.o \xmark & le*no*cí*nio \cmark \\
leoa & le*o*a \cmark & le*o-a \xmark \\
lepidopterologia & le*pi*dop*te*ro*lo*gi*a \cmark & le*pi*dop*te*ro*lo*gi-a \xmark \\
leporídeo & le*po*rí*de.o \xmark & le*po*rí*deo \cmark \\
lésbio & lés*bi.o \xmark & lés*bio \cmark \\
letargia & le*tar*gi*a \cmark & le*tar*gi-a \xmark \\
letícia & le*tí*ci.a \xmark & le*tí*cia \cmark \\
leucemia & leu*ce*mi*a \cmark & leu*ce*mi-a \xmark \\
leucopenia & leu*co*pe*ni*a \cmark & leu*co*pe*ni-a \xmark \\
leucoplasia & leu*co*pla*si*a \cmark & leu*co*pla*si-a \xmark \\
leucotomia & leu*co*to*mi*a \cmark & leu*co*to*mi-a \xmark \\
lexicografia & le*xi*co*gra*fi*a \cmark & le*xi*co*gra*fi-a \xmark \\
lexicologia & le*xi*co*lo*gi*a \cmark & le*xi*co*lo*gi-a \xmark \\
lezíria & le*zí*ri.a \xmark & le*zí*ria \cmark \\
lia & li*a \cmark & li-a \xmark \\
libertário & li*ber*tá*ri.o \xmark & li*ber*tá*rio \cmark \\
líbio & lí*bi.o \xmark & lí*bio \cmark \\
licantropia & li*can*tro*pi*a \cmark & li*can*tro*pi-a \xmark \\
lício & lí*ci.o \xmark & lí*cio \cmark \\
limnologia & lim*no*lo*gi*a \cmark & lim*no*lo*gi-a \xmark \\
linóleo & li*nó*le*o \cmark & li*nó*le-o \xmark \\
lírio & lí*ri*o \cmark & lí*ri-o \xmark \\
lisboa & lis*bo*a \cmark & lis*bo-a \xmark \\
lisimáquia & li*si*má*qui*a \cmark & li*si*má*qui-a \xmark \\
litania & li*ta*ni*a \cmark & li*ta*ni-a \xmark \\
litíase & li*tí-a*se \xmark & li*tí*a*se \cmark \\
litígio & li*tí*gi.o \xmark & li*tí*gio \cmark \\
lítio & lí*ti.o \xmark & lí*tio \cmark \\
litispendência & li*tis*pen*dên*ci*a \cmark & li*tis*pen*dên*ci-a \xmark \\
litografia & li*to*gra*fi*a \cmark & li*to*gra*fi-a \xmark \\
litologia & li*to*lo*gi*a \cmark & li*to*lo*gi-a \xmark \\
litotripsia & li*to*trip*si*a \cmark & li*to*trip*si-a \xmark \\
liturgia & li*tur*gi*a \cmark & li*tur*gi-a \xmark \\
livraria & li*vra*ri*a \cmark & li*vra*ri-a \xmark \\
lixívia & li*xí*vi.a \xmark & li*xí*via \cmark \\
loa & lo*a \cmark & lo-a \xmark \\
lobectomia & lo*bec*to*mi*a \cmark & lo*bec*to*mi-a \xmark \\
lobélia & lo*bé*li*a \cmark & lo*bé*li-a \xmark \\
locatário & lo*ca*tá*ri.o \xmark & lo*ca*tá*rio \cmark \\
logosofia & lo*go*so*fi*a \cmark & lo*go*so*fi-a \xmark \\
lombalgia & lom*bal*gi*a \cmark & lom*bal*gi-a \xmark \\
loteria & lo*te*ri*a \cmark & lo*te*ri-a \xmark \\
lua & lu*a \cmark & lu-a \xmark \\
ludoterapia & lu*do*te*ra*pi*a \cmark & lu*do*te*ra*pi-a \xmark \\
luís & lu-ís \xmark & lu*ís \cmark \\
luminescência & lu*mi*nes*cên*ci.a \xmark & lu*mi*nes*cên*cia \cmark \\
lunário & lu*ná*ri.o \xmark & lu*ná*rio \cmark \\
lusíada & lu*sí-a*da \xmark & lu*sí*a*da \cmark \\
lusofobia & lu*so*fo*bi*a \cmark & lu*so*fo*bi-a \xmark \\
lusofonia & lu*so*fo*ni*a \cmark & lu*so*fo*ni-a \xmark \\
lutécio & lu*té*ci.o \xmark & lu*té*cio \cmark \\
luteína & lu*te-í*na \xmark & lu*te*í*na \cmark \\
luteria & lu*te*ri*a \cmark & lu*te*ri-a \xmark \\
luxúria & lu*xú*ri.a \xmark & lu*xú*ria \cmark \\
macadâmia & ma*ca*dâ*mi.a \xmark & ma*ca*dâ*mia \cmark \\
macambúzio & ma*cam*bú*zi.o \xmark & ma*cam*bú*zio \cmark \\
macaúba & ma*ca-ú*ba \xmark & ma*ca*ú*ba \cmark \\
macedônia & ma*ce-dô*ni.a \xmark & ma*ce*dô*nia \cmark \\
maceió & ma*cei-ó \xmark & ma*cei*ó \cmark \\
macio & ma*ci*o \cmark & ma*ci-o \xmark \\
maçonaria & ma*ço*na*ri*a \cmark & ma*ço*na*ri-a \xmark \\
maçônico & ma-çô*ni*co \xmark & ma*çô*ni*co \cmark \\
macrobiótica & ma*cro*bi-ó*ti*ca \xmark & ma*cro*bi*ó*ti*ca \cmark \\
macrobiótico & ma*cro*bi-ó*ti*co \xmark & ma*cro*bi*ó*ti*co \cmark \\
macroeconomia & ma*cro*e*co*no*mi*a \cmark & ma*cro*e*co*no*mi-a \xmark \\
macroeconômico & ma*cro*e*co-nô*mi*co \xmark & ma*cro*e*co*nô*mi*co \cmark \\
macronúcleo & ma*cro*nú*cle.o \xmark & ma*cro*nú*cleo \cmark \\
macrorregião & ma*cror*re*gi-ão \xmark & ma*cror*re*gi*ão \cmark \\
maestria & ma*es*tri*a \cmark & ma*es*tri-a \xmark \\
máfia & má*fi.a \xmark & má*fia \cmark \\
mafuá & ma*fu-á \xmark & ma*fu*á \cmark \\
magia & ma*gi*a \cmark & ma*gi-a \xmark \\
magistério & ma*gis*té*ri.o \xmark & ma*gis*té*rio \cmark \\
magnésio & mag*né*si.o \xmark & mag*né*sio \cmark \\
magnificência & mag*ni*fi*cên*ci.a \xmark & mag*ni*fi*cên*cia \cmark \\
magnólia & mag*nó*li*a \cmark & mag*nó*li-a \xmark \\
mágoa & má*go.a \xmark & má*goa \cmark \\
maia & mai*a \cmark & mai-a \xmark \\
maiólica & mai-ó*li*ca \xmark & mai*ó*li*ca \cmark \\
maio & mai*o \cmark & mai-o \xmark \\
maioria & mai*o*ri*a \cmark & mai*o*ri-a \xmark \\
maís & ma-ís \xmark & ma*ís \cmark \\
maiúscula & mai-ús*cu*la \xmark & mai*ús*cu*la \cmark \\
maiúsculo & mai-ús*cu*lo \xmark & mai*ús*cu*lo \cmark \\
majoritário & ma*jo*ri*tá*ri.o \xmark & ma*jo*ri*tá*rio \cmark \\
malacologia & ma*la*co*lo*gi*a \cmark & ma*la*co*lo*gi-a \xmark \\
malafaia & ma*la*fai*a \cmark & ma*la*fai-a \xmark \\
malaio & ma*lai*o \cmark & ma*lai-o \xmark \\
malauiano & ma*lau-i*a*no \xmark & ma*lau*i*a*no \cmark \\
maleável & ma*le-á*vel \xmark & ma*le*á*vel \cmark \\
maledicência & ma*le*di*cên*ci.a \xmark & ma*le*di*cên*cia \cmark \\
malícia & ma*lí*ci.a \xmark & ma*lí*cia \cmark \\
malthusianismo & mal-thu*si*a*nis*mo \xmark & mal-thu*si*a*nis*mo \xmark \\
malthusiano & mal-thu*si*a*no \xmark & mal-thu*si*a*no \xmark \\
malvasia & mal*va*si*a \cmark & mal*va*si-a \xmark \\
mamário & ma*má*ri.o \xmark & ma*má*rio \cmark \\
mamografia & ma*mo*gra*fi*a \cmark & ma*mo*gra*fi-a \xmark \\
mancebia & man*ce*bi*a \cmark & man*ce*bi-a \xmark \\
mandaçaia & man*da*çai*a \cmark & man*da*çai-a \xmark \\
mandrião & man*dri-ão \xmark & man*dri*ão \cmark \\
maneia & ma*nei*a \cmark & ma*nei-a \xmark \\
maníaco & ma*ní-a*co \xmark & ma*ní*a*co \cmark \\
mania & ma*ni*a \cmark & ma*ni-a \xmark \\
maniqueísmo & ma*ni*que-ís*mo \xmark & ma*ni*que*ís*mo \cmark \\
maniqueísta & ma*ni*que-ís*ta \xmark & ma*ni*que*ís*ta \cmark \\
manômetro & ma-nô*me*tro \xmark & ma*nô*me*tro \cmark \\
mantô & man-tô \xmark & man*tô \cmark \\
manuseio & ma*nu*sei*o \cmark & ma*nu*sei-o \xmark \\
maquia & ma*qui*a \cmark & ma*qui-a \xmark \\
maquinaria & ma*qui*na*ri*a \cmark & ma*qui*na*ri-a \xmark \\
maquinário & ma*qui*ná*ri.o \xmark & ma*qui*ná*rio \cmark \\
marambaia & ma*ram*bai*a \cmark & ma*ram*bai-a \xmark \\
marcenaria & mar*ce*na*ri*a \cmark & mar*ce*na*ri-a \xmark \\
marchetaria & mar*che*ta*ri*a \cmark & mar*che*ta*ri-a \xmark \\
márcio & már*ci.o \xmark & már*cio \cmark \\
maresia & ma*re*si*a \cmark & ma*re*si-a \xmark \\
maria & ma*ri*a \cmark & ma*ri-a \xmark \\
marinharia & ma*ri*nha*ri*a \cmark & ma*ri*nha*ri-a \xmark \\
marquês & mar-quês \xmark & mar*quês \cmark \\
martírio & mar*tí*ri.o \xmark & mar*tí*rio \cmark \\
maruim & ma*ru-im \xmark & ma*ru-im \xmark \\
masdeísmo & mas*de-ís*mo \xmark & mas*de*ís*mo \cmark \\
mastectomia & mas*tec*to*mi*a \cmark & mas*tec*to*mi-a \xmark \\
mastologia & mas*to*lo*gi*a \cmark & mas*to*lo*gi-a \xmark \\
matéria & ma*té*ri.a \xmark & ma*té*ria \cmark \\
matricídio & ma*tri*cí*di.o \xmark & ma*tri*cí*dio \cmark \\
meão & me-ão \xmark & me*ão \cmark \\
média & mé*di.a \xmark & mé*dia \cmark \\
mediático & me*di-á*ti*co \xmark & me*di*á*ti*co \cmark \\
mediocracia & me*di*o*cra*ci*a \cmark & me*di*o*cra*ci-a \xmark \\
medíocre & me*dí-o*cre \xmark & me*dí*o*cre \cmark \\
médio & mé*di.o \xmark & mé*dio \cmark \\
mediterrâneo & me*di*ter*râ*ne.o \xmark & me*di*ter*râ*neo \cmark \\
megalomaníaco & me*ga*lo*ma*ní-a*co \xmark & me*ga*lo*ma*ní*a*co \cmark \\
megalomania & me*ga*lo*ma*ni*a \cmark & me*ga*lo*ma*ni-a \xmark \\
megatério & me*ga*té*ri.o \xmark & me*ga*té*rio \cmark \\
megawatt & me*ga-watt \xmark & me*ga-watt \xmark \\
meia & mei*a \cmark & mei-a \xmark \\
meio & mei*o \cmark & mei-o \xmark \\
melancia & me*lan*ci*a \cmark & me*lan*ci-a \xmark \\
melancolia & me*lan*co*li*a \cmark & me*lan*co*li-a \xmark \\
melhoria & me*lho*ri*a \cmark & me*lho*ri-a \xmark \\
melífluo & me*lí*flu.o \xmark & me*lí*fluo \cmark \\
melodia & me*lo*di*a \cmark & me*lo*di-a \xmark \\
melofobia & me*lo*fo*bi*a \cmark & me*lo*fo*bi-a \xmark \\
melomania & me*lo*ma*ni*a \cmark & me*lo*ma*ni-a \xmark \\
memória & me*mó*ri.a \xmark & me*mó*ria \cmark \\
mendelévio & men*de*lé*vi.o \xmark & men*de*lé*vio \cmark \\
meneio & me*nei*o \cmark & me*nei-o \xmark \\
menorragia & me*nor*ra*gi*a \cmark & me*nor*ra*gi-a \xmark \\
mênstruo & mêns*tru.o \xmark & mêns*truo \cmark \\
mercadologia & mer*ca*do*lo*gi*a \cmark & mer*ca*do*lo*gi-a \xmark \\
mercadoria & mer*ca*do*ri*a \cmark & mer*ca*do*ri-a \xmark \\
mercancia & mer*can*ci*a \cmark & mer*can*ci-a \xmark \\
mercearia & mer*ce*a*ri*a \cmark & mer*ce*a*ri-a \xmark \\
mercenário & mer*ce*ná*ri.o \xmark & mer*ce*ná*rio \cmark \\
mercúrio & mer*cú*ri.o \xmark & mer*cú*rio \cmark \\
mergulhia & mer*gu*lhi*a \cmark & mer*gu*lhi-a \xmark \\
meritocracia & me*ri*to*cra*ci*a \cmark & me*ri*to*cra*ci-a \xmark \\
merovíngio & me*ro*vín*gi.o \xmark & me*ro*vín*gio \cmark \\
mesquinharia & mes*qui*nha*ri*a \cmark & mes*qui*nha*ri-a \xmark \\
metafonia & me*ta*fo*ni*a \cmark & me*ta*fo*ni-a \xmark \\
metalografia & me*ta*lo*gra*fi*a \cmark & me*ta*lo*gra*fi-a \xmark \\
metalurgia & me*ta*lur*gi*a \cmark & me*ta*lur*gi-a \xmark \\
metaplasia & me*ta*pla*si*a \cmark & me*ta*pla*si-a \xmark \\
meteórico & me*te-ó*ri*co \xmark & me*te*ó*ri*co \cmark \\
meteorologia & me*te*o*ro*lo*gi*a \cmark & me*te*o*ro*lo*gi-a \xmark \\
metodologia & me*to*do*lo*gi*a \cmark & me*to*do*lo*gi-a \xmark \\
metonímia & me*to*ní*mi.a \xmark & me*to*ní*mia \cmark \\
metrologia & me*tro*lo*gi*a \cmark & me*tro*lo*gi-a \xmark \\
metrorragia & me*tror*ra*gi*a \cmark & me*tror*ra*gi-a \xmark \\
mialgia & mi*al*gi*a \cmark & mi*al*gi-a \xmark \\
micélio & mi*cé*li.o \xmark & mi*cé*lio \cmark \\
micologia & mi*co*lo*gi*a \cmark & mi*co*lo*gi-a \xmark \\
microbiólogo & mi*cro*bi-ó*lo*go \xmark & mi*cro*bi*ó*lo*go \cmark \\
micróbio & mi*cró*bi.o \xmark & mi*cró*bio \cmark \\
microcefalia & mi*cro*ce*fa*li*a \cmark & mi*cro*ce*fa*li-a \xmark \\
microcirurgia & mi*cro*ci*rur*gi*a \cmark & mi*cro*ci*rur*gi-a \xmark \\
microfonia & mi*cro*fo*ni*a \cmark & mi*cro*fo*ni-a \xmark \\
microfotografia & mi*cro*fo*to*gra*fi*a \cmark & mi*cro*fo*to*gra*fi-a \xmark \\
microscopia & mi*cros*co*pi*a \cmark & mi*cros*co*pi-a \xmark \\
microscópio & mi*cros*có*pi.o \xmark & mi*cros*có*pio \cmark \\
mictório & mic*tó*ri.o \xmark & mic*tó*rio \cmark \\
micuim & mi*cu-im \xmark & mi*cu-im \xmark \\
mídia & mí*di.a \xmark & mí*dia \cmark \\
midríase & mi*drí-a*se \xmark & mi*drí*a*se \cmark \\
migratório & mi*gra*tó*ri.o \xmark & mi*gra*tó*rio \cmark \\
milenário & mi*le*ná*ri.o \xmark & mi*le*ná*rio \cmark \\
milionário & mi*li*o*ná*ri*o \cmark & mi*li*o*ná*ri-o \xmark \\
militância & mi*li*tân*ci.a \xmark & mi*li*tân*cia \cmark \\
mimeógrafo & mi*me-ó*gra*fo \xmark & mi*me*ó*gra*fo \cmark \\
mineralogia & mi*ne*ra*lo*gi*a \cmark & mi*ne*ra*lo*gi-a \xmark \\
minério & mi*né*ri.o \xmark & mi*né*rio \cmark \\
minifúndio & mi*ni*fún*di.o \xmark & mi*ni*fún*dio \cmark \\
minissérie & mi*nis*sé*ri.e \xmark & mi*nis*sé*rie \cmark \\
ministério & mi*nis*té*ri.o \xmark & mi*nis*té*rio \cmark \\
minoria & mi*no*ri*a \cmark & mi*no*ri-a \xmark \\
minoritário & mi*no*ri*tá*ri.o \xmark & mi*no*ri*tá*rio \cmark \\
minúcia & mi*nú*ci.a \xmark & mi*nú*cia \cmark \\
minudência & mi*nu*dên*ci.a \xmark & mi*nu*dên*cia \cmark \\
míope & mí-o*pe \xmark & mí*o*pe \cmark \\
miopia & mi*o*pi*a \cmark & mi*o*pi-a \xmark \\
miríade & mi*rí-a*de \xmark & mi*rí*a*de \cmark \\
misantropia & mi*san*tro*pi*a \cmark & mi*san*tro*pi-a \xmark \\
miscelânea & mis*ce*lâ*ne.a \xmark & mis*ce*lâ*nea \cmark \\
miséria & mi*sé*ri.a \xmark & mi*sé*ria \cmark \\
misericórdia & mi*se*ri*cór*di.a \xmark & mi*se*ri*cór*dia \cmark \\
misoginia & mi*so*gi*ni*a \cmark & mi*so*gi*ni-a \xmark \\
missionário & mis*si*o*ná*ri.o \xmark & mis*si*o*ná*rio \cmark \\
mistério & mis*té*ri.o \xmark & mis*té*rio \cmark \\
mitocôndria & mi*to-côn*dri.a \xmark & mi*to*côn*dria \cmark \\
mitografia & mi*to*gra*fi*a \cmark & mi*to*gra*fi-a \xmark \\
mitologia & mi*to*lo*gi*a \cmark & mi*to*lo*gi-a \xmark \\
mitomania & mi*to*ma*ni*a \cmark & mi*to*ma*ni-a \xmark \\
miúdo & mi-ú*do \xmark & mi*ú*do \cmark \\
mixaria & mi*xa*ri*a \cmark & mi*xa*ri-a \xmark \\
mixórdia & mi*xór*di.a \xmark & mi*xór*dia \cmark \\
mnemônica & m.ne-mô*ni*ca \xmark & mne*mô*ni*ca \cmark \\
mnemônico & m.ne-mô*ni*co \xmark & mne*mô*ni*co \cmark \\
moa & mo*a \cmark & mo-a \xmark \\
mobília & mo*bí*li.a \xmark & mo*bí*lia \cmark \\
mobiliário & mo*bi*li-á*ri.o \xmark & mo*bi*li*á*rio \cmark \\
modéstia & mo*dés*ti.a \xmark & mo*dés*tia \cmark \\
moído & mo-í*do \xmark & mo*í*do \cmark \\
moinha & mo-i*nha \xmark & mo*i*nha \cmark \\
moinho & mo-i*nho \xmark & mo*i*nho \cmark \\
moio & moi*o \cmark & moi-o \xmark \\
moldávio & mol*dá*vi.o \xmark & mol*dá*vio \cmark \\
moléstia & mo*lés*ti.a \xmark & mo*lés*tia \cmark \\
momentâneo & mo*men*tâ*ne.o \xmark & mo*men*tâ*neo \cmark \\
monarquia & mo*nar*qui*a \cmark & mo*nar*qui-a \xmark \\
monetário & mo*ne*tá*ri.o \xmark & mo*ne*tá*rio \cmark \\
monitoria & mo*ni*to*ri*a \cmark & mo*ni*to*ri-a \xmark \\
monocromia & mo*no*cro*mi*a \cmark & mo*no*cro*mi-a \xmark \\
monofonia & mo*no*fo*ni*a \cmark & mo*no*fo*ni-a \xmark \\
monografia & mo*no*gra*fi*a \cmark & mo*no*gra*fi-a \xmark \\
monomaníaco & mo*no*ma*ní-a*co \xmark & mo*no*ma*ní*a*co \cmark \\
monomania & mo*no*ma*ni*a \cmark & mo*no*ma*ni-a \xmark \\
monômero & mo-nô*me*ro \xmark & mo*nô*me*ro \cmark \\
monopólio & mo*no*pó*li.o \xmark & mo*no*pó*lio \cmark \\
monossomia & mo*nos*so*mi*a \cmark & mo*nos*so*mi-a \xmark \\
monoteísmo & mo*no*te-ís*mo \xmark & mo*no*te*ís*mo \cmark \\
monoteísta & mo*no*te-ís*ta \xmark & mo*no*te*ís*ta \cmark \\
monotonia & mo*no*to*ni*a \cmark & mo*no*to*ni-a \xmark \\
montaria & mon*ta*ri*a \cmark & mon*ta*ri-a \xmark \\
montepio & mon*te*pi*o \cmark & mon*te*pi-o \xmark \\
moquém & mo-quém \xmark & mo*quém \cmark \\
moradia & mo*ra*di*a \cmark & mo*ra*di-a \xmark \\
moratória & mo*ra*tó*ri.a \xmark & mo*ra*tó*ria \cmark \\
mordomia & mor*do*mi*a \cmark & mor*do*mi-a \xmark \\
moreia & mo*rei*a \cmark & mo*rei-a \xmark \\
morfologia & mor*fo*lo*gi*a \cmark & mor*fo*lo*gi-a \xmark \\
morrião & mor*ri-ão \xmark & mor*ri*ão \cmark \\
mostruário & mos*tru-á*ri.o \xmark & mos*tru*á*rio \cmark \\
mouraria & mou*ra*ri*a \cmark & mou*ra*ri-a \xmark \\
movelaria & mo*ve*la*ri*a \cmark & mo*ve*la*ri-a \xmark \\
mucuim & mu*cu-im \xmark & mu*cu-im \xmark \\
mulataria & mu*la*ta*ri*a \cmark & mu*la*ta*ri-a \xmark \\
mulherio & mu*lhe*ri*o \cmark & mu*lhe*ri-o \xmark \\
multimídia & mul*ti*mí*di.a \xmark & mul*ti*mí*dia \cmark \\
multitudinário & mul*ti*tu*di*ná*ri*o \cmark & mul*ti*tu*di*ná*ri-o \xmark \\
múmia & mú*mi.a \xmark & mú*mia \cmark \\
município & mu*ni*cí*pi.o \xmark & mu*ni*cí*pio \cmark \\
murmúrio & mur*mú*ri.o \xmark & mur*mú*rio \cmark \\
museologia & mu*se*o*lo*gi*a \cmark & mu*se*o*lo*gi-a \xmark \\
museólogo & mu*se-ó*lo*go \xmark & mu*se*ó*lo*go \cmark \\
musicografia & mu*si*co*gra*fi*a \cmark & mu*si*co*gra*fi-a \xmark \\
musicologia & mu*si*co*lo*gi*a \cmark & mu*si*co*lo*gi-a \xmark \\
musicoterapia & mu*si*co*te*ra*pi*a \cmark & mu*si*co*te*ra*pi-a \xmark \\
mustelídeo & mus*te*lí*de.o \xmark & mus*te*lí*deo \cmark \\
mutuário & mu*tu-á*ri.o \xmark & mu*tu*á*rio \cmark \\
mútuo & mú*tu.o \xmark & mú*tuo \cmark \\
nanotecnologia & na*no*tec*no*lo*gi*a \cmark & na*no*tec*no*lo*gi-a \xmark \\
napoleão & na*po*le-ão \xmark & na*po*le*ão \cmark \\
natalício & na*ta*lí*ci.o \xmark & na*ta*lí*cio \cmark \\
naufrágio & nau*frá*gi.o \xmark & nau*frá*gio \cmark \\
navio & na*vi*o \cmark & na*vi-o \xmark \\
necessário & ne*ces*sá*ri.o \xmark & ne*ces*sá*rio \cmark \\
necrofagia & ne*cro*fa*gi*a \cmark & ne*cro*fa*gi-a \xmark \\
necrofilia & ne*cro*fi*li*a \cmark & ne*cro*fi*li-a \xmark \\
necrologia & ne*cro*lo*gi*a \cmark & ne*cro*lo*gi-a \xmark \\
necromancia & ne*cro*man*ci*a \cmark & ne*cro*man*ci-a \xmark \\
necropsia & ne*crop*si*a \cmark & ne*crop*si-a \xmark \\
necrotério & ne*cro*té*ri.o \xmark & ne*cro*té*rio \cmark \\
neerlandês & ne*er*lan*dês \cmark & ne*er*lan*dês \cmark \\
nefrectomia & ne*frec*to*mi*a \cmark & ne*frec*to*mi-a \xmark \\
nefrologia & ne*fro*lo*gi*a \cmark & ne*fro*lo*gi-a \xmark \\
nefropatia & ne*fro*pa*ti*a \cmark & ne*fro*pa*ti-a \xmark \\
negligência & ne*gli*gên*ci.a \xmark & ne*gli*gên*cia \cmark \\
negócio & ne*gó*ci.o \xmark & ne*gó*cio \cmark \\
neocolonialismo & ne*o*co*lo*ni*a*lis*mo \cmark & ne-o*co*lo*ni*a*lis*mo \xmark \\
neodímio & ne*o*dí*mi.o \xmark & ne-o*dí*mio \xmark \\
neofascista & ne*o*fas*cis*ta \cmark & ne-o*fas*cis*ta \xmark \\
neófito & ne-ó*fi*to \xmark & ne*ó*fi*to \cmark \\
neolatino & ne*o*la*ti*no \cmark & ne-o*la*ti*no \xmark \\
neoliberalismo & ne*o*li*be*ra*lis*mo \cmark & ne-o*li*be*ra*lis*mo \xmark \\
neoliberal & ne*o*li*be*ral \cmark & ne-o*li*be*ral \xmark \\
neolítico & ne*o*lí*ti*co \cmark & ne-o*lí*ti*co \xmark \\
neologismo & ne*o*lo*gis*mo \cmark & ne-o*lo*gis*mo \xmark \\
neonatologia & ne*o*na*to*lo*gi*a \cmark & ne-o*na*to*lo*gi-a \xmark \\
neonato & ne*o*na*to \cmark & ne-o*na*to \xmark \\
neonazismo & ne*o*na*zis*mo \cmark & ne-o*na*zis*mo \xmark \\
neonazista & ne*o*na*zis*ta \cmark & ne-o*na*zis*ta \xmark \\
neon & ne*on \cmark & ne-on \xmark \\
néon & né-on \xmark & né*on \cmark \\
neoplasia & ne*o*pla*si*a \cmark & ne-o*pla*si-a \xmark \\
neoplatonismo & ne*o*pla*to*nis*mo \cmark & ne-o*pla*to*nis*mo \xmark \\
neozelandês & ne*o*ze*lan*dês \cmark & ne-o*ze*lan*dês \xmark \\
nério & né*ri.o \xmark & né*rio \cmark \\
néscio & nés*ci.o \xmark & nés*cio \cmark \\
neuralgia & neu*ral*gi*a \cmark & neu*ral*gi-a \xmark \\
neurastenia & neu*ras*te*ni*a \cmark & neu*ras*te*ni-a \xmark \\
neurobiologia & neu*ro*bi*o*lo*gi*a \cmark & neu*ro*bi*o*lo*gi-a \xmark \\
neurocirurgia & neu*ro*ci*rur*gi*a \cmark & neu*ro*ci*rur*gi-a \xmark \\
neurocirurgião & neu*ro*ci*rur*gi-ão \xmark & neu*ro*ci*rur*gi*ão \cmark \\
neurologia & neu*ro*lo*gi*a \cmark & neu*ro*lo*gi-a \xmark \\
neuropatia & neu*ro*pa*ti*a \cmark & neu*ro*pa*ti-a \xmark \\
neuropatologia & neu*ro*pa*to*lo*gi*a \cmark & neu*ro*pa*to*lo*gi-a \xmark \\
neutropenia & neu*tro*pe*ni*a \cmark & neu*tro*pe*ni-a \xmark \\
névoa & né*vo.a \xmark & né*voa \cmark \\
nevralgia & ne*vral*gi*a \cmark & ne*vral*gi-a \xmark \\
ninfomaníaca & nin*fo*ma*ní-a*ca \xmark & nin*fo*ma*ní*a*ca \cmark \\
ninfomaníaco & nin*fo*ma*ní-a*co \xmark & nin*fo*ma*ní*a*co \cmark \\
ninfomania & nin*fo*ma*ni*a \cmark & nin*fo*ma*ni-a \xmark \\
nobélio & no*bé*li.o \xmark & no*bé*lio \cmark \\
nobiliarquia & no*bi*li*ar*qui*a \cmark & no*bi*li*ar*qui-a \xmark \\
nobiliárquico & no*bi*li-ár*qui*co \xmark & no*bi*li*ár*qui*co \cmark \\
nomografia & no*mo*gra*fi*a \cmark & no*mo*gra*fi-a \xmark \\
nonagenário & no*na*ge*ná*ri.o \xmark & no*na*ge*ná*rio \cmark \\
noologia & no*o*lo*gi*a \cmark & no*o*lo*gi-a \xmark \\
nosologia & no*so*lo*gi*a \cmark & no*so*lo*gi-a \xmark \\
nostalgia & nos*tal*gi*a \cmark & nos*tal*gi-a \xmark \\
notário & no*tá*ri.o \xmark & no*tá*rio \cmark \\
notícia & no*tí*ci.a \xmark & no*tí*cia \cmark \\
noticiário & no*ti*ci-á*ri.o \xmark & no*ti*ci*á*rio \cmark \\
notocórdio & no*to*cór*di*o \cmark & no*to*cór*di-o \xmark \\
notório & no*tó*ri.o \xmark & no*tó*rio \cmark \\
núbio & nú*bi.o \xmark & nú*bio \cmark \\
núcleo & nú*cle.o \xmark & nú*cleo \cmark \\
núncio & nún*ci.o \xmark & nún*cio \cmark \\
núpcias & núp*ci.as \xmark & núp*ci.as \xmark \\
obediência & o*be*di*ên*ci.a \xmark & o*be*di*ên*cia \cmark \\
obituário & o*bi*tu-á*ri.o \xmark & o*bi*tu*á*rio \cmark \\
oblívio & o*blí*vi.o \xmark & o*blí*vio \cmark \\
oboé & o*bo-é \xmark & o*bo*é \cmark \\
oboísta & o*bo-ís*ta \xmark & o*bo*ís*ta \cmark \\
obrigatório & o*bri*ga*tó*ri.o \xmark & o*bri*ga*tó*rio \cmark \\
obséquio & ob*sé*qui.o \xmark & ob*sé*quio \cmark \\
observância & ob*ser*vân*ci.a \xmark & ob*ser*vân*cia \cmark \\
observatório & ob*ser*va*tó*ri.o \xmark & ob*ser*va*tó*rio \cmark \\
obsolescência & ob*so*les*cên*ci.a \xmark & ob*so*les*cên*cia \cmark \\
obstetrícia & obs*te*trí*ci.a \xmark & obs*te*trí*cia \cmark \\
obstruído & obs*tru-í*do \xmark & obs*tru*í*do \cmark \\
obstruir & obs*tru-ir \xmark & obs*tru*ir \cmark \\
óbvio & ób*vi.o \xmark & ób*vio \cmark \\
ocasião & o*ca*si-ão \xmark & o*ca*si*ão \cmark \\
oceânico & o*ce-â*ni*co \xmark & o*ce*â*ni*co \cmark \\
oceanografia & o*ce*a*no*gra*fi*a \cmark & o*ce*a*no*gra*fi-a \xmark \\
oceanologia & o*ce*a*no*lo*gi*a \cmark & o*ce*a*no*lo*gi-a \xmark \\
ócio & ó*ci.o \xmark & ó*cio \cmark \\
oclocracia & o*clo*cra*ci*a \cmark & o*clo*cra*ci-a \xmark \\
ocluir & o*clu-ir \xmark & o*clu*ir \cmark \\
octaédrico & oc*ta-é*dri*co \xmark & oc*ta*é*dri*co \cmark \\
octogenário & oc*to*ge*ná*ri.o \xmark & oc*to*ge*ná*rio \cmark \\
odeão & o*de-ão \xmark & o*de*ão \cmark \\
ódio & ó*di.o \xmark & ó*dio \cmark \\
odisseia & o*dis*sei*a \cmark & o*dis*sei-a \xmark \\
odômetro & o-dô*me*tro \xmark & o*dô*me*tro \cmark \\
odontologia & o*don*to*lo*gi*a \cmark & o*don*to*lo*gi-a \xmark \\
ofertório & o*fer*tó*ri.o \xmark & o*fer*tó*rio \cmark \\
ofício & o*fí*ci.o \xmark & o*fí*cio \cmark \\
ofidiário & o*fi*di-á*ri.o \xmark & o*fi*di*á*rio \cmark \\
ofídio & o*fí*di.o \xmark & o*fí*dio \cmark \\
oftalmia & of*tal*mi*a \cmark & of*tal*mi-a \xmark \\
oftalmologia & of*tal*mo*lo*gi*a \cmark & of*tal*mo*lo*gi-a \xmark \\
oftalmoplegia & of*tal*mo*ple*gi*a \cmark & of*tal*mo*ple*gi-a \xmark \\
oftalmoscopia & of*tal*mos*co*pi*a \cmark & of*tal*mos*co*pi-a \xmark \\
olaia & o*lai*a \cmark & o*lai-a \xmark \\
olaria & o*la*ri*a \cmark & o*la*ri-a \xmark \\
oleícola & o*le-í*co*la \xmark & o*le*í*co*la \cmark \\
oleína & o*le-í*na \xmark & o*le*í*na \cmark \\
oligarquia & o*li*gar*qui*a \cmark & o*li*gar*qui-a \xmark \\
oligofrenia & o*li*go*fre*ni*a \cmark & o*li*go*fre*ni-a \xmark \\
oligopólio & o*li*go*pó*li.o \xmark & o*li*go*pó*lio \cmark \\
olimpíada & o*lim*pí-a*da \xmark & o*lim*pí*a*da \cmark \\
oncologia & on*co*lo*gi*a \cmark & on*co*lo*gi-a \xmark \\
ondulatório & on*du*la*tó*ri.o \xmark & on*du*la*tó*rio \cmark \\
onomatopeia & o*no*ma*to*pei*a \cmark & o*no*ma*to*pei-a \xmark \\
ontogenia & on*to*ge*ni*a \cmark & on*to*ge*ni-a \xmark \\
ontologia & on*to*lo*gi*a \cmark & on*to*lo*gi-a \xmark \\
oócito & o-ó*ci*to \xmark & o*ó*ci*to \cmark \\
operário & o*pe*rá*ri.o \xmark & o*pe*rá*rio \cmark \\
opinião & o*pi*ni-ão \xmark & o*pi*ni*ão \cmark \\
ópio & ó*pi.o \xmark & ó*pio \cmark \\
opróbrio & o*pró*bri.o \xmark & o*pró*brio \cmark \\
optometria & op*to*me*tri*a \cmark & op*to*me*tri-a \xmark \\
opulência & o*pu*lên*ci.a \xmark & o*pu*lên*cia \cmark \\
oratória & o*ra*tó*ri.a \xmark & o*ra*tó*ria \cmark \\
oratório & o*ra*tó*ri.o \xmark & o*ra*tó*rio \cmark \\
orçamentário & or*ça*men*tá*ri.o \xmark & or*ça*men*tá*rio \cmark \\
ordálio & or*dá*li.o \xmark & or*dá*lio \cmark \\
ordinária & or*di*ná*ri.a \xmark & or*di*ná*ria \cmark \\
ordinário & or*di*ná*ri.o \xmark & or*di*ná*rio \cmark \\
orfeão & or*fe-ão \xmark & or*fe*ão \cmark \\
organologia & or*ga*no*lo*gi*a \cmark & or*ga*no*lo*gi-a \xmark \\
orgia & or*gi*a \cmark & or*gi-a \xmark \\
oriá & o*ri-á \xmark & o*ri*á \cmark \\
orifício & o*ri*fí*ci.o \xmark & o*ri*fí*cio \cmark \\
oriundo & o*ri-un*do \xmark & o*ri*un*do \cmark \\
ornitofilia & or*ni*to*fi*li*a \cmark & or*ni*to*fi*li-a \xmark \\
ornitologia & or*ni*to*lo*gi*a \cmark & or*ni*to*lo*gi-a \xmark \\
orogenia & o*ro*ge*ni*a \cmark & o*ro*ge*ni-a \xmark \\
orografia & o*ro*gra*fi*a \cmark & o*ro*gra*fi-a \xmark \\
ortodontia & or*to*don*ti*a \cmark & or*to*don*ti-a \xmark \\
ortodôntico & or*to-dôn*ti*co \xmark & or*to*dôn*ti*co \cmark \\
ortodoxia & or*to*do*xi*a \cmark & or*to*do*xi-a \xmark \\
ortodromia & or*to*dro*mi*a \cmark & or*to*dro*mi-a \xmark \\
ortopedia & or*to*pe*di*a \cmark & or*to*pe*di-a \xmark \\
ortopraxia & or*to*pra*xi*a \cmark & or*to*pra*xi-a \xmark \\
oscilatório & os*ci*la*tó*ri.o \xmark & os*ci*la*tó*rio \cmark \\
ósmio & ós*mi.o \xmark & ós*mio \cmark \\
ósseo & ós*se.o \xmark & ós*seo \cmark \\
osteíte & os*te-í*te \xmark & os*te*í*te \cmark \\
osteopatia & os*te*o*pa*ti*a \cmark & os*te*o*pa*ti-a \xmark \\
otalgia & o*tal*gi*a \cmark & o*tal*gi-a \xmark \\
otária & o*tá*ri.a \xmark & o*tá*ria \cmark \\
otário & o*tá*ri.o \xmark & o*tá*rio \cmark \\
otorrinolaringologia & o*tor*ri*no*la*rin*go*lo*gi*a \cmark & o*tor*ri*no*la*rin*go*lo*gi-a \xmark \\
ourivesaria & ou*ri*ve*sa*ri*a \cmark & ou*ri*ve*sa*ri-a \xmark \\
ousadia & ou*sa*di*a \cmark & ou*sa*di-a \xmark \\
ouvidoria & ou*vi*do*ri*a \cmark & ou*vi*do*ri-a \xmark \\
paciência & pa*ci*ên*ci.a \xmark & pa*ci*ên*cia \cmark \\
padaria & pa*da*ri*a \cmark & pa*da*ri-a \xmark \\
pagadoria & pa*ga*do*ri*a \cmark & pa*ga*do*ri-a \xmark \\
painço & pa-in*ço \xmark & pa-in*ço \xmark \\
paio & pai*o \cmark & pai-o \xmark \\
país & pa-ís \xmark & pa*ís \cmark \\
palácio & pa*lá*ci.o \xmark & pa*lá*cio \cmark \\
paládio & pa*lá*di.o \xmark & pa*lá*dio \cmark \\
palavrório & pa*la*vró*ri.o \xmark & pa*la*vró*rio \cmark \\
paleógrafo & pa*le-ó*gra*fo \xmark & pa*le*ó*gra*fo \cmark \\
palinologia & pa*li*no*lo*gi*a \cmark & pa*li*no*lo*gi-a \xmark \\
pálio & pá*li.o \xmark & pá*lio \cmark \\
palmatória & pal*ma*tó*ri.a \xmark & pal*ma*tó*ria \cmark \\
panaceia & pa*na*cei*a \cmark & pa*na*cei-a \xmark \\
panaria & pa*na*ri*a \cmark & pa*na*ri-a \xmark \\
pancadaria & pan*ca*da*ri*a \cmark & pan*ca*da*ri-a \xmark \\
pandemia & pan*de*mi*a \cmark & pan*de*mi-a \xmark \\
panfletário & pan*fle*tá*ri*o \cmark & pan*fle*tá*ri-o \xmark \\
pantagruélico & pan*ta*gru-é*li*co \xmark & pan*ta*gru*é*li*co \cmark \\
panteão & pan*te-ão \xmark & pan*te*ão \cmark \\
panteísmo & pan*te-ís*mo \xmark & pan*te*ís*mo \cmark \\
panteísta & pan*te-ís*ta \xmark & pan*te*ís*ta \cmark \\
papagaio & pa*pa*gai*o \cmark & pa*pa*gai-o \xmark \\
papaia & pa*pai*a \cmark & pa*pai-a \xmark \\
papelaria & pa*pe*la*ri*a \cmark & pa*pe*la*ri-a \xmark \\
papua & pa*pu*a \cmark & pa*pu-a \xmark \\
paquímetro & pa-quí*me*tro \xmark & pa*quí*me*tro \cmark \\
paradisíaco & pa*ra*di*sí-a*co \xmark & pa*ra*di*sí*a*co \cmark \\
parafernália & pa*ra*fer*ná*li.a \xmark & pa*ra*fer*ná*lia \cmark \\
parafilia & pa*ra*fi*li*a \cmark & pa*ra*fi*li-a \xmark \\
paraguaia & pa*ra*guai*a \cmark & pa*ra*guai-a \xmark \\
paraguaio & pa*ra*guai*o \cmark & pa*ra*guai-o \xmark \\
paraibano & pa*ra-i*ba*no \xmark & pa*ra-i*ba*no \xmark \\
paraíba & pa*ra-í*ba \xmark & pa*ra*í*ba \cmark \\
paraíso & pa*ra-í*so \xmark & pa*ra*í*so \cmark \\
paralisia & pa*ra*li*si*a \cmark & pa*ra*li*si-a \xmark \\
paranoia & pa*ra*noi*a \cmark & pa*ra*noi-a \xmark \\
paranomásia & pa*ra*no*má*si.a \xmark & pa*ra*no*má*sia \cmark \\
paraplegia & pa*ra*ple*gi*a \cmark & pa*ra*ple*gi-a \xmark \\
parasitologia & pa*ra*si*to*lo*gi*a \cmark & pa*ra*si*to*lo*gi-a \xmark \\
parceria & par*ce*ri*a \cmark & par*ce*ri-a \xmark \\
páreo & pá*re.o \xmark & pá*reo \cmark \\
paresia & pa*re*si*a \cmark & pa*re*si-a \xmark \\
parestesia & pa*res*te*si*a \cmark & pa*res*te*si-a \xmark \\
pária & pá*ri.a \xmark & pá*ria \cmark \\
paritário & pa*ri*tá*ri*o \cmark & pa*ri*tá*ri-o \xmark \\
parlatório & par*la*tó*ri.o \xmark & par*la*tó*rio \cmark \\
parnaíba & par*na-í*ba \xmark & par*na*í*ba \cmark \\
paródia & pa*ró*di.a \xmark & pa*ró*dia \cmark \\
paroníquia & pa*ro*ní*qui*a \cmark & pa*ro*ní*qui-a \xmark \\
paronomásia & pa*ro*no*má*si.a \xmark & pa*ro*no*má*sia \cmark \\
paróquia & pa*ró*qui.a \xmark & pa*ró*quia \cmark \\
parquê & par-quê \xmark & par*quê \cmark \\
parquímetro & par-quí*me*tro \xmark & par*quí*me*tro \cmark \\
parricídio & par*ri*cí*di.o \xmark & par*ri*cí*dio \cmark \\
particípio & par*ti*cí*pi.o \xmark & par*ti*cí*pio \cmark \\
partidário & par*ti*dá*ri.o \xmark & par*ti*dá*rio \cmark \\
parusia & pa*ru*si*a \cmark & pa*ru*si-a \xmark \\
parvoíce & par*vo-í*ce \xmark & par*vo*í*ce \cmark \\
pascácio & pas*cá*ci.o \xmark & pas*cá*cio \cmark \\
passamanaria & pas*sa*ma*na*ri*a \cmark & pas*sa*ma*na*ri-a \xmark \\
passeio & pas*sei*o \cmark & pas*sei-o \xmark \\
pastelaria & pas*te*la*ri*a \cmark & pas*te*la*ri-a \xmark \\
pastoreio & pas*to*rei*o \cmark & pas*to*rei-o \xmark \\
patchuli & pat*chu*li \cmark & pat-chu*li \xmark \\
patenteável & pa*ten*te-á*vel \xmark & pa*ten*te*á*vel \cmark \\
pátio & pá*ti.o \xmark & pá*tio \cmark \\
patoá & pa*to-á \xmark & pa*to*á \cmark \\
patogenia & pa*to*ge*ni*a \cmark & pa*to*ge*ni-a \xmark \\
patologia & pa*to*lo*gi*a \cmark & pa*to*lo*gi-a \xmark \\
pátria & pá*tri.a \xmark & pá*tria \cmark \\
patrícia & pa*trí*ci.a \xmark & pa*trí*cia \cmark \\
patrício & pa*trí*ci.o \xmark & pa*trí*cio \cmark \\
pátrio & pá*tri.o \xmark & pá*trio \cmark \\
patriótico & pa*tri-ó*ti*co \xmark & pa*tri*ó*ti*co \cmark \\
patroa & pa*tro*a \cmark & pa*tro-a \xmark \\
patuá & pa*tu-á \xmark & pa*tu*á \cmark \\
patuleia & pa*tu*lei*a \cmark & pa*tu*lei-a \xmark \\
paul & pa-ul \xmark & pa-ul \xmark \\
pavio & pa*vi*o \cmark & pa*vi-o \xmark \\
pavoa & pa*vo*a \cmark & pa*vo-a \xmark \\
paxiúba & pa*xi-ú*ba \xmark & pa*xi*ú*ba \cmark \\
peão & pe-ão \xmark & pe*ão \cmark \\
peã & pe-ã \xmark & pe*ã \cmark \\
pecíolo & pe*cí-o*lo \xmark & pe*cí*o*lo \cmark \\
pecuária & pe*cu-á*ri.a \xmark & pe*cu*á*ria \cmark \\
pecuário & pe*cu-á*ri.o \xmark & pe*cu*á*rio \cmark \\
pecúlio & pe*cú*li.o \xmark & pe*cú*lio \cmark \\
pecúnia & pe*cú*ni.a \xmark & pe*cú*nia \cmark \\
pecuniário & pe*cu*ni-á*ri.o \xmark & pe*cu*ni*á*rio \cmark \\
pedágio & pe*dá*gi.o \xmark & pe*dá*gio \cmark \\
pedagogia & pe*da*go*gi*a \cmark & pe*da*go*gi-a \xmark \\
pederastia & pe*de*ras*ti*a \cmark & pe*de*ras*ti-a \xmark \\
pediatria & pe*di*a*tri*a \cmark & pe*di*a*tri-a \xmark \\
pediátrico & pe*di-á*tri*co \xmark & pe*di*á*tri*co \cmark \\
pedofilia & pe*do*fi*li*a \cmark & pe*do*fi*li-a \xmark \\
pedofobia & pe*do*fo*bi*a \cmark & pe*do*fo*bi-a \xmark \\
pedologia & pe*do*lo*gi*a \cmark & pe*do*lo*gi-a \xmark \\
pedraria & pe*dra*ri*a \cmark & pe*dra*ri-a \xmark \\
peia & pei*a \cmark & pei-a \xmark \\
peixaria & pei*xa*ri*a \cmark & pei*xa*ri-a \xmark \\
pelúcia & pe*lú*ci.a \xmark & pe*lú*cia \cmark \\
pendência & pen*dên*ci.a \xmark & pen*dên*cia \cmark \\
penedia & pe*ne*di*a \cmark & pe*ne*di-a \xmark \\
peneplanície & pe*ne*pla*ní*ci.e \xmark & pe*ne*pla*ní*cie \cmark \\
penitenciária & pe*ni*ten*ci-á*ri.a \xmark & pe*ni*ten*ci*á*ria \cmark \\
penitenciário & pe*ni*ten*ci-á*ri.o \xmark & pe*ni*ten*ci*á*rio \cmark \\
pentarquia & pen*tar*qui*a \cmark & pen*tar*qui-a \xmark \\
penúria & pe*nú*ri.a \xmark & pe*nú*ria \cmark \\
perdoável & per*do-á*vel \xmark & per*do*á*vel \cmark \\
perdulário & per*du*lá*ri.o \xmark & per*du*lá*rio \cmark \\
perfídia & per*fí*di.a \xmark & per*fí*dia \cmark \\
perfumaria & per*fu*ma*ri*a \cmark & per*fu*ma*ri-a \xmark \\
perícia & pe*rí*ci.a \xmark & pe*rí*cia \cmark \\
periélio & pe*ri-é*li.o \xmark & pe*ri*é*lio \cmark \\
periferia & pe*ri*fe*ri*a \cmark & pe*ri*fe*ri-a \xmark \\
períneo & pe*rí*ne.o \xmark & pe*rí*neo \cmark \\
periódico & pe*ri-ó*di*co \xmark & pe*ri*ó*di*co \cmark \\
período & pe*rí-o*do \xmark & pe*rí*o*do \cmark \\
peripécia & pe*ri*pé*ci.a \xmark & pe*ri*pé*cia \cmark \\
periscópio & pe*ris*có*pi.o \xmark & pe*ris*có*pio \cmark \\
periurbano & pe*ri-ur*ba*no \xmark & pe*ri*ur*ba*no \cmark \\
permeável & per*me-á*vel \xmark & per*me*á*vel \cmark \\
permeio & per*mei*o \cmark & per*mei-o \xmark \\
pérsio & pér*si.o \xmark & pér*sio \cmark \\
persistência & per*sis*tên*ci.a \xmark & per*sis*tên*cia \cmark \\
pertinácia & per*ti*ná*ci.a \xmark & per*ti*ná*cia \cmark \\
pertinência & per*ti*nên*ci.a \xmark & per*ti*nên*cia \cmark \\
perua & pe*ru*a \cmark & pe*ru-a \xmark \\
pescaria & pes*ca*ri*a \cmark & pes*ca*ri-a \xmark \\
pessoa & pes*so*a \cmark & pes*so-a \xmark \\
pestilência & pes*ti*lên*ci.a \xmark & pes*ti*lên*cia \cmark \\
pétreo & pé*tre.o \xmark & pé*treo \cmark \\
petróleo & pe*tró*le.o \xmark & pe*tró*leo \cmark \\
petrologia & pe*tro*lo*gi*a \cmark & pe*tro*lo*gi-a \xmark \\
petroquímica & pe*tro-quí*mi*ca \xmark & pe*tro*quí*mi*ca \cmark \\
petroquímico & pe*tro-quí*mi*co \xmark & pe*tro*quí*mi*co \cmark \\
petulância & pe*tu*lân*ci.a \xmark & pe*tu*lân*cia \cmark \\
petúnia & pe*tú*ni*a \cmark & pe*tú*ni-a \xmark \\
pia & pi*a \cmark & pi-a \xmark \\
piá & pi-á \xmark & pi*á \cmark \\
piauiense & pi*au-i*en*se \xmark & pi*au*i*en*se \cmark \\
picardia & pi*car*di*a \cmark & pi*car*di-a \xmark \\
pictografia & pic*to*gra*fi*a \cmark & pic*to*gra*fi-a \xmark \\
picuinha & pi*cu-i*nha \xmark & pi*cu*i*nha \cmark \\
piério & pi-é*ri.o \xmark & pi*é*rio \cmark \\
pífio & pí*fi*o \cmark & pí*fi-o \xmark \\
pigmeia & pig*mei*a \cmark & pig*mei-a \xmark \\
píleo & pí*le*o \cmark & pí*le-o \xmark \\
pilhéria & pi*lhé*ri*a \cmark & pi*lhé*ri-a \xmark \\
pindaíba & pin*da-í*ba \xmark & pin*da*í*ba \cmark \\
pintainho & pin*ta-i*nho \xmark & pin*ta*i*nho \cmark \\
pio & pi*o \cmark & pi-o \xmark \\
piraí & pi*ra-í \xmark & pi*ra*í \cmark \\
pirataria & pi*ra*ta*ri*a \cmark & pi*ra*ta*ri-a \xmark \\
piromancia & pi*ro*man*ci*a \cmark & pi*ro*man*ci-a \xmark \\
piromaníaco & pi*ro*ma*ní-a*co \xmark & pi*ro*ma*ní*a*co \cmark \\
piromania & pi*ro*ma*ni*a \cmark & pi*ro*ma*ni-a \xmark \\
pirotecnia & pi*ro*tec*ni*a \cmark & pi*ro*tec*ni-a \xmark \\
piruá & pi*ru-á \xmark & pi*ru*á \cmark \\
piscatória & pis*ca*tó*ri.a \xmark & pis*ca*tó*ria \cmark \\
pisoteio & pi*so*tei*o \cmark & pi*so*tei-o \xmark \\
pitiríase & pi*ti*rí-a*se \xmark & pi*ti*rí*a*se \cmark \\
pium & pi-um \xmark & pi*um \cmark \\
pivô & pi-vô \xmark & pi*vô \cmark \\
pixaim & pi*xa-im \xmark & pi*xa-im \xmark \\
pizzaria & piz*za*ri*a \cmark & piz*za*ri-a \xmark \\
plagiário & pla*gi-á*ri.o \xmark & pla*gi*á*rio \cmark \\
plágio & plá*gi.o \xmark & plá*gio \cmark \\
planetário & pla*ne*tá*ri.o \xmark & pla*ne*tá*rio \cmark \\
planície & pla*ní*ci.e \xmark & pla*ní*cie \cmark \\
plantio & plan*ti*o \cmark & plan*ti-o \xmark \\
plateia & pla*tei*a \cmark & pla*tei-a \xmark \\
platônico & pla-tô*ni*co \xmark & pla*tô*ni*co \cmark \\
platô & pla-tô \xmark & pla*tô \cmark \\
plebeísmo & ple*be-ís*mo \xmark & ple*be*ís*mo \cmark \\
plenário & ple*ná*ri.o \xmark & ple*ná*rio \cmark \\
plenilúnio & ple*ni*lú*ni.o \xmark & ple*ni*lú*nio \cmark \\
pleurisia & pleu*ri*si*a \cmark & pleu*ri*si-a \xmark \\
plúmbeo & plúm*be.o \xmark & plúm*beo \cmark \\
plutocracia & plu*to*cra*ci*a \cmark & plu*to*cra*ci-a \xmark \\
pluviometria & plu*vi*o*me*tri*a \cmark & plu*vi*o*me*tri-a \xmark \\
pneumática & p.neu*má*ti*ca \xmark & pneu*má*ti*ca \cmark \\
pneumático & p.neu*má*ti*co \xmark & pneu*má*ti*co \cmark \\
pneumatologia & p.neu*ma*to*lo*gi*a \xmark & pneu*ma*to*lo*gi-a \xmark \\
pneumococo & p.neu*mo*co*co \xmark & pneu*mo*co*co \cmark \\
pneumoconiose & p.neu*mo*co*ni*o*se \xmark & pneu*mo*co*ni*o*se \cmark \\
pneumogástrico & p.neu*mo*gás*tri*co \xmark & pneu*mo*gás*tri*co \cmark \\
pneumologia & p.neu*mo*lo*gi*a \xmark & pneu*mo*lo*gi-a \xmark \\
pneumonia & p.neu*mo*ni*a \xmark & pneu*mo*ni-a \xmark \\
pneumônico & p.neu-mô*ni*co \xmark & pneu*mô*ni*co \cmark \\
pneumotórax & p.neu*mo*tó*rax \xmark & pneu*mo*tó*rax \cmark \\
pneu & p.neu \xmark & pneu \cmark \\
poderio & po*de*ri*o \cmark & po*de*ri-o \xmark \\
pódio & pó*di.o \xmark & pó*dio \cmark \\
podologia & po*do*lo*gi*a \cmark & po*do*lo*gi-a \xmark \\
poesia & po*e*si*a \cmark & po*e*si-a \xmark \\
poética & po-é*ti*ca \xmark & po*é*ti*ca \cmark \\
poético & po-é*ti*co \xmark & po*é*ti*co \cmark \\
poia & poi*a \cmark & poi-a \xmark \\
poio & poi*o \cmark & poi-o \xmark \\
poliandria & po*li*an*dri*a \cmark & po*li*an*dri-a \xmark \\
polia & po*li*a \cmark & po*li-a \xmark \\
polícia & po*lí*ci.a \xmark & po*lí*cia \cmark \\
policitemia & po*li*ci*te*mi*a \cmark & po*li*ci*te*mi-a \xmark \\
policromia & po*li*cro*mi*a \cmark & po*li*cro*mi-a \xmark \\
polidipsia & po*li*dip*si*a \cmark & po*li*dip*si-a \xmark \\
poliéster & po*li-és*ter \xmark & po*li*és*ter \cmark \\
polifagia & po*li*fa*gi*a \cmark & po*li*fa*gi-a \xmark \\
polifônico & po*li-fô*ni*co \xmark & po*li*fô*ni*co \cmark \\
poliginia & po*li*gi*ni*a \cmark & po*li*gi*ni-a \xmark \\
polimorfia & po*li*mor*fi*a \cmark & po*li*mor*fi-a \xmark \\
poliploidia & po*li*ploi*di*a \cmark & po*li*ploi*di-a \xmark \\
polirritmia & po*lir*rit*mi*a \cmark & po*lir*rit*mi-a \xmark \\
polissacarídeo & po*lis*sa*ca*rí*de.o \xmark & po*lis*sa*ca*rí*deo \cmark \\
polissemia & po*lis*se*mi*a \cmark & po*lis*se*mi-a \xmark \\
politeísmo & po*li*te-ís*mo \xmark & po*li*te*ís*mo \cmark \\
politeísta & po*li*te-ís*ta \xmark & po*li*te*ís*ta \cmark \\
poliúria & po*li-ú*ri.a \xmark & po*li*ú*ria \cmark \\
poluir & po*lu-ir \xmark & po*lu*ir \cmark \\
pontaria & pon*ta*ri*a \cmark & pon*ta*ri-a \xmark \\
pontifício & pon*ti*fí*ci.o \xmark & pon*ti*fí*cio \cmark \\
poraquê & po*ra-quê \xmark & po*ra*quê \cmark \\
porcaria & por*ca*ri*a \cmark & por*ca*ri-a \xmark \\
porfia & por*fi*a \cmark & por*fi-a \xmark \\
pornografia & por*no*gra*fi*a \cmark & por*no*gra*fi-a \xmark \\
pornô & por-nô \xmark & por*nô \cmark \\
porquê & por-quê \xmark & por*quê \cmark \\
portaria & por*ta*ri*a \cmark & por*ta*ri-a \xmark \\
portfólio & port*fó*li.o \xmark & port*fó*lio \cmark \\
portuário & por*tu-á*ri.o \xmark & por*tu*á*rio \cmark \\
posologia & po*so*lo*gi*a \cmark & po*so*lo*gi-a \xmark \\
possuído & pos*su-í*do \xmark & pos*su*í*do \cmark \\
possuidor & pos*su-i*dor \xmark & pos*su*i*dor \cmark \\
possuir & pos*su-ir \xmark & pos*su*ir \cmark \\
potássio & po*tás*si.o \xmark & po*tás*sio \cmark \\
potência & po*tên*ci.a \xmark & po*tên*cia \cmark \\
pousio & pou*si*o \cmark & pou*si-o \xmark \\
pradaria & pra*da*ri*a \cmark & pra*da*ri-a \xmark \\
praia & prai*a \cmark & prai-a \xmark \\
prataria & pra*ta*ri*a \cmark & pra*ta*ri-a \xmark \\
preâmbulo & pre-âm*bu*lo \xmark & pre*âm*bu*lo \cmark \\
preá & pre-á \xmark & pre*á \cmark \\
precário & pre*cá*ri.o \xmark & pre*cá*rio \cmark \\
precatório & pre*ca*tó*ri.o \xmark & pre*ca*tó*rio \cmark \\
precípuo & pre*cí*pu.o \xmark & pre*cí*puo \cmark \\
predominância & pre*do*mi*nân*ci.a \xmark & pre*do*mi*nân*cia \cmark \\
predomínio & pre*do*mí*ni.o \xmark & pre*do*mí*nio \cmark \\
preeminência & pre*e*mi*nên*ci.a \xmark & pre*e*mi*nên*cia \cmark \\
preênsil & pre-ên*sil \xmark & pre*ên*sil \cmark \\
prefácio & pre*fá*ci.o \xmark & pre*fá*cio \cmark \\
preferência & pre*fe*rên*ci.a \xmark & pre*fe*rên*cia \cmark \\
preia & prei*a \cmark & prei-a \xmark \\
prejuízo & pre*ju-í*zo \xmark & pre*ju*í*zo \cmark \\
prelazia & pre*la*zi*a \cmark & pre*la*zi-a \xmark \\
prélio & pré*li.o \xmark & pré*lio \cmark \\
prelúdio & pre*lú*di.o \xmark & pre*lú*dio \cmark \\
prenúncio & pre*nún*ci.o \xmark & pre*nún*cio \cmark \\
preparatório & pre*pa*ra*tó*ri.o \xmark & pre*pa*ra*tó*rio \cmark \\
preponderância & pre*pon*de*rân*ci.a \xmark & pre*pon*de*rân*cia \cmark \\
prepotência & pre*po*tên*ci.a \xmark & pre*po*tên*cia \cmark \\
presbiopia & pres*bi*o*pi*a \cmark & pres*bi*o*pi-a \xmark \\
presbitério & pres*bi*té*ri.o \xmark & pres*bi*té*rio \cmark \\
presciência & pres*ci*ên*ci.a \xmark & pres*ci*ên*cia \cmark \\
presépio & pre*sé*pi.o \xmark & pre*sé*pio \cmark \\
presidência & pre*si*dên*ci.a \xmark & pre*si*dên*cia \cmark \\
presidenciável & pre*si*den*ci-á*vel \xmark & pre*si*den*ci*á*vel \cmark \\
presságio & pres*sá*gi.o \xmark & pres*sá*gio \cmark \\
prestígio & pres*tí*gi.o \xmark & pres*tí*gio \cmark \\
pretoria & pre*to*ri*a \cmark & pre*to*ri-a \xmark \\
pretório & pre*tó*ri.o \xmark & pre*tó*rio \cmark \\
previdenciário & pre*vi*den*ci-á*ri.o \xmark & pre*vi*den*ci*á*rio \cmark \\
prévio & pré*vi.o \xmark & pré*vio \cmark \\
primário & pri*má*ri.o \xmark & pri*má*rio \cmark \\
primatologia & pri*ma*to*lo*gi*a \cmark & pri*ma*to*lo*gi-a \xmark \\
primazia & pri*ma*zi*a \cmark & pri*ma*zi-a \xmark \\
primórdio & pri*mór*di.o \xmark & pri*mór*dio \cmark \\
princípio & prin*cí*pi.o \xmark & prin*cí*pio \cmark \\
prioritário & pri*o*ri*tá*ri.o \xmark & pri*o*ri*tá*rio \cmark \\
privilégio & pri*vi*lé*gi.o \xmark & pri*vi*lé*gio \cmark \\
proa & pro*a \cmark & pro-a \xmark \\
procaína & pro*ca-í*na \xmark & pro*ca*í*na \cmark \\
procariótico & pro*ca*ri-ó*ti*co \xmark & pro*ca*ri*ó*ti*co \cmark \\
proctologia & proc*to*lo*gi*a \cmark & proc*to*lo*gi-a \xmark \\
procuradoria & pro*cu*ra*do*ri*a \cmark & pro*cu*ra*do*ri-a \xmark \\
prodígio & pro*dí*gi.o \xmark & pro*dí*gio \cmark \\
proeminência & pro*e*mi*nên*ci.a \xmark & pro*e*mi*nên*cia \cmark \\
profecia & pro*fe*ci*a \cmark & pro*fe*ci-a \xmark \\
proficiência & pro*fi*ci*ên*ci.a \xmark & pro*fi*ci*ên*cia \cmark \\
profícuo & pro*fí*cu.o \xmark & pro*fí*cuo \cmark \\
profilaxia & pro*fi*la*xi*a \cmark & pro*fi*la*xi-a \xmark \\
prolegômenos & pro*le-gô*me*nos \xmark & pro*le*gô*me*nos \cmark \\
proletário & pro*le*tá*ri.o \xmark & pro*le*tá*rio \cmark \\
promontório & pro*mon*tó*ri.o \xmark & pro*mon*tó*rio \cmark \\
promotoria & pro*mo*to*ri*a \cmark & pro*mo*to*ri-a \xmark \\
pronúncia & pro*nún*ci.a \xmark & pro*nún*cia \cmark \\
propício & pro*pí*ci.o \xmark & pro*pí*cio \cmark \\
proprietário & pro*pri*e*tá*ri.o \xmark & pro*pri*e*tá*rio \cmark \\
próprio & pró*pri.o \xmark & pró*prio \cmark \\
prosaísmo & pro*sa-ís*mo \xmark & pro*sa*ís*mo \cmark \\
prosódia & pro*só*di.a \xmark & pro*só*dia \cmark \\
prosopagnosia & pro*so*pag*no*si*a \cmark & pro*so*pag*no*si-a \xmark \\
prosopopeia & pro*so*po*pei*a \cmark & pro*so*po*pei-a \xmark \\
prostatectomia & pros*ta*tec*to*mi*a \cmark & pros*ta*tec*to*mi-a \xmark \\
prostituição & pros*ti*tu-i*ção \xmark & pros*ti*tu*i*ção \cmark \\
prostituir & pros*ti*tu-ir \xmark & pros*ti*tu*ir \cmark \\
protactínio & pro*tac*tí*ni.o \xmark & pro*tac*tí*nio \cmark \\
proteína & pro*te-í*na \xmark & pro*te*í*na \cmark \\
proteólise & pro*te-ó*li*se \xmark & pro*te*ó*li*se \cmark \\
prótio & pró*ti.o \xmark & pró*tio \cmark \\
protofonia & pro*to*fo*ni*a \cmark & pro*to*fo*ni-a \xmark \\
protozoário & pro*to*zo-á*ri.o \xmark & pro*to*zo*á*rio \cmark \\
protuberância & pro*tu*be*rân*ci.a \xmark & pro*tu*be*rân*cia \cmark \\
provedoria & pro*ve*do*ri*a \cmark & pro*ve*do*ri-a \xmark \\
provérbio & pro*vér*bi.o \xmark & pro*vér*bio \cmark \\
providência & pro*vi*dên*ci.a \xmark & pro*vi*dên*cia \cmark \\
província & pro*vín*ci.a \xmark & pro*vín*cia \cmark \\
provisório & pro*vi*só*ri.o \xmark & pro*vi*só*rio \cmark \\
provocatório & pro*vo*ca*tó*ri.o \xmark & pro*vo*ca*tó*rio \cmark \\
prudência & pru*dên*ci.a \xmark & pru*dên*cia \cmark \\
pseudônimo & p.seu-dô*ni*mo \xmark & pseu*dô*ni*mo \cmark \\
psicanálise & p.si*ca*ná*li*se \xmark & psi*ca*ná*li*se \cmark \\
psicanalista & p.si*ca*na*lis*ta \xmark & psi*ca*na*lis*ta \cmark \\
psicanalítico & p.si*ca*na*lí*ti*co \xmark & psi*ca*na*lí*ti*co \cmark \\
psicodrama & p.si*co*dra*ma \xmark & psi*co*dra*ma \cmark \\
psicografia & p.si*co*gra*fi*a \xmark & psi*co*gra*fi-a \xmark \\
psicógrafo & p.si*có*gra*fo \xmark & psi*có*gra*fo \cmark \\
psicolinguista & p.si*co*lin*guis*ta \xmark & psi*co*lin*guis*ta \cmark \\
psicologia & p.si*co*lo*gi*a \xmark & psi*co*lo*gi-a \xmark \\
psicologicamente & p.si*co*lo*gi*ca*men*te \xmark & psi*co*lo*gi*ca*men*te \cmark \\
psicológico & p.si*co*ló*gi*co \xmark & psi*co*ló*gi*co \cmark \\
psicologismo & p.si*co*lo*gis*mo \xmark & psi*co*lo*gis*mo \cmark \\
psicologista & p.si*co*lo*gis*ta \xmark & psi*co*lo*gis*ta \cmark \\
psicologizar & p.si*co*lo*gi*zar \xmark & psi*co*lo*gi*zar \cmark \\
psicólogo & p.si*có*lo*go \xmark & psi*có*lo*go \cmark \\
psicométrico & p.si*co*mé*tri*co \xmark & psi*co*mé*tri*co \cmark \\
psicomotor & p.si*co*mo*tor \xmark & psi*co*mo*tor \cmark \\
psicomotricidade & p.si*co*mo*tri*ci*da*de \xmark & psi*co*mo*tri*ci*da*de \cmark \\
psicopata & p.si*co*pa*ta \xmark & psi*co*pa*ta \cmark \\
psicopático & p.si*co*pá*ti*co \xmark & psi*co*pá*ti*co \cmark \\
psicopatologia & p.si*co*pa*to*lo*gi*a \xmark & psi*co*pa*to*lo*gi-a \xmark \\
psicopatológico & p.si*co*pa*to*ló*gi*co \xmark & psi*co*pa*to*ló*gi*co \cmark \\
psicopedagogia & p.si*co*pe*da*go*gi*a \xmark & psi*co*pe*da*go*gi-a \xmark \\
psicopedagógico & p.si*co*pe*da*gó*gi*co \xmark & psi*co*pe*da*gó*gi*co \cmark \\
psicopedagogo & p.si*co*pe*da*go*go \xmark & psi*co*pe*da*go*go \cmark \\
psicose & p.si*co*se \xmark & psi*co*se \cmark \\
psicossexual & p.si*cos*se*xu*al \xmark & psi*cos*se*xu*al \cmark \\
psicossomático & p.si*cos*so*má*ti*co \xmark & psi*cos*so*má*ti*co \cmark \\
psicotécnico & p.si*co*téc*ni*co \xmark & psi*co*téc*ni*co \cmark \\
psicoterapeuta & p.si*co*te*ra*peu*ta \xmark & psi*co*te*ra*peu*ta \cmark \\
psicoterapia & p.si*co*te*ra*pi*a \xmark & psi*co*te*ra*pi-a \xmark \\
psicoterápico & p.si*co*te*rá*pi*co \xmark & psi*co*te*rá*pi*co \cmark \\
psicótico & p.si*có*ti*co \xmark & psi*có*ti*co \cmark \\
psicotrópico & p.si*co*tró*pi*co \xmark & psi*co*tró*pi*co \cmark \\
psique & p.si*que \xmark & psi*que \cmark \\
psiquiatra & p.si*qui*a*tra \xmark & psi*qui*a*tra \cmark \\
psiquiatria & p.si*qui*a*tri*a \xmark & psi*qui*a*tri-a \xmark \\
psiquiátrico & p.si*qui-á*tri*co \xmark & psi*qui*á*tri*co \cmark \\
psíquico & p.sí*qui*co \xmark & psí*qui*co \cmark \\
psiquismo & p.si*quis*mo \xmark & psi*quis*mo \cmark \\
psitacídeo & p.si*ta*cí*de.o \xmark & psi*ta*cí*deo \cmark \\
psitaciforme & p.si*ta*ci*for*me \xmark & psi*ta*ci*for*me \cmark \\
psitacismo & p.si*ta*cis*mo \xmark & psi*ta*cis*mo \cmark \\
psoas & p.so*as \xmark & p.so*as \xmark \\
psoríase & p.so*rí-a*se \xmark & p.so*rí*a*se \xmark \\
pterossauro & p.te*ros*sau*ro \xmark & pte*ros*sau*ro \cmark \\
ptose & p.to*se \xmark & pto*se \cmark \\
pua & pu*a \cmark & pu-a \xmark \\
pubescência & pu*bes*cên*ci.a \xmark & pu*bes*cên*cia \cmark \\
publicitário & pu*bli*ci*tá*ri*o \cmark & pu*bli*ci*tá*ri-o \xmark \\
pudicícia & pu*di*cí*ci.a \xmark & pu*di*cí*cia \cmark \\
pulôver & pu-lô*ver \xmark & pu*lô*ver \cmark \\
purgatório & pur*ga*tó*ri.o \xmark & pur*ga*tó*rio \cmark \\
quadricromia & qua*dri*cro*mi*a \cmark & qua*dri*cro*mi-a \xmark \\
quadringentenário & qua*drin*gen*te*ná*ri.o \xmark & qua*drin*gen*te*ná*rio \cmark \\
quadrívio & qua*drí*vi.o \xmark & qua*drí*vio \cmark \\
quantia & quan*ti*a \cmark & quan*ti-a \xmark \\
quartzito & quart*zi*to \cmark & quart*zi*to \cmark \\
quartzo & quart*zo \cmark & quart*zo \cmark \\
quaternário & qua*ter*ná*ri.o \xmark & qua*ter*ná*rio \cmark \\
quéchua & qué*chu*a \cmark & qué*chu-a \xmark \\
queijaria & quei*ja*ri*a \cmark & quei*ja*ri-a \xmark \\
questionário & ques*ti*o*ná*ri.o \xmark & ques*ti*o*ná*rio \cmark \\
questiúncula & ques*ti-ún*cu*la \xmark & ques*ti*ún*cu*la \cmark \\
quíchua & quí*chu.a \xmark & quí*chua \cmark \\
quilômetro & qui-lô*me*tro \xmark & qui*lô*me*tro \cmark \\
quilowatt & qui*lo-watt \xmark & qui*lo-watt \xmark \\
quimioterapia & qui*mi*o*te*ra*pi*a \cmark & qui*mi*o*te*ra*pi-a \xmark \\
quinário & qui*ná*ri.o \xmark & qui*ná*rio \cmark \\
quinoa & qui*no*a \cmark & qui*no-a \xmark \\
quinquênio & quin-quê*ni.o \xmark & quin*quê*nio \cmark \\
quinquilharia & quin*qui*lha*ri*a \cmark & quin*qui*lha*ri-a \xmark \\
quiproquó & qui*pro-quó \xmark & qui*pro*quó \cmark \\
quiromancia & qui*ro*man*ci*a \cmark & qui*ro*man*ci-a \xmark \\
raciocínio & ra*ci*o*cí*ni.o \xmark & ra*ci*o*cí*nio \cmark \\
radiestesia & ra*di*es*te*si*a \cmark & ra*di*es*te*si-a \xmark \\
radiografia & ra*di*o*gra*fi*a \cmark & ra*di*o*gra*fi-a \xmark \\
radiólise & ra*di-ó*li*se \xmark & ra*di*ó*li*se \cmark \\
radiologia & ra*di*o*lo*gi*a \cmark & ra*di*o*lo*gi-a \xmark \\
rádio & rá*di.o \xmark & rá*dio \cmark \\
radiotelescópio & ra*di*o*te*les*có*pi.o \xmark & ra*di*o*te*les*có*pio \cmark \\
radioterapia & ra*di*o*te*ra*pi*a \cmark & ra*di*o*te*ra*pi-a \xmark \\
ráfia & rá*fi*a \cmark & rá*fi-a \xmark \\
raia & rai*a \cmark & rai-a \xmark \\
rainha & ra-i*nha \xmark & ra*i*nha \cmark \\
raio & rai*o \cmark & rai-o \xmark \\
raiz & ra-iz \xmark & ra*iz \cmark \\
ramaria & ra*ma*ri*a \cmark & ra*ma*ri-a \xmark \\
ranário & ra*ná*ri.o \xmark & ra*ná*rio \cmark \\
rancharia & ran*cha*ri*a \cmark & ran*cha*ri-a \xmark \\
randômico & ran-dô*mi*co \xmark & ran*dô*mi*co \cmark \\
rapsódia & rap*só*di.a \xmark & rap*só*dia \cmark \\
raquítico & ra-quí*ti*co \xmark & ra*quí*ti*co \cmark \\
rataria & ra*ta*ri*a \cmark & ra*ta*ri-a \xmark \\
rateio & ra*tei*o \cmark & ra*tei-o \xmark \\
ravióli & ra*vi-ó*li \xmark & ra*vi*ó*li \cmark \\
razia & ra*zi*a \cmark & ra*zi-a \xmark \\
razoável & ra*zo-á*vel \xmark & ra*zo*á*vel \cmark \\
rebeldia & re*bel*di*a \cmark & re*bel*di-a \xmark \\
rebelião & re*be*li-ão \xmark & re*be*li*ão \cmark \\
recaída & re*ca-í*da \xmark & re*ca*í*da \cmark \\
recair & re*ca-ir \xmark & re*ca*ir \cmark \\
receio & re*cei*o \cmark & re*cei-o \xmark \\
receituário & re*cei*tu-á*ri.o \xmark & re*cei*tu*á*rio \cmark \\
recheio & re*chei*o \cmark & re*chei-o \xmark \\
recôncavo & re-côn*ca*vo \xmark & re*côn*ca*vo \cmark \\
recôndito & re-côn*di*to \xmark & re*côn*di*to \cmark \\
reconstituição & re*cons*ti*tu-i*ção \xmark & re*cons*ti*tu*i*ção \cmark \\
reconstituinte & re*cons*ti*tu-in*te \xmark & re*cons*ti*tu*in*te \cmark \\
reconstituir & re*cons*ti*tu-ir \xmark & re*cons*ti*tu*ir \cmark \\
reconstruído & re*cons*tru-í*do \xmark & re*cons*tru*í*do \cmark \\
reconstruir & re*cons*tru-ir \xmark & re*cons*tru*ir \cmark \\
recorrência & re*cor*rên*ci.a \xmark & re*cor*rên*cia \cmark \\
recreio & re*crei*o \cmark & re*crei-o \xmark \\
recria & re*cri*a \cmark & re*cri-a \xmark \\
recua & re*cu*a \cmark & re*cu-a \xmark \\
recuo & re*cu*o \cmark & re*cu-o \xmark \\
rédea & ré*de*a \cmark & ré*de-a \xmark \\
redistribuir & re*dis*tri*bu-ir \xmark & re*dis*tri*bu*ir \cmark \\
reengenharia & re*en*ge*nha*ri*a \cmark & re*en*ge*nha*ri-a \xmark \\
reentrância & re*en*trân*ci.a \xmark & re*en*trân*cia \cmark \\
reenvio & re*en*vi*o \cmark & re*en*vi-o \xmark \\
refeitório & re*fei*tó*ri.o \xmark & re*fei*tó*rio \cmark \\
referência & re*fe*rên*ci.a \xmark & re*fe*rên*cia \cmark \\
refinaria & re*fi*na*ri*a \cmark & re*fi*na*ri-a \xmark \\
reflexologia & re*fle*xo*lo*gi*a \cmark & re*fle*xo*lo*gi-a \xmark \\
refluir & re*flu-ir \xmark & re*flu*ir \cmark \\
refúgio & re*fú*gi.o \xmark & re*fú*gio \cmark \\
regalia & re*ga*li*a \cmark & re*ga*li-a \xmark \\
regateio & re*ga*tei*o \cmark & re*ga*tei-o \xmark \\
regência & re*gên*ci.a \xmark & re*gên*cia \cmark \\
região & re*gi-ão \xmark & re*gi*ão \cmark \\
régia & ré*gi.a \xmark & ré*gia \cmark \\
régio & ré*gi.o \xmark & ré*gio \cmark \\
reimplantação & re-im*plan*ta*ção \xmark & re*im*plan*ta*ção \cmark \\
reimplantar & re-im*plan*tar \xmark & re*im*plan*tar \cmark \\
reimportar & re-im*por*tar \xmark & re*im*por*tar \cmark \\
reimpressão & re-im*pres*são \xmark & re*im*pres*são \cmark \\
reimprimir & re-im*pri*mir \xmark & re*im*pri*mir \cmark \\
reincidente & re-in*ci*den*te \xmark & re*in*ci*den*te \cmark \\
reincidir & re-in*ci*dir \xmark & re*in*ci*dir \cmark \\
reincorporação & re-in*cor*po*ra*ção \xmark & re*in*cor*po*ra*ção \cmark \\
reindexação & re-in*de*xa*ção \xmark & re-in*de*xa*ção \xmark \\
reingressar & re-in*gres*sar \xmark & re*in*gres*sar \cmark \\
reingresso & re-in*gres*so \xmark & re*in*gres*so \cmark \\
reinscrever & re-ins*cre*ver \xmark & re*ins*cre*ver \cmark \\
reinserção & re-in*ser*ção \xmark & re*in*ser*ção \cmark \\
reinserir & re-in*se*rir \xmark & re*in*se*rir \cmark \\
reinstalação & re-ins*ta*la*ção \xmark & re*ins*ta*la*ção \cmark \\
reinstalar & re-ins*ta*lar \xmark & re*ins*ta*lar \cmark \\
reinstituir & re-ins*ti*tu-ir \xmark & re*ins*ti*tu*ir \cmark \\
reintegração & re-in*te*gra*ção \xmark & re*in*te*gra*ção \cmark \\
reintegrar & re-in*te*grar \xmark & re*in*te*grar \cmark \\
reintrodução & re-in*tro*du*ção \xmark & re*in*tro*du*ção \cmark \\
reintroduzir & re-in*tro*du*zir \xmark & re*in*tro*du*zir \cmark \\
reinventar & re-in*ven*tar \xmark & re*in*ven*tar \cmark \\
reinvestir & re-in*ves*tir \xmark & re*in*ves*tir \cmark \\
reitoria & rei*to*ri*a \cmark & rei*to*ri-a \xmark \\
relatoria & re*la*to*ri*a \cmark & re*la*to*ri-a \xmark \\
relatório & re*la*tó*ri.o \xmark & re*la*tó*rio \cmark \\
relevância & re*le*vân*ci.a \xmark & re*le*vân*cia \cmark \\
relicário & re*li*cá*ri.o \xmark & re*li*cá*rio \cmark \\
religião & re*li*gi-ão \xmark & re*li*gi*ão \cmark \\
relíquia & re*lí*qui.a \xmark & re*lí*quia \cmark \\
relógio & re*ló*gi.o \xmark & re*ló*gio \cmark \\
relutância & re*lu*tân*ci.a \xmark & re*lu*tân*cia \cmark \\
reminiscência & re*mi*nis*cên*ci.a \xmark & re*mi*nis*cên*cia \cmark \\
renúncia & re*nún*ci.a \xmark & re*nún*cia \cmark \\
repertório & re*per*tó*ri.o \xmark & re*per*tó*rio \cmark \\
repetência & re*pe*tên*ci.a \xmark & re*pe*tên*cia \cmark \\
replantio & re*plan*ti*o \cmark & re*plan*ti-o \xmark \\
repositório & re*po*si*tó*ri.o \xmark & re*po*si*tó*rio \cmark \\
represália & re*pre*sá*li.a \xmark & re*pre*sá*lia \cmark \\
reprografia & re*pro*gra*fi*a \cmark & re*pro*gra*fi-a \xmark \\
repúdio & re*pú*di.o \xmark & re*pú*dio \cmark \\
repugnância & re*pug*nân*ci.a \xmark & re*pug*nân*cia \cmark \\
reservatório & re*ser*va*tó*ri.o \xmark & re*ser*va*tó*rio \cmark \\
residência & re*si*dên*ci.a \xmark & re*si*dên*cia \cmark \\
resiliência & re*si*li*ên*ci.a \xmark & re*si*li*ên*cia \cmark \\
resistência & re*sis*tên*ci.a \xmark & re*sis*tên*cia \cmark \\
respiratório & res*pi*ra*tó*ri.o \xmark & res*pi*ra*tó*rio \cmark \\
resquício & res-quí*ci.o \xmark & res*quí*cio \cmark \\
ressonância & res*so*nân*ci.a \xmark & res*so*nân*cia \cmark \\
restituição & res*ti*tu-i*ção \xmark & res*ti*tu*i*ção \cmark \\
restituir & res*ti*tu-ir \xmark & res*ti*tu*ir \cmark \\
retardatário & re*tar*da*tá*ri.o \xmark & re*tar*da*tá*rio \cmark \\
reticência & re*ti*cên*ci.a \xmark & re*ti*cên*cia \cmark \\
retinopatia & re*ti*no*pa*ti*a \cmark & re*ti*no*pa*ti-a \xmark \\
retraído & re*tra-í*do \xmark & re*tra*í*do \cmark \\
retrair & re*tra-ir \xmark & re*tra*ir \cmark \\
retribuição & re*tri*bu-i*ção \xmark & re*tri*bu*i*ção \cmark \\
retribuir & re*tri*bu-ir \xmark & re*tri*bu*ir \cmark \\
reumatologia & reu*ma*to*lo*gi*a \cmark & reu*ma*to*lo*gi-a \xmark \\
reunir & re-u*nir \xmark & re-u*nir \xmark \\
reurbanizar & re-ur*ba*ni*zar \xmark & re-ur*ba*ni*zar \xmark \\
revelia & re*ve*li*a \cmark & re*ve*li-a \xmark \\
reverência & re*ve*rên*ci.a \xmark & re*ve*rên*cia \cmark \\
revolucionário & re*vo*lu*ci*o*ná*ri.o \xmark & re*vo*lu*ci*o*ná*rio \cmark \\
ria & ri*a \cmark & ri-a \xmark \\
ridicularia & ri*di*cu*la*ri*a \cmark & ri*di*cu*la*ri-a \xmark \\
rinoplastia & ri*no*plas*ti*a \cmark & ri*no*plas*ti-a \xmark \\
rio & ri*o \cmark & ri-o \xmark \\
robô & ro-bô \xmark & ro*bô \cmark \\
rocio & ro*ci*o \cmark & ro*ci-o \xmark \\
rodeio & ro*dei*o \cmark & ro*dei-o \xmark \\
ródio & ró*di.o \xmark & ró*dio \cmark \\
rodízio & ro*dí*zi.o \xmark & ro*dí*zio \cmark \\
rodopio & ro*do*pi*o \cmark & ro*do*pi-o \xmark \\
rodoviário & ro*do*vi-á*ri.o \xmark & ro*do*vi*á*rio \cmark \\
rodovia & ro*do*vi*a \cmark & ro*do*vi-a \xmark \\
roído & ro-í*do \xmark & ro*í*do \cmark \\
romaria & ro*ma*ri*a \cmark & ro*ma*ri-a \xmark \\
rosácea & ro*sá*ce.a \xmark & ro*sá*cea \cmark \\
rosário & ro*sá*ri.o \xmark & ro*sá*rio \cmark \\
roséola & ro*sé-o*la \xmark & ro*sé*o*la \cmark \\
róseo & ró*se.o \xmark & ró*seo \cmark \\
rossio & ros*si*o \cmark & ros*si-o \xmark \\
rouparia & rou*pa*ri*a \cmark & rou*pa*ri-a \xmark \\
ruão & ru-ão \xmark & ru*ão \cmark \\
rua & ru*a \cmark & ru-a \xmark \\
rubéola & ru*bé-o*la \xmark & ru*bé*o*la \cmark \\
rubídio & ru*bí*di.o \xmark & ru*bí*dio \cmark \\
rufião & ru*fi-ão \xmark & ru*fi*ão \cmark \\
ruído & ru-í*do \xmark & ru*í*do \cmark \\
ruim & ru-im \xmark & ru-im \xmark \\
ruína & ru-í*na \xmark & ru*í*na \cmark \\
ruindade & ru-in*da*de \xmark & ru*in*da*de \cmark \\
ruinoso & ru-i*no*so \xmark & ru*i*no*so \cmark \\
ruir & ru-ir \xmark & ru*ir \cmark \\
rupia & ru*pi*a \cmark & ru*pi-a \xmark \\
rutênio & ru*tê*ni.o \xmark & ru*tê*nio \cmark \\
rutherford & ru-ther*ford \xmark & ru-ther*ford \xmark \\
sabedoria & sa*be*do*ri*a \cmark & sa*be*do*ri-a \xmark \\
sabiá & sa*bi-á \xmark & sa*bi*á \cmark \\
sábio & sá*bi.o \xmark & sá*bio \cmark \\
sacaria & sa*ca*ri*a \cmark & sa*ca*ri-a \xmark \\
sacerdócio & sa*cer*dó*ci.o \xmark & sa*cer*dó*cio \cmark \\
sacramentário & sa*cra*men*tá*ri*o \cmark & sa*cra*men*tá*ri-o \xmark \\
sacrário & sa*crá*ri.o \xmark & sa*crá*rio \cmark \\
sacrifício & sa*cri*fí*ci.o \xmark & sa*cri*fí*cio \cmark \\
sacrilégio & sa*cri*lé*gi.o \xmark & sa*cri*lé*gio \cmark \\
sacristia & sa*cris*ti*a \cmark & sa*cris*ti-a \xmark \\
saia & sai*a \cmark & sai-a \xmark \\
saída & sa-í*da \xmark & sa*í*da \cmark \\
saído & sa-í*do \xmark & sa*í*do \cmark \\
saio & sai*o \cmark & sai-o \xmark \\
sair & sa-ir \xmark & sa*ir \cmark \\
saí & sa-í \xmark & sa*í \cmark \\
salafrário & sa*la*frá*ri.o \xmark & sa*la*frá*rio \cmark \\
salário & sa*lá*ri.o \xmark & sa*lá*rio \cmark \\
saliência & sa*li*ên*ci.a \xmark & sa*li*ên*cia \cmark \\
salmodia & sal*mo*di*a \cmark & sal*mo*di-a \xmark \\
saloio & sa*loi*o \cmark & sa*loi-o \xmark \\
salomônico & sa*lo-mô*ni*co \xmark & sa*lo*mô*ni*co \cmark \\
salsicharia & sal*si*cha*ri*a \cmark & sal*si*cha*ri-a \xmark \\
samambaia & sa*mam*bai*a \cmark & sa*mam*bai-a \xmark \\
samário & sa*má*ri.o \xmark & sa*má*rio \cmark \\
sambaíba & sam*ba-í*ba \xmark & sam*ba*í*ba \cmark \\
sandália & san*dá*li.a \xmark & san*dá*lia \cmark \\
sanduíche & san*du-í*che \xmark & san*du*í*che \cmark \\
sangria & san*gri*a \cmark & san*gri-a \xmark \\
sanguinário & san*gui*ná*ri.o \xmark & san*gui*ná*rio \cmark \\
sanguínea & san*guí*ne.a \xmark & san*guí*nea \cmark \\
sanitário & sa*ni*tá*ri.o \xmark & sa*ni*tá*rio \cmark \\
santuário & san*tu-á*ri.o \xmark & san*tu*á*rio \cmark \\
sapataria & sa*pa*ta*ri*a \cmark & sa*pa*ta*ri-a \xmark \\
sapiência & sa*pi*ên*ci.a \xmark & sa*pi*ên*cia \cmark \\
saponáceo & sa*po*ná*ce.o \xmark & sa*po*ná*ceo \cmark \\
sapucaia & sa*pu*cai*a \cmark & sa*pu*cai-a \xmark \\
saquê & sa-quê \xmark & sa*quê \cmark \\
sardônico & sar-dô*ni*co \xmark & sar*dô*ni*co \cmark \\
satiríase & sa*ti*rí-a*se \xmark & sa*ti*rí*a*se \cmark \\
satisfatório & sa*tis*fa*tó*ri.o \xmark & sa*tis*fa*tó*rio \cmark \\
satrapia & sa*tra*pi*a \cmark & sa*tra*pi-a \xmark \\
saúde & sa-ú*de \xmark & sa*ú*de \cmark \\
saúva & sa-ú*va \xmark & sa*ú*va \cmark \\
sebastião & se*bas*ti-ão \xmark & se*bas*ti*ão \cmark \\
seborreia & se*bor*rei*a \cmark & se*bor*rei-a \xmark \\
secretaria & se*cre*ta*ri*a \cmark & se*cre*ta*ri-a \xmark \\
secretária & se*cre*tá*ri.a \xmark & se*cre*tá*ria \cmark \\
secretário & se*cre*tá*ri.o \xmark & se*cre*tá*rio \cmark \\
sectário & sec*tá*ri.o \xmark & sec*tá*rio \cmark \\
secundário & se*cun*dá*ri.o \xmark & se*cun*dá*rio \cmark \\
securitário & se*cu*ri*tá*ri.o \xmark & se*cu*ri*tá*rio \cmark \\
sedentário & se*den*tá*ri.o \xmark & se*den*tá*rio \cmark \\
seio & sei*o \cmark & sei-o \xmark \\
selaria & se*la*ri*a \cmark & se*la*ri-a \xmark \\
selenografia & se*le*no*gra*fi*a \cmark & se*le*no*gra*fi-a \xmark \\
selvageria & sel*va*ge*ri*a \cmark & sel*va*ge*ri-a \xmark \\
selvajaria & sel*va*ja*ri*a \cmark & sel*va*ja*ri-a \xmark \\
semanário & se*ma*ná*ri.o \xmark & se*ma*ná*rio \cmark \\
sematologia & se*ma*to*lo*gi*a \cmark & se*ma*to*lo*gi-a \xmark \\
semiárido & se*mi-á*ri*do \xmark & se*mi*á*ri*do \cmark \\
semicolcheia & se*mi*col*chei*a \cmark & se*mi*col*chei-a \xmark \\
seminário & se*mi*ná*ri.o \xmark & se*mi*ná*rio \cmark \\
semiologia & se*mi*o*lo*gi*a \cmark & se*mi*o*lo*gi-a \xmark \\
senescência & se*nes*cên*ci.a \xmark & se*nes*cên*cia \cmark \\
senhoria & se*nho*ri*a \cmark & se*nho*ri-a \xmark \\
senhorio & se*nho*ri*o \cmark & se*nho*ri-o \xmark \\
sensaboria & sen*sa*bo*ri*a \cmark & sen*sa*bo*ri-a \xmark \\
sépia & sé*pi.a \xmark & sé*pia \cmark \\
septicemia & sep*ti*ce*mi*a \cmark & sep*ti*ce*mi-a \xmark \\
sequência & se-quên*ci.a \xmark & se*quên*cia \cmark \\
sequoia & se*quoi*a \cmark & se*quoi-a \xmark \\
sereia & se*rei*a \cmark & se*rei-a \xmark \\
série & sé*ri.e \xmark & sé*rie \cmark \\
serigrafia & se*ri*gra*fi*a \cmark & se*ri*gra*fi-a \xmark \\
sério & sé*ri.o \xmark & sé*rio \cmark \\
serôdio & se-rô*di.o \xmark & se*rô*dio \cmark \\
serologia & se*ro*lo*gi*a \cmark & se*ro*lo*gi-a \xmark \\
serralharia & ser*ra*lha*ri*a \cmark & ser*ra*lha*ri-a \xmark \\
serrania & ser*ra*ni*a \cmark & ser*ra*ni-a \xmark \\
serventia & ser*ven*ti*a \cmark & ser*ven*ti-a \xmark \\
serventuário & ser*ven*tu-á*ri.o \xmark & ser*ven*tu*á*rio \cmark \\
sérvio & sér*vi.o \xmark & sér*vio \cmark \\
servofreio & ser*vo*frei*o \cmark & ser*vo*frei-o \xmark \\
sesmaria & ses*ma*ri*a \cmark & ses*ma*ri-a \xmark \\
sesquicentenário & ses*qui*cen*te*ná*ri.o \xmark & ses*qui*cen*te*ná*rio \cmark \\
sestércio & ses*tér*ci.o \xmark & ses*tér*cio \cmark \\
sexagenário & se*xa*ge*ná*ri.o \xmark & se*xa*ge*ná*rio \cmark \\
sexologia & se*xo*lo*gi*a \cmark & se*xo*lo*gi-a \xmark \\
siálico & si-á*li*co \xmark & si*á*li*co \cmark \\
sialorreia & si*a*lor*rei*a \cmark & si*a*lor*rei-a \xmark \\
sibilância & si*bi*lân*ci.a \xmark & si*bi*lân*cia \cmark \\
siderurgia & si*de*rur*gi*a \cmark & si*de*rur*gi-a \xmark \\
signatário & sig*na*tá*ri.o \xmark & sig*na*tá*rio \cmark \\
significância & sig*ni*fi*cân*ci.a \xmark & sig*ni*fi*cân*cia \cmark \\
silêncio & si*lên*ci.o \xmark & si*lên*cio \cmark \\
silício & si*lí*ci.o \xmark & si*lí*cio \cmark \\
sílvia & síl*vi.a \xmark & síl*via \cmark \\
simbologia & sim*bo*lo*gi*a \cmark & sim*bo*lo*gi-a \xmark \\
simetria & si*me*tri*a \cmark & si*me*tri-a \xmark \\
símio & sí*mi.o \xmark & sí*mio \cmark \\
simonia & si*mo*ni*a \cmark & si*mo*ni-a \xmark \\
simpatia & sim*pa*ti*a \cmark & sim*pa*ti-a \xmark \\
simpatria & sim*pa*tri*a \cmark & sim*pa*tri-a \xmark \\
simpósio & sim*pó*si.o \xmark & sim*pó*sio \cmark \\
simultâneo & si*mul*tâ*ne.o \xmark & si*mul*tâ*neo \cmark \\
sincronia & sin*cro*ni*a \cmark & sin*cro*ni-a \xmark \\
sindicância & sin*di*cân*ci.a \xmark & sin*di*cân*cia \cmark \\
sinergia & si*ner*gi*a \cmark & si*ner*gi-a \xmark \\
sinestesia & si*nes*te*si*a \cmark & si*nes*te*si-a \xmark \\
sinfonia & sin*fo*ni*a \cmark & sin*fo*ni-a \xmark \\
sinfônica & sin-fô*ni*ca \xmark & sin*fô*ni*ca \cmark \\
sinfônico & sin-fô*ni*co \xmark & sin*fô*ni*co \cmark \\
sinologia & si*no*lo*gi*a \cmark & si*no*lo*gi-a \xmark \\
sinonímia & si*no*ní*mi.a \xmark & si*no*ní*mia \cmark \\
sinônimo & si-nô*ni*mo \xmark & si*nô*ni*mo \cmark \\
sintomatologia & sin*to*ma*to*lo*gi*a \cmark & sin*to*ma*to*lo*gi-a \xmark \\
sintonia & sin*to*ni*a \cmark & sin*to*ni-a \xmark \\
siríaco & si*rí-a*co \xmark & si*rí*a*co \cmark \\
sírio & sí*ri.o \xmark & sí*rio \cmark \\
sítio & sí*ti.o \xmark & sí*tio \cmark \\
sizígia & si*zí*gi.a \xmark & si*zí*gia \cmark \\
soberania & so*be*ra*ni*a \cmark & so*be*ra*ni-a \xmark \\
sobressaia & so*bres*sai*a \cmark & so*bres*sai-a \xmark \\
sobressair & so*bres*sa-ir \xmark & so*bres*sa*ir \cmark \\
sobrevivência & so*bre*vi*vên*ci.a \xmark & so*bre*vi*vên*cia \cmark \\
sobrevoo & so*bre*vo*o \cmark & so*bre*vo-o \xmark \\
sóbrio & só*bri.o \xmark & só*brio \cmark \\
societário & so*ci*e*tá*ri.o \xmark & so*ci*e*tá*rio \cmark \\
sociologia & so*ci*o*lo*gi*a \cmark & so*ci*o*lo*gi-a \xmark \\
sociólogo & so*ci-ó*lo*go \xmark & so*ci*ó*lo*go \cmark \\
sociopatia & so*ci*o*pa*ti*a \cmark & so*ci*o*pa*ti-a \xmark \\
sócio & só*ci.o \xmark & só*cio \cmark \\
sodalício & so*da*lí*ci.o \xmark & so*da*lí*cio \cmark \\
sódio & só*di.o \xmark & só*dio \cmark \\
sodomia & so*do*mi*a \cmark & so*do*mi-a \xmark \\
solário & so*lá*ri.o \xmark & so*lá*rio \cmark \\
solidário & so*li*dá*ri.o \xmark & so*li*dá*rio \cmark \\
sólio & só*li.o \xmark & só*lio \cmark \\
solitária & so*li*tá*ri.a \xmark & so*li*tá*ria \cmark \\
solitário & so*li*tá*ri.o \xmark & so*li*tá*rio \cmark \\
solstício & sols*tí*ci.o \xmark & sols*tí*cio \cmark \\
somatório & so*ma*tó*ri.o \xmark & so*ma*tó*rio \cmark \\
sombrio & som*bri*o \cmark & som*bri-o \xmark \\
sonolência & so*no*lên*ci.a \xmark & so*no*lên*cia \cmark \\
sonômetro & so-nô*me*tro \xmark & so*nô*me*tro \cmark \\
sonoplastia & so*no*plas*ti*a \cmark & so*no*plas*ti-a \xmark \\
sorologia & so*ro*lo*gi*a \cmark & so*ro*lo*gi-a \xmark \\
soroterapia & so*ro*te*ra*pi*a \cmark & so*ro*te*ra*pi-a \xmark \\
sorteio & sor*tei*o \cmark & sor*tei-o \xmark \\
sortilégio & sor*ti*lé*gi.o \xmark & sor*ti*lé*gio \cmark \\
sorveteria & sor*ve*te*ri*a \cmark & sor*ve*te*ri-a \xmark \\
sósia & só*si.a \xmark & só*sia \cmark \\
soslaio & sos*lai*o \cmark & sos*lai-o \xmark \\
soviético & so*vi-é*ti*co \xmark & so*vi*é*ti*co \cmark \\
stalinismo & s.ta*li*nis*mo \xmark & sta*li*nis*mo \cmark \\
stalinista & s.ta*li*nis*ta \xmark & sta*li*nis*ta \cmark \\
suaíli & su*a-í*li \xmark & su*a*í*li \cmark \\
suão & su-ão \xmark & su*ão \cmark \\
suástica & su-ás*ti*ca \xmark & su*ás*ti*ca \cmark \\
suã & su-ã \xmark & su*ã \cmark \\
subaquático & su*ba-quá*ti*co \xmark & su*ba*quá*ti*co \cmark \\
subcategoria & sub*ca*te*go*ri*a \cmark & sub*ca*te*go*ri-a \xmark \\
subcutâneo & sub*cu*tâ*ne.o \xmark & sub*cu*tâ*neo \cmark \\
subdiácono & sub*di-á*co*no \xmark & sub*di*á*co*no \cmark \\
subespécie & su*bes*pé*ci.e \xmark & su*bes*pé*cie \cmark \\
subliminar & su.b-li*mi*nar \xmark & su.b-li*mi*nar \xmark \\
sublingual & su.b-lin*gual \xmark & su.b-lin*gual \xmark \\
sublinha & su.b-li*nha \xmark & su.b-li*nha \xmark \\
subliteratura & su.b-li*te*ra*tu*ra \xmark & su.b-li*te*ra*tu*ra \xmark \\
sublocação & su.b-lo*ca*ção \xmark & sub*lo*ca*ção \cmark \\
sublocar & su.b-lo*car \xmark & sub*lo*car \cmark \\
sublunar & su.b-lu*nar \xmark & sub*lu*nar \cmark \\
subsecretário & sub*se*cre*tá*ri.o \xmark & sub*se*cre*tá*rio \cmark \\
subserviência & sub*ser*vi*ên*ci.a \xmark & sub*ser*vi*ên*cia \cmark \\
subsidiário & sub*si*di-á*ri.o \xmark & sub*si*di*á*rio \cmark \\
subsídio & sub*sí*di.o \xmark & sub*sí*dio \cmark \\
subsistência & sub*sis*tên*ci.a \xmark & sub*sis*tên*cia \cmark \\
substância & subs*tân*ci.a \xmark & subs*tân*cia \cmark \\
substituição & subs*ti*tu-i*ção \xmark & subs*ti*tu*i*ção \cmark \\
substituído & subs*ti*tu-í*do \xmark & subs*ti*tu*í*do \cmark \\
substituir & subs*ti*tu-ir \xmark & subs*ti*tu*ir \cmark \\
substituível & subs*ti*tu-í*vel \xmark & subs*ti*tu*í*vel \cmark \\
subterrâneo & sub*ter*râ*ne.o \xmark & sub*ter*râ*neo \cmark \\
subtraído & sub*tra-í*do \xmark & sub*tra*í*do \cmark \\
subtrair & sub*tra-ir \xmark & sub*tra*ir \cmark \\
subúrbio & su*búr*bi.o \xmark & su*búr*bio \cmark \\
sucedâneo & su*ce*dâ*ne.o \xmark & su*ce*dâ*neo \cmark \\
sucessório & su*ces*só*ri.o \xmark & su*ces*só*rio \cmark \\
súcia & sú*ci.a \xmark & sú*cia \cmark \\
sudário & su*dá*ri.o \xmark & su*dá*rio \cmark \\
suéter & su-é*ter \xmark & su*é*ter \cmark \\
suficiência & su*fi*ci*ên*ci.a \xmark & su*fi*ci*ên*cia \cmark \\
sufragâneo & su*fra*gâ*ne.o \xmark & su*fra*gâ*neo \cmark \\
sufrágio & su*frá*gi.o \xmark & su*frá*gio \cmark \\
suíça & su-í*ça \xmark & su*í*ça \cmark \\
suíço & su-í*ço \xmark & su*í*ço \cmark \\
suingue & su-in*gue \xmark & su*in*gue \cmark \\
suíno & su-í*no \xmark & su*í*no \cmark \\
suíte & su-í*te \xmark & su*í*te \cmark \\
sumário & su*má*ri.o \xmark & su*má*rio \cmark \\
sumério & su*mé*ri.o \xmark & su*mé*rio \cmark \\
sumô & su-mô \xmark & su*mô \cmark \\
supedâneo & su*pe*dâ*ne.o \xmark & su*pe*dâ*neo \cmark \\
superavitário & su*pe*ra*vi*tá*ri.o \xmark & su*pe*ra*vi*tá*rio \cmark \\
supercampeão & su*per*cam*pe-ão \xmark & su*per*cam*pe*ão \cmark \\
superfície & su*per*fí*ci.e \xmark & su*per*fí*cie \cmark \\
supérfluo & su*pér*flu.o \xmark & su*pér*fluo \cmark \\
superintendência & su*pe*rin*ten*dên*ci.a \xmark & su*pe*rin*ten*dên*cia \cmark \\
supersônico & su*per-sô*ni*co \xmark & su*per*sô*ni*co \cmark \\
superveniência & su*per*ve*ni*ên*ci.a \xmark & su*per*ve*ni*ên*cia \cmark \\
supremacia & su*pre*ma*ci*a \cmark & su*pre*ma*ci-a \xmark \\
surucuá & su*ru*cu-á \xmark & su*ru*cu*á \cmark \\
suspensório & sus*pen*só*ri.o \xmark & sus*pen*só*rio \cmark \\
sutiã & su*ti-ã \xmark & su*ti*ã \cmark \\
tabacaria & ta*ba*ca*ri*a \cmark & ta*ba*ca*ri-a \xmark \\
tabelião & ta*be*li-ão \xmark & ta*be*li*ão \cmark \\
tabuão & ta*bu-ão \xmark & ta*bu*ão \cmark \\
tabua & ta*bu*a \cmark & ta*bu-a \xmark \\
tacômetro & ta-cô*me*tro \xmark & ta*cô*me*tro \cmark \\
tainha & ta-i*nha \xmark & ta*i*nha \cmark \\
taiuiá & tai-ui-á \xmark & tai*ui*á \cmark \\
taiwanês & tai-wa*nês \xmark & tai-wa*nês \xmark \\
talassemia & ta*las*se*mi*a \cmark & ta*las*se*mi-a \xmark \\
talassocracia & ta*las*so*cra*ci*a \cmark & ta*las*so*cra*ci-a \xmark \\
talassoterapia & ta*las*so*te*ra*pi*a \cmark & ta*las*so*te*ra*pi-a \xmark \\
talião & ta*li-ão \xmark & ta*li*ão \cmark \\
tálio & tá*li.o \xmark & tá*lio \cmark \\
tanatofobia & ta*na*to*fo*bi*a \cmark & ta*na*to*fo*bi-a \xmark \\
tanatologia & ta*na*to*lo*gi*a \cmark & ta*na*to*lo*gi-a \xmark \\
tangência & tan*gên*ci.a \xmark & tan*gên*cia \cmark \\
tanoaria & ta*no*a*ri*a \cmark & ta*no*a*ri-a \xmark \\
tapuia & ta*pui*a \cmark & ta*pui-a \xmark \\
tapuio & ta*pui*o \cmark & ta*pui-o \xmark \\
taquicardia & ta*qui*car*di*a \cmark & ta*qui*car*di-a \xmark \\
taquigrafia & ta*qui*gra*fi*a \cmark & ta*qui*gra*fi-a \xmark \\
taquígrafo & ta-quí*gra*fo \xmark & ta*quí*gra*fo \cmark \\
taquímetro & ta-quí*me*tro \xmark & ta*quí*me*tro \cmark \\
taquipneia & ta*quip*nei*a \cmark & ta*quip*nei-a \xmark \\
tardio & tar*di*o \cmark & tar*di-o \xmark \\
tarô & ta-rô \xmark & ta*rô \cmark \\
tataravô & ta*ta*ra-vô \xmark & ta*ta*ra*vô \cmark \\
tatuí & ta*tu-í \xmark & ta*tu*í \cmark \\
taumaturgia & tau*ma*tur*gi*a \cmark & tau*ma*tur*gi-a \xmark \\
tauromaquia & tau*ro*ma*qui*a \cmark & tau*ro*ma*qui-a \xmark \\
tautologia & tau*to*lo*gi*a \cmark & tau*to*lo*gi-a \xmark \\
taxionomia & ta*xi*o*no*mi*a \cmark & ta*xi*o*no*mi-a \xmark \\
taxonomia & ta*xo*no*mi*a \cmark & ta*xo*no*mi-a \xmark \\
taxonômico & ta*xo-nô*mi*co \xmark & ta*xo*nô*mi*co \cmark \\
tcheco & t.che*co \xmark & tche*co \cmark \\
tecnécio & tec*né*ci.o \xmark & tec*né*cio \cmark \\
tecnocracia & tec*no*cra*ci*a \cmark & tec*no*cra*ci-a \xmark \\
tecnologia & tec*no*lo*gi*a \cmark & tec*no*lo*gi-a \xmark \\
tédio & té*di.o \xmark & té*dio \cmark \\
teia & tei*a \cmark & tei-a \xmark \\
teimosia & tei*mo*si*a \cmark & tei*mo*si-a \xmark \\
teísmo & te-ís*mo \xmark & te*ís*mo \cmark \\
teísta & te-ís*ta \xmark & te*ís*ta \cmark \\
teiú & tei-ú \xmark & tei*ú \cmark \\
telecinesia & te*le*ci*ne*si*a \cmark & te*le*ci*ne*si-a \xmark \\
teleconferência & te*le*con*fe*rên*ci.a \xmark & te*le*con*fe*rên*cia \cmark \\
teledramaturgia & te*le*dra*ma*tur*gi*a \cmark & te*le*dra*ma*tur*gi-a \xmark \\
telefonia & te*le*fo*ni*a \cmark & te*le*fo*ni-a \xmark \\
telefônico & te*le-fô*ni*co \xmark & te*le*fô*ni*co \cmark \\
telegrafia & te*le*gra*fi*a \cmark & te*le*gra*fi-a \xmark \\
telemetria & te*le*me*tri*a \cmark & te*le*me*tri-a \xmark \\
teleologia & te*le*o*lo*gi*a \cmark & te*le*o*lo*gi-a \xmark \\
teleósteo & te*le-ós*te.o \xmark & te*le*ós*teo \cmark \\
telepatia & te*le*pa*ti*a \cmark & te*le*pa*ti-a \xmark \\
telescópio & te*les*có*pi.o \xmark & te*les*có*pio \cmark \\
telúrio & te*lú*ri.o \xmark & te*lú*rio \cmark \\
temário & te*má*ri.o \xmark & te*má*rio \cmark \\
temerário & te*me*rá*ri.o \xmark & te*me*rá*rio \cmark \\
templário & tem*plá*ri.o \xmark & tem*plá*rio \cmark \\
temporário & tem*po*rá*ri.o \xmark & tem*po*rá*rio \cmark \\
tendência & ten*dên*ci.a \xmark & ten*dên*cia \cmark \\
teníase & te*ní-a*se \xmark & te*ní*a*se \cmark \\
teocracia & te*o*cra*ci*a \cmark & te*o*cra*ci-a \xmark \\
teofania & te*o*fa*ni*a \cmark & te*o*fa*ni-a \xmark \\
teófobo & te-ó*fo*bo \xmark & te*ó*fo*bo \cmark \\
teologia & te*o*lo*gi*a \cmark & te*o*lo*gi-a \xmark \\
teólogo & te-ó*lo*go \xmark & te*ó*lo*go \cmark \\
teônimo & te-ô*ni*mo \xmark & te*ô*ni*mo \cmark \\
teoria & te*o*ri*a \cmark & te*o*ri-a \xmark \\
teórica & te-ó*ri*ca \xmark & te*ó*ri*ca \cmark \\
teórico & te-ó*ri*co \xmark & te*ó*ri*co \cmark \\
teosofia & te*o*so*fi*a \cmark & te*o*so*fi-a \xmark \\
terapia & te*ra*pi*a \cmark & te*ra*pi-a \xmark \\
teratologia & te*ra*to*lo*gi*a \cmark & te*ra*to*lo*gi-a \xmark \\
térbio & tér*bi.o \xmark & tér*bio \cmark \\
terciário & ter*ci-á*ri.o \xmark & ter*ci*á*rio \cmark \\
tércia & tér*ci.a \xmark & tér*cia \cmark \\
terminologia & ter*mi*no*lo*gi*a \cmark & ter*mi*no*lo*gi-a \xmark \\
termografia & ter*mo*gra*fi*a \cmark & ter*mo*gra*fi-a \xmark \\
termologia & ter*mo*lo*gi*a \cmark & ter*mo*lo*gi-a \xmark \\
termômetro & ter-mô*me*tro \xmark & ter*mô*me*tro \cmark \\
termoquímica & ter*mo-quí*mi*ca \xmark & ter*mo*quí*mi*ca \cmark \\
termoquímico & ter*mo-quí*mi*co \xmark & ter*mo*quí*mi*co \cmark \\
térreo & tér*re.o \xmark & tér*reo \cmark \\
território & ter*ri*tó*ri.o \xmark & ter*ri*tó*rio \cmark \\
tertúlia & ter*tú*li.a \xmark & ter*tú*lia \cmark \\
tesouraria & te*sou*ra*ri*a \cmark & te*sou*ra*ri-a \xmark \\
testamentário & tes*ta*men*tá*ri.o \xmark & tes*ta*men*tá*rio \cmark \\
tetania & te*ta*ni*a \cmark & te*ta*ni-a \xmark \\
tetracampeão & te*tra*cam*pe-ão \xmark & te*tra*cam*pe*ão \cmark \\
tetraédrico & te*tra-é*dri*co \xmark & te*tra*é*dri*co \cmark \\
tetraplegia & te*tra*ple*gi*a \cmark & te*tra*ple*gi-a \xmark \\
tetrarquia & te*trar*qui*a \cmark & te*trar*qui-a \xmark \\
tetravô & te*tra-vô \xmark & te*tra*vô \cmark \\
teúrgico & te-úr*gi*co \xmark & te*úr*gi*co \cmark \\
teutônico & teu-tô*ni*co \xmark & teu*tô*ni*co \cmark \\
tia & ti*a \cmark & ti-a \xmark \\
tíbia & tí*bi.a \xmark & tí*bia \cmark \\
tíbio & tí*bi.o \xmark & tí*bio \cmark \\
tília & tí*li.a \xmark & tí*lia \cmark \\
timocracia & ti*mo*cra*ci*a \cmark & ti*mo*cra*ci-a \xmark \\
tinturaria & tin*tu*ra*ri*a \cmark & tin*tu*ra*ri-a \xmark \\
tio & ti*o \cmark & ti-o \xmark \\
tipografia & ti*po*gra*fi*a \cmark & ti*po*gra*fi-a \xmark \\
tipoia & ti*poi*a \cmark & ti*poi-a \xmark \\
tiptologia & tip*to*lo*gi*a \cmark & tip*to*lo*gi-a \xmark \\
tirania & ti*ra*ni*a \cmark & ti*ra*ni-a \xmark \\
tirocínio & ti*ro*cí*ni.o \xmark & ti*ro*cí*nio \cmark \\
tiromancia & ti*ro*man*ci*a \cmark & ti*ro*man*ci-a \xmark \\
tiroteio & ti*ro*tei*o \cmark & ti*ro*tei-o \xmark \\
titânio & ti*tâ*ni.o \xmark & ti*tâ*nio \cmark \\
titia & ti*ti*a \cmark & ti*ti-a \xmark \\
titio & ti*ti*o \cmark & ti*ti-o \xmark \\
tiú & ti-ú \xmark & ti*ú \cmark \\
tmese & t.me*se \xmark & tme*se \cmark \\
toa & to*a \cmark & to-a \xmark \\
toboágua & to*bo-á*gua \xmark & to*bo*á*gua \cmark \\
tocaia & to*cai*a \cmark & to*cai-a \xmark \\
todavia & to*da*vi*a \cmark & to*da*vi-a \xmark \\
tolerância & to*le*rân*ci.a \xmark & to*le*rân*cia \cmark \\
tomografia & to*mo*gra*fi*a \cmark & to*mo*gra*fi-a \xmark \\
tonelaria & to*ne*la*ri*a \cmark & to*ne*la*ri-a \xmark \\
topázio & to*pá*zi.o \xmark & to*pá*zio \cmark \\
topiaria & to*pi*a*ri*a \cmark & to*pi*a*ri-a \xmark \\
topografia & to*po*gra*fi*a \cmark & to*po*gra*fi-a \xmark \\
topologia & to*po*lo*gi*a \cmark & to*po*lo*gi-a \xmark \\
toponímia & to*po*ní*mi.a \xmark & to*po*ní*mia \cmark \\
topônimo & to-pô*ni*mo \xmark & to*pô*ni*mo \cmark \\
toracoscopia & to*ra*cos*co*pi*a \cmark & to*ra*cos*co*pi-a \xmark \\
toracotomia & to*ra*co*to*mi*a \cmark & to*ra*co*to*mi-a \xmark \\
torquês & tor-quês \xmark & tor*quês \cmark \\
torreão & tor*re-ão \xmark & tor*re*ão \cmark \\
tosquia & tos*qui*a \cmark & tos*qui-a \xmark \\
totalitário & to*ta*li*tá*ri.o \xmark & to*ta*li*tá*rio \cmark \\
toxicologia & to*xi*co*lo*gi*a \cmark & to*xi*co*lo*gi-a \xmark \\
toxicomania & to*xi*co*ma*ni*a \cmark & to*xi*co*ma*ni-a \xmark \\
toxicômano & to*xi-cô*ma*no \xmark & to*xi*cô*ma*no \cmark \\
trácio & trá*ci*o \cmark & trá*ci-o \xmark \\
tragédia & tra*gé*di.a \xmark & tra*gé*dia \cmark \\
tragediógrafo & tra*ge*di-ó*gra*fo \xmark & tra*ge*di*ó*gra*fo \cmark \\
tragicomédia & tra*gi*co*mé*di.a \xmark & tra*gi*co*mé*dia \cmark \\
traído & tra-í*do \xmark & tra*í*do \cmark \\
traíra & tra-í*ra \xmark & tra*í*ra \cmark \\
trair & tra-ir \xmark & tra*ir \cmark \\
tramoia & tra*moi*a \cmark & tra*moi-a \xmark \\
transcendência & trans*cen*dên*ci.a \xmark & trans*cen*dên*cia \cmark \\
transeunte & tran*se-un*te \xmark & tran*se-un*te \xmark \\
transferência & trans*fe*rên*ci.a \xmark & trans*fe*rên*cia \cmark \\
transigência & tran*si*gên*ci.a \xmark & tran*si*gên*cia \cmark \\
transitório & tran*si*tó*ri.o \xmark & tran*si*tó*rio \cmark \\
transoceânico & tran*so*ce-â*ni*co \xmark & tran*so*ce*â*ni*co \cmark \\
transumância & tran*su*mân*ci.a \xmark & tran*su*mân*cia \cmark \\
trapézio & tra*pé*zi.o \xmark & tra*pé*zio \cmark \\
traqueia & tra*quei*a \cmark & tra*quei-a \xmark \\
traumatologia & trau*ma*to*lo*gi*a \cmark & trau*ma*to*lo*gi-a \xmark \\
travessia & tra*ves*si*a \cmark & tra*ves*si-a \xmark \\
tríade & trí-a*de \xmark & trí*a*de \cmark \\
triarquia & tri*ar*qui*a \cmark & tri*ar*qui-a \xmark \\
tributário & tri*bu*tá*ri.o \xmark & tri*bu*tá*rio \cmark \\
tricampeão & tri*cam*pe-ão \xmark & tri*cam*pe*ão \cmark \\
tricentenário & tri*cen*te*ná*ri.o \xmark & tri*cen*te*ná*rio \cmark \\
tricotilomania & tri*co*ti*lo*ma*ni*a \cmark & tri*co*ti*lo*ma*ni-a \xmark \\
tricotomia & tri*co*to*mi*a \cmark & tri*co*to*mi-a \xmark \\
tricô & tri-cô \xmark & tri*cô \cmark \\
tricromia & tri*cro*mi*a \cmark & tri*cro*mi-a \xmark \\
tríduo & trí*du.o \xmark & trí*duo \cmark \\
trigonometria & tri*go*no*me*tri*a \cmark & tri*go*no*me*tri-a \xmark \\
trilogia & tri*lo*gi*a \cmark & tri*lo*gi-a \xmark \\
trinitário & tri*ni*tá*ri.o \xmark & tri*ni*tá*rio \cmark \\
trio & tri*o \cmark & tri-o \xmark \\
tripanossomíase & tri*pa*nos*so*mí-a*se \xmark & tri*pa*nos*so*mí*a*se \cmark \\
trisavô & tri*sa-vô \xmark & tri*sa*vô \cmark \\
trissomia & tris*so*mi*a \cmark & tris*so*mi-a \xmark \\
trítio & trí*ti.o \xmark & trí*tio \cmark \\
triunfador & tri-un*fa*dor \xmark & tri*un*fa*dor \cmark \\
triunfalismo & tri-un*fa*lis*mo \xmark & tri*un*fa*lis*mo \cmark \\
triunfalista & tri-un*fa*lis*ta \xmark & tri*un*fa*lis*ta \cmark \\
triunfal & tri-un*fal \xmark & tri*un*fal \cmark \\
triunfante & tri-un*fan*te \xmark & tri*un*fan*te \cmark \\
triunfar & tri-un*far \xmark & tri*un*far \cmark \\
triunfo & tri-un*fo \xmark & tri*un*fo \cmark \\
triúnviro & tri-ún*vi*ro \xmark & tri*ún*vi*ro \cmark \\
trívio & trí*vi.o \xmark & trí*vio \cmark \\
trombocitopenia & trom*bo*ci*to*pe*ni*a \cmark & trom*bo*ci*to*pe*ni-a \xmark \\
truculência & tru*cu*lên*ci.a \xmark & tru*cu*lên*cia \cmark \\
truísmo & tru-ís*mo \xmark & tru*ís*mo \cmark \\
tua & tu*a \cmark & tu-a \xmark \\
tuim & tu-im \xmark & tu*im \cmark \\
túlio & tú*li.o \xmark & tú*lio \cmark \\
tumultuário & tu*mul*tu-á*ri.o \xmark & tu*mul*tu*á*rio \cmark \\
tzarista & t.za*ris*ta \xmark & tza*ris*ta \cmark \\
ufania & u*fa*ni*a \cmark & u*fa*ni-a \xmark \\
uísque & u-ís*que \xmark & u*ís*que \cmark \\
uíste & u-ís*te \xmark & u*ís*te \cmark \\
umbaúba & um*ba-ú*ba \xmark & um*ba*ú*ba \cmark \\
umbria & um*bri*a \cmark & um*bri-a \xmark \\
úmbrio & úm*bri.o \xmark & úm*brio \cmark \\
união & u*ni-ão \xmark & u*ni*ão \cmark \\
unicórnio & u*ni*cór*ni.o \xmark & u*ni*cór*nio \cmark \\
unitário & u*ni*tá*ri.o \xmark & u*ni*tá*rio \cmark \\
universitário & u*ni*ver*si*tá*ri.o \xmark & u*ni*ver*si*tá*rio \cmark \\
urânio & u*râ*ni.o \xmark & u*râ*nio \cmark \\
ureia & u*rei*a \cmark & u*rei-a \xmark \\
uremia & u*re*mi*a \cmark & u*re*mi-a \xmark \\
urgência & ur*gên*ci.a \xmark & ur*gên*cia \cmark \\
urinário & u*ri*ná*ri.o \xmark & u*ri*ná*rio \cmark \\
urnário & ur*ná*ri*o \cmark & ur*ná*ri-o \xmark \\
urologia & u*ro*lo*gi*a \cmark & u*ro*lo*gi-a \xmark \\
uropígio & u*ro*pí*gi.o \xmark & u*ro*pí*gio \cmark \\
urticária & ur*ti*cá*ri.a \xmark & ur*ti*cá*ria \cmark \\
uruguaio & u*ru*guai*o \cmark & u*ru*guai-o \xmark \\
usuário & u*su-á*ri.o \xmark & u*su*á*rio \cmark \\
usucapião & u*su*ca*pi-ão \xmark & u*su*ca*pi*ão \cmark \\
usufruir & u*su*fru-ir \xmark & u*su*fru*ir \cmark \\
usurário & u*su*rá*ri.o \xmark & u*su*rá*rio \cmark \\
utensílio & u*ten*sí*li.o \xmark & u*ten*sí*lio \cmark \\
utilitário & u*ti*li*tá*ri.o \xmark & u*ti*li*tá*rio \cmark \\
utopia & u*to*pi*a \cmark & u*to*pi-a \xmark \\
uvaia & u*vai*a \cmark & u*vai-a \xmark \\
úvea & ú*ve*a \cmark & ú*ve-a \xmark \\
uveíte & u*ve-í*te \xmark & u*ve*í*te \cmark \\
vacância & va*cân*ci.a \xmark & va*cân*cia \cmark \\
vacaria & va*ca*ri*a \cmark & va*ca*ri-a \xmark \\
vacúolo & va*cú-o*lo \xmark & va*cú*o*lo \cmark \\
vácuo & vá*cu.o \xmark & vá*cuo \cmark \\
vadia & va*di*a \cmark & va*di-a \xmark \\
vaia & vai*a \cmark & vai-a \xmark \\
valência & va*lên*ci.a \xmark & va*lên*cia \cmark \\
valentia & va*len*ti*a \cmark & va*len*ti-a \xmark \\
valia & va*li*a \cmark & va*li-a \xmark \\
vanádio & va*ná*di.o \xmark & va*ná*dio \cmark \\
vanglória & van*gló*ri.a \xmark & van*gló*ria \cmark \\
vária & vá*ri.a \xmark & vá*ria \cmark \\
variável & va*ri-á*vel \xmark & va*ri*á*vel \cmark \\
varíola & va*rí-o*la \xmark & va*rí*o*la \cmark \\
vário & vá*ri.o \xmark & vá*rio \cmark \\
varonia & va*ro*ni*a \cmark & va*ro*ni-a \xmark \\
várzea & vár*ze.a \xmark & vár*zea \cmark \\
vascaíno & vas*ca-í*no \xmark & vas*ca*í*no \cmark \\
vasectomia & va*sec*to*mi*a \cmark & va*sec*to*mi-a \xmark \\
vazia & va*zi*a \cmark & va*zi-a \xmark \\
vazio & va*zi*o \cmark & va*zi-o \xmark \\
veemência & ve*e*mên*ci.a \xmark & ve*e*mên*cia \cmark \\
veia & vei*a \cmark & vei-a \xmark \\
veículo & ve-í*cu*lo \xmark & ve*í*cu*lo \cmark \\
velário & ve*lá*ri.o \xmark & ve*lá*rio \cmark \\
velhacaria & ve*lha*ca*ri*a \cmark & ve*lha*ca*ri-a \xmark \\
velharia & ve*lha*ri*a \cmark & ve*lha*ri-a \xmark \\
velório & ve*ló*ri.o \xmark & ve*ló*rio \cmark \\
venéreo & ve*né*re.o \xmark & ve*né*reo \cmark \\
ventania & ven*ta*ni*a \cmark & ven*ta*ni-a \xmark \\
ventriloquia & ven*tri*lo*qui*a \cmark & ven*tri*lo*qui-a \xmark \\
veraneio & ve*ra*nei*o \cmark & ve*ra*nei-o \xmark \\
verborragia & ver*bor*ra*gi*a \cmark & ver*bor*ra*gi-a \xmark \\
vestiário & ves*ti-á*ri.o \xmark & ves*ti*á*rio \cmark \\
véstia & vés*ti*a \cmark & vés*ti-a \xmark \\
vestígio & ves*tí*gi.o \xmark & ves*tí*gio \cmark \\
vestuário & ves*tu-á*ri.o \xmark & ves*tu*á*rio \cmark \\
veterinária & ve*te*ri*ná*ri.a \xmark & ve*te*ri*ná*ria \cmark \\
veterinário & ve*te*ri*ná*ri.o \xmark & ve*te*ri*ná*rio \cmark \\
viário & vi-á*ri.o \xmark & vi*á*rio \cmark \\
viável & vi-á*vel \xmark & vi*á*vel \cmark \\
via & vi*a \cmark & vi-a \xmark \\
vibrião & vi*bri-ão \xmark & vi*bri*ão \cmark \\
vicário & vi*cá*ri.o \xmark & vi*cá*rio \cmark \\
vício & ví*ci.o \xmark & ví*cio \cmark \\
vidência & vi*dên*ci.a \xmark & vi*dên*cia \cmark \\
videoconferência & vi*de*o*con*fe*rên*ci.a \xmark & vi*de*o*con*fe*rên*cia \cmark \\
vídeo & ví*de.o \xmark & ví*deo \cmark \\
vidraçaria & vi*dra*ça*ri*a \cmark & vi*dra*ça*ri-a \xmark \\
vidraria & vi*dra*ri*a \cmark & vi*dra*ri-a \xmark \\
viés & vi-és \xmark & vi*és \cmark \\
vigário & vi*gá*ri.o \xmark & vi*gá*rio \cmark \\
vigência & vi*gên*ci.a \xmark & vi*gên*cia \cmark \\
vigia & vi*gi*a \cmark & vi*gi-a \xmark \\
vigilância & vi*gi*lân*ci.a \xmark & vi*gi*lân*cia \cmark \\
vigília & vi*gí*li.a \xmark & vi*gí*lia \cmark \\
vilania & vi*la*ni*a \cmark & vi*la*ni-a \xmark \\
vilipêndio & vi*li*pên*di.o \xmark & vi*li*pên*dio \cmark \\
violáceo & vi*o*lá*ce.o \xmark & vi*o*lá*ceo \cmark \\
violência & vi*o*lên*ci.a \xmark & vi*o*lên*cia \cmark \\
virologia & vi*ro*lo*gi*a \cmark & vi*ro*lo*gi-a \xmark \\
virulência & vi*ru*lên*ci.a \xmark & vi*ru*lên*cia \cmark \\
visionário & vi*si*o*ná*ri.o \xmark & vi*si*o*ná*rio \cmark \\
vistoria & vis*to*ri*a \cmark & vis*to*ri-a \xmark \\
vitalício & vi*ta*lí*ci.o \xmark & vi*ta*lí*cio \cmark \\
vitória & vi*tó*ri.a \xmark & vi*tó*ria \cmark \\
vítreo & ví*tre.o \xmark & ví*treo \cmark \\
vitriólico & vi*tri-ó*li*co \xmark & vi*tri*ó*li*co \cmark \\
vitupério & vi*tu*pé*ri.o \xmark & vi*tu*pé*rio \cmark \\
viúva & vi-ú*va \xmark & vi*ú*va \cmark \\
viúvo & vi-ú*vo \xmark & vi*ú*vo \cmark \\
vivência & vi*vên*ci.a \xmark & vi*vên*cia \cmark \\
vocabulário & vo*ca*bu*lá*ri.o \xmark & vo*ca*bu*lá*rio \cmark \\
volfrâmio & vol*frâ*mi.o \xmark & vol*frâ*mio \cmark \\
voluntário & vo*lun*tá*ri.o \xmark & vo*lun*tá*rio \cmark \\
voo & vo*o \cmark & vo-o \xmark \\
vovô & vo-vô \xmark & vo*vô \cmark \\
voyeurismo & voy-eu*ris*mo \xmark & voy-eu*ris*mo \xmark \\
voyeurista & voy-eu*ris*ta \xmark & voy-eu*ris*ta \xmark \\
voyeurístico & voy-eu*rís*ti*co \xmark & voy-eu*rís*ti*co \xmark \\
vozerio & vo*ze*ri*o \cmark & vo*ze*ri-o \xmark \\
vulcanologia & vul*ca*no*lo*gi*a \cmark & vul*ca*no*lo*gi-a \xmark \\
xaria & xa*ri*a \cmark & xa*ri-a \xmark \\
xátria & xá*tri.a \xmark & xá*tria \cmark \\
xenofilia & xe*no*fi*li*a \cmark & xe*no*fi*li-a \xmark \\
xenofobia & xe*no*fo*bi*a \cmark & xe*no*fo*bi-a \xmark \\
xenomania & xe*no*ma*ni*a \cmark & xe*no*ma*ni-a \xmark \\
xeroftalmia & xe*rof*tal*mi*a \cmark & xe*rof*tal*mi-a \xmark \\
xerografia & xe*ro*gra*fi*a \cmark & xe*ro*gra*fi-a \xmark \\
xilocaína & xi*lo*ca-í*na \xmark & xi*lo*ca*í*na \cmark \\
xilografia & xi*lo*gra*fi*a \cmark & xi*lo*gra*fi-a \xmark \\
xilomancia & xi*lo*man*ci*a \cmark & xi*lo*man*ci-a \xmark \\
xintoísmo & xin*to-ís*mo \xmark & xin*to*ís*mo \cmark \\
xintoísta & xin*to-ís*ta \xmark & xin*to*ís*ta \cmark \\
zaragatoa & za*ra*ga*to*a \cmark & za*ra*ga*to-a \xmark \\
zodíaco & zo*dí-a*co \xmark & zo*dí*a*co \cmark \\
zombaria & zom*ba*ri*a \cmark & zom*ba*ri-a \xmark \\
zoofilia & zo*o*fi*li*a \cmark & zo*o*fi*li-a \xmark \\
zoogeografia & zo*o*ge*o*gra*fi*a \cmark & zo*o*ge*o*gra*fi-a \xmark \\
zoolagnia & zo*o*lag*ni*a \cmark & zo*o*lag*ni-a \xmark \\
zoologia & zo*o*lo*gi*a \cmark & zo*o*lo*gi-a \xmark \\
zoósporo & zo-ós*po*ro \xmark & zo*ós*po*ro \cmark \\
zootecnia & zo*o*tec*ni*a \cmark & zo*o*tec*ni-a \xmark \\
zoo & zo*o \cmark & zo-o \xmark \\
zurraria & zur*ra*ri*a \cmark & zur*ra*ri-a \xmark \\
total correct & 1331 & 1992 \\

\end{longtable}


\printbibliography

\end{document}
