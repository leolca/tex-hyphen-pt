\section{Creating the dictionary}\label{sec-dictionay}

A bash script, named \texttt{gethyphenations.sh}, was developed to crawl six
online dictionaries and extract word hyphenations whenever available. The
process involves launching parallel threads for each word in the list, querying
a specific online dictionary, and retrieving the corresponding hyphenation. For
this purpose, individual scripts were created for each dictionary:
\texttt{getmichaelishyphenation.sh} for \href{https://michaelis.uol.com.br/}{Michaelis},
\texttt{getpriberamhyphenation.sh} for \href{https://dicionario.priberam.org/}{Priberam},
\texttt{getwiktionaryhyphenation.sh} for \href{https://pt.wiktionary.org}{Wikcionário},
\texttt{getauletehyphenation.sh} for \href{https://aulete.com.br/}{Aulete},
\texttt{getportalhyphenations.sh} for \href{http://www.portaldalinguaportuguesa.org/}{Portal da Língua Portuguesa}, and
\texttt{getdiciohyphenation.sh} for \href{https://www.dicio.com.br/}{Dicio}.
This systematic approach ensures efficient and accurate extraction of
hyphenations from various online sources. The resulting data is stored in a CSV\footnote{
    Comma-separated values (CSV) is a text file format defined in \href{https://datatracker.ietf.org/doc/html/rfc4180}{RFC 4180}.
    In CSV, commas are used to separate values, and each line in the file represents a new record. Records consist of fields separated by delimiters, typically a single comma.
} file named \texttt{hyphenations.csv}.






\subsection{Establishing a gold standard}

To establish our gold standard for hyphenation, we will compare the
hyphenations offered by all six dictionaries. First, we will determine the total count
of unique hyphenations across these dictionaries using the script
\texttt{hyphenation\-agreements.sh}.
It's worth noting that certain dictionaries annotate words with a colon (`:') to
indicate potential variations in pronunciation or production.
Particularly, `apparent proparoxytones' might appear marked by this colon sign, indicating 
the possibility of a diphthongs or hiatuses in the word ending.
In Portuguese, words ending in \emph{ea}, \emph{eo}, \emph{ia}, \emph{ie},
\emph{io}, \emph{ua}, \emph{ue}, or \emph{uo}, with the stressed syllable
before these endings, can be considered either paroxytones ending in a rising
diphthong or proparoxytones ending in a hiatus. For instance, words like \emph{his-tó-ri:a}
and \emph{on-du-la-tó-ri:o} are examples of words marked as apparent
proparoxytones, suggesting they could be regarded as paroxytones (\emph{his-tó-ria}
and \emph{on-du-la-tó-rio}) or proparoxytones (\emph{his-tó-ri-a} and
\emph{on-du-la-tó-ri-o}). The script \texttt{hyphenationagreements.sh} 
offers four potential approaches to handle these apparent proparoxytones:
retaining the hyphenations with the colon mark to indicate an apparent proparoxytone; 
selecting only the paroxytone or only the proparoxytone counterpart; or counting both paroxytones and proparoxytones. 
Given the lack of consensus in the literature, our initial approach involves 
counting both hyphenations, and we will detail below how we resolve this issue. 
In our hyphenation dictionary, we have identified \NumberOfAppProp{} words marked as apparent proparoxytones.

As our initial hyphenation reference, we will consider those cases where all
six dictionaries are in agreement. This comprehensive list comprises
\num{\NumberOfSixAgrees} words. Their corresponding hyphenations are
meticulously compiled and stored in the file \texttt{hyphenations6.dic}. This
curated list serves as a benchmark for evaluating the accuracy and
effectiveness of hyphenation rules generated through computational methods. It
will establish a reliable baseline for our hyphenation analysis, enabling us to
assess and refine our methodologies effectively.

In case of disagreement, we must exercise caution and examine each instance.
We might be tempted to choose the majority vote approach, simply selecting
those hyphenations that have the most supporters, but it is unclear 
how the hyphenations on those online dictionaries were curated; 
many might use algorithmic approaches, leading to potential flaws shared
among them. 
For example, 5 dictionaries use the hyphenation 
\emph{quart-zo} while the dictionary Aulete (originally from Portugal) favors the hyphenation \emph{quar-tzo},
which can be explained by the lack of an epenthetic vowel in European Portuguese\footnote{
The online dictionary Aulete has its origins on the printed version known as \emph{Dicionário Caldas Aulete}.
It is a dictionary from Portugal, reflecting the European Portuguese dialect.} \parencite{mateus}.
% Epêntese Vocálica e Restrições de Acento no Português do Sul do Brasil, Gisela COLLISCHONN  - Universidade Federal do Rio Grande do Sul (UFRGS)
% https://core.ac.uk/download/pdf/293615216.pdf
Similarly, \emph{ap-nei-a} or \emph{a-pnei-a} and \emph{disp-nei-a} or \emph{dis-pnei-a} should
be accepted considering, since there could be an epenthetic vowel (Brazilian Portuguese) or not (European Portuguese).
The same happens with \emph{hiperalgesia}, already mentioned in Section \ref{sec-grammar-port}.
Among other examples, we might also highlight \emph{neerlandês}, which appears 
hyphenated as \emph{ne-er-lan-dês} in five dictionaries and as \emph{neer-lan-dês} in one.
If we consider its pronunciation, we would prefer the former hyphenation, but the
first one conforms better to regular written syllables in Portuguese.
%The same happens with \emph{de-sem-ba-i-nhar} and \emph{de-sem-bai-nhar},
%the former one receiving the majority of the votes.
Some strings might accept more than one hyphenation, since they represent different words. For example, 
\emph{sub-li-nha} and \emph{su-bli-nha}, where the former is the substantive \emph{underline} and the latter is a flexed form of the verb \emph{to underline}.
%\todo{Aline, segue a lista dos casos duvidosos que encontrei com votos 5 a 1. Marquei em negrito aqueles
%    que considero ser a forma correta.
%    quart-zo(5), \textbf{quar-tzo}(1) (assim também trocar quart-zi-to por \textbf{quar-tzi-to)}; \textbf{sub-li-nha}(5), \textbf{su-bli-nha}(1); 
%    de-sem-ba-i-nhar(5), \textbf{de-sem-bai-nhar}(1); ap-nei-a(5), \textbf{a-pnei-a}(1); 
%    disp-nei-a(5), dispneia(1) (\textbf{dis-pnei-a}, forma não encontrada nos dicionários); 
%    hi-pe-ral-ge-si-a(5), \textbf{hi-per-al-ge-si-a}(1);
%    in-de-is-cen-te(5), \textbf{in-deis-cen-te}(1); jo-in-vi-len-se(5), \textbf{join-vi-len-se}(1); 
%    ne-er-lan-dês(5), \textbf{neer-lan-dês}(1). Os demais casos de 5 a 1 são proparoxítonas aparentes.
%}

The procedure to create the second list of hyphenated words involved several steps:
\begin{enumerate*}[label=\arabic*)]
  \item select those hyphenations agreed by five dictionaries, that have no alternative hyphenation proposal (5 to 0 votes);
  \item from those with alternative proposals, exam whether they constituted apparent proparoxytones (these cases could arise from instances with 5 to 1 votes or 6 to 1 votes, as explained earlier), 
      \begin{enumerate*}[label=\alph*)]
	  \item if the word had a diacritic, select the paroxytone option (e.g. \emph{po-lí-cia} instead of \emph{po-lí-ci-a} and \emph{pe-rí-neo} instead of \emph{pe-rí-ne-o}),
	  \item if not, choose the proparoxytone option (e.g. \emph{au-to-ri-a} instead of \emph{au-to-ri-a} and \emph{mar-ce-nei-ro} instead of \emph{mar-ce-neiro})\footnote{We just have to make sure we are not selecting those cases where there is a sequence of the form \texttt{[gq]u-[aeio]} (using regex notation), as in \emph{a-pa-zi-gu-ar} and \emph{en-xa-gu-ar}.}.
	      % [gq]u-[aeio]
      \end{enumerate*}
  \item if the word did not fall into the apparent proparoxytone category, select the hyphenation with five votes.
\end{enumerate*}
This procedure was implemented in the script \texttt{get51.sh}. 
Only a few hyphenations required manual correction, including the following that were added:
\emph{quar-tzo}, \emph{quar-tzi-to}, \emph{su-bli-nha}, \emph{a-pnei-a}, \emph{dis-pnei-a}, 
\emph{hi-per-al-ge-si-a}, and \emph{neer-lan-dês}.
The final list, containing \num{\NumberOfFiveAgrees} additional hyphenations, is stored in \texttt{hyphenations5.dic}, 
bringing the total number of hyphenations across both lists to \num{\NumberOfSixFiveAgrees}.

% results from script ./get51.sh
% cases where majority vote might go wrong:
% sub-li-nha(5),su-bli-nha(1) % both are valid
% de-sem-ba-i-nhar(5),de-sem-bai-nhar(1) % de-sem-bai-nhar seems to be the correct choice
% ap-nei-a(5),a-pnei-a(1) % ??
% disp-nei-a(5),dispneia(1) or dis-pnei-a?
% hi-pe-ral-ge-si-a(5),hi-per-al-ge-si-a(1)
% in-de-is-cen-te(5),in-deis-cen-te(1)
% jo-in-vi-len-se(5),join-vi-len-se(1)
% ne-er-lan-dês(5),neer-lan-dês(1)
% the remaining are apparent proparoxytones


For those hyphenations with four votes, we adopted a similar approach to the
one used previously. Many words garnered only four votes, indicating a lack of
alternative hyphenations provided by any dictionary, and were thus initially
hyphenated as proparoxytones. These instances were typically distinguished by
diacritics in the third-to-last syllable and a \emph{V-V} sequence at the end.
To address this, we converted these words into paroxytones by eliminating the
final hyphen. In cases where alternative hyphenations were available, we
assessed the presence of diacritics. If diacritics were detected, we chose the
paroxytone option; otherwise, we defaulted to the proparoxytone option. In
instances with multiple alternative options, we selected the option with the
highest frequency of votes. This procedure is implemented in the script
\texttt{get4.sh}.

% analisar os resultados das hifenizações através da análise de frequência das sílabas
% as silabas de baixa frequência podem ser indicativos de erros... verificar a origem delas
% cat ../data/hyphenations6.dic ../data/hyphenations5.dic ../data/hyphenations4.dic | tr '-' '\n' | sort | uniq -c | sort -nr > /tmp/syls
% grep -P "\s+1\s[^0-9]+$" /tmp/syls | tr -d '1' | sed 's/^\s\+//g' | while read syl; do grep "^$syl-\|-$syl-\|-$syl$" ../data/hyphenations6.dic; done > /tmp/rare
% grep -P "\s+1\s[^0-9]+$" /tmp/syls | tr -d '1' | sed 's/^\s\+//g' | while read syl; do grep "^$syl-\|-$syl-\|-$syl$" ../data/hyphenations5.dic; done >> /tmp/rare
% grep -P "\s+1\s[^0-9]+$" /tmp/syls | tr -d '1' | sed 's/^\s\+//g' | while read syl; do grep "^$syl-\|-$syl-\|-$syl$" ../data/hyphenations4.dic; done >> /tmp/rare
% grep -P "\s+1\s[^0-9]+$" /tmp/syls | tr -d '1' | sed 's/^\s\+//g' | while read syl; do grep --color=always "^$syl-\|-$syl-\|-$syl$" <(cat ../data/hyphenations6.dic ../data/hyphenations5.dic ../data/hyphenations4.dic); done | ansi2html > /tmp/rare.html


% fun-ci-o-ná-ri-o > fun-ci-o-ná-rio
% co-mí-ci-o > co-mí-cio
% con-vê-ni-o > con-vê-nio
% cri-té-ri-o > cri-té-rio
% ple-ná-ri-o > ple-ná-rio
% cre-di-á-ri-o > cre-di-á-rio
% es-tá-gi-o > es-tá-gio
% in-ter-ban-cá-ri-o > in-ter-ban-cá-rio
% ur-gên-ci-a > ur-gên-cia
% i-mo-bi-li-á-ri-a > i-mo-bi-li-á-ria
% re-vo-lu-ci-o-ná-ri-o > re-vo-lu-ci-o-ná-rio
% ve-ló-ri-o > ve-ló-rio
% es-có-ci-a > es-có-cia
% for-mu-lá-ri-o > for-mu-lá-rio
% or-ça-men-tá-ri-o > or-ça-men-tá-rio



% ag-nós-ti-co(5),a-gnós-ti-co(1)   << ?
% co-i-bir(5),coi-bir(1)            << both are correct?
% drui-da(5),dru-i-da(1)            << drui-da is correct, since an hiatus would require an accent dru-í-da
% flui-do(5),flu-i-do(1)            << as there is no diacritic it should be flui-do
%					fluido (sem acento) - substantivo: flui-do (ditongo) -- exemplo: O fluido de freio acabou.
%					fluido (com acento) - forma verbal: flu-í-do (hiato) -- exemplo: A discussão para aprovar reformas no país não tem fluído como esperavam os governantes.
% sub-li-nha(5),su-bli-nha(1)       << sub-linha or su-blinha (sublinhar) 
% man-ti-do(5) mantido(1)           << mantido clearly was not hyphenated 
% quart-zo(5),quar-tzo(1)           << quar-tzo sounds like the correct choice
% den-ti-frí-ci-o(5),den-ti-frí-cio(1) << it should be marked as apparent proparoxytones but no one did


%compare_strings() {
%    # Trim leading and trailing whitespace from the first string
%    string1_trimmed="${1#"${1%%[![:space:]]*}"}"
%    string1_trimmed="${string1_trimmed%"${string1_trimmed##*[![:space:]]}"}"
%
%    # Trim leading and trailing whitespace from the second string
%    string2_trimmed="${2#"${2%%[![:space:]]*}"}"
%    string2_trimmed="${string2_trimmed%"${string2_trimmed##*[![:space:]]}"}"
%
%    if [ "$string1_trimmed" == "$string2_trimmed" ]; then
%        return 0  # Strings are equal
%    else
%        return 1  # Strings are different
%    fi
%}

% grep "(5)" ../data/hyphagreements | grep "(1)" | head | tr -d '()0-9' | while read line; do word1=$(echo $line | cut -d, -f1); word2=$(echo $line | cut -d, -f2); count1=$(grep -o '-' <<< "$word1" | wc -l); count2=$(grep -o '-' <<< "$word2" | wc -l); if [ $count1 -gt $count2 ]; then w1=$(echo $word1 | sed 's/\(.*\)-/\1/'); compare_strings "$w1" "word2"; result=$?; if [ $result -ne 0 ]; then echo "c1: $line $w1 $word2"; od -c <<< "$w1"; od -c <<< "$word2"; fi; elif [ $count1 -lt $count2 ]; then w2=$(echo $word2 | sed 's/\(.*\)-/\1/'); compare_strings "$w2" "word1"; result=$?; if [ $result -ne 0 ]; then echo "c2: $line $word1 $w2"; fi; else compare_strings "$word1" "word2"; result=$?; if [ $result -ne 0 ]; then echo $line; fi; fi; done


After completing the aforementioned procedures to create our golden hyphenation
dictionary, manual corrections may still be necessary to ensure its accuracy
and reliability. These manual adjustments, which encompass the corrections
described earlier, are documented in the file \texttt{replacements.txt}.
Utilizing the script \texttt{manualcorrections.sh}, we meticulously apply these
corrections to refine our hyphenation dictionary and enhance its effectiveness.



% results
% ./gocreateresultfile.sh ../data/hyphenations6.dic  ../data/default.TeX.pt-br.patterns > ../data/results_hyph6_dfpatt



