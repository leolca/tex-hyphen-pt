\section{Hyphenation rule patches}\label{sec-hyph-rpatches}
We systematize the set of new rules in this section, and provide a few examples for
each instance. These rules are intended as a complement for the default \TeX{} hyphenation
rules created by \cite{hyphpt}. 


\begin{rules}
\refstepcounter{numberRulesGroups}\item\label{rulegrp_gno} rule: \texttt{.g2no}, \texttt{.g2nó}, \texttt{.g2nô} -- \emph{gnomo}, \emph{gnóstico}, \emph{gnômon}

\refstepcounter{numberRulesGroups}\item\label{rulegrp_tc} rule: \texttt{t2c} -- \emph{tchau}, \emph{tcheco}

\refstepcounter{numberRulesGroups}\item\label{rulegrp_pneu} rule: \texttt{1p2neu} -- \emph{pneumonia}, \emph{pneumotórax}, \emph{pneumático}, \emph{hidropneumático}

\refstepcounter{numberRulesGroups}\item\label{rulegrp_tm} rule: \texttt{.t2m} -- \emph{tmese}
   
\refstepcounter{numberRulesGroups}\item\label{rulegrp_pt} rule: \texttt{.p2t} -- \emph{ptose}, \emph{pterossauro}

\refstepcounter{numberRulesGroups}\item\label{rulegrp_mn} rule: \texttt{.m2n} -- \emph{mnemônico}

\refstepcounter{numberRulesGroups}\item\label{rulegrp_cza} rule: \texttt{c2za} -- \emph{czar}

\refstepcounter{numberRulesGroups}\item\label{rulegrp_s2} rule: \texttt{.s2} -- \emph{stalinismo}

\refstepcounter{numberRulesGroups}\item\label{rulegrp_t2} rule: \texttt{.t2} -- \emph{tsunami}, \emph{tzarista}

\refstepcounter{numberRulesGroups}\item\label{rulegrp_psi} rule: \texttt{.p2si}, \texttt{.p2sí} -- \emph{psicologia}, \emph{psíquico} % update (word-initial pattern, exception not necessary (?)) 
%\item\label{rulegrp_psi} regra: \texttt{p2si}, \texttt{p2sí} -- psicologia, psíquico\\
%exceção: \texttt{p3sia}, \texttt{p3sin} -- epilepsia, rodopsina

\refstepcounter{numberRulesGroups}\item\label{rulegrp_sub} rule: \texttt{su2b3r}, \texttt{su2b3l} -- \emph{sublunar}, \emph{subrotina} \\
exception: \texttt{.su3b4li} -- \emph{sublinhar}, \emph{sublimar}
%exceção: \texttt{su3b4li1nh}, \texttt{su3b4li1ma}, \texttt{su3b4li1me}, \texttt{su3b4li1mid} -- sublinhar, sublimar, sublime, sublimidade

\refstepcounter{numberRulesGroups}\item\label{rulegrp_neo} rule: \texttt{.ne4o} -- \emph{neoliberal}, \emph{neonazista}

\refstepcounter{numberRulesGroups}\item\label{rulegrp_pseu} rule: \texttt{1p2seu1d} -- \emph{pseudônimo}

\refstepcounter{numberRulesGroups}\item\label{rulegrp_qu} rule: \texttt{1qu} -- \emph{enquanto}, \emph{inquieto}, \emph{farquhar}, \emph{qubit}

\refstepcounter{numberRulesGroups}\item\label{rulegrp_air} rule: \texttt{a1ir.}, \texttt{u1ir.} -- \emph{sair}, \emph{extrair}, \emph{diminuir}, \emph{incluir}

\refstepcounter{numberRulesGroups}\item\label{rulegrp_ain} rule: \texttt{a1ind}, \texttt{a1i1nh} -- \emph{ainda}, \emph{rainha}

\refstepcounter{numberRulesGroups}\item\label{rulegrp_eim} rule: \texttt{e1imp} -- \emph{reimpresso}, \emph{teleimpressor}%, coimbra, coimperador
%exceção: \texttt{que2i}, \texttt{te2im} -- queimar, queimadura, teimosia, tirateima

\refstepcounter{numberRulesGroups}\item\label{rulegrp_ein} rule: \texttt{e1inc}, \texttt{e1inf}, \texttt{e1ing}, \texttt{e1ins}, \texttt{e1int}, \texttt{e1inv} -- \emph{reincidência}, \emph{reinfecção}, \emph{reingressa}, \emph{reinserção}, \emph{reintegração}, \emph{reinventar} 
%exceção: \texttt{e1in2st}, \texttt{e1in2sc} -- reinstalado, reinscrever

%\item\label{rulegrp_unit} rule: \texttt{u1int}, \texttt{u1ind} -- \emph{contribuinte}, \emph{excluindo}
%não precisa, temos a regra u1in

\refstepcounter{numberRulesGroups}\item\label{rulegrp_aiz} rule: \texttt{u1iz.}, \texttt{a1iz.} -- \emph{juiz}, \emph{raiz}

\refstepcounter{numberRulesGroups}\item\label{rulegrp_proi} rule: \texttt{pro1i1b} -- \emph{proibição}

\refstepcounter{numberRulesGroups}\item\label{rulegrp_Vi} rule: \texttt{tu1i}, \texttt{bu1i}, \texttt{nu1i}, \texttt{o1im}, \texttt{o1in}, \texttt{u1in}, \texttt{su1i}, \texttt{í1e}, \texttt{ju1i}, \texttt{fu1i}, \texttt{du1i}, \texttt{do1im}, \texttt{au1i}, \texttt{u1i1ç} -- \emph{intuitivo}, \emph{contribuidor}, \emph{ingenuidade}, \emph{coimbra}, \emph{coincide}, \emph{ruindade}, \emph{suicida}, \emph{píer}, \emph{juizado}, \emph{fuinha}, \emph{assiduidade}, \emph{amendoim}, \emph{cacauicultor}, \emph{constituição} \\
exception: \texttt{tu2id}, \texttt{tu2it}, \texttt{co2ima}, \texttt{o2i1na} -- \emph{gratuidade}, \emph{intuito}, \emph{coima}, \emph{boina}

\refstepcounter{numberRulesGroups}\item\label{rulegrp_aa} rule: \texttt{a1â}, \texttt{a1ã}, \texttt{a1é}, \texttt{a1í}, \texttt{a1ó}, \texttt{a1ô}, \texttt{a1ú}, \texttt{e1á}, \texttt{e1â}, \texttt{e1ã}, \texttt{e1é}, \texttt{e1ê}, \texttt{e1í}, \texttt{e1ó}, \texttt{e1ô}, \texttt{e1ú}, \texttt{é1o}, \texttt{i1á}, \texttt{i1ã}, \texttt{i1é}, \texttt{i1í}, \texttt{i1ó}, \texttt{i1u}, \texttt{i1ú}, \texttt{í1a}, \texttt{í1o}, \texttt{o1á}, \texttt{o1ã}, \texttt{o1é}, \texttt{o1ê}, \texttt{o1í}, \texttt{o1ó}, \texttt{u1á}, \texttt{u1ã}, \texttt{u1â}, \texttt{u1é}, \texttt{u1ê}, \texttt{u1í}, \texttt{ú1o} --  \emph{abraâmico}, \emph{abraão}, \emph{aéreo}, \emph{país}, \emph{caótico}, \emph{faraônico}, \emph{saúde}, \emph{balneário}, \emph{oceânico}, \emph{campeã}, \emph{feérico}, \emph{preênsil}, \emph{veículo}, \emph{teórico}, \emph{napoleônico}, \emph{conteúdo}, \emph{néon}, \emph{diário}, \emph{região}, \emph{soviético}, \emph{iídiche}, \emph{periódico}, \emph{feiura}, \emph{viúva}, \emph{maníaco}, \emph{íon}, \emph{razoável}, \emph{joão}, \emph{poético}, \emph{boêmia}, \emph{heroísmo}, \emph{alcoólico}, \emph{usuário}, \emph{itapuã}, \emph{lituânia}, \emph{suécia}, \emph{cauê}, \emph{suíça}, \emph{flúor} \\
exception: \texttt{1gu2á}, \texttt{1gu2ã}, \texttt{1gu2é}, \texttt{1gu2ê}, \texttt{1gu2í}, \texttt{1qu2á}, \texttt{1qu2ã}, \texttt{1qu2â}, \texttt{1qu2é}, \texttt{1qu2ê}, \texttt{1qu2í}, -- \emph{jaraguá}, \emph{saguão}, \emph{alguém}, \emph{português}, \emph{linguística}, \emph{aquático}, \emph{camaquã}, \emph{equânime}, \emph{inquérito}, \emph{sequência}, \emph{química} 
% the inclusion of rules e1é, é1o, i1í and í1a cause problem and if exceptions are necessary

%\item\label{rulegrp_gua} rule: \texttt{1gu2á}, \texttt{1gu2ã}, \texttt{1qu2ã} -- \emph{jaraguá}, \emph{saguão}, \emph{quão}

\refstepcounter{numberRulesGroups}\item\label{rulegrp_bo} rule: \texttt{1bô}, \texttt{1cô}, \texttt{1çô}, \texttt{1dô}, \texttt{1fô}, \texttt{1gô}, \texttt{1lô}, \texttt{1pô}, \texttt{1mô}, \texttt{1nô}, \texttt{1rô}, \texttt{1sô}, \texttt{1tô}, \texttt{1vô}, \texttt{1xô}, \texttt{1zô} -- \emph{robô}, \emph{recôncavo}, \emph{maçônico}, \emph{judô}, \emph{telefônica}, \emph{xangô}, \emph{camelô}, \emph{capô}, \emph{sumô}, \emph{econômico}, \emph{subsônico}, \emph{tarô}, \emph{chatô}, \emph{vovô}, \emph{saxônia}, \emph{amazônia}

\refstepcounter{numberRulesGroups}\item\label{rulegrp_Vf} rule: \texttt{4a.}, \texttt{4e.}, \texttt{4o.} -- \emph{secretária}, \emph{planície}, \emph{paratormônio}

  % co1in (parace não ser necessário, já que o1in faz o seu papel)
  % cu1i (não faze sentido, já que temos cu2i)
  % ra1is (excluir, só existe a arrais e me parece que a forma correta é ar-rais)

%rules not necessary
%\item\label{rulegrp_cui} rule: \texttt{cu2i} -- \emph{cuidado} \\
%exception: \texttt{cu3in} -- \emph{picuinha}


%\item\label{rulegrp_aui} regra: \texttt{au1i1c}, \texttt{du1i1c}, \texttt{u1i1ç}, \texttt{u1i1d}, \texttt{cu2i}, \texttt{dru2i}, \texttt{flu2id}, \texttt{bu1i1n}, \texttt{cu1i1n}, \texttt{fu1i1n}, \texttt{nu1i1n}, \texttt{ru1i1na}, \texttt{ru1i1no}, \texttt{su1i1ti}, \texttt{tu1i1ti}, \texttt{u1iz} -- cacauicultor, sanduicheira, constituição, continuidade, cuidador, druida, fluido, tabuinha, picuinha, fuinha, genuinamente, arruinado, arruinou, jesuitismo, intuitivo, juizado \\ 
%exceção: \texttt{cu3i1da1de}, \texttt{bu1i2n1d}, \texttt{bu1i2n1t}  -- acuidade, contribuindo, contribuinte
% not necessary

%\item\label{rulegrp_oin} regra: \texttt{o1in} -- coincidência, agroindustrial, bioinformática, amendoins, gastrointestinal
%exceção: \texttt{oi1na} -- boina
% already includede

%\item\label{rulegrp_zz} regra: \texttt{z1z}, \texttt{p1p}, \texttt{z2z.} -- pizza, shopping, jazz
% not necessary

%\item\label{rulegrp_so} regra: \texttt{1sô} -- subsônico
% already included above

%\item\label{rulegrp_ein} regra: \texttt{e1in1c}, \texttt{e1in1f}, \texttt{e1in1g}, \texttt{e1in1s}, \texttt{e1in1t}, \texttt{e1in1v} -- reincidência, reinfecção, reingressa, reinserção, reintegração, reinventar \\
%exceção: \texttt{e1in2st}, \texttt{e1in2sc} -- reinstalado, reinscrever

%\item\label{rulegrp_ia} regra: \texttt{í1a} -- baía
% already included above

%\item\label{rulegrp_oin} regra: \texttt{o1i1nh} -- moinho
% not necessary, we have o1in and 1n2h

%\item\label{rulegrp_aui} regra: \texttt{au1i1c}, \texttt{du1i1c}, \texttt{u1i1ç}, \texttt{u1i1d}, \texttt{cu2i}, \texttt{dru2i}, \texttt{flu2id}, \texttt{bu1i1n}, \texttt{cu1i1n}, \texttt{fu1i1n}, \texttt{nu1i1n}, \texttt{ru1i1na}, \texttt{ru1i1no}, \texttt{su1i1ti}, \texttt{tu1i1ti}, \texttt{u1iz} -- cacauicultor, sanduicheira, constituição, continuidade, cuidador, druida, fluido, tabuinha, picuinha, fuinha, genuinamente, arruinado, arruinou, jesuitismo, intuitivo, juizado \\ 
%exceção: \texttt{cu3i1da1de}, \texttt{bu1i2n1d}, \texttt{bu1i2n1t}  -- acuidade, contribuindo, contribuinte
% not necessary

%\item\label{rulegrp_iur} regra: \texttt{i1ur} -- diurno
% not necessary

%\item\label{rulegrp_uo} regra: \texttt{ú1o} -- flúor
% not necessary

%\item\label{rulegrp_iun} regra: \texttt{i1un} -- triunfar
% not necessary

%\item\label{rulegrp_quo} regra: \texttt{1qu4ó}, \texttt{1qu4â} -- quórum, equânime
% created rule u1â and 1qu2â
% three is no word with u-ó, so rule u1ó is not necessary, therefore 1qu2ó is also not necessary

%\item\label{rulegrp_eo} regra: \texttt{é3o} -- alvéolo
% rule é1o included 

%\item\label{rulegrp_ia} \st{regra: \texttt{i4a.}, \texttt{i4e.}, \texttt{i4o.}, \texttt{o4a.}, \texttt{u4a.} -- economia, espécie, vazio, destoa, institua}
% not necessary

%\item\label{rulegrp_lo} regra: \texttt{1lô} -- camelô

%\item\label{rulegrp_co} regra: \texttt{1cô} -- recôncavo

%\item\label{rulegrp_cco} regra: \texttt{1çô} -- maçônico
% included

% As sequências gráficas ia, ie, io, oa, ua quando em posição
% final átona, que são normalmente pronunciadas como ditongos
% crescentes, mas que podem corresponder foneticamente a duas
% vogais (hiato), sobretudo quando pronunciadas pausadamente:
% vitó-ria, espé-cie, exercí-cio, nó-doa, lé-gua, tê-nue, ambí-guo.
%\setcounter{numberRulesGroups}{\value{enumi}}
\end{rules}


% script to extract the rules above (inside \texttt{ })
% head -n 142 ../article/hyphenpatches.tex | grep -v "^%" | grep -oP '\\texttt\{.*?\}' | sed 's/\\texttt{//; s/}//' 
% meld <(head -n 142 ../article/hyphenpatches.tex | grep -v "^%" | grep -oP '\\texttt\{.*?\}' | sed 's/\\texttt{//; s/}//' | sort) <(sort ../data/patch.TeX.pt-br.patterns.v2)

The \NumberOfNewRules{} rules were grouped above in a list of
\thenumberRulesGroups\ types of rules. They may be further organized into five
large groups. The first, which comprises rules %\ref{rulegrp_gno} to \ref{rulegrp_t2}, 
\cref{rulegrp_gno,rulegrp_tc,rulegrp_pneu,rulegrp_tm,rulegrp_pt,rulegrp_mn,rulegrp_cza,rulegrp_s2,rulegrp_t2},
includes consonant clusters such as \emph{czar}, \emph{ptose}
and \emph{gnomo}. 
%They, unlike the examples that will be exposed in section \ref{secxxx} (exceptions - item 9 currently), present a set of derived words,
%which makes their marking advantageous in view of the number of cases that are included in this marking rule. 
The second group, comprising rules
%\ref{rulegrp_psi} to \ref{rulegrp_eim}, 
\cref{rulegrp_psi,rulegrp_sub,rulegrp_neo,rulegrp_pseu},
delimits the morphological boundary
between prefixes and radicals. As noted, although phonological issues guide the
separation of numerous words in Portuguese, there are also those that are
guided by morphology. This is the case of words that have the prefixes
\emph{sub-} and \emph{re-}, such as \emph{sublunar} and \emph{reinserção}. The
third group, comprising rules \cref{rulegrp_qu,rulegrp_air,rulegrp_ain,rulegrp_eim,rulegrp_ein,rulegrp_aiz,rulegrp_proi,rulegrp_Vi,rulegrp_aa}, seeks to
understand a set of words that have vowel combinations that do not follow the
general rules. This is because the Portuguese language has vowel encounters
with the second vowel graphically marked that can be separated, forming
hiatuses, such as \emph{caótico}, \emph{balneário} and \emph{razoável}, while
there are also words with a similar structure that constitute a diphthong, such
as \emph{português}, \emph{alguém} and \emph{linguística}. It is remarkable, of
course, that the latter are formed by the digraphs \emph{qu-} and \emph{gu-},
while the former by vowels other than \emph{i} and \emph{u}. The fourth group, 
in turn, which comprises \cref{rulegrp_aa,rulegrp_bo}\footnote{Note that \cref{rulegrp_aa} is in the intersection
between the third and fourth group of rules.}, 
which are counterparts of rules that were already in the
default rules, but did not contemplate the cases with certain accents. They
were then added to encompass words such as \emph{camelô}, \emph{recôncavo},
\emph{amazônia}, and \emph{maçônico}. 
The fifth, and last group, has just a single instance, \cref{rulegrp_Vf},
which represents our choice in how to deal the apparent proparoxytones,
avoiding a final hyphenation with the vowels \emph{a}, \emph{e}, or \emph{o}.

%It is important to highlight that the words included in these rules are, in
%general, of low frequency and were incorporated into the Portuguese language
%without a phonotactic adaptation, which causes these idiosyncrasies and
%exceptions to the language. They are characterized as marked cases, since it is
%not possible to defend that they reproduce a phonological pattern of the
%language because they are not productive, that is, they are not taken as an
%example and/or derive new words.


The set made of \NumberOfNewRules{} rules is presented below:
\begin{multicols}{6}
    \footnotesize
    \verbatiminput{../data/patch.TeX.pt-br.patterns}
\end{multicols}

\Cref{tab-resultados} presents the hyphenation errors in a set of 876 words.
We have omitted those words where both rules let to a correct hyphenation and
those that have incorrect hyphenation in a vowel in the last position. 
Once again, we are using the \emph{patgen} notation to mark correct, incorrect,
and missing hyphenation points: \texttt{*}, \texttt{.}, and \texttt{-}, respectively.
Incorrect or missing hyphenation points are considered errors. In these cases,
we place mark \xmark. Correct hyphenations are marked with \cmark.
The default rules achieved 17 correct hyphenations in this set, against 811 
of the patched rules.
%A Tabela \ref{tab-resultados} apresenta os erros cometidos no conjunto das
%37798 palavras analisadas. As palavras omitidas são aquelas que foram
%hifenizadas corretamente pelos dois conjuntos de regra. Utilizamos aqui a
%marcação adotada pelo \emph{Patgen}: \texttt{*} para indicar pontos de
%hifenização corretamente marcados, \texttt{.} para indicar pontos de hifenização
%marcados erroneamente e \texttt{-} para indicar pontos de hifenização não
%encontrados. 
%Os pontos marcados erroneamente e aqueles não encontrados são
%considerados erros. Quando ocorre algum erro de hifenização, marcaremos a
%palavra com \xmark. Quando não houver erro algum, marcaremos a palavra com
%\cmark. Ao final, podemos verificar que as regras padrão apresentaram 1368 erros
%(3,62\%) e as novas regras 132 erros (0,35\%).



