\section{Portuguese spelling system}\label{sec-grammar-port}

Portuguese employs an alphabetical writing system, which means its spelling is
guided by phonological principles \parencite{cagliari2015}. The correlation between
spelling and pronunciation influences word hyphenation, as words are divided
into syllables based on the phonemic system. It is important to note that
different languages follow diverse principles for word division. For example,
English is primarily guided by morphological principles, evident in words like
\emph{walk-ing}, \emph{un-happy}, \emph{work-s}, and \emph{ear-ly}.
Additionally, other factors also impact word hyphenation in English, such as 
the distinction between long and short vowels, which function within the context
of open or closed syllable, respectively; and the presence of doubled consonants and
digraphs \parencite{lin2011,yavas2020}.
Although each language has its driving principles for the hyphenation process, 
multiple factors come into play, leading to various solutions in certain scenarios.
For instance, in Portuguese, the phonological principle would lead to \emph{hipe-rativo}, while 
the morphological principle would lead to \emph{hiper-ativo}. Both approaches seem valid
and are indeed found in online dictionaries\footnote{Looking at words with prefixes, 
it seems that even on those words, the phonological approach was predominant on hyphenation,
but we cannot tell if it is a byproduct of automatic hyphenation that might be used
to hyphenate words on online dictionaries.}.
Some rather rare words, such as \emph{hiperalgesia}, may not be subject to morphological 
influences. Since it has a low frequency of occurrence, the individual may not be
aware of its morphological components and therefore hyphenate it as \emph{hipe-ralgesia}.
The form \emph{hiper-algesia} could also be accepted, emphasizing its morphological constituents. 
Furthermore, there exist numerous exceptions that may be categorized into rules.
% https://english.stackexchange.com/questions/385/what-are-the-rules-for-splitting-words-at-the-end-of-a-line

% $ grep "^hiper[aeiou]" ../data/hyphenations.csv  | grep "-" | grep -o "hi-pe-r[aeiou]\|hi-per-[aeiou]" | sort | uniq -c
% 16 hi-pe-ra
%  1 hi-per-a
% 29 hi-pe-re
% 20 hi-pe-ri
%  2 hi-per-i
%  2 hi-pe-ro
%  2 hi-pe-ru

Merely stating that an orthographic system is guided by phonological issues
does not necessarily mean that its hyphenation rules directly mirror the
phonetic counterpart. This is notably apparent in Portuguese, where a strict
one-to-one correspondence between letters and sounds is not always observed.
The orthographic system operates according to its specific rules. For example,
consider consonant clusters that create a single sound (digraphs) in words like
%\emph{\underline{ch}ave}, 
\emph{a\underline{ch}ado},
\emph{i\underline{lh}a}, 
%\emph{\underline{sh}ampoo}, 
\emph{su\underline{sh}i},
\emph{ca\underline{rr}o}, and \emph{ma\underline{ss}a}.
While these digraphs are pronounced as one sound within a single syllable,
their representation in writing determines how they are divided. Specifically,
different consonants within a digraph must remain together, whereas identical
consonants are separated. As a result, we observe hyphenations like
%\emph{cha-ve}, 
\emph{a-cha-do},
\emph{i-lha}, and 
%\emph{sham-poo}, 
\emph{su-shi},
but \emph{car-ro} and \emph{mas-sa}. 

%When there is no space left in line for a whole word, it might be split in two
%using a hyphen as an indication of such procedure. 

In Portuguese, hyphenations are allowed on syllables boundaries and, in
general, follow phonological principles.  According to the Grammar
\parencite{cunha2016,bergstrom2011,cegalla2008}, some rules might still apply:
\begin{description}
    \item[Non-Splitting Rules]\  

\begin{enumerate}
    \item\label{rule-di-triphthong} diphthong or triphthong should not be split
	(e.g. \emph{m\underline{ui}-to}, \emph{Pa-ra-g\underline{uai}});
    \item\label{rule-unstressed} the sequences \emph{ia}, \emph{ie}, \emph{io}, \emph{oa}, \emph{ua},
	\emph{ue} and \emph{uo}, when in final unstressed position, should not be split 
	(e.g. \emph{gló-r\underline{ia}}, \emph{vi-tó-r\underline{ia}}, 
	\emph{cá-r\underline{ie}}, \emph{es-pé-c\underline{ie}}, 
	\emph{Má-r\underline{io}}, \emph{má-g\underline{oa}}, 
	\emph{ré-g\underline{ua}}, \emph{tê-n\underline{ue}}, 
	\emph{con-tí-g\underline{uo}}, \emph{am-bí-g\underline{uo}});
    \item\label{rule-c-clusters} consonant clusters starting a syllable should not be
	split (e.g. \emph{\underline{pn}eu-má-ti-co}, \emph{\underline{ps}i-có-lo-go}, 
	\emph{\underline{mn}e-mô-ni-co});
    \item\label{rules-digraphs-ns} the digraphs	\emph{ch}, \emph{lh}, \emph{nh}
	should not be split (e.g. \emph{ra-\underline{ch}ar},
	\emph{a-bro-\underline{lh}os}, \emph{ma-\underline{nh}ã};
    \item\label{rule-guqu} bigrams like \emph{gu} and \emph{qu} whose vowel \emph{u} is not 
	pronounced are never separated from the vowel or diphthong that follows it 
	(e.g. \emph{U-ru-\underline{guai}}, \emph{pe-\underline{que}});
    \item\label{rule-nasalization} since they are digraphs, a vowel and its following 
	nasalization marker (a graphic nasal consonant) should not be split (e.g. \emph{\underline{am}-bição},
	\emph{m\underline{an}-cha});
    \item\label{rule-decreasing} decreasing diphthongs should not be split (e.g. 
	\emph{\underline{ai}-ro-so}, \emph{c\underline{au}-te-la}, \emph{ca-d\underline{ei}-ra}, \emph{cha-p\underline{éu}},
	\emph{o-ra-ç\underline{ão}}), \emph{n\underline{oi}-te}, \emph{ca-la-b\underline{ou}-ço}, \emph{as-te-r\underline{ói}-de},
  \emph{re-tri-b\underline{ui}});
    \item\label{rule-rising} rising diphthong should not be split (e.g.
  \emph{his-tó-r\underline{ia}}, \emph{má-g\underline{oa}}, 
  \emph{sé-\underline{rio}}, \emph{gló-r\underline{ia}}, \emph{fre-q\underline{ue}n-te},
  \emph{pá-tr\underline{ia}});
    \item\label{rule-singlev} disyllables whose syllable has a single vowel should 
	not be split (e.g. \emph{\underline{a}to}, \emph{ru\underline{a}}, 
	\emph{\underline{ó}dio}, \emph{\underline{u}nha});
    \item\label{rule-orphan} words with more than two syllables, when divided, cannot 
	isolate a syllable composed of a single vowel (e.g. \emph{\underline{a}gos-to}, 
	\emph{la-go\underline{a}}, \emph{\underline{i}da-de});
\end{enumerate}

\item[Splitting Rules]\ 

%    \hspace{10em} \hbox to 5cm{\leaders\hbox to 10pt{\hss . \hss}\hfil} 
\begin{enumerate}[resume]
    \item\label{rule-hiatus} hiatus vowels and those vowel sequences where each vowel
	belongs to a different syllable should be split (e.g.
	\emph{sa\underline{-ú-}de}, \emph{ra\underline{-i-}nha}, \emph{d\underline{o-e}r},
	\emph{v\underline{o-o}s}), the same procedure is used splitting diphthongs
	in different syllables (e.g. \emph{c\underline{ai-ai}s}) or diphthong and
	vowel in different syllables (e.g. \emph{en-s\underline{ai-o}s});
    \item\label{rule-consonants} consonant sequences, when in different syllables, should
	be split (e.g. \emph{a\underline{f-t}a}, \emph{a\underline{b-d}i-car},
	\emph{re\underline{s-c}i-são}, \emph{a\underline{b-s}o-lu-to});
    \item\label{rule-digraphs} the following consonant digraphs should be split:
	\emph{rr}, \emph{ss}, \emph{mm}, \emph{nn}, \emph{sc}, \emph{sç} and
	\emph{xc} (e.g. \emph{te\underline{r-r}a}, \emph{pro-fe\underline{s-s}or},
  \emph{co-mu\underline{\emph{m-m}}en-te}, \emph{co\underline{n-n}os-co}\footnote{It is noteworthy that the forms \emph{conosco} and \emph{comummente} are not used in all varieties of Portuguese. For example, Europeans adopt these forms, while Brazilians do not.},
	\emph{de\underline{s-c}er}, \emph{cre\underline{s-ç}a}, \emph{e\underline{x-c}e-der}).
\end{enumerate}
\end{description}

Rules \ref{rule-singlev} and \ref{rule-orphan} are primarily aimed at ensuring
proper readability of the text, aligning with \TeX{} approach to deal with
widows and orphans. As mentioned in \Cref{sec-intro}, the variables
\verb|lefthyphenmin| and \verb|righthyphenmin| control the minimum length for
fragments of hyphenated words.  When those variables are set to values greater
than one, Rules \ref{rule-singlev} and \ref{rule-orphan} become fiddling rules
in \TeX{} hyphenation, and they could be disregarded. Notwithstanding, the full
hyphenation of words is useful, particularly in text-to-speech applications
\parencite{libossek2000,trogkanis2010}.

Additionally, it is advisable to refrain from
splitting disyllables consisting of four letters (e.g., \emph{para},
\emph{como}, \emph{cede}). This considerations also lead to more aesthetically
pleasing and intelligible text, and \TeX{}'s control of isolated fragments,
through the variables \verb|\lefthyphenmin| and \verb|\righthyphenmin|, already
address this issue.

%Note that rules \ref{rule-singlev} and \ref{rule-orphan} are fiddling rules in
%\TeX{} hyphenation, since \TeX{} already has directives to avoid hyphenated
%widows and orphans.  As mentioned in \Cref{sec-intro}, the variables
%\verb|lefthyphenmin| and \verb|righthyphenmin| are used to control the minimum
%length for fragments of hyphenated words.  For this reason, we will not take
%into account rules \ref{rule-singlev} and \ref{rule-orphan} during the analysis
%and development of the hyphenation rules.

In some situations a hyphen should be repeated at the start of the following
line. They are: 
\begin{enumerate}
\item cases where compound words using a hyphen are split across lines (e.g.,
    \emph{couve-/-flor}, \emph{ex-/-presidente}); and
\item cases where splitting a pronoun could result in a different meaning (e.g.,
    \emph{prazer de ver-/-me}\footnote{Possible conveyed meanings: 
    \emph{pleasure in seeing me} or \emph{worm's pleasure}.}). 
\end{enumerate}

%To ensure proper hyphenation and readability, certain guidelines should be
%followed. Avoiding a single vowel at the end or start of a line is crucial
%(e.g., \emph{san-guí-nea} should be split only on the provided hyphens,
%avoiding a split that leaves the final vowel \emph{a} alone at the start of a
%new line: \emph{sanguíne-a}). Additionally, it is advisable to refrain from
%splitting disyllables consisting of four letters (e.g., \emph{para},
%\emph{como}, \emph{cede}).  These considerations lead to more aesthetically
%pleasing and intelligible text.

Systematizing the rules that guide syllable boundaries and hyphenation in
Portuguese is a fundamental step to understand and improve the \TeX{}
hyphenation rules. To ensure the accuracy of hyphenation rules and effectively
compare their results, a hyphenation spelling dictionary serves as an
indispensable reference. This dictionary provides a comprehensive listing of
correctly hyphenated words, enabling the computation of the correctness of each
set of hyphenation rules. By consulting the hyphenation spelling dictionary,
one can verify whether the hyphenation patterns generated by a set of rules
align with established standards. This process involves analyzing how well the
hyphenation rules adhere to the accepted conventions of the language, ensuring
that they effectively segment words without compromising readability or
consistency. Moreover, by comparing the results obtained from different sets of
hyphenation rules against the entries in the spelling dictionary, we can assess
the efficacy and reliability of each approach. In section \ref{sec-dictionay},
we will define the specifications of this reference dictionary, outlining its
role in subsequent rule updates and rules generation.


