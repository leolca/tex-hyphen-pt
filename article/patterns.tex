\section{Patterns for \TeX{} hyphenation}\label{sec-patt4TeX} 

To simplify, if we consider only the Latin alphabet, with no diacritcs, the patterns used 
in \TeX{} hyphenation are of the form: \texttt{\^{}\textbackslash{}.?[0-9]?([a-z]+[0-9]?)+\textbackslash{}.?\$},
where we have described it using a regular expression\footnote{
    Regular expressions (regex) are powerful search patterns used in text processing to find, match, and manipulate strings of text. They are a fundamental tool in programming and are supported in many programming languages.
    The regex given here uses the Perl syntax and might be broken in the following
    parts: \texttt{\^{}} and \texttt{\$} mark the start and end of the string;
    \texttt{\textbackslash{}.?} specifies an optional period; \texttt{[0-9]?} allows for an optional single digit;
    \texttt{([a-z]+[0-9]?)+} matches sequences of lowercase letters interleaved with optional digits.
}.
One example of such pattern is \texttt{4z1z2}, which is composed of a sequence
of letters and numbers. In general, we use characters/symbols from the language
alphabet, along with Hindu-Arabic numbers, to express either hyphenation
facilitation or inhibition.  Odd numbers indicate a good hyphenation point,
whereas even numbers indicate a bad place to break.  The given example states
that the sequence has a good breaking point between the first and the second
\emph{z} and an hyphenation should be inhibited before the first \emph{z} and
after the second \emph{z}. For example, the hyphenation of the word
\emph{piz-za}, \emph{fiz-zle} and \emph{mez-zanine} use this rule, where we see
the hyphen placed between the two \emph{z}'s and no hyphen before, nor after
the \emph{z}'s. Patterns may also use period symbol (\emph{.}) to indicate word
boundaries. The pattern \texttt{.sh2} applies to beginning of words, implying
that the \emph{s} and the \emph{h} should stick together in beginning of a word
and an hyphenation should also be inhibited after the \emph{h}. For example,
this pattern is used in \emph{Sher-lock}.

Hyphenation rules are organized in levels, from 1 to 9, where odd numbers
represent hyphenating levels and even numbers represent inhibiting levels. Each
level works as an exception level of it predecessor. For example, the rule
\texttt{sh1er} indicate a good hyphenation point between the \emph{h} and the
\emph{e} in the sequence \emph{sher}. A rule at a higher level, as
\texttt{.sh2}, implies an exception to the lower level rule. When we see
\emph{sher}, in the beginning of a word, the rule \texttt{.sh2} applies and
hyphenation proposed by the lower level rule \texttt{sh1er} should be hindered.
That is the case in the hyphenation of the word \emph{Sher-lock}. The full
example is provided in \Cref{sherlockhyphenation}, where we might see all pertinent
English rules taking place in the hyphenation of \emph{Sher-lock}.

\noindent\begin{minipage}{\linewidth}
\begin{lstlisting}[language={}, caption={Example of rules applyied in the
hyphenation of the word \emph{Sherlock}. Example done using a port of \TeX{}'s
hyphenation algorithm to Go provided at
\url{https://github.com/speedata/hyphenation}.}, label=sherlockhyphenation]
   .   s   h   e   r   l   o   c   k   .
     0   0   2   |   |   |   |   |   |    .sh2
     0   2   0   |   |   |   |   |   |    s2h
     0   0   1   0   0   |   |   |   |    sh1er
     |   |   |   0   1   0   |   |   |    r1l
     |   |   |   0   3   0   4   |   |    r3lo4
     |   |   |   |   |   |   0   0   1    ck1
max: 0   2   2   0   3   0   4   0   1
final: s   h   e   r - l   o   c   k -
\end{lstlisting}
\end{minipage}

In summary, a pattern will consist of a string made of characters (from the language
alphabet) possibly with a number in between, expressing the
hyphenation/inhibition level. Occasionally word boundaries marker (the period)
is used at the pattern edges. When there is no number between characters in a pattern,
a zero is assumed, which means \emph{undefined} and no hyphenation point will
be suggest at that location.



