\section{Creating a new set of rules using Patgen}
In this section, we describe the process of creating a new set of hyphenation rules using \emph{patgen}.
This involves defining specific parameters, choosing a reference dictionary with correct hyphenations, and 
using a translation file tailored for \emph{patgen}. In the previous \Cref{sec-patgen} we have detailed 
how \emph{patgen} works.

A partial of the content of the translation file is shown in \Cref{lst-trafile}.

\begin{minipage}{\linewidth}
\lstinputlisting[language={}, numbers=left, numberstyle=\tiny, linerange={1-10}, basicstyle=\small\ttfamily, caption={Portuguese translation file.}, label={lst-trafile}]{../data/portuguese.tra}
\ldots
\lstinputlisting[language={}, numbers=left, numberstyle=\tiny, linerange={39-41}, firstnumber=39, basicstyle=\small\ttfamily]{../data/portuguese.tra}
\end{minipage}


%\emph{Patgen} uses a few parameters along its executions: 
%\begin{description}
%    \item[hyph\_start and hyph\_finish]: Two numbers between 1 and 9 (separated by a space), representing the desired pattern levels
%	in the final set of rules. Odd pattern levels are hyphenating levels and even pattern levels are inhibiting levels. 
%	Higher level numbers prevail over lower ones, creating exceptions, exceptions over exceptions, and so on... 
%	\texttt{hyph\_start} and \texttt{hyph\_finish} specify the first and last levels, respectively, to be considered during the rule creation process.
%
%    \item[pat\_start and pat\_finish]: Patterns at each level are chosen in order of increasing pattern length (usually starting with length 2). 
%	This is controlled by the parameters \texttt{pat\_start} and \texttt{pat\_finish} specified at the beginning of each level. 
%	These are the minimum and maximum lengths of patterns we are interesting in. Their values range between 1 and 15.
%
%    \item[good weight, bad weight, threshold]: Each level of patterns is tested over all words in the dictionary.
%	A pattern is only if it satisfies the following formula: $\alpha \times \#\text{good matches} - \beta \times \#\text{bad matches} \geq \eta$, where
%	$\alpha$ is the \texttt{good weight}, $\beta$ is the \texttt{bad weight} and $\eta$ is the \texttt{threshold}.
%
%\end{description}
%
%The syntax to run \emph{patgen} is described in \Cref{lst-patgen-syntax}
%\begin{lstlisting}[language={bash}, basicstyle=\small\ttfamily, caption={Syntax to run patgen.}, label={lst-patgen-syntax}]
%patgen dictionary_file initial_pattern_set output_file translation_file
%\end{lstlisting}

The parameters mentioned in \Cref{sec-patgen} to run \emph{patgen} are provided interactively during the execution of \emph{patgen}.
We might automatize this process using a script. The \Cref{lst-patgentest} presents the code block
responsible for passing these parameters to \emph{patgen}.
\lstinputlisting[language={bash}, linerange={76-83}, basicstyle=\small\ttfamily, breaklines=true, caption={Portuguese translation file.}, label={lst-patgentest}]{../scripts/patgentest.sh}
As an example of execution, when choosing \texttt{hyph\_start}=1, \texttt{hyph\_finish}=9, \texttt{pat\_start}=2, \texttt{pat\_finish}=5,
\texttt{good weight}=1, \texttt{bad weight}=1, and \texttt{threshold}=1, we achieve 99.97\%(86799) correct 
hyphenations, 0.01\%(12) incorrect hyphenations, and 0.03\%(25) missing hyphenations.
It produced a set of 799 rules, where just 123 are present in the handcrafted set of rules.

We have also used the approach used by \textcite{sojka2019} in creating the script \texttt{make-full-pattern}.
In this approach, the parameters are loaded from a file and might be distinct for each level.
We start from an empty set and create progressive higher order rules, always using the rules from the previous
step as the starting point at the next step.
Using this approach with the 8 level parameters they used to create hyphenation rules to German,
we achieved 99.98\%(86820) correct hyphenations, 0.01\%(9) incorrect hyphenations, and 0.00\%(4) missed hyphenations.
The result was a set of 593 rules, with only 69 in common with those handcrafted rules proposed earlier.


% patgen portuguese_utf8.dic portuguese.0.pat portuguese_utf8.out portuguese_utf8.tra

% $ cat ../data/hyphenations6.dic ../data/hyphenations5.dic ../data/hyphenations4.dic > ../data/portuguese.dic
% $ ./patgentest.sh -w "1 2 1" -h "1 9" -p "2 5"
% 86799 good, 12 bad, 25 missed
% 99.97 %, 0.01 %, 0.03 %
% 799 rules
% 123 lines in common with handcrafted rules

% $ ./patgentest.sh -w "1 1 1" -h "1 9" -p "2 5"
% 86791 good, 5 bad, 33 missed
% 99.96 %, 0.01 %, 0.04 %
% 704 rules
% 81 lines in common with handcrafted rules

% $ ./patgentest.sh -w "1 1 2" -h "1 9" -p "2 5"
% 86671 good, 93 bad, 153 missed
% 99.82 %, 0.11 %, 0.18 %
% 498 rules
% 72 lines in common with handcrafted rules

% make-full-pattern.sh DICTIONARY_FILE TRANSLATE_FILE PARAMETERS_FILE
%
% ./make-full-pattern.sh ../data/portuguese.dic ../data/portuguese.tra german8levels-orig.par
% 86820 good, 9 bad, 4 missed
% 100.00 %, 0.01 %, 0.00 %
% 593 rules
% 69 lines in common with handcrafted rules
%
% ./make-full-pattern.sh ../data/portuguese.dic ../data/portuguese.tra german8levels-orig.par ../data/patched-default.TeX.pt-br.patterns.v2
% 86818 good, 7 bad, 6 missed
% 99.99 %, 0.01 %, 0.01 %
% 1260 rules
% 421 lines in common with handcrafted rules
% 
% ./make-full-pattern.sh ../data/portuguese.dic ../data/portuguese.tra scannell.par
% 86815 good, 10 bad, 9 missed
% 99.99 %, 0.01 %, 0.01 %
% 1104 rules
% 44 lines in common with handcrafted rules
%
% ./make-full-pattern.sh ../data/portuguese.dic ../data/portuguese.tra scannell.par ../data/patched-default.TeX.pt-br.patterns.v2
% 86812 good, 7 bad, 12 missed
% 99.99 %, 0.01 %, 0.01 %
% 764 rules
% 429 lines in common with handcrafted rules

% ./apparentproparoxytonesfix.sh ../data/portuguese.dic
% ./make-full-pattern.sh ../data/portuguese_apparentproparoxytone.dic ../data/portuguese.tra german8levels-orig.par
% 86766 good, 9 bad, 4 missed
% 100.00 %, 0.01 %, 0.00 %
% 547 rules
% 69 lines in common with handcrafted rules 

% ./make-full-pattern.sh ../data/portuguese_apparentproparoxytone.dic ../data/portuguese.tra scannell.par
% 86767 good, 10 bad, 3 missed
% 100.00 %, 0.01 %, 0.00 % 
% 1065 rules
% 44 lines in common with handcrafted rules

% ./make-full-pattern.sh ../data/portuguese_apparentproparoxytone.dic ../data/portuguese.tra german8levels-orig.par ../data/patched-default.TeX.pt-br.patterns.v2
% 86764 good, 7 bad, 6 missed
% 99.99 %, 0.01 %, 0.01 %
% 1182 rules 
% 421 lines in common with handcrafted rules


% ./gocreateresultfile.sh ../data/portuguese.dic portuguese.out > /tmp/portuguese_patgen_001.dic;
% ./resultfilemetrics.sh -f /tmp/portuguese_patgen_001.dic;
% correct: 32447; wrong: 12 (12); missing: 25 (25)
% 32447 good, 12 bad, 25 missed
% 99.89 %, 0.04 %, 0.07 %

% $ cat <(printf "1 5\n") <(printf -- '2 5\n1 2 1\n%.0s' {1..5}) <(printf "y") | patgen ../data/portuguese.dic ../data/patched-default.TeX.pt-br.patterns.v2 portuguese.out.2 ../data/portuguese.tra
% 86803 good, 429 bad, 21 missed
% 99.98 %, 0.49 %, 0.02 %



% create a test set using words with three hyphenations (3) in dictionary
% grep "(3)" ../data/hyphagreements | cut -d, -f1 | grep -v "(4)" | sed 's/([0-9])//g' > ../data/hyphenations3.dic
% ./apparentproparoxytonesfix.sh ../data/hyphenations3.dic
% ../data/hyphenations3_apparentproparoxytone.dic
% ./gocreateresultfile.sh ../data/hyphenations3_apparentproparoxytone.dic pattern.final > /tmp/portuguese3patfinal.dic;
% ./resultfilemetrics.sh -f /tmp/portuguese3patfinal.dic;
% correct: 647; wrong: 47 (46,1); missing: 7089 (1899,2088,1588,946,423,122,21,1,1)
%
% ./gocreateresultfile.sh ../data/hyphenations3_apparentproparoxytone.dic output_w111_h19_p25 > /tmp/portuguese3w111_h19_p25.dic;
% ./resultfilemetrics.sh -f /tmp/portuguese3w111_h19_p25.dic
% correct: 7465; wrong: 150 (143,6,1); missing: 183 (181,2) 
%
% ./gocreateresultfile.sh ../data/hyphenations3_apparentproparoxytone.dic ../data/patched-default.TeX.pt-br.patterns.v2 > /tmp/portuguese3patchv2.dic
% ./resultfilemetrics.sh -f /tmp/portuguese3patchv2.dic
% correct: 7335; wrong: 42 (40,2); missing: 385 (381,4)


The results from the set of rules crafted by \emph{patgen} seems outstanding, but we must be 
skeptical of its generalization power. To inquire on this matter, we decided to retrieve the
list of words where just three hyphenations were found in the dictionaries. We have also dealt
with apparent proparoxytones, to be consistent in our approach.  
In this new set, the number of correct, incorrect and missing hyphenations for each set of rules
is given \Cref{tbl-comres-hy3}.
\begin{table}[h]
\centering
\caption{Compared results between two use cases of \emph{patgen} and the handcrafted rule set.}\label{tbl-comres-hy3}
\begin{tabular}{llll}
rule set          & correct & incorrect & missing \\
\hline
fixed parameters  & 7465    & 150       & 183 \\
make-full-pattern & 647     & 47        & 7089 \\
handcrafted       & 7335    & 42        & 385 
\end{tabular}
\end{table}
It is important to hightlight that, among those 385 missing hyphenations in the handcrafted rules,
278 cases correspond to the last hyphen, leaving a single vowel as a final syllable.
Comparing to the other set of rules, we have 7 (in 183, for fixed parameters case), and 257 (in 7089, 
for the \texttt{make-full-pattern} script).

Establishing this comparison leave us no doubt that our handcrafted rules achieved the best
performance, also retaining a small number of useful rules, what leads to good generalization.


% $ grep "-" /tmp/portuguese3patchv2.dic | grep "\-[aeiouáéíóúãõâêô]$" | wc -l
% 278
% $ grep "-" /tmp/portuguese3w111_h19_p25.dic | grep "\-[aeiouáéíóúãõâêô]$" | wc -l
% 7


